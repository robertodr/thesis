\chapter{Response Theory and Molecular Properties}\label{ch:molprop}

\begin{epigraphs}
\qitem{
White light goin' messin' up my brain

White light it's gonna drive me insane
}{
--- \textsc{The Velvet Underground}
}
\qitem{What's the frequency, Kenneth? [...]

          I never understood the frequency, uh-huh}{
          --- \textsc{R.E.M.}}
\end{epigraphs}

The scope of quantum chemistry is to understand and predict molecular
properties, but so far we have made no mention of properties other than
the energy.
The experimentalists' view of molecular systems is built mainly around the use
of spectroscopic techniques that explore the interaction of light and matter.
When a system is exposed to an external perturbing electromagnetic field, it will
respond with a detectable change in its properties.\autocite{Pedersen2012-il, Jaszunski2012-wy}
By "external" we mean an electromagnetic field which is "weak" when
compared to the electron--electron and electron--nuclei interactions in
the molecule.
\emph{Response theory} is the missing link between theory and
experiment. It allows to describe perturbation-induced changes in
molecular properties in terms of \emph{response functions}, built solely
by means of \emph{unperturbed} molecular states and energies.
Moreover, excitation energies and transition moments can be inferred
from response functions without recurring to an explicit modelling of
excited states.
Section \ref{sec:exact-response} will give a brief introduction to exact
state response theory for isolated molecules. The concepts of
quasienergy,\autocite{Christiansen1998-pe} variational perturbation
theory\autocite{Helgaker1992-ph} and pole-and-residue
analysis of response functions\autocite{Olsen1985-nr} will be introduced.
I will explicitly derive the \acrshort{SCF} response function for a
quantum/classical polarizable Hamiltonian in Section
\ref{sec:csm-response}. The derivation will employ the
density matrix-based, \acrshort{AO} formalism introduced by
\citeauthor{Thorvaldsen2008-sg} in \noparcite[ref.][]{Thorvaldsen2008-sg}.
This formalism is suitable for the derivation of arbitrary-order
response functions and their residues, also when perturbation-dependent
basis sets are considered.\autocite{Friese2015-kb, Friese2015-bu}
Such a general derivation for the \acrshort{PCM} is presented in
\paper{V}.

The response treatment of molecular properties has its roots in
time-dependent perturbation theory\autocite{Konishi2009-zb} and has been
continually developed in quantum chemistry for the past 30 years:
the exposition will be necessarily brief and incomplete, the reader is
referred to the cited works in the literature for a more detailed
presentation of the topic.\autocite{Olsen1985-nr, Helgaker1992-ph, Olsen1995-pf,
Christiansen1998-pe, Norman2011-ad, Helgaker2012-cz, Pawlowski2015-sq}

\section{Exact State Response Theory}\label{sec:exact-response}

General theory~\autocite{Olsen1985-nr, Helgaker1992-ph, Olsen1995-pf,
Christiansen1998-pe, Norman2011-ad, Helgaker2012-cz, Pawlowski2015-sq}
Solution strategy for response equations~\autocite{Saad2003-oa,
Saad2011-gm, Kauczor2011-rd, Malmqvist2013-vw}
Open-ended \acrshort{AO}-based \acrshort{SCF} response~\autocite{
Larsen2000-hj, Kjaergaard2008-hy, Thorvaldsen2008-sg,
Kristensen2008-hv, Ringholm2014-gx}
Open-ended \acrshort{SCF} residues~\autocite{Friese2015-kb}


\section{SCF Response Theory for Quantum/Classical Polarizable Hamiltonians}\label{sec:csm-response}

\acrshort{PCM}~\autocite{Cammi1994-qj, Cammi1996-wf, Cammi1996-vx,
Cammi1999-rb, Cammi2003-qy, Frediani2005-nc, Ferrighi2010-pm}
\acrshort{FQ} and \acrshort{FQ}/\acrshort{PCM}~\autocite{Lipparini2012-hx, Lipparini2012-tl, Lipparini2013-ud}
\acrshort{PE}~\autocite{Olsen2010-wa}
\acrshort{QM}/\acrshort{MM}/Continuum~\autocite{Steindal2011-ki, Caprasecca2012-ir}
Paper V~\autocite{pcm-openrsp}
