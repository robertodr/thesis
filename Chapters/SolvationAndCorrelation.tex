% mainfile: ../RobertoDiRemigioPhDThesis.tex
%************************************************
\chapter{Electron Correlation and Solvation}\label{ch:solvation-correlation}
\footnotetext{Potius sero quam numquam \\
\hspace*{15pt} --- \textsc{Titus Livius}, \\ \hspace*{15pt}
\textit{Ab Urbe Condita}}

I will present results on the formulation of quantum/classical
polarizable models when the quantum part of the system includes a
many-body description of electron correlation either by a \gls{MBPT},
\gls{CC} or hybrid approach.
The contents of this Chapter are the result of an ongoing collaboration
with prof. T. Daniel Crawford (Virginia Tech) and Dr. Andrew
C. Simmonett (NIH).

I will leverage the variational formulation of quantum/classical
polarizable Hamiltonians described in Section \ref{sec:variational}.
This is an element of novelty with respect to the existing literature.
I will show how a fully variational framework enables transparent
derivations of \acrshort{MBPT} and \acrshort{CC} theory, encompassing
previous work\autocite{Christiansen1999-tj, Cammi2009-gu,
Caricato2010-hx, Caricato2011-tx, Olivares_del_Valle1991-of,
Aguilar1991-vq, Olivares_del_Valle1991-tq, Olivares_del_Valle1993-xq,
Olivares_del_Valle1993-ra, Surjan1983-oc, Angyan1991-mr, Angyan1993-ay,
Angyan2008-nj, Nielsen2001-yl, Kongsted2003-py, Sneskov2010-dz,
Sneskov2011-jm, Schwabe2012-cf}
and solving some inconsistencies in the formulation of \acrshort{MBPT}
noted by \citeauthor{Angyan1995-co}.\autocite{Angyan1995-co,
Lipparini2009-io}

Sections \ref{sec:effective-cc-lagrangian} and \ref{sec:source-terms}
will introduce the effective \acrshort{CC} Lagrangian and the relevant
notation for the source terms in the classical energy functional,
respectively.
In Section \ref{sec:pcm-cc-models} I will derive the governing equation
for the effective Lagrangian, showing how previous work on
continuum\autocite{Christiansen1999-tj, Cammi2009-gu, Caricato2011-tx}
and explicit models\autocite{Kongsted2003-py, Sneskov2011-jm} can easily
be derived in the variational framework.

The second part of this Chapter will be devoted to the derivation of
\acrshort{MBPT} for quantum/classical polarizable Hamiltonians.
I will exploit the \acrshort{CC} ansatz for the wave function and formulate
\acrshort{MBPT} in terms of the effective \acrshort{CC} Lagrangian, an approach
guaranteeing term-by-term size-extensivity.

Iterative and noniterative approximations to the \acrshort{CCSD} and
\acrshort{CCSDT} models will be the subject of Section
\ref{sec:cc-approximate-quantum-classical} where the equations governing
the \acrshort{CC2} and \acrshort{CC3} models will be derived.
Eventually, I will describe schemes for the noniterative inclusion of
the effect of connected triples excitations on top of the
\acrshort{CCSD} wave function.
The symmetric and asymmetric (T) corrections will be derived.

\subsection*{Notation}

For ease of reference, I will repeat some of the notational conventions
introduced in Chapter \ref{ch:QM} here.
The cluster operator at truncation level $\mathcal{M}$ is:
\begin{equation}
  T = \sum_{u=1}^\mathcal{M} T_u =
  \sum_{u=1}^\mathcal{M}\sum_{\mu_u}\tamp{\mu_u}\cluster{\mu_u},
  \tag{\ref{eq:cluster} from Chapter \ref{ch:QM}}
\end{equation}
where $\tamp{\mu_u}$ and $\cluster{\mu_u}$ are the cluster amplitudes
and cluster operators for the $\mu_u$-th excited determinant at the
$u$-th excitation level, respectively.
The similarity transformed counterpart of the operator $O$ is denoted
by $\overline{O}$:
\begin{equation}
  \overline{O} =\expo{-T}O\expo{T}.
  \tag{\ref{eq:similarity-transformation} from Chapter \ref{ch:QM}}
\end{equation}
We recall that hermiticity of $O$ is not preserved by the similarity
transformation.
Manipulation of similarity transformed operators involves the use of
their \acrlong{BCH} expansions:
\begin{equation}
  \begin{aligned}
  \overline{O}
  &=
  O + \tilde{O} \\
  &=
  O + \BCHfirst{O}{T} + \BCHsecond{O}{T} + \ldots
  \end{aligned}
  \tag{\ref{eq:bch} from Chapter \ref{ch:QM}}
\end{equation}
since we are concerned with one- and two-electron operators only,
such expansions will truncate after the third and fifth terms,
respectively.\autocite{Helgaker2000-tz}
The similarity transformation by means of the singles cluster operator
only, \ie the $T_1$-transformation, will be denoted by a "check" on top
of the operator:
\begin{equation}
  \check{O} = \expo{-T_1}O\expo{T_1}.
  \tag{\ref{eq:t1-transformation} from Chapter \ref{ch:QM}}
\end{equation}
The $T_1$-transformation preserves the particle rank of the operator
$O$, see Appendix \ref{app:mathematical-results} for details.
Finally, the $u$-th excited component of the \acrlong{CC} left state
will be denoted as:
\begin{equation}
  \bra{\tbar{u}} = \sum_{\mu_u}\tbar{\mu_u}\bra{\mu_u},
  \tag{\ref{eq:left-state} from Chapter \ref{ch:QM}}
\end{equation}
where $\tbar{\mu_u}$ are Lagrange multipliers.

\section{Effective Coupled Cluster Lagrangian}
\label{sec:effective-cc-lagrangian}

Our purpose in this Section is to derive an \emph{effective} \acrshort{CC}
Lagrangian that takes into account the coupling of the correlated
electronic state with the classical polarizable
environment.
This can be readily obtained\autocite{Lipparini2016-mo} by coupling the
usual Lagrangian in Eq. \eqref{eq:cc-lagrangian} with the classical
polarization functional in Eq. \eqref{eq:pcm-mm-functional}.
Care must however be taken in the definition of the classical
variational degrees of freedom as to not erroneously include
contributions from the reference state in the subsequent correlated
treatment.

Once again, let us assume that the reference state is a closed-shell
\acrshort{HF} determinant. Further, we assume that the reference determinant
was optimized in the presence of the classical polarizable environment.
Then the polarization degree of freedom corresponding to the environment
satisfies:
\begin{equation}
  \V\p_\mathrm{HF} + \s_\mathrm{HF} = 0
\end{equation}
and corresponds to the \emph{reference reaction field}.
In the subsequent correlated treatment one can separate the reaction
field into reference and correlated components:
\begin{equation}
  \p^\mathrm{tot} = \p_\mathrm{HF} + \p
\end{equation}
and similarly for the source term:
\begin{equation}
  \begin{aligned}
    \s(\tamp{}, \tbar{})_\mathcal{M} &= \s_\mathrm{HF} + \s_{\mathrm{N}}(\tamp{}, \tbar{})_\mathcal{M} \\
                     &=
    \braket{\mathrm{HF} | \s | \mathrm{HF}}
  + \braket{\mathrm{HF} | \tilde{\s} | \mathrm{HF}}
  + \sum_{u=1}^{\mathcal{M}}\braket{\tbar{u} | \overline{\s} |
  \mathrm{HF}}.
  \end{aligned}
\end{equation}
The \acrshort{BCH} expansions of the source term has been rewritten as:
\begin{equation}\label{eq:source-BCH}
  \overline{\s} = \s + \tilde{\s} = \s + \commutator{\s}{T}
  + \frac{1}{2}\commutator{\commutator{\s}{T}}{T},
\end{equation}
taking advantage of the fact that $\s$ is a
nondiagonal, one-electron operators and hence its commutator
expansion truncates at the third term. Moreover, normal-ordering of the
operators has been introduced.\autocite{Crawford2000-ey, Shavitt2009-mr}
This is equivalent to applying a shift to the correlation part of the
source term one-electron operators to remove the reference source terms:
\begin{equation}
  \begin{aligned}
    \s_{\mathrm{N}}(\tamp{}, \tbar{})_\mathcal{M}
  &=
   \braket{\mathrm{HF} | \tilde{\s} | \mathrm{HF}}
  + \sum_{u=1}^{\mathcal{M}}\braket{\tbar{u} | \overline{\s} | \mathrm{HF}} \\
  &=
    \braket{\mathrm{HF} | \overline{\s} - \s_\mathrm{HF} | \mathrm{HF}}
  + \sum_{u=1}^{\mathcal{M}}\braket{\tbar{u} | \overline{\s} -
  \s_\mathrm{HF} | \mathrm{HF}}.
  \end{aligned}
\end{equation}

Eventually, the polarization functional can be rewritten as:
\begin{equation}
  \begin{aligned}
  U_\mathrm{pol}
  &=
  \frac{1}{2}({}^t\p_\mathrm{HF}+{}^t\p)\V(\p_\mathrm{HF}+\p) \\
  &+ ({}^t\p_\mathrm{HF}+{}^t\p)(\s_\mathrm{HF}+\s_{\mathrm{N}}(\tamp{},\tbar{})_\mathcal{M}) \\
  &=
    \frac{1}{2}{}^t\p_\mathrm{HF}\V\p_\mathrm{HF} + {}^t\p_\mathrm{HF}\s_\mathrm{HF}
  + \frac{1}{2}{}^t\p_\mathrm{HF}\V\p + \frac{1}{2}{}^t\p\V\p_\mathrm{HF} \\
  &+ {}^t\p_\mathrm{HF}\s_{\mathrm{N}}(\tamp{},\tbar{})_\mathcal{M}
  + {}^t\p\s_\mathrm{HF}
  + \frac{1}{2}{}^t\p\V\p + {}^t\p\s_{\mathrm{N}}(\tamp{},\tbar{})_\mathcal{M} \\
  &=
    U_\mathrm{pol}^\mathrm{ref}
  + \frac{1}{2}{}^t\p\V\p + {}^t\p\s_{\mathrm{N}}(\tamp{},\tbar{})_\mathcal{M}
  + {}^t\p_\mathrm{HF}\s_{\mathrm{N}}(\tamp{},\tbar{})_\mathcal{M}.
  \end{aligned}
\end{equation}
The reference polarization energy $U_\mathrm{pol}^\mathrm{ref}$ still
appears among the terms in the functional but will obviously not enter
in the optimization of the \acrshort{CC} wave function.
One can then write the effective Lagrangian as:
\begin{equation}
  \begin{aligned}
    \lag(\tamp{}, \tbar{}, \p)_\mathcal{M} &=
  \braket{\mathrm{HF} | \overline{H_0} | \mathrm{HF}}
  + \sum_{u=1}^{\mathcal{M}}\braket{\tamp{u} | \overline{H_0} | \mathrm{HF}} \\
  &+
  \frac{1}{2}{}^t\p\V\p +
  {}^t\p\s_{\mathrm{N}}(\tamp{},\tbar{})_\mathcal{M} \\
  &+ {}^t\p_\mathrm{HF}\s_{\mathrm{N}}(\tamp{},\tbar{})_\mathcal{M}
  + U_\mathrm{pol}^\mathrm{ref}
  \end{aligned}
\end{equation}
In accordance with previous work,\autocite{Cammi2009-gu,
Caricato2011-tx} the Hamiltonian $H_0$ is augmented with the
${}^t\p_\mathrm{HF}\s_{\mathrm{N}}(\tamp{},\tbar{})$ term to yield the
\acrlong{PTE} (\acrshort{PTE}) Hamiltonian -- $H$ -- and eventually the
\acrlong{PTED} (\acrshort{PTED}) Lagrangian:\autocite{Olivares_del_Valle1991-of,
Aguilar1991-vq, Olivares_del_Valle1991-tq, Olivares_del_Valle1993-xq,
Olivares_del_Valle1993-ra, Lipparini2009-io}
\begin{equation}\label{eq:pted-cc}
  \lag(\tamp{}, \tbar{}, \p)_\mathcal{M} =
  {}^\mathrm{PTE}\lag(\tamp{}, \tbar{})_\mathcal{M}
  + \frac{1}{2}{}^t\p\V\p +
  {}^t\p\s_{\mathrm{N}}(\tamp{},\tbar{})_\mathcal{M},
\end{equation}
where the \acrshort{PTE} Lagrangian has been introduced:
\begin{equation}
  {}^\mathrm{PTE}\lag(\tamp{}, \tbar{})_\mathcal{M}
  =
  \braket{\mathrm{HF} | \overline{H} | \mathrm{HF}}
  + \sum_{u=1}^{\mathcal{M}}\braket{\tbar{u} | \overline{H} | \mathrm{HF}}
  + U_\mathrm{pol}^\mathrm{ref}.
\end{equation}

As noted by \citeauthor{Cammi2009-gu} for the implicit \acrshort{PCM} model,
the \acrshort{PTE} Hamiltonian is computed in \acrshort{MP}-partitioned form by
employing the \enquote{solvated} orbitals and Fock matrix:
\begin{equation}
  H = F + \Phi.
\end{equation}
$F$ is a diagonal one-electron operator and $\Phi$ is the
fluctuation potential.
The simple nature of the Fock operator lets us re-express the \acrshort{PTE}
Lagrangian as follows:
\begin{equation}
  \begin{aligned}
  {}^\mathrm{PTE}\lag(\tamp{}, \tbar{})_\mathcal{M}
  &=
  G_0
  + \sum_{u=1}^{\mathcal{M}}\tbar{\mu_u}\denom{\mu_u}\tamp{\mu_u}
  + \braket{\mathrm{HF} | \overline{\Phi} | \mathrm{HF}} \\
  &+ \sum_{u=1}^{\mathcal{M}}\braket{\tbar{u} | \overline{\Phi} | \mathrm{HF}}
\end{aligned}
\end{equation}
where $G_0 = E_0 + U_\mathrm{pol}^\mathrm{ref}$ is the reference free
energy.

\section{Source Terms in the Classical Energy Functional}
\label{sec:source-terms}

Let us take a closer look at the source term in the classical energy
functional.
It is important to single out which terms contribute in the various
expectation values where the similarity-transformed sources Eq.
\eqref{eq:source-BCH} are involved.
This analysis is carried out in detail in Appendix \ref{sec:CC-expval}
and we report here the final results for the \gls{CCS}, \gls{CCSD} and
\gls{CCSDT} models.

For the \gls{CCS} model ($\mathcal{M} = 1$) one has:
\begin{equation}\label{eq:source-ccs}
  \begin{aligned}
    \s_{\mathrm{N}}(\tamp{}, \tbar{})_\mathrm{CCS} &=
  \braket{\mathrm{HF} | \BCHfirst{\s}{T_1} | \mathrm{HF}}
  + \braket{\tbar{1} | \check{\s} | \mathrm{HF}} \\
  &=
  \braket{\mathrm{HF} | \BCHfirst{\s}{T_1} | \mathrm{HF}}
  + \braket{\tbar{1} | \s | \mathrm{HF}} \\
  &+ \braket{\tbar{1} | \BCHfirst{\s}{T_1} | \mathrm{HF}}
  + \frac{1}{2}\braket{\tbar{1} | \commutator{\commutator{\s}{T_1}}{T_1} | \mathrm{HF}}
  \end{aligned}
\end{equation}
while for the \gls{CCSD} model ($\mathcal{M} = 2$):
\begin{equation}\label{eq:source-ccsd}
  \begin{aligned}
  \s_{\mathrm{N}}(\tamp{}, \tbar{})_\mathrm{CCSD} &=
  \s_{\mathrm{N}}(\tamp{}, \tbar{})_\mathrm{CCS}
  + \braket{\tbar{1} | \BCHfirst{\s}{T_2} | \mathrm{HF}} \\
  &+ \braket{\tbar{2} | \BCHfirst{\s}{T_2} | \mathrm{HF}}
  + \braket{\tbar{2} | \commutator{\commutator{\s}{T_1}}{T_2} | \mathrm{HF}}
  \end{aligned}
\end{equation}
Eventually, within the \gls{CCSDT} model ($\mathcal{M} = 3$) four more
terms are added:
\begin{equation}\label{eq:source-ccsdt}
  \begin{aligned}
    \s_{\mathrm{N}}(\tamp{}, \tbar{})_\mathrm{CCSDT} &=
  \s_{\mathrm{N}}(\tamp{}, \tbar{})_\mathrm{CCSD}
  + \braket{\tbar{2} | \BCHfirst{\s}{T_3} | \mathrm{HF}} \\
  &+ \braket{\tbar{3} | \BCHfirst{\s}{T_3} | \mathrm{HF}} \\
  &+ \frac{1}{2}
  \braket{\tbar{3} | \commutator{\commutator{\s}{T_2}}{T_2} |
  \mathrm{HF}} \\
  &+ \braket{\tbar{3} | \commutator{\commutator{\s}{T_1}}{T_3} |
  \mathrm{HF}}
  \end{aligned}
\end{equation}

\section{Governing Equations and Their Approximations}\label{sec:pcm-cc-models}

\begarrowthought{Differentiation of the effective Lagrangian in Eq.
\eqref{eq:pted-cc}}
with respect to the variational parameters $\tamp{\mu_i}$,
$\tbar{\mu_i}$ and $\p$ yields the \acrshort{PTED}-\acrshort{CC}
equations:
\begin{subequations}\label{eq:pted}
  \begin{align}
   \tampEq{\mu_q}(\tamp{}, \tbar{}, \p)  &=
   \denom{\mu_q}\tamp{\mu_q} + \braket{\mu_q | \overline{\Phi} | \mathrm{HF}}
   + {}^t\p\braket{\mu_q | \overline{\s} | \mathrm{HF}}
             = 0 \label{eq:pted-cc-amplitudes}\\
   \tbarEq{\mu_q}(\tamp{}, \tbar{}, \p)
    &=
    \denom{\mu_q}\tbar{\mu_q} +
    \braket{\mathrm{HF} | \commutator{\overline{\Phi}}{\cluster{\mu_q}} | \mathrm{HF}} +
    \sum_{u=1}^{\mathcal{M}}\braket{\tbar{u} |
    \commutator{\overline{\Phi}}{\cluster{\mu_q}} | \mathrm{HF}}
    \nonumber \\
    &+
    {}^t\p\braket{\mathrm{HF} | \commutator{\s}{\cluster{\mu_q}} | \mathrm{HF}}\delta_{\mu_q\mu_1}
    +
    {}^t\p\sum_{u=1}^{\mathcal{M}}\braket{\tbar{u} |
    \commutator{\s}{\cluster{\mu_q}} | \mathrm{HF}} \nonumber \\
    &+
    {}^t\p\sum_{u=1}^{\mathcal{M}}\braket{\tbar{u} | \commutator{\commutator{\s}{T}}{\cluster{\mu_q}} | \mathrm{HF}}
             = 0 \label{eq:pted-cc-multipliers}\\
    \Omega_\p(\tamp{}, \tbar{}, \p)
    &=
    \V\p + \s_{\mathrm{N}}(\tamp{}, \tbar{})_\mathcal{M} = 0
    \label{eq:pted-cc-polarization}
  \end{align}
\end{subequations}
A coupling of the amplitudes and multipliers equations is introduced by
the presence of the $\s_{\mathrm{N}}(\tamp{}, \tbar{})_\mathcal{M}$ expectation value
in the polarization equation.
This requires a proper macroiteration/microiteration self-consistency
scheme for its implementation.\autocite{Cammi2009-gu, Caricato2010-hx}
As a consequence, a single-point \acrshort{PTED}-\acrshort{CC} will suffer from a
$2\times$ prefactor in its computational cost with respect to an
\emph{in vacuo} \acrshort{CC} calculation.

\newthought{It is thus convenient to devise approximations} that are able to simplify the
Lagrangian and the governing equations by breaking or weakening the coupling.
While Eq. \eqref{eq:pted-cc-polarization} is \emph{directly} coupled to
the multipliers equation, since the source terms directly depends on
amplitudes and multipliers, it is only \emph{indirectly} coupled to the
amplitudes equation, where only the polarization enters.
Moreover, the leading terms in the correlated source operator
expectation value prominently involve the
singles cluster operator and the singles multiplier state.
Three approximations have currently been proposed: \gls{PTES}, \gls{PTE(S)}
and \gls{PTE}.

\subsection*{PTES Scheme}

The first approximate scheme truncates the source term to its
\acrshort{CCS} expectation value, yielding the effective
Lagrangian:\autocite{Caricato2011-tx, Krause2016-ee}
\begin{equation}\label{eq:ptes-cc}
  \begin{aligned}
  \hspace{-20pt}  {}^{\mathrm{PTES}}\lag(\tamp{}, \tbar{},
  \p)_\mathcal{M} &=
  {}^\mathrm{PTE}\lag(\tamp{}, \tbar{})_\mathcal{M} \\
  &+ \frac{1}{2}{}^t\p\V\p + {}^t\p\s_{\mathrm{N}}(\tamp{},
  \tbar{})_\mathrm{CCS}
\end{aligned}
\end{equation}
and the corresponding governing equations:
\begin{subequations}\label{eq:ptes}
  \begin{align}
   \tampEq{\mu_q}(\tamp{}, \tbar{}, \p)  &=
   \denom{\mu_q}\tamp{\mu_q} + \braket{\mu_q | \overline{\Phi} | \mathrm{HF}}
   + {}^t\p\braket{\mu_q | \check{\s} | \mathrm{HF}}\delta_{\mu_q\mu_1}
             = 0 \\
   \tbarEq{\mu_q}(\tamp{}, \tbar{}, \p)
    &=
    \denom{\mu_q}\tbar{\mu_q} +
    \braket{\mathrm{HF} | \commutator{\overline{\Phi}}{\cluster{\mu_q}} | \mathrm{HF}} +
    \sum_{u=1}^{\mathcal{M}}\braket{\tbar{u} |
    \commutator{\overline{\Phi}}{\cluster{\mu_q}} | \mathrm{HF}}
    \nonumber \\
    &+
    {}^t\p\braket{\mathrm{HF} | \commutator{\s}{\cluster{\mu_q}} | \mathrm{HF}}\delta_{\mu_q\mu_1}
    +
    {}^t\p\braket{\tbar{1} |
    \commutator{\s}{\cluster{\mu_q}} | \mathrm{HF}}\delta_{\mu_q\mu_1}
     \nonumber \\
    &+
    {}^t\p\braket{\tbar{1} |
    \commutator{\commutator{\s}{T_1}}{\cluster{\mu_q}} | \mathrm{HF}}\delta_{\mu_q\mu_1}
             = 0 \\
    \Omega_\p(\tamp{}, \tbar{}, \p)
    &=
    \V\p + \s_{\mathrm{N}}(\tamp{}, \tbar{})_\mathrm{CCS} = 0
    \label{eq:ptes-cc-polarization-amp}
  \end{align}
\end{subequations}
These equations are, however, still coupled due to the presence of the
multipliers in the \acrshort{CCS} source term in the polarization equation.
To break the coupling, we split the system of equations
\eqref{eq:ptes} into \emph{two} systems of equations, one for the
amplitudes:
\begin{subequations}\label{eq:ptes-amp}
  \begin{align}
   {}^{\mathrm{PTES}}\tampEq{\mu_q}(\tamp{}, \tbar{}, \p)  &=
   \denom{\mu_q}\tamp{\mu_q} + \braket{\mu_q | \overline{\Phi} | \mathrm{HF}}
   + {}^t\p\braket{\mu_q | \check{\s} | \mathrm{HF}}\delta_{\mu_q\mu_1}
             = 0 \label{eq:ptes-cc-amplitudes} \\
    {}^{\mathrm{PTES}}\Omega_\p(\tamp{}, \p)
    &=
    \V\p +
    \braket{\mathrm{HF} | \check{\s}_{\mathrm{N}} | \mathrm{HF}} = 0
  \end{align}
\end{subequations}
where we further approximate the source term
$\s_{\mathrm{N}}(\tamp{},\tbar{})_\mathrm{CCS}$ with its
multiplier-independent part:
\begin{equation}\label{eq:pte-s-source}
  \s_{\mathrm{N}}(\tamp{}, \tbar{})_\mathrm{CCS}
  \simeq
  \braket{\mathrm{HF} | \BCHfirst{\s}{T_1} | \mathrm{HF}}
  =
  \braket{\mathrm{HF} | \check{\s}_{\mathrm{N}} | \mathrm{HF}},
\end{equation}
and one for the multipliers:
\begin{subequations}\label{eq:ptes-mult}
  \begin{align}
   {}^{\mathrm{PTES}}\tbarEq{\mu_q}(\tamp{}, \tbar{}, \p)
    &=
    \denom{\mu_q}\tbar{\mu_q} +
    \braket{\mathrm{HF} | \commutator{\overline{\Phi}}{\cluster{\mu_q}} | \mathrm{HF}} +
    \sum_{u=1}^{\mathcal{M}}\braket{\tbar{u} |
    \commutator{\overline{\Phi}}{\cluster{\mu_q}} | \mathrm{HF}}
    \nonumber \\
    &+
    {}^t\p\braket{\mathrm{HF} | \commutator{\s}{\cluster{\mu_q}} | \mathrm{HF}}\delta_{\mu_q\mu_1}
    +
    {}^t\p\braket{\tbar{1} |
    \commutator{\s}{\cluster{\mu_q}} | \mathrm{HF}}\delta_{\mu_q\mu_1}
     \nonumber \\
    &+
    {}^t\p\braket{\tbar{1} |
    \commutator{\commutator{\s}{T_1}}{\cluster{\mu_q}} | \mathrm{HF}}\delta_{\mu_q\mu_1}
             = 0 \label{eq:ptes-cc-multipliers} \\
    {}^{\mathrm{PTES}}\Omega_\p(\tamp{}, \tbar{}, \p)
    &=
    \V\p + \s_{\mathrm{N}}(\tamp{}, \tbar{})_\mathrm{CCS} = 0
    \label{eq:ptes-cc-polarization-mult}
  \end{align}
\end{subequations}
where no truncation for the source term is introduced.
Both the amplitudes and multipliers equations are formally identical to
the \acrshort{PTED} equations \eqref{eq:pted} but are coupled to
approximate polarization equations.
For the amplitudes equation \eqref{eq:ptes-cc-amplitudes}, one computes
the polarization from a modified \acrshort{CCS} expectation value of
the source term. This adds a negligible $O(N^5)$ term to the iterative
solution of the equations.
The amplitudes equations are instead coupled to the polarization
obtained using the full \acrshort{CCS} expectation value of the source
term. This adds another $O(N^5)$ step to the iterative
procedure, which is, overall, negligible.
Let us notice that this scheme is not consistent with electrostatics, as
the amplitudes and multipliers are optimized in the presence of two
different polarizations.

\subsection*{PTE(S) Scheme}

The \acrshort{PTES} approximation can be brought one step further, by
truncating the \acrshort{CCS} expectation value to its multiplier-independent
term already in the effective Lagrangian.
The coupling between the amplitudes and multipliers is broken, while
still improving upon the reference reaction field with correlated
contributions.\autocite{Caricato2011-tx}
The effective Lagrangian would then read as:
\begin{equation}\label{eq:pte-s-cc}
    {}^{\mathrm{PTE(S)}}\lag(\tamp{}, \tbar{}, \p)_\mathcal{M} =
  {}^\mathrm{PTE}\lag(\tamp{}, \tbar{})_\mathcal{M}
  + \frac{1}{2}{}^t\p\V\p + {}^t\p\braket{\mathrm{HF} | \check{\s}_{\mathrm{N}} | \mathrm{HF}}
\end{equation}
which uncouples all equations:
\begin{subequations}
  \begin{align}
  {}^{\mathrm{PTE(S)}}\tampEq{\mu_q}(\tamp{}, \tbar{})  &=
   \denom{\mu_q}\tamp{\mu_q} + \braket{\mu_q | \overline{\Phi} | \mathrm{HF}}
   = 0 \\
   {}^{\mathrm{PTE(S)}}\tbarEq{\mu_q}(\tamp{}, \tbar{}, \p) &=
    \denom{\mu_q}\tbar{\mu_q} +
    \braket{\mathrm{HF} | \commutator{\overline{\Phi}}{\cluster{\mu_q}} | \mathrm{HF}} +
    \sum_{u=1}^{\mathcal{M}}\braket{\tbar{u} |
    \commutator{\overline{\Phi}}{\cluster{\mu_q}} | \mathrm{HF}}
    \nonumber \\
    &+
    {}^t\p\braket{\mathrm{HF} | \commutator{\s}{\cluster{\mu_q}} | \mathrm{HF}}\delta_{\mu_q\mu_1}
    = 0 \\
    {}^{\mathrm{PTE(S)}}\Omega_\p(\tamp{}, \tbar{}, \p)
    &=
    \V\p + \braket{\mathrm{HF} | \check{\s}_{\mathrm{N}} | \mathrm{HF}} = 0
    \label{eq:pte-s-polarization}
  \end{align}
\end{subequations}
The amplitude equations are formally equivalent to those derived in the
\acrshort{PTE} scheme, \emph{vide infra}.
The equations for the multipliers are also formally equivalent to the
\acrshort{PTE} equations, apart from the equation determining the
singles multipliers which features an additional term involving the
polarization:
${}^t\p\braket{\mathrm{HF} | \commutator{\s}{\cluster{\mu_q}} | \mathrm{HF}}\delta_{\mu_q\mu_1}$.
This term is not, however, coupled to the other equations given that
$\p$ is fixed once the singles amplitudes have been determined.

Practical implementation of the \acrshort{PTE(S)} scheme will solve the
\acrshort{PTE}-like amplitude equation, form the source term
$\braket{\mathrm{HF} | \check{\s}_\mathrm{N} | \mathrm{HF}}$, solve for
the polarization degrees of freedom and calculate the polarization
energy as a correction to the \acrshort{CC} energy.
Only when a first-order molecular property, such as the molecular
gradient, is needed, will the multipliers equations need to be solved.
In other words, only the $T_1$-dependent part of the \acrshort{CC} density is
used to define the classical sources.

\subsection*{PTE Scheme}

Complete neglect of the polarization functional leads to the most
drastic approximation to the full quantum/classical polarizable
effective Lagrangian, the \acrshort{PTE} scheme.
This is equivalent to differentiating the polarization-independent
\acrshort{PTE} effective Lagrangian:
\begin{subequations}
  \begin{align}
 {}^{\mathrm{PTE}}\tampEq{\mu_q}(\tamp{}, \tbar{}, \p)  &=
   \denom{\mu_q}\tamp{\mu_q} + \braket{\mu_q | \overline{\Phi} | \mathrm{HF}}
    = 0 \\
   {}^{\mathrm{PTE}}\tbarEq{\mu_q}(\tamp{}, \tbar{}, \p)
    &=
    \denom{\mu_q}\tbar{\mu_q} +
    \braket{\mathrm{HF} | \commutator{\overline{\Phi}}{\cluster{\mu_q}} | \mathrm{HF}} +
    \sum_{u=1}^{\mathcal{M}}\braket{\tbar{u} |
    \commutator{\overline{\Phi}}{\cluster{\mu_q}} | \mathrm{HF}}
    = 0
  \end{align}
\end{subequations}
The \acrshort{PTE} scheme naturally preserves the scaling of the underlying
\acrshort{CC} method, since no coupling between the \acrshort{CC} amplitudes and
multipliers equations is introduced.
The scheme only requires access to a
reference state optimized in the presence of the classical medium and is
thus readily implemented.
The \acrshort{PTE} model offers an efficient approximation to the full
\acrshort{PTED} model with a computational cost comparable to that of
\emph{in vacuo} \acrshort{CC}
theory.\autocite{Olivares_del_Valle1991-of, Begue2005-qn,
Hasegawa2007-jt}

However, the polarization included in the correlation treatment is the
same as for the reference determinant, an approximation which appears
questionable from the point of view of classical electrostatics. There
is, in fact, no relaxation of the reference reaction field due to the
correlated description of the electronic density.

\section[MBPT for Quantum/Classical Polarizable Hamiltonians]{
MBPT for Quantum/Classical Polarizable Hamiltonians}\label{sec:mbpt-polarizable}

We will now develop a perturbative many-body expansion of the effective
\acrshort{PTED}-\acrshort{CC} Lagrangian \eqref{eq:pted-cc}
The use of a \acrshort{CC} ansatz for the wave function ensures term-by-term
size-extensivity of the energy at all orders.\autocite{Helgaker2000-tz}
The fluctuation potential $\Phi$ is assumed as the perturbation, hence
it will be considered of order 1: $\Phi = O(1)$.
We still assume a closed-shell \acrshort{HF} reference function and
perturbation orders will be counted accordingly.
A superscript index in square brackets - $[i]$ -  will denote an $i$-th
order contribution.
We further assume that real wave functions are used.

The structure of the stationarity conditions,
Eqs. \eqref{eq:pted-cc-amplitudes} and \eqref{eq:pted-cc-multipliers},
already shows that $\tamp{\mu_u}^{[0]} = 0, \,\forall u$ and
$\tbar{\mu_u}^{[0]} = 0, \, \forall u$.
Thus cluster operators can be of order 1 and higher. We introduce the
following notation for the $i$-th order cluster operators and
multipliers states:
\begin{alignat}{2}
  T_u^{[i]} = \sum_{\mu_u}\tamp{\mu_u}^{[i]}\cluster{\mu_u},
  \quad&
  \bra{\tbar{u}^{[i]}} = \sum_{\mu_u}\tbar{\mu_u}^{[i]}\bra{\mu_u}
\end{alignat}

The \acrshort{PTED}-\acrshort{CC} equations are expanded in orders of the
perturbation and terms are collected order by order. The use of an
effective, variational Lagrangian implies the validity of the $2n+1$
rule for the amplitudes $\tamp{\mu_u}$ and polarization $\p$ and of the
$2n+2$ rule for the multipliers $\tbar{\mu_u}$.
We can thus derive energy corrections up to fifth order by means of the
amplitudes and polarization up to and including second order terms and
the multipliers up to and including second order.

The polarization equation couples to the multipliers and amplitudes
equations \emph{via} the source term.
Its perturbative expansion will be given as:
\begin{equation}\label{eq:source-term-pt}
  \s_{\mathrm{N}}(\tamp{},\tbar{})_\mathcal{M} =
  \s_{\mathrm{N}}^{[0]}
  + \s_{\mathrm{N}}^{[1]}
  + \s_{\mathrm{N}}^{[2]}
  + \ldots
\end{equation}
and correspondingly for the polarization:
\begin{equation}\label{eq:pol-eq-pt}
  \V\p^{[i]} + \s_{\mathrm{N}}^{[i]} = 0
\end{equation}

The source operator $\s$ will be considered as \emph{zeroth-order} in
the perturbation.
The order of the contributions to
$\s_{\mathrm{N}}(\tamp{},\tbar{})_\mathcal{M}$ will be solely determined
by the amplitudes and multipliers. From the structure of
\eqref{eq:pted-cc-polarization}, it is already clear that
$\s_{\mathrm{N}}(\tamp{},\tbar{})_\mathcal{M}$ is at least first order
in the fluctuation potential:
\begin{equation}
  \s_{\mathrm{N}}(\tamp{},\tbar{})_\mathcal{M} =
  \s_{\mathrm{N}}^{[1]}
  + \s_{\mathrm{N}}^{[2]}
  + \ldots
\end{equation}
which also implies:
\begin{equation}
  \V\p^{[0]} + \s_{\mathrm{N}}^{[0]} = 0 \Rightarrow  \p^{[0]} = 0
\end{equation}

\subsection{First order equations}\label{sec:first-order-pt}

\paragraph*{Amplitudes}
\begin{equation}\label{eq:1st-order-amp}
  \Omega_{\mu_q}^{[1]} = \denom{\mu_q}\tamp{\mu_q}^{[1]}
  + \braket{\mu_q | \Phi | \mathrm{HF}}
  + {}^t\p^{[1]}\braket{\mu_q | \s | \mathrm{HF}}\delta_{\mu_q\mu_1}
  = 0
\end{equation}
Explicitly considering the singles, doubles and triples excitation manifolds:
\begin{subequations}
  \begin{align}
  \Omega_{\mu_1}^{[1]} &= \denom{\mu_1}\tamp{\mu_1}^{[1]}
  + {}^t\p^{[1]}\braket{\mu_1 | \s | \mathrm{HF}}
  = 0 \label{eq:1st-order-amp-singles} \\
  \Omega_{\mu_2}^{[1]} &= \denom{\mu_2}\tamp{\mu_2}^{[1]}
  + \braket{\mu_2 | \Phi | \mathrm{HF}}
  = 0 \\
  \Omega_{\mu_3}^{[1]} &= \denom{\mu_3}\tamp{\mu_3}^{[1]} = 0
  \end{align}
\end{subequations}
Having chosen a closed-shell \acrshort{HF} as reference,
the term $\braket{\mu_1 | \Phi | \mathrm{HF}}$ is zero in the
first-order singles equation. However, due to the quantum/classical
coupling, there will still be nonzero first-order singles amplitudes.

\paragraph*{Multipliers}
\begin{equation}%\label{eq:1st-order-mult}
  \bar{\Omega}_{\mu_q}^{[1]} =
    \denom{\mu_q}\tbar{\mu_q}^{[1]}
    + \braket{\mathrm{HF} | \commutator{\Phi}{\cluster{\mu_q}} | \mathrm{HF}}
    + {}^t\p^{[1]}\braket{\mathrm{HF} |
    \commutator{\s}{\cluster{\mu_q}} | \mathrm{HF}}\delta_{\mu_q\mu_1} =0
\end{equation}
and expanding the commutators:
\begin{equation}\label{eq:1st-order-mult}
  \bar{\Omega}_{\mu_q}^{[1]} =
    \denom{\mu_q}\tbar{\mu_q}^{[1]}
    + \braket{\mathrm{HF} | \Phi | \mu_q}
    + {}^t\p^{[1]}\braket{\mathrm{HF} | \s | \mu_q}\delta_{\mu_q\mu_1} =0
\end{equation}
This clearly shows that Eq. \eqref{eq:1st-order-amp}
and Eq. \eqref{eq:1st-order-mult} are complex conjugates. Under the
assumption of real wave functions we can also conclude that:
\begin{equation}
  \tamp{\mu_u}^{[1]} = \tbar{\mu_u}^{[1]},\,\,\forall u
\end{equation}
Thus, as for the amplitudes, despite the closed-shell \acrshort{HF} reference, the singles
multipliers will already appear in first order, due to the
quantum/classical coupling.
Moreover, as is the case \emph{in vacuo}, there are no triples contributions to
the first order amplitudes and multipliers: $\tamp{\mu_3}^{[1]} = \tbar{\mu_3}^{[1]} = 0$.

\paragraph*{Polarization}
\begin{equation}%\label{eq:1st-order-pol}
  \V\p^{[1]} + \braket{\mathrm{HF} | \commutator{\s}{T_1^{[1]}}| \mathrm{HF}}
  + \braket{\tbar{1}^{[1]} | \s | \mathrm{HF} } = 0
\end{equation}
Under the assumption of real wave functions, the equivalence of
first order amplitudes and multipliers yields:
\begin{equation}
  \braket{\mathrm{HF} | \commutator{\s}{T_1^{[1]}}| \mathrm{HF}}
  = \braket{\tbar{1}^{[1]} | \s | \mathrm{HF}}
\end{equation}
so that the first order polarization equation becomes:
\begin{equation}\label{eq:1st-order-pol}
  \V\p^{[1]} + 2\braket{\mathrm{HF} | \commutator{\s}{T_1^{[1]}}| \mathrm{HF}} = 0
\end{equation}

\begin{comment}
\begin{equation}
  \begin{aligned}
  \braket{\mathrm{HF} | \commutator{\s}{T_1^{[1]}}| \mathrm{HF}}
  %%%% Steps
  &= \left/ \text{Expand commutator}, T_1^{[1]} = \sum_{\mu_1}\ket{\mu_1}\tamp{\mu_1} \right/
  =
  \sum_{\mu_1}\braket{\mathrm{HF}|\s|\mu_1}\tamp{\mu_1}^{[1]} \\
  %%%% Steps
  &= \left/ \text{Complex conjugation and real wave functions} \right/ \\
  &= \sum_{\mu_1}\tbar{\mu_1}^{[1]}\braket{\mu_1 | \s | \mathrm{HF}} \\
  %%%% Steps
  &= \left/
  \bra{\tbar{1}^{[1]}} = \sum_{\mu_1}\tbar{\mu_1}^{[1]}\bra{\mu_1} \right/
  = \braket{\tbar{1}^{[1]} | \s | \mathrm{HF}}
 \end{aligned}
\end{equation}
\end{comment}

\subsection{Second order equations}\label{sec:second-order-pt}

\paragraph*{Amplitudes}
\begin{equation}\label{eq:2nd-order-amp}
  \begin{aligned}
  \Omega_{\mu_q}^{[2]} &= \denom{\mu_q}\tamp{\mu_q}^{[2]}
  + \braket{\mu_q | \commutator{\Phi}{T^{[1]}} | \mathrm{HF}}
  + {}^t\p^{[2]}\braket{\mu_q | \s | \mathrm{HF}}\delta_{\mu_q\mu_1} \\
  &+ {}^t\p^{[1]}\braket{\mu_q | \commutator{\s}{T^{[1]}} |
  \mathrm{HF}}
  = 0
  \end{aligned}
\end{equation}
As already noted elsewhere,\autocite{Koch1997-nm, Helgaker2000-tz}
the second term in Eq. \eqref{eq:2nd-order-amp} can involve no higher
than triple excitations.
Moreover, the triples first appear to second order and are not
\emph{directly} affected by the quantum/classical coupling, as can be
seen by explicitly considering the singles, doubles and triples
excitation manifolds:
\begin{subequations}
  \begin{align}
  \Omega_{\mu_1}^{[2]} &= \denom{\mu_1}\tamp{\mu_1}^{[2]}
  + \braket{\mu_1 | \commutator{\Phi}{T^{[1]}} | \mathrm{HF}}
  + {}^t\p^{[2]}\braket{\mu_1 | \s | \mathrm{HF}} \nonumber \\
  &+ {}^t\p^{[1]}\braket{\mu_1 | \commutator{\s}{T^{[1]}} |
  \mathrm{HF}}
  = 0 \\
  \Omega_{\mu_2}^{[2]} &= \denom{\mu_2}\tamp{\mu_2}^{[2]}
  + \braket{\mu_2 | \commutator{\Phi}{T^{[1]}} | \mathrm{HF}}
  + {}^t\p^{[1]}\braket{\mu_2 | \commutator{\s}{T_2^{[1]}} |
  \mathrm{HF}}
  = 0 \\
  \Omega_{\mu_3}^{[2]} &= \denom{\mu_3}\tamp{\mu_3}^{[2]}
  + \braket{\mu_3 | \commutator{\Phi}{T_2^{[1]}} | \mathrm{HF}}
  = 0
  \end{align}
\end{subequations}

\paragraph*{Multipliers}
\begin{equation}\label{eq:2nd-order-mult}
  \begin{aligned}
  \bar{\Omega}_{\mu_q}^{[2]} &=
    \denom{\mu_q}\tbar{\mu_q}^{[2]}
    + \braket{\mathrm{HF} | \commutator{\commutator{\Phi}{T^{[1]}}}{\cluster{\mu_q}} | \mathrm{HF}}
    + \sum_{u=1}^\mathcal{M}\braket{\tbar{u}^{[1]} |
    \commutator{\Phi}{\cluster{\mu_q}} | \mathrm{HF}} \\
    &+ {}^t\p^{[2]}\braket{\mathrm{HF} |
    \commutator{\s}{\cluster{\mu_q}} | \mathrm{HF}}\delta_{\mu_q\mu_1}
    + {}^t\p^{[1]}
    \sum_{u=1}^\mathcal{M}\braket{\tbar{u}^{[1]} |
    \commutator{\s}{\cluster{\mu_q}} | \mathrm{HF}}
    =0
  \end{aligned}
\end{equation}
Since first-order singles amplitudes are now nonzero, the double
commutator term will contribute to the singles multipliers equation.
This is markedly in contrast with the derivation \emph{in vacuo}, see
Section \ref{sec:mbpt}.\autocite{Koch1997-nm, Helgaker2000-tz}
For the singles, doubles and triples manifolds the multipliers equations
to second order are given as:
\begin{subequations}
  \begin{align}
    \bar{\Omega}_{\mu_1}^{[2]} &=
      \denom{\mu_1}\tbar{\mu_1}^{[2]}
      + \braket{\mathrm{HF} |
      \commutator{\commutator{\Phi}{T_1^{[1]}}}{\cluster{\mu_1}} | \mathrm{HF}}
      + \sum_{u=1}^\mathcal{M}\braket{\tbar{u}^{[1]} |
      \commutator{\Phi}{\cluster{\mu_1}} | \mathrm{HF}} \nonumber \\
      &+ {}^t\p^{[2]}\braket{\mathrm{HF} |
      \commutator{\s}{\cluster{\mu_1}} | \mathrm{HF}}
      + {}^t\p^{[1]}
      \braket{\tbar{1}^{[1]} |
      \commutator{\s}{\cluster{\mu_1}} | \mathrm{HF}}
      =0 \\
    \bar{\Omega}_{\mu_2}^{[2]} &=
      \denom{\mu_2}\tbar{\mu_2}^{[2]}
      + \sum_{u=1}^\mathcal{M}\braket{\tbar{u}^{[1]} |
      \commutator{\Phi}{\cluster{\mu_2}} | \mathrm{HF}}
      + {}^t\p^{[1]}
      \sum_{u=1}^\mathcal{M}\braket{\tbar{u}^{[1]} |
      \commutator{\s}{\cluster{\mu_2}} | \mathrm{HF}}
      =0 \\
    \bar{\Omega}_{\mu_3}^{[2]} &=
      \denom{\mu_3}\tbar{\mu_3}^{[2]}
      + \sum_{u=1}^\mathcal{M}\braket{\tbar{u}^{[1]} |
      \commutator{\Phi}{\cluster{\mu_3}} | \mathrm{HF}}
      =0
  \end{align}
\end{subequations}
\citeauthor{Koch1997-nm} showed that, assuming real wave functions, the
\emph{in vacuo} second order amplitudes and multipliers equations are
the complex conjugates of each other for the singles, doubles and
triples manifolds.\autocite{Koch1997-nm, Helgaker2000-tz,
Shavitt2009-mr}
This symmetry is a consequence of choosing a closed-shell \acrshort{HF}
determinant as reference function\autocite{Shavitt2009-mr} and is
\emph{broken} by the quantum/classical coupling.
As an example, consider the triples multipliers equation. \emph{In vacuo}
one has:
\begin{equation}\tag{\ref{eq:2nd-order} from Chapter \ref{ch:QM}}
  \denom{\mu_3}\tamp{\mu_3}^{[2]}
  = - \braket{\mu_i | \commutator{\Phi}{T_2^{[1]}} | \mathrm{HF}}
  = - \braket{\tbar{2}^{[1]} | \commutator{\Phi}{\cluster{\mu_3}} |
  \mathrm{HF}}
  = \denom{\mu_3}\tbar{\mu_3}^{[2]},
\end{equation}
whereas now the first-order singles multiplier state also contributes to
the second-order triples multipliers.

\paragraph*{Polarization}
\begin{equation}\label{eq:2nd-order-pol}
  \begin{aligned}
  \V\p^{[2]} &+
  \braket{\mathrm{HF} | \commutator{\s}{T_1^{[2]}}| \mathrm{HF}}
  + \braket{\tbar{1}^{[2]} | \s | \mathrm{HF} } \\
  &+ \braket{\tbar{1}^{[1]} | \commutator{\s}{T^{[1]}} | \mathrm{HF}}
  + \braket{\tbar{2}^{[1]} | \commutator{\s}{T_2^{[1]}} | \mathrm{HF}}
  = 0
  \end{aligned}
\end{equation}
There are no explicit contributions from the triples and thus the
\acrshort{CCSD} polarization is correct through fourth order in the
fluctuation potential.

\subsection{Free energies up to fifth order}\label{sec:energies-pt}

From the expansion of the effective Lagrangian we can obtain free energy
corrections up to the desired order. Given the variational nature of the
Lagrangian, only terms fulfilling the $2n+1$ and $2n+2$ rules will
appear in the free energy corrections.
To zeroth-order all variational parameters are zero. To first order we
thus have:
\begin{equation}\label{eq:pted-mbpt-1st}
  G^{[1]} = \braket{\mathrm{HF} | \Phi | \mathrm{HF}}
\end{equation}
The reference energy can thus be recovered as usual:
\begin{equation}
  G_\mathrm{HF} = G^{[0]} + G^{[1]}
\end{equation}
where $G^{[0]}$ includes the quantum/classical coupling.

We introduce the following notation:
\begin{equation}
  \s_{\mathrm{N}}^{[m]}(\tamp{}^{[p]}, \tbar{}^{[q]}),
\end{equation}
for a source term of $m$-th order formed by amplitudes up to and
including $p$-th order and by multipliers up to and including $q$-th order.
Red boxes will appear around quantum/classical coupling terms that
involve the triples manifold.

The second order energy correction will be formed including first order
amplitudes and polarization, while using zeroth order multipliers:
\begin{equation}\label{eq:pted-mbpt-2nd}
  \begin{aligned}
  G^{[2]} &=
  \braket{\mathrm{HF} | \commutator{\Phi}{T^{[1]}} | \mathrm{HF}}
  + \frac{1}{2}{}^t\p^{[1]}\V\p^{[1]} +
  {}^t\p^{[1]}\s_{\mathrm{N}}^{[1]}(\tamp{}^{[1]}, \tbar{}^{[0]}) \\
  &= E^{[2]}
  +
  \frac{1}{2}{}^t\p^{[1]}\V\p^{[1]}
  +
  {}^t\p^{[1]}\braket{\mathrm{HF} | \commutator{\s}{T^{[1]}_1} | \mathrm{HF}}
  = E^{[2]}
  \end{aligned}
\end{equation}
The polarization terms were eliminated by virtue of Eq.
\eqref{eq:1st-order-pol}. Thus, at least formally, $G^{[2]}$ has the same expression as
\emph{in vacuo} and is consistent with \emph{both} electrostatics and
perturbation theory.
We remark that this would not be the case if one had based the
development of \acrshort{MBPT} on the use of a nonlinear
Hamiltonian.\autocite{Angyan1995-co}

The third order energy correction will be formed including first order
amplitudes, multipliers and polarization:
\begin{equation}\label{eq:pted-mbpt-3rd}
  \begin{aligned}
    G^{[3]} &=
    \sum_{u=1}^2\braket{\tbar{u}^{[1]} | \commutator{\Phi}{T^{[1]}} | \mathrm{HF}}
    + {}^t\p^{[1]}\s_{\mathrm{N}}^{[2]}(\tamp{}^{[1]}, \tbar{}^{[1]}) \\
    &=
    E^{[3]}
    + {}^t\p^{[1]}
    \left\lbrace
      \braket{\tbar{1}^{[1]} | \commutator{\s}{T^{[1]}} | \mathrm{HF}}
    + \braket{\tbar{2}^{[1]} | \commutator{\s}{T^{[1]}_2} | \mathrm{HF}}
    \right\rbrace
  \end{aligned}
\end{equation}

The fourth order energy corrections will be formed including second
order amplitudes and polarization, while using first order multipliers:
\begin{equation}\label{eq:pted-mbpt-4th}
  \begin{aligned}
    G^{[4]} &=
    \sum_{u=1}^2\braket{\tbar{u}^{[1]} | \commutator{\Phi}{T^{[2]}} | \mathrm{HF}}
    +
    \sum_{u=1}^2\braket{\tbar{u}^{[1]} |
    \frac{1}{2}\commutator{\commutator{\Phi}{T^{[1]}}}{T^{[1]}} | \mathrm{HF}}
    \\
    &+\frac{1}{2}{}^t\p^{[2]}\V\p^{[2]}
    + {}^t\p^{[2]}\s_{\mathrm{N}}^{[2]}(\tamp{}^{[2]}, \tbar{}^{[1]})
    + {}^t\p^{[1]}\s_{\mathrm{N}}^{[3]}(\tamp{}^{[2]}, \tbar{}^{[1]}) \\
    &= E^{[4]}
    + \frac{1}{2}{}^t\p^{[2]}\V\p^{[2]} \\
    &+{}^t\p^{[2]}
    \left\lbrace
    \braket{\mathrm{HF} | \commutator{\s}{T_1^{[2]}} | \mathrm{HF}}
    +
    \sum_{u=1}^2\braket{\tbar{u}^{[1]}|
    \commutator{\s}{T^{[1]}} | \mathrm{HF} }
    \right\rbrace \\
    &+{}^t\p^{[1]}
    \left\lbrace
    \sum_{u=1}^2\braket{\tbar{u}^{[1]}|
    \commutator{\s}{T^{[2]}} | \mathrm{HF}}
    +
    \sum_{u=1}^2\braket{\tbar{u}^{[1]}|
    \frac{1}{2}
    \commutator{\commutator{\s}{T^{[1]}}}{T^{[1]}}
    |
    \mathrm{HF}}
    \right\rbrace \\
  &= E^{[4]}
    + \frac{1}{2}{}^t\p^{[2]}\V\p^{[2]}
    + {}^t\p^{[2]}
      \braket{\mathrm{HF} | \commutator{\s}{T_1^{[2]}} | \mathrm{HF}} \\
    &+ {}^t\p^{[2]}
      \braket{\tbar{1}^{[1]}| \commutator{\s}{T^{[1]}} | \mathrm{HF} }
    + {}^t\p^{[2]}
       \braket{\tbar{2}^{[1]}| \commutator{\s}{T_2^{[1]}} | \mathrm{HF} } \\
    &+
    {}^t\p^{[1]}
    \braket{\tbar{1}^{[1]} | \commutator{\s}{T_1^{[2]} + T_2^{[2]}} | \mathrm{HF}}
    +
    {}^t\p^{[1]}
    \braket{\tbar{2}^{[1]} | \commutator{\s}{T_2^{[2]} + T_3^{[2]}} | \mathrm{HF}}
    \\
    &+
    {}^t\p^{[1]}
    \braket{\tbar{1}^{[1]} |
    \frac{1}{2}\commutator{\commutator{\s}{T_1^{[1]}}}{T_1^{[1]}}
    |
    \mathrm{HF}}
    +
    {}^t\p^{[1]}
    \braket{\tbar{2}^{[1]} |
    \commutator{\commutator{\s}{T_1^{[1]}}}{T_2^{[1]}}
    |
    \mathrm{HF}}
  \end{aligned}
\end{equation}
Employing Eq. \eqref{eq:2nd-order-pol}, the terms involving the
second order polarization can be simplified further:
\begin{equation}
  \begin{aligned}
  G^{[4]}
  &= E^{[4]}
    + \frac{1}{2}{}^t\p^{[2]}
      \braket{\mathrm{HF} | \commutator{\s}{T_1^{[2]}} | \mathrm{HF}}
    - \frac{1}{2}{}^t\p^{[2]}
       \braket{\tbar{1}^{[2]}| \s | \mathrm{HF}} \\
    &+ \frac{1}{2}{}^t\p^{[2]}
      \braket{\tbar{1}^{[1]}| \commutator{\s}{T^{[1]}} | \mathrm{HF} }
    + \frac{1}{2}{}^t\p^{[2]}
       \braket{\tbar{2}^{[1]}| \commutator{\s}{T_2^{[1]}} | \mathrm{HF} } \\
    &+ \frac{1}{2}{}^t\p^{[2]}\braket{\tbar{1}^{[1]}| \commutator{\s}{T^{[1]}} | \mathrm{HF} }
    + \frac{1}{2}{}^t\p^{[2]}\braket{\tbar{2}^{[1]}| \commutator{\s}{T_2^{[1]}} | \mathrm{HF} } \\
    &+
    {}^t\p^{[1]}
    \braket{\tbar{1}^{[1]} | \commutator{\s}{T_1^{[2]} + T_2^{[2]}} | \mathrm{HF}}
    +
    \highlight{brewerRed}{
    {}^t\p^{[1]}
    \braket{\tbar{2}^{[1]} | \commutator{\s}{T_2^{[2]} + T_3^{[2]}} | \mathrm{HF}}
    }
    \\
    &+
    {}^t\p^{[1]}
    \braket{\tbar{1}^{[1]} |
    \frac{1}{2}
    \commutator{\commutator{\s}{T_1^{[1]}}}{T_1^{[1]}}
    |
    \mathrm{HF}}
    +
    {}^t\p^{[1]}
    \braket{\tbar{2}^{[1]} |
    \commutator{\commutator{\s}{T_1^{[1]}}}{T_2^{[1]}}
    |
    \mathrm{HF}}
  \end{aligned}
\end{equation}
Notice that the term $\frac{1}{2}{}^t\p^{[2]}\braket{\tbar{1}^{[2]}| \s | \mathrm{HF}}$
has not been included, since it would violate the $2n+2$ rule for the
multipliers. Moreover, there is only one term that involves both the
triples amplitudes and the polarization.

Eventually, the fifth order energy correction is formed including second order
amplitudes, multipliers and polarization:
\begin{equation}\label{eq:pted-mbpt-5th}
  \begin{aligned}
    G^{[5]} &=
    \braket{\mathrm{HF} | \frac{1}{2}
    \commutator{\commutator{\Phi}{T^{[2]}}}{T^{[2]}}
    | \mathrm{HF}}
    +
    \sum_{u=1}^2\braket{\tbar{u}^{[1]}|
    \commutator{\commutator{\Phi}{T^{[2]}}}{T^{[1]}}
    | \mathrm{HF}}
    \\
    &+
    \sum_{u=1}^3\braket{\tbar{u}^{[2]}|
      \commutator{\Phi}{T^{[2]}}
    | \mathrm{HF}}
    +
    \sum_{u=1}^3\braket{\tbar{u}^{[2]}|
    \frac{1}{2}
    \commutator{\commutator{\Phi}{T^{[1]}}}{T^{[1]}}
    | \mathrm{HF}} \\
    &+ {}^t\p^{[2]}\s_{\mathrm{N}}^{[3]}(\tamp{}^{[2]}, \tbar{}^{[2]})
    + {}^t\p^{[1]}\s_{\mathrm{N}}^{[4]}(\tamp{}^{[2]}, \tbar{}^{[2]}) \\
    &= E^{[5]}
    + {}^t\p^{[2]}
    \sum_{u=1}^2\braket{\tbar{u}^{[1]} |
    \commutator{\s}{T^{[2]}} | \mathrm{HF}} \\
    &+ {}^t\p^{[2]}
    \sum_{u=1}^2\braket{\tbar{u}^{[1]} |
    \frac{1}{2}\commutator{\commutator{\s}{T^{[1]}}}{T^{[1]}}
    | \mathrm{HF}}
    + {}^t\p^{[2]}
    \sum_{u=1}^3\braket{ \tbar{u}^{[2]}
                                 | \commutator{\s}{T^{[1]}} | \mathrm{HF}}
    \\
    &+ {}^t\p^{[1]}
    \sum_{u=1}^3\braket{ \tbar{u}^{[2]}
    \commutator{\s}{T^{[2]}} | \mathrm{HF}}
    + {}^t\p^{[1]}
    \sum_{u=1}^3\braket{ \tbar{u}^{[2]}
    \frac{1}{2}\commutator{\commutator{\s}{T^{[1]}}}{T^{[1]}}
    | \mathrm{HF}} \\
    &+ {}^t\p^{[1]}
    \sum_{u=1}^2\braket{\tbar{u}^{[1]} |
    \commutator{\commutator{\s}{T^{[1]}}}{T^{[2]}}
    | \mathrm{HF}}
    \\
  &= E^{[5]}
    + {}^t\p^{[2]}\braket{\tbar{1}^{[1]} | \commutator{\s}{T_1^{[2]} +
    T_2^{[2]}} | \mathrm{HF}} \\
    &+ \highlight{brewerRed}{
    {}^t\p^{[2]}\braket{\tbar{2}^{[1]} | \commutator{\s}{T_2^{[2]} + T_3^{[2]}} | \mathrm{HF}}
    } \\
    &+ {}^t\p^{[2]}\braket{\tbar{1}^{[2]} | \commutator{\s}{T^{[1]}} | \mathrm{HF}}
    + {}^t\p^{[2]}
    \braket{\tbar{1}^{[1]} |
    \frac{1}{2}\commutator{\commutator{\s}{T_1^{[1]}}}{T_1^{[1]}}
    | \mathrm{HF}} \\
    &+ {}^t\p^{[2]}
    \braket{\tbar{2}^{[1]} |
    \commutator{\commutator{\s}{T_1^{[1]}}}{T_2^{[1]}}
    | \mathrm{HF}}
    +
    {}^t\p^{[1]}
    \braket{\tbar{1}^{[2]} | \commutator{\s}{T_1^{[2]} + T_2^{[2]}} |
    \mathrm{HF}} \\
    &+
    \highlight{brewerRed}{
    {}^t\p^{[1]}
    \braket{\tbar{2}^{[2]} | \commutator{\s}{T_2^{[2]} + T_3^{[2]}} | \mathrm{HF}}
    }
    +
    \highlight{brewerRed}{
    {}^t\p^{[1]}
    \braket{\tbar{3}^{[2]} | \commutator{\s}{T_3^{[2]}} | \mathrm{HF}}
    } \\
    &+
    {}^t\p^{[1]}
    \braket{\tbar{1}^{[2]} |
    \frac{1}{2}\commutator{\commutator{\s}{T_1^{[1]}}}{T_1^{[1]}}
    | \mathrm{HF}}
    +
    {}^t\p^{[1]}
    \braket{\tbar{2}^{[2]} |
    \commutator{\commutator{\s}{T_1^{[1]}}}{T_2^{[1]}}
    | \mathrm{HF}} \\
    &+
    \highlight{brewerRed}{
    {}^t\p^{[1]}
    \braket{\tbar{3}^{[2]} |
    \frac{1}{2}\commutator{\commutator{\s}{T_2^{[1]}}}{T_2^{[1]}}
    | \mathrm{HF}}
    } \\
    &+ {}^t\p^{[1]}
    \braket{\tbar{1}^{[1]} |
    \commutator{\commutator{\s}{T_1^{[1]}}}{T_1^{[2]}}
    | \mathrm{HF}}
    + {}^t\p^{[1]}
    \braket{\tbar{2}^{[1]} |
    \commutator{\commutator{\s}{T_1^{[1]}}}{T_2^{[2]}}
    | \mathrm{HF}} \\
    &+ {}^t\p^{[1]}
    \braket{\tbar{2}^{[1]} |
    \commutator{\commutator{\s}{T_2^{[1]}}}{T_1^{[2]}}
    | \mathrm{HF}}
  \end{aligned}
\end{equation}
Similarly to the exposition in Chapter \ref{ch:QM}, these expressions
will form the basis for the development of approximations
beyond \acrshort{CCSD} including connected triples.

The \acrshort{PTED}-\acrshort{CCSD} scheme for the quantum/classical coupling
only includes singles and doubles in its excitation manifold and is thus
correct up to third order in perturbation theory.
Already in fourth order connected triples make their appearance and any
model going beyond third order must then take their effect into account.
The fourth and fifth order contributions to the energy due to connected
triples given in Eqs.
\eqref{eq:4th-order-triples} and \eqref{eq:5th-order-triples} will be
augmented by additional terms due to the nonzero first order singles and
the quantum/classical polarizable coupling.
To fourth order these are:
\begin{equation}\label{eq:triples-fourth-order}
  \begin{aligned}
  G_T^{[4]} &=
  \highlight{brewerYellow}{\braket{\tbar{1}^{[1]} | \commutator{\Phi}{T_3^{[2]}} | \mathrm{HF}}}
  + \braket{\tbar{2}^{[1]}| \commutator{\Phi}{T_3^{[2]}} | \mathrm{HF}} \\
  &+ {}^t\p^{[1]}\braket{\tbar{2}^{[1]} | \commutator{\s}{T_3^{[2]}} |
  \mathrm{HF}},
  \end{aligned}
\end{equation}
while the fifth order contributions are:
\begin{equation}\label{eq:triples-fifth-order}
  \begin{aligned}
  G_T^{[5]} &=
    \highlight{brewerYellow}{
    \braket{\tbar{2}^{[1]} |
    \commutator{\commutator{\Phi}{T_3^{[2]}}}{T_1^{[1]}}
    | \mathrm{HF}}
    }
    +
    \braket{\tbar{1}^{[2]} | \commutator{\Phi}{T_3^{[2]}} | \mathrm{HF}} \\
  &+ \braket{\tbar{2}^{[2]} | \commutator{\Phi}{T_3^{[2]}} | \mathrm{HF}} \\
  &+ \braket{\tbar{3}^{[2]} | \commutator{\Phi}{T_2^{[2]}} | \mathrm{HF}}
  + \braket{\tbar{3}^{[2]} | \commutator{\Phi}{T_3^{[2]}} | \mathrm{HF}}
  \\
    &+
    \frac{1}{2}\braket{\tbar{3}^{[2]} |
    \commutator{\commutator{\Phi}{T_2^{[1]}}}{T_2^{[1]}}
    | \mathrm{HF}}
  +
  \highlight{brewerYellow}{
    \braket{\tbar{3}^{[2]} |
    \commutator{\commutator{\Phi}{T_1^{[1]}}}{T_2^{[1]}}
    | \mathrm{HF}}
   }
   \\
   &+
    {}^t\p^{[2]}\braket{\tbar{2}^{[1]} | \commutator{\s}{T_3^{[2]}} | \mathrm{HF}}
  + {}^t\p^{[1]}\braket{\tbar{2}^{[2]} | \commutator{\s}{T_3^{[2]}} | \mathrm{HF}} \\
  &+ {}^t\p^{[1]}\braket{\tbar{3}^{[2]} | \commutator{\s}{T_3^{[2]}} | \mathrm{HF}}
  + {}^t\p^{[1]}\braket{\tbar{3}^{[2]} |
    \frac{1}{2}\commutator{\commutator{\s}{T_2^{[1]}}}{T_2^{[1]}}
    | \mathrm{HF}}
  \end{aligned}
\end{equation}
We have highlighted the vacuum-like terms that appear due to the nonzero
first order singles amplitudes and multipliers in yellow.

\subsection{Approximations to PTED}

Perturbative expansions can also be developed starting from the
approximations to \acrshort{PTED} presented in Section
\ref{sec:pcm-cc-models}.
For the \acrshort{PTE} scheme, the only change with respect to the
\emph{in vacuo} theory as presented in Section \ref{sec:mbpt} is the use
of the solvated Fock matrix.\autocite{Olivares_del_Valle1991-of,
Angyan1995-co, Lipparini2009-io}

In the \acrshort{PTE(S)} scheme, the amplitudes equations are uncoupled
from the multipliers and polarization equations. \acrshort{MBPT} for
this approximate model can be developed similarly to that in the
\acrshort{PTE} scheme.
The first and second order parameters are determined by:
\begin{subequations}\label{eq:pte-s-mbpt}
  \begin{align}
  \denom{\mu_2}\tamp{\mu_2}^{[1]} &=
  - \braket{\mu_2 | \Phi | \mathrm{HF}}
  =
  - \braket{\mathrm{HF} | \commutator{\Phi}{\cluster{\mu_2}} | \mathrm{HF}}
  =
  \denom{\mu_2}\tbar{\mu_2}^{[1]}
  \\
    \denom{\mu_q}\tamp{\mu_q}^{[2]}
    &= -
      \braket{\mu_q | \commutator{\Phi}{T_2^{[1]}} | \mathrm{HF}}
  \\
    \denom{\mu_q}\tbar{\mu_q}^{[2]}
    &=
    - \left(
      \braket{\tbar{2}^{[1]} | \commutator{\Phi}{\cluster{\mu_q}} | \mathrm{HF}}
      + {}^t\p^{[2]}\braket{\mathrm{HF} |
      \commutator{\s}{\cluster{\mu_q}} | \mathrm{HF}}\delta_{\mu_q\mu_1}
    \right)
    = 0
    \label{eq:pte-s-2nd-order-mult}
  \\
  \V\p^{[2]} &+ \braket{\mathrm{HF} | \commutator{\s}{T_1^{[2]}}| \mathrm{HF}} = 0
  \label{eq:pte-s-2nd-order-pol}
  \end{align}
\end{subequations}
Notice that the first order singles amplitudes and multipliers are now
zero, as is the first order polarization. The second order singles
multipliers are coupled to the second order polarization. Notice however
that $\p^{[2]}$ only depends on $T_1^{[2]}$ and can thus be computed
on-the-fly, without iterating between Eqs.
\eqref{eq:pte-s-2nd-order-mult} and \eqref{eq:pte-s-2nd-order-pol}
Starting from Eqs. \eqref{eq:mbpt-1st}--\eqref{eq:mbpt-5th}, the free energy
corrections in the \acrshort{PTE(S)} scheme are:
\begin{subequations}
  \begin{align}
    G^{[1]} &= E^{[1]} \label{eq:pte-s-mbpt-1st} \\
    G^{[2]} &= E^{[2]} \label{eq:pte-s-mbpt-2nd} \\
    G^{[3]} &= E^{[3]} \label{eq:pte-s-mbpt-3rd} \\
    G^{[4]} &= E^{[4]} + \frac{1}{2}
    {}^t\p^{[2]}\braket{\mathrm{HF} | \BCHfirst{\s}{T_1^{[2]}}
    |\mathrm{HF}} \label{eq:pte-s-mbpt-4th} \\
    G^{[5]} &= E^{[5]} \label{eq:pte-s-mbpt-5th}
  \end{align}
\end{subequations}

In the \acrshort{PTES} scheme, two different polarizations are
introduced to achieve an approximate decoupling of the governing
equations. \acrshort{MBPT} can be developed by considering the
Lagrangian \eqref{eq:ptes-cc} to form the energy corrections, Eq.
\eqref{eq:ptes-amp} for the order analysis of the amplitudes, Eq.
\eqref{eq:ptes-mult} for the order analysis of the multipliers.
The order analysis of the polarization will be developed based on
Eq. \eqref{eq:ptes-cc-polarization-mult}, \ie~the polarization coupled
to the multipliers.
To first order:
\begin{subequations}\label{eq:ptes-1st-mbpt}
  \begin{align}
  \denom{\mu_2}\tamp{\mu_2}^{[1]} &=
  - \braket{\mu_2 | \Phi | \mathrm{HF}}
  =
  - \braket{\mathrm{HF} | \commutator{\Phi}{\cluster{\mu_2}} | \mathrm{HF}}
  =
  \denom{\mu_2}\tbar{\mu_2}^{[1]}
  \\
  \denom{\mu_1}\tbar{\mu_1}^{[1]} &=
  - {}^t\p^{[1]}\braket{\mathrm{HF} |
      \commutator{\s}{\cluster{\mu_q}} | \mathrm{HF}}\delta_{\mu_q\mu_1}
  \\
  \V\p^{[1]} &+ \braket{\tbar{1}^{[1]} | \s | \mathrm{HF}} = 0
  \label{eq:ptes-1st-order-pol}
  \end{align}
\end{subequations}
while to second order:
\begin{subequations}\label{eq:ptes-2nd-mbpt}
  \begin{align}
    \denom{\mu_q}\tamp{\mu_q}^{[2]}
    &= -
      \braket{\mu_q | \commutator{\Phi}{T_2^{[1]}} | \mathrm{HF}}
  \\
    \denom{\mu_q}\tbar{\mu_q}^{[2]}
    &=
    - \braket{\tbar{2}^{[1]} | \commutator{\Phi}{\cluster{\mu_q}} | \mathrm{HF}}
    - {}^t\p^{[2]}\braket{\mathrm{HF} |
      \commutator{\s}{\cluster{\mu_q}} | \mathrm{HF}}\delta_{\mu_q\mu_1}
      \nonumber \\
    &- {}^t\p^{[1]}\braket{\tbar{1}^{[1]} | \commutator{\s}{\cluster{\mu_q}} | \mathrm{HF}}
    = 0
    \label{eq:ptes-2nd-order-mult}
  \\
  \V\p^{[2]} &+ \braket{\mathrm{HF} | \commutator{\s}{T_1^{[2]}}| \mathrm{HF}}
  + \braket{\tbar{1}^{[2]} | \s | \mathrm{HF}} = 0
  \label{eq:ptes-2nd-order-pol}
  \end{align}
\end{subequations}
We can now develop the free energy corrections in the \acrshort{PTES}
scheme based on Eqs. \eqref{eq:mbpt-1st}--\eqref{eq:mbpt-5th}
\begin{subequations}
  \begin{align}
    G^{[1]} &= E^{[1]} \label{eq:ptes-mbpt-1st} \\
    G^{[2]} &= E^{[2]} \label{eq:ptes-mbpt-2nd} \\
    G^{[3]} &= E^{[3]} \label{eq:ptes-mbpt-3rd} \\
    G^{[4]} &= E^{[4]} + \frac{1}{2}
    {}^t\p^{[2]}\braket{\mathrm{HF} | \BCHfirst{\s}{T_1^{[2]}}
    |\mathrm{HF}} \label{eq:ptes-mbpt-4th} \\
    G^{[5]} &= E^{[5]}
    + {}^t\p^{[2]}\braket{\tbar{1}^{[1]} | \BCHfirst{\s}{T_1^{[2]}}
    \label{eq:ptes-mbpt-5th}
    |\mathrm{HF}}
  \end{align}
\end{subequations}

\section[Approximate Coupled Cluster Methods]{
Approximate Coupled Cluster Methods with Quantum/Classical Polarizable Hamiltonians
}\label{sec:cc-approximate-quantum-classical}

In complete analogy to the discussion in Section
\ref{sec:cc-approximate}, we can derive iterative and noniterative
schemes for the approximate inclusion of higher order excitations in the
\acrshort{CC} hierarchy.
Starting from the \acrshort{CC2} Lagrangian, Eq.
\eqref{eq:cc2-lagrangian}, we can write down the corresponding
\acrshort{PTED} effective Lagrangian:\autocite{Sneskov2011-jm,
Schwabe2012-cf}
\begin{equation}\label{eq:pted-cc2-lagrangian}
  \begin{aligned}
  \lag(\tamp{}, \tbar{}, \p)_\mathrm{CC2}
  &=
  E_0
  + \sum_{u=1}^{2}\tbar{\mu_u}\denom{\mu_u}\tamp{\mu_u}
  + \braket{\mathrm{HF} | \check{\Phi} + \commutator{\check{\Phi}}{Q_2} | \mathrm{HF}} \\
  &+ \braket{\tbar{1} |
  \check{\Phi} + \commutator{\check{\Phi}}{Q_2}
  | \mathrm{HF}}
  + \braket{\tbar{2} | \check{\Phi} | \mathrm{HF}}
  + \frac{1}{2}{}^t\p\V\p +
  {}^t\p\s_{\mathrm{N}}(\tamp{},\tbar{})_\mathrm{CC2}
  \end{aligned}
\end{equation}
where the source term is formed according to Eq.
\eqref{eq:cc2-expectation-value}
The multipliers equations now become:
\begin{subequations}\label{eq:pted-cc2-amplitudes}
  \begin{align}
    \denom{\mu_1}\tamp{\mu_1} &+
    \braket{\mu_1 | \check{\Phi} +
    \commutator{\check{\Phi}}{Q_2} | \mathrm{HF}}
    + {}^t\p\braket{\mu_1 | \check{\s} +
    \commutator{\check{\s}}{Q_2} | \mathrm{HF}}
    = 0
    \label{eq:pted-cc2-singles-amplitudes} \\
    \denom{\mu_2}\tamp{\mu_2} &+ \braket{\mu_2 | \check{\Phi} | \mathrm{HF}}
    + {}^t\p\braket{\mu_2 | \commutator{\check{\s}}{Q_2} | \mathrm{HF}}
    = 0
    \label{eq:pted-cc2-doubles-amplitudes}
  \end{align}
\end{subequations}
while for the multipliers one has:
\begin{subequations}\label{eq:pted-cc2-multipliers}
  \begin{align}
    \denom{\mu_1}\tbar{\mu_1} &+
     \braket{\mathrm{HF} | \commutator{\check{\Phi}}{\cluster{\mu_1}} | \mathrm{HF}}
   + \braket{\tbar{1} |
       \commutator{\check{\Phi}}{\cluster{\mu_1}}
     + \commutator{\commutator{\check{\Phi}}{\cluster{\mu_1}}}{Q_2}
     | \mathrm{HF}} \nonumber \\
   &+ \braket{\tbar{2} | \commutator{\check{\Phi}}{\cluster{\mu_1}} |
   \mathrm{HF}}
   + {}^t\p\braket{\mathrm{HF} |
   \commutator{\check{\s}_\mathrm{N}}{\cluster{\mu_1}} | \mathrm{HF}}  \nonumber \\
   &+ {}^t\p\braket{\tbar{1} | \commutator{\check{\s}}{\cluster{\mu_1}} | \mathrm{HF}}
   + {}^t\p\braket{\tbar{2} | \commutator{\commutator{\check{\s}}{\cluster{\mu_1}}}{Q_2} | \mathrm{HF}}
   = 0 \label{eq:pted-cc2-singles-multipliers} \\
    \denom{\mu_2}\tbar{\mu_2} &+
    \braket{\mathrm{HF} | \commutator{\check{\Phi}}{\cluster{\mu_2}} | \mathrm{HF}}
   + \braket{\tbar{1} |
       \commutator{\check{\Phi}}{\cluster{\mu_2}}
     | \mathrm{HF}} \nonumber \\
   &+ {}^t\p\braket{\tbar{1} | \commutator{\check{\s}}{\cluster{\mu_2}} | \mathrm{HF}}
   + {}^t\p\braket{\tbar{2} | \commutator{\check{\s}}{\cluster{\mu_2}} | \mathrm{HF}}
     = 0 \label{eq:pted-cc2-doubles-multipliers}
  \end{align}
\end{subequations}
Finally, the polarization is obtained by solving:
\begin{equation}\label{eq:pted-cc2-polarization}
  \V\p + \s_{\mathrm{N}}(\tamp{},\tbar{})_\mathrm{CC2} = 0
\end{equation}

The \acrshort{PTED}-\acrshort{CC3} model is obtained similarly. The
source term is formed according to Eq. \eqref{eq:cc3-expectation-value}.
Thus, adding the polarization functional to the Lagrangian
\eqref{eq:cc3-lagrangian}, we obtain the effective Lagrangian:
\begin{equation}\label{eq:pted-cc3-lagrangian}
  \begin{aligned}
  \lag(\tamp{}, \tbar{}, \p)_\mathrm{CC3}
  &=
  E_0
  + \sum_{u=1}^{3}\tbar{\mu_u}\denom{\mu_u}\tamp{\mu_u}
  + \braket{\mathrm{HF} | \check{\Phi} + \commutator{\check{\Phi}}{T_2} | \mathrm{HF}} \\
  &+ \braket{\tbar{1} |
  \check{\Phi} + \commutator{\check{\Phi}}{T_2} + \commutator{\check{\Phi}}{Q_3}
  | \mathrm{HF}} \\
  &+ \braket{\tbar{2} |
  \check{\Phi} + \commutator{\check{\Phi}}{T_2}
  + \frac{1}{2}\commutator{\commutator{\check{\Phi}}{T_2}}{T_2} + \commutator{\check{\Phi}}{Q_3}
  | \mathrm{HF}} \\
  &+ \braket{\tbar{3} | \commutator{\check{\Phi}}{T_2} | \mathrm{HF}}
  + \frac{1}{2}{}^t\p\V\p +
  {}^t\p\s_{\mathrm{N}}(\tamp{},\tbar{})_\mathrm{CC3}
  \end{aligned}
\end{equation}
The polarization equation is straightforwardly:
\begin{equation}\label{eq:pted-cc3-polarization}
  \V\p + \s_{\mathrm{N}}(\tamp{},\tbar{})_\mathrm{CC3} = 0
\end{equation}
The amplitudes are determined by:
\begin{subequations}\label{eq:pted-cc3-amplitudes}
  \begin{align}
    \denom{\mu_1}\tamp{\mu_1} &+ \braket{\mu_1 |
    \check{\Phi} + \commutator{\check{\Phi}}{T_2}
    + \commutator{\check{\Phi}}{Q_3}
    | \mathrm{HF}} \nonumber \\
    &+ {}^t\p\braket{\mu_1 | \check{\s} +
    \commutator{\check{\s}}{T_2} | \mathrm{HF}}
    = 0 \label{eq:pted-cc3-singles-amplitudes} \\
    \denom{\mu_2}\tamp{\mu_2} &+ \braket{\mu_2 |
    \check{\Phi} + \commutator{\check{\Phi}}{T_2}
    + \frac{1}{2}\commutator{\commutator{\check{\Phi}}{T_2}}{T_2}
    + \commutator{\check{\Phi}}{Q_3}
    | \mathrm{HF}} \nonumber  \\
    &+
    {}^t\p\braket{\mu_2 | \commutator{\check{\s}}{T_2} +
    \commutator{\check{\s}}{Q_3} | \mathrm{HF}}
    = 0 \label{eq:pted-cc3-doubles-amplitudes} \\
    \denom{\mu_3}\tamp{\mu_3} &+ \braket{\mu_3 |
    \commutator{\check{\Phi}}{T_2}
    | \mathrm{HF}}
    + {}^t\p\braket{\mu_3 | \commutator{\check{\s}}{Q_3}
     + \frac{1}{2}\commutator{\commutator{\check{\s}}{T_2}}{T_2}
    | \mathrm{HF}}
    = 0 \label{eq:pted-cc3-triples-amplitudes}
    \end{align}
\end{subequations}
and finally the multipliers obey:
\begin{subequations}\label{eq:pted-cc3-multipliers}
  \begin{align}
    \denom{\mu_1}\tbar{\mu_1} &+
     \braket{\mathrm{HF} | \commutator{\check{\Phi}}{\cluster{\mu_1}} | \mathrm{HF}}
   + \braket{\tbar{1} |
       \commutator{\check{\Phi}}{\cluster{\mu_1}}
     + \commutator{\commutator{\check{\Phi}}{\cluster{\mu_1}}}{T_2}
     | \mathrm{HF}} \nonumber \\
   &+ \braket{\tbar{2} |
       \commutator{\check{\Phi}}{\cluster{\mu_1}}
     + \commutator{\commutator{\check{\Phi}}{\cluster{\mu_1}}}{T_2}
     + \commutator{\commutator{\check{\Phi}}{\cluster{\mu_1}}}{Q_3}
   | \mathrm{HF}} \nonumber \\
   &+ \braket{\tbar{3} |
      \commutator{\commutator{\check{\Phi}}{\cluster{\mu_1}}}{T_2}
     | \mathrm{HF}}
   + {}^t\p\braket{\mathrm{HF} |
   \commutator{\check{\s}_\mathrm{N}}{\cluster{\mu_1}} | \mathrm{HF}}  \nonumber \\
   &+ {}^t\p\braket{\tbar{1} | \commutator{\check{\s}}{\cluster{\mu_1}} | \mathrm{HF}}
   + {}^t\p\braket{\tbar{2} |
   \commutator{\commutator{\check{\s}}{\cluster{\mu_1}}}{Q_2} |
   \mathrm{HF}} \nonumber \\
   &+ {}^t\p\braket{\tbar{3} | \commutator{\commutator{\check{\s}}{\cluster{\mu_1}}}{Q_3} | \mathrm{HF}}
     = 0 \label{eq:pted-cc3-singles-multipliers} \\
    \denom{\mu_2}\tbar{\mu_2} &+
    \braket{\mathrm{HF} | \commutator{\check{\Phi}}{\cluster{\mu_2}} | \mathrm{HF}}
   + \braket{\tbar{1} |
       \commutator{\check{\Phi}}{\cluster{\mu_2}}
     | \mathrm{HF}} \nonumber \\
  &+ \braket{\tbar{2} |
       \commutator{\check{\Phi}}{\cluster{\mu_2}}
     + \commutator{\commutator{\check{\Phi}}{\cluster{\mu_2}}}{T_2}
     | \mathrm{HF}}
  + \braket{\tbar{3} |
       \commutator{\check{\Phi}}{\cluster{\mu_2}}
     | \mathrm{HF}} \nonumber \\
   &+ {}^t\p\braket{\tbar{1} | \commutator{\check{\s}}{\cluster{\mu_2}} | \mathrm{HF}}
   + {}^t\p\braket{\tbar{2} | \commutator{\check{\s}}{\cluster{\mu_2}} | \mathrm{HF}}
   \nonumber \\
   &+ {}^t\p\braket{\tbar{3} |
   \commutator{\commutator{\check{\s}}{\cluster{\mu_2}}}{T_2} | \mathrm{HF}}
    = 0 \label{eq:pted-cc3-doubles-multipliers} \\
    \denom{\mu_3}\tbar{\mu_3} &+
    \braket{\tbar{1} |
       \commutator{\check{\Phi}}{\cluster{\mu_3}}
     | \mathrm{HF}}
  + \braket{\tbar{2} |
       \commutator{\check{\Phi}}{\cluster{\mu_3}}
     | \mathrm{HF}} \nonumber \\
   &+ {}^t\p\braket{\tbar{2} | \commutator{\check{\s}}{\cluster{\mu_3}} | \mathrm{HF}}
   + {}^t\p\braket{\tbar{3} | \commutator{\check{\s}}{\cluster{\mu_3}} | \mathrm{HF}}
    = 0 \label{eq:pted-cc3-triples-multipliers}
  \end{align}
\end{subequations}

We can construct the \emph{noniterative}
$\Lambda\text{CCSD}(\text{T})$\autocite{Kucharski1998-qq,
Kucharski1998-oi, Crawford1998-vj} and
$\text{CCSD}(\text{T})$\autocite{Raghavachari1989-bn} methods as shown
in Chapter \ref{ch:QM}.
We propose to include all the fourth order terms in Eq.
\eqref{eq:triples-fourth-order} and the following two terms from the
fifth order connected triples corrections in Eq.
\eqref{eq:triples-fifth-order}:
\begin{alignat}{2}
 \braket{\tbar{1}^{[2]} | \commutator{\Phi}{T_3^{[2]}} | \mathrm{HF}},
    \quad&
 {}^t\p^{[1]}\braket{\tbar{2}^{[2]} | \commutator{\s}{T_3^{[2]}} | \mathrm{HF}}
\end{alignat}
Inserting the converged \acrshort{CCSD} parameters, denoted with a $*$
superscript, yields the following expressions:
\begin{subequations}\label{eq:g-parentheses-t}
 \begin{align}
  G_{\Lambda(\text{T})}
  &=
    \sum_{u=1}^2\braket{\tbar{u}^{*}| \commutator{\Phi}{T_3^{*}} | \mathrm{HF}}
    + {}^t\p^{*}\braket{\tbar{2}^{*} | \commutator{\s}{T_3^{*}} | \mathrm{HF}} \\
  G_{(\text{T})}
 &=
    \sum_{u=1}^2\braket{\tamp{u}^{*}|\commutator{\Phi}{T_3^{*}} | \mathrm{HF}}
    + {}^t\p^{*}\braket{\tamp{2}^{*} | \commutator{\s}{T_3^{*}} | \mathrm{HF}}
 \end{align}
\end{subequations}
We recall that both the left and right state are used in the
$\Lambda\text{CCSD}(\text{T})$ method, while $\text{CCSD}(\text{T})$
uses the right state only, replacing the multipliers with the
amplitudes.
Further insight into the proposed correction can be gained by
inserting the \acrshort{MBPT} expansion of the converged \acrshort{CCSD}
parameters:\autocite{Stanton1997-lx, Koch1997-nm,
Eriksen2014-gy, Eriksen2015-il, Kristensen2016-od}
First of all, we notice that $T_3^{*} = T_3^{[2]} + \tilde{T}_3 + O(3)$
contains higher than second order contributions. In particular, $\tilde{T}_3$
is a third order contributions to the connected triples arising from
second order connected doubles:
\begin{equation}
  \denom{\mu_3}\tamp{\mu_3}^{*} =
  - \braket{\mu_3 |\commutator{\Phi}{T_2^{[1]}} | \mathrm{HF}}
  - \braket{\mu_3 | \commutator{\Phi}{T_2^{[2]}} | \mathrm{HF}}
  + O(3)
\end{equation}
such that the third order $\tilde{T}_3$ can re rewritten as:
\begin{equation}
  \tilde{T}_3
  = \sum_{\mu_3}
  \left(-\denom{\mu_3}^{-1}\braket{\mu_3 |
  \commutator{\Phi}{T_2^{[2]}}
  | \mathrm{HF}}\right)
  \cluster{\mu_3}
\end{equation}

The converged \acrshort{PTED}-\acrshort{PTED} parameters are correct up
to second order in the fluctuation potential:
\begin{subequations}
  \begin{align}
    \tamp{\mu_i}^{*} &= \tamp{\mu_i}^{[1]} + \tamp{\mu_i}^{[2]} + O(3)
    \label{eq:tamp-star} \\
    \tbar{\mu_i}^{*} &= \tbar{\mu_i}^{[1]} + \tbar{\mu_i}^{[2]} + O(3)
    \label{eq:tbar-star} \\
    \p^{*} &= \p^{[1]} + \p^{[2]} + O(3)
    \label{eq:p-star}
  \end{align}
\end{subequations}
Using the equations above one can expand the energy correction as
follows:
\begin{equation}
  \begin{aligned}
  G_{\Lambda(\text{T})} &=
  \sum_{u=1}^2 \braket{\tbar{u}^{[1]} | \commutator{\Phi}{T_3^{[2]}} | \mathrm{HF}}
  +
  \braket{\tbar{3}^{[2]} | \commutator{\Phi}{T_2^{[2]}} | \mathrm{HF}}
  +
  \sum_{u=1}^2 \braket{\tbar{u}^{[2]} | \commutator{\Phi}{T_3^{[2]}} | \mathrm{HF}}
  \\
  &+
  {}^t\p^{[1]} \braket{\tbar{2}^{[1]} | \commutator{\s}{T_3^{[2]}} | \mathrm{HF}}
  +
  {}^t\p^{[1]} \braket{\tbar{2}^{[2]} | \commutator{\s}{T_3^{[2]}} | \mathrm{HF}}
   \\
  &+
  {}^t\p^{[1]} \braket{\tbar{2}^{[1]} | \commutator{\s}{\tilde{T}_3} | \mathrm{HF}}
  +
  {}^t\p^{[2]} \braket{\tbar{2}^{[1]} | \commutator{\s}{T_3^{[2]}} | \mathrm{HF}}
  + O(6)
  \end{aligned}
\end{equation}

\section[Approximate CC and Approximate Couplings]{Approximate Coupled Cluster Methods and Approximate
Quantum/Classical Coupling Schemes}\label{sec:approximate-everything}

The iterative \acrshort{CC2} and \acrshort{CC3} and the noniterative
$\Lambda\text{CCSD}(\text{T})$ and $\text{CCSD}(\text{T})$ methods can
be formulated, \emph{mutatis mutandis}, also within the approximate
quantum/classical coupling schemes mentioned previously.\autocite{Caricato2011-tx, Schwabe2012-cf, Krause2016-ee}

Let us first consider the noniterative models for the inclusion of
connected triples.
The \acrshort{PTE} scheme is particularly simple: the equations are
unchanged with respect to vacuum theory, the only change being in the
definition of the \glspl{MO} and the Fock matrix.\autocite{Cammi2009-gu,
Begue2005-qn, Hasegawa2007-jt}
Comparing the fourth and fifth order \acrshort{MBPT} energy corrections
in the \acrshort{PTE(S)}, Eqs. \eqref{eq:pte-s-mbpt-1st}--\eqref{eq:pte-s-mbpt-5th}, and
\acrshort{PTES}, Eqs. \eqref{eq:ptes-mbpt-1st}--\eqref{eq:ptes-mbpt-5th}, schemes to
those obtained for the \acrshort{PTED} scheme, Eqs.
\eqref{eq:pted-mbpt-1st}--\eqref{eq:pted-mbpt-5th},
we can see that there are no contributions from the triples to the
quantum/classical terms involving the polarization.
Thus, also for \acrshort{PTE(S)} and \acrshort{PTES} the asymmetric and
symmetric triples corrections are formally identical to the vacuum
expression given in equations \eqref{eq:parentheses-t}
The converged \acrshort{CCSD} parameters are optimized in the presence of an
approximate polarization and are used to form the triples free energy
corrections.
Thus, despite the absence of explicit polarization terms in the
\acrshort{PTE(S)} and \acrshort{PTES} triples free energy corrections,
the effect of the polarization on the approximate triples amplitudes is
implicitly included.

The theory and implementation for iterative \acrshort{CC} methods with
quantum/classical polarizable Hamiltonians has already been presented in
the context of the multipolar continuum and \acrlong{PE}
models.\autocite{Kongsted2002-zm, Osted2003-qc, Sneskov2011-jm,
Schwabe2012-cf}
Once again, the \acrshort{PTE} scheme can be trivially derived from the
\acrshort{PTED} effective Lagrangians \eqref{eq:pted-cc2-lagrangian} and
\eqref{eq:pted-cc3-lagrangian} by neglecting the polarization components.
In the \acrshort{PTE(S)} scheme, we replace the expectation value of the
source with its \acrshort{CCS} counterpart and further neglect the
multiplier-dependent part, Eq. \eqref{eq:pte-s-source}.
For \acrshort{CC2}, the amplitudes equations are the same as \emph{in
vacuo}, Eqs. \eqref{eq:cc2-amplitudes} The multipliers are instead
determined by:
\begin{subequations}\label{eq:pte-s-cc2-multipliers}
  \begin{align}
    \denom{\mu_1}\tbar{\mu_1} &+
     \braket{\mathrm{HF} | \commutator{\check{\Phi}}{\cluster{\mu_1}} | \mathrm{HF}}
   + \braket{\tbar{1} |
       \commutator{\check{\Phi}}{\cluster{\mu_1}}
     + \commutator{\commutator{\check{\Phi}}{\cluster{\mu_1}}}{Q_2}
     | \mathrm{HF}} \nonumber \\
   &+ \braket{\tbar{2} | \commutator{\check{\Phi}}{\cluster{\mu_1}} |
   \mathrm{HF}}
   + {}^t\p\braket{\mathrm{HF} |
   \commutator{\check{\s}_\mathrm{N}}{\cluster{\mu_1}} | \mathrm{HF}}
   = 0 \label{eq:pte-s-cc2-singles-multipliers} \\
    \denom{\mu_2}\tbar{\mu_2} &+
    \braket{\mathrm{HF} | \commutator{\check{\Phi}}{\cluster{\mu_2}} | \mathrm{HF}}
   + \braket{\tbar{1} |
       \commutator{\check{\Phi}}{\cluster{\mu_2}}
     | \mathrm{HF}}
     = 0 \label{eq:pte-s-cc2-doubles-multipliers}
  \end{align}
\end{subequations}
The \acrshort{CC3} amplitudes are once again the same as \emph{in
vacuo}, Eqs. \eqref{eq:cc3-amplitudes}
The multipliers equations are:
\begin{subequations}\label{eq:pte-s-cc3-multipliers}
  \begin{align}
    \denom{\mu_1}\tbar{\mu_1} &+
     \braket{\mathrm{HF} | \commutator{\check{\Phi}}{\cluster{\mu_1}} | \mathrm{HF}}
   + \braket{\tbar{1} |
       \commutator{\check{\Phi}}{\cluster{\mu_1}}
     + \commutator{\commutator{\check{\Phi}}{\cluster{\mu_1}}}{T_2}
     | \mathrm{HF}} \nonumber \\
   &+ \braket{\tbar{2} |
       \commutator{\check{\Phi}}{\cluster{\mu_1}}
     + \commutator{\commutator{\check{\Phi}}{\cluster{\mu_1}}}{T_2}
     + \commutator{\commutator{\check{\Phi}}{\cluster{\mu_1}}}{Q_3}
   | \mathrm{HF}} \nonumber \\
   &+ \braket{\tbar{3} |
      \commutator{\commutator{\check{\Phi}}{\cluster{\mu_1}}}{T_2}
     | \mathrm{HF}}
   + {}^t\p\braket{\mathrm{HF} |
   \commutator{\check{\s}_\mathrm{N}}{\cluster{\mu_1}} | \mathrm{HF}}
     = 0 \label{eq:pte-s-cc3-singles-multipliers} \\
    \denom{\mu_2}\tbar{\mu_2} &+
    \braket{\mathrm{HF} | \commutator{\check{\Phi}}{\cluster{\mu_2}} | \mathrm{HF}}
   + \braket{\tbar{1} |
       \commutator{\check{\Phi}}{\cluster{\mu_2}}
     | \mathrm{HF}} \nonumber \\
  &+ \braket{\tbar{2} |
       \commutator{\check{\Phi}}{\cluster{\mu_2}}
     + \commutator{\commutator{\check{\Phi}}{\cluster{\mu_2}}}{T_2}
     | \mathrm{HF}}
  + \braket{\tbar{3} |
       \commutator{\check{\Phi}}{\cluster{\mu_2}}
     | \mathrm{HF}}
    = 0 \label{eq:pte-s-cc3-doubles-multipliers} \\
    \denom{\mu_3}\tbar{\mu_3} &+
    \braket{\tbar{1} |
       \commutator{\check{\Phi}}{\cluster{\mu_3}}
     | \mathrm{HF}}
  + \braket{\tbar{2} |
       \commutator{\check{\Phi}}{\cluster{\mu_3}}
    | \mathrm{HF}}
    = 0 \label{eq:pte-s-cc3-triples-multipliers}
  \end{align}
\end{subequations}
In both \acrlong{CC} models, the polarization is still determined by
Eq. \eqref{eq:pte-s-polarization} and only the singles multipliers
equation is modified.

In the \acrshort{PTES} scheme, we replace the expectation value of the
source with its \acrshort{CCS} counterpart. The singles amplitudes equation
thus contains a quantum/classical coupling term. However, to avoid
the coupling with the multipliers equations, an additional
approximation is introduced in the polarization equation, where the
source term is truncated to its multiplier-independent part.
As already noted, this leads to the use of \emph{two} different
polarization equations when solving for the amplitudes and the
multipliers.
Within the \acrshort{CC2} model we have the following amplitudes
equations:
\begin{subequations}\label{eq:ptes-cc2-amp}
  \begin{align}
   \denom{\mu_1}\tamp{\mu_1} &+ \braket{\mu_1 | \check{\Phi} +
   \BCHfirst{\check{\Phi}}{Q_2} | \mathrm{HF}}
   + {}^t\p\braket{\mu_1 | \check{\s} | \mathrm{HF}}
             = 0 \label{eq:ptes-cc2-singles-amplitudes} \\
   \denom{\mu_2}\tamp{\mu_2} &+ \braket{\mu_2 | \check{\Phi} | \mathrm{HF}}
             = 0 \label{eq:ptes-cc2-doubles-amplitudes} \\
    \V\p &+
    \braket{\mathrm{HF} | \check{\s}_{\mathrm{N}} | \mathrm{HF}} = 0
  \end{align}
\end{subequations}
For the \acrshort{CC2} multipliers one has:
\begin{subequations}\label{eq:ptes-cc2-mult}
  \begin{align}
    \denom{\mu_1}\tbar{\mu_1} &+
    \braket{\mathrm{HF} | \commutator{\check{\Phi}}{\cluster{\mu_1}} | \mathrm{HF}} +
    \braket{\tbar{1} |
    \commutator{\check{\Phi}}{\cluster{\mu_1}}
    + \commutator{\commutator{\check{\Phi}}{\cluster{\mu_1}}}{Q_2}
    | \mathrm{HF}}
    \nonumber \\
    &+ \braket{\tbar{2} |
    \commutator{\check{\Phi}}{\cluster{\mu_1}}
    | \mathrm{HF}}
    \nonumber \\
    &+
    {}^t\p\braket{\mathrm{HF} | \commutator{\check{\s}_\mathrm{N}}{\cluster{\mu_1}} | \mathrm{HF}}
    +
    {}^t\p\braket{\tbar{1} |
    \commutator{\check{\s}}{\cluster{\mu_1}} | \mathrm{HF}}
             = 0 \label{eq:ptes-cc2-singles-multipliers} \\
    \denom{\mu_2}\tbar{\mu_2} &+
    \braket{\mathrm{HF} | \commutator{\check{\Phi}}{\cluster{\mu_2}} | \mathrm{HF}} +
    \braket{\tbar{1} |
    \commutator{\check{\Phi}}{\cluster{\mu_2}} | \mathrm{HF}}
             = 0 \label{eq:ptes-cc2-doubles-multipliers} \\
    \V\p &+ \s_{\mathrm{N}}(\tamp{}, \tbar{})_\mathrm{CCS} = 0
    \label{eq:ptes-cc2-polarization-mult}
  \end{align}
\end{subequations}
We note that the \acrshort{PTES} approximation is crucial
in the implementation of \acrshort{PE}-\acrshort{CC2} presented by
\citeauthor{Schwabe2012-cf} in \noparcite[ref.][]{Schwabe2012-cf}
The resolution of the identity can in fact still be applied when
assuming the \acrshort{PTES} approximation.
Moreover, the simplified \acrshort{PE}-\acrshort{CC}
(s\acrshort{PE}\acrshort{CC}) Lagrangian of \citeauthor{Krause2016-ee}
is essentially the \acrshort{PTE(S)} approximation for
\acrshort{PE}-\acrshort{CC}.\autocite{Krause2016-ee}

Finally, for the \acrshort{CC3} model the amplitudes are determined by:
\begin{subequations}\label{eq:ptes-cc3-amplitudes}
  \begin{align}
    \denom{\mu_1}\tamp{\mu_1} &+ \braket{\mu_1 |
    \check{\Phi} + \commutator{\check{\Phi}}{T_2}
    + \commutator{\check{\Phi}}{Q_3}
    | \mathrm{HF}}
    + {}^t\p\braket{\mu_1 | \check{\s} | \mathrm{HF}}
    = 0 \label{eq:ptes-cc3-singles-amplitudes} \\
    \denom{\mu_2}\tamp{\mu_2} &+ \braket{\mu_2 |
    \check{\Phi} + \commutator{\check{\Phi}}{T_2}
    + \frac{1}{2}\commutator{\commutator{\check{\Phi}}{T_2}}{T_2}
    + \commutator{\check{\Phi}}{Q_3}
    | \mathrm{HF}}
    = 0 \label{eq:ptes-cc3-doubles-amplitudes} \\
    \denom{\mu_3}\tamp{\mu_3} &+ \braket{\mu_3 |
    \commutator{\check{\Phi}}{T_2}
    | \mathrm{HF}}
    = 0 \label{eq:ptes-cc3-triples-amplitudes} \\
    \V\p &+
    \braket{\mathrm{HF} | \check{\s}_{\mathrm{N}} | \mathrm{HF}} = 0
    \end{align}
\end{subequations}
and the multipliers by:
\begin{subequations}\label{eq:ptes-cc3-multipliers}
  \begin{align}
    \denom{\mu_1}\tbar{\mu_1} &+
     \braket{\mathrm{HF} | \commutator{\check{\Phi}}{\cluster{\mu_1}} | \mathrm{HF}}
   + \braket{\tbar{1} |
       \commutator{\check{\Phi}}{\cluster{\mu_1}}
     + \commutator{\commutator{\check{\Phi}}{\cluster{\mu_1}}}{T_2}
     | \mathrm{HF}} \nonumber \\
   &+ \braket{\tbar{2} |
       \commutator{\check{\Phi}}{\cluster{\mu_1}}
     + \commutator{\commutator{\check{\Phi}}{\cluster{\mu_1}}}{T_2}
     + \commutator{\commutator{\check{\Phi}}{\cluster{\mu_1}}}{Q_3}
   | \mathrm{HF}} \nonumber \\
   &+ \braket{\tbar{3} |
      \commutator{\commutator{\check{\Phi}}{\cluster{\mu_1}}}{T_2}
     | \mathrm{HF}}
   + {}^t\p\braket{\mathrm{HF} |
   \commutator{\check{\s}_\mathrm{N}}{\cluster{\mu_1}} | \mathrm{HF}}  \nonumber \\
   &+ {}^t\p\braket{\tbar{1} | \commutator{\check{\s}}{\cluster{\mu_1}} | \mathrm{HF}}
     = 0 \label{eq:ptes-cc3-singles-multipliers} \\
    \denom{\mu_2}\tbar{\mu_2} &+
    \braket{\mathrm{HF} | \commutator{\check{\Phi}}{\cluster{\mu_2}} | \mathrm{HF}}
   + \braket{\tbar{1} |
       \commutator{\check{\Phi}}{\cluster{\mu_2}}
     | \mathrm{HF}} \nonumber \\
  &+ \braket{\tbar{2} |
       \commutator{\check{\Phi}}{\cluster{\mu_2}}
     + \commutator{\commutator{\check{\Phi}}{\cluster{\mu_2}}}{T_2}
     | \mathrm{HF}}
  + \braket{\tbar{3} |
       \commutator{\check{\Phi}}{\cluster{\mu_2}}
     | \mathrm{HF}}
    = 0 \label{eq:ptes-cc3-doubles-multipliers} \\
    \denom{\mu_3}\tbar{\mu_3} &+
    \braket{\tbar{1} |
       \commutator{\check{\Phi}}{\cluster{\mu_3}}
     | \mathrm{HF}}
  + \braket{\tbar{2} |
       \commutator{\check{\Phi}}{\cluster{\mu_3}}
     | \mathrm{HF}}
    = 0 \label{eq:ptes-cc3-triples-multipliers} \\
    \V\p &+ \s_{\mathrm{N}}(\tamp{}, \tbar{})_\mathrm{CCS} = 0
    \label{eq:ptes-cc3-polarization-mult}
  \end{align}
\end{subequations}
