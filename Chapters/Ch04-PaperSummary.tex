% mainfile: ../RobertoDiRemigioPhDThesis.tex
%************************************************
\chapter{Summary of the publications}\label{ch:publications-summary}

In the following, I provide a brief summary of the motivations, contents and
conclusions of the papers this thesis is based on.
In addition, I give a description of my contributions to the work leading to
the publication of each of the papers.
\todo[inline]{Write something along these lines: "My coauthors have all read and approved."}

\section*{Software}

The codes used in the papers are all interfaced to the \pcmsolver library,
of which I am the main developer and maintainer.

\section*{Paper I. Continuum solvation in the relativistic regime}

Systems containing heavy elements are notoriously challenging for quantum chemistry.
In addition to their significantly large sizes, one has to properly include the
effect of special relativity in the quantum chemical description in order to
achieve at least qualitative agreement with experiment.
Additional complications arise if one needs to also include environment effects.
Continuum models arguably represent a cost-effective strategy to achieve a first
approximation of these effects.
In this paper, we presented the first derivation and implementation of the \acs{PCM}
coupled to a \acs{SCF} description of the solute in the fully relativistic, four-component
regime.

I contributed the theoretical derivation of the quantum/classical polarizable
terms in a four-component \acs{SCF} framework, for energies and linear response
properties. I devised the coupling of the four-component program \DIRAC with
\pcmsolver by providing the implementation and testing of:
\begin{itemize}
  \item \acs{MEP} integrals for four-component wave functions,
  \item the additional Fock matrix contributions,
  \item the additional terms in the response equations.
\end{itemize}
I performed all the calculations and large part of the data analysis
for the results reported in the paper.
Finally, I wrote the first draft of the paper and coordinated editing of
all subsequent versions.

\section*{Paper II. The wavelet Galerkin boundary element method for PCM}

Continuum solvation models are inherently parametrized. Apart from solvent permittivities,
the atomic radii and molecular surface definition play a crucial role
in determining the performance of the models.
However, the numerical accuracy of the \acs{BEM} procedure used to numerically solve the
underlying \acs{BIE} is a not-so-often studied aspect of these models.
Traditionally, collocation methods have been used, but these require parametrization of some of the necessary surface integrals.
Galerkin methods do not suffer from such a limitation and additionally preserve symmetry of the underlying
boundary integral operators.
The use of biorthogonal wavelet bases as finite elements achieves sparsity in
the \acs{BEM} procedure, due to their intrinsically hierarchical structure and
the existence of \emph{a priori} and \emph{a posteriori} matrix compression
estimates.
Thus, wavelet Galerkin \acs{BEM} represents a valid alternative to traditional collocation methods,
both to achieve a better computational scaling and to provide accurate, benchmark results.

This work builds upon and extends previous work by~\citeauthor{Weijo2010-hy}
\citeauthor{Weijo2010-hy} used \ac{PWC} bases to show that superior accuracy
and convergence could be obtained for the calculation of molecular energies.
In this work we showed that faster convergence can be achieved when \ac{PWL} bases are used instead.
Moreover, the same holds for the calculation of static electric properties.
Notably, the traditional collocation solver cannot guarantee the same accuracy,
even for very large finite element bases. This suggests that, in some cases,
\acs{BEM} collocation methodologies might slow down or even prevent the
convergence of the quantum mechanical response equations solvers.

For this paper, I provided template interface and test sets for the cavity
generator\autocite{Harbrecht2009-no, Harbrecht2011-dk}
and wavelet Galerkin \acs{BEM} solver\autocite{Harbrecht2004-uo, Harbrecht2006-ug, Dahmen2006-pj}
with \pcmsolver.
These were used to interface with the new C++ implementation of the wavelet
Galerkin solvers of Monica Bugeanu.
I implemented the interface between the \LSDALTON quantum chemistry software
package and the \pcmsolver software library. The interface allows to run \acs{HF} and
\acs{KS}-\acs{DFT} single-point and linear response calculations.
I, together with coauthor Krzysztof Mozgawa, performed the benchmark quantum
chemical calculations presented in the paper.
Finally, I coordinated the editing of all manuscript drafts. I wrote Sections 2.1 and 2.3,
while Sections 3 and 4 were co-written with the first author, Monica Bugeanu.
I performed most of the data analysis and produced tables and graphs.

\begin{figure}[tb]
 \centering
  \missingfigure{Take figure from paper!}
  \caption[Convergence of $\alpha_\mathrm{iso}$ with respect to the number of molecular
  electrostatic potential (MEP) evaluation points on the cavity surface.]
  {Convergence of $\alpha_\mathrm{iso}$ with respect to the number of
   molecular electrostatic potential (MEP) evaluation points on the cavity
   surface.
   The values reported are in $a_0^3$ and refer to benzene.
   The lower axis reports the average area for the collocation
   tesselation, while the upper axis refers to the patch level in the
   wavelet Galerkin discretization. The annotation report the number of
   MEP evaluation points.
   The compression parameter triplet was set to $a = 2.5,\ d^\prime =
   2.0,\ b = 0.01$ for the piecewise constant wavelets and to $a = 2.5,\
   d^\prime = 3.0,\ b = 0.01$ for piecewise bilinear wavelets.
   The limit value is extrapolated from the results obtained from the
   piecewise bilinear wavelet Galerkin scheme.}
  \label{fig:alpha_convergence}
\end{figure}


The interface to \LSDALTON was later released in version ...
providing \acs{PCM} capabilities to the software package.

\section*{Paper III: Continuum modelling of non homogeneous environments}


%************************************************
%*****************************************
%*****************************************
%*****************************************
%*****************************************
%*****************************************




