% mainfile: ../RobertoDiRemigioPhDThesis.tex
\chapter{Continuum solvation models}\label{ch:CSM}


\begin{itemize}
    \item why and how continuum solvation models
    \item derivation of \acs{IEF}-\acs{PCM}, importance of Green's
      functions.
    \item \ac{BEM} and wavelet \acs{BEM} I think it's necessary to
      mention also ddCOSMO and ddPCM as alternative strategies (put a
      reference)
    \item variational formulation (no derivations, just mention the
      relevant literature)
\end{itemize}

\pagebreak

\section[Variational Formulation of Classical Polarizable Models]{
A Unifying View of Classical Polarizable Models within a Variational Formulation}
\label{sec:variational}

Variational functional for implicit models, \acs{PCM} in this case:
\begin{equation}
 U_\mathrm{PCM} = \frac{1}{2}\scalprod{\sigma}{\PCM\sigma} + \scalprod{\esp}{\sigma}
\end{equation}

Variational functional for explicit models, could be MMpol or PE or FQ:
\begin{equation}
  U_\mathrm{MM} = \frac{1}{2}\kappa\MM\kappa + \kappa\zeta
\end{equation}

The PCM equations will be written in the ``complete basis'' meaning that
we will introduce the usual boundary-element method (BEM) discretization
at the very end of the derivation. In other words, we will be working
with the exact integral equation and not with its discretized
counterpart. As a consequence, the apparent surface charge $\sigma$ and
the electrostatic potential $\esp$ will have a \emph{continuous}
dependence on a ``cavity surface'' index $\vect{s}$. Whenever a
charge-potential product is present it is then to be interpreted as the
\emph{surface integral}, i.e. the scalar product in the suitable,
infinite-dimensional vector space on the cavity boundary $\Gamma$. The
following shorthand notation will be adopted:
\begin{equation}
 \sigma\esp = \scalprod{\sigma}{\esp}
\end{equation}

Coupled implicit/explicit polarization energy functional:
\begin{equation}\label{eq:pcm-mm-functional}
  U_\mathrm{pol} =
   \frac{1}{2}\sigma\PCM\sigma + \sigma\esp
 + \frac{1}{2}\kappa\MM\kappa + \kappa\zeta
 + \sigma\bi{X}\kappa
\end{equation}
where one has:
\begin{equation}
  \sigma\MMPCM\kappa = \kappa\MMPCM^\dagger\sigma
\end{equation}
and $\MMPCM$ is the implicit/explicit interaction kernel. This is
charge/dipole or charge/charge interaction kernel for the MMpol and PE models
or the FQ model, respectively.
The global minimum of the convex functional is found by differentiating
it with respect to the variational degrees of freedom, \ie{} the
\acl{ASC} (\acs{ASC}) density $\sigma$ and the classical point
charges/dipoles $\kappa$. This leads to the coupled equations:
\begin{align}
  \PCM\sigma + \esp + \MMPCM\kappa &= 0 \\
  \MM\kappa  + \zeta + \MMPCM^\dagger\sigma &= 0
\end{align}
or in the commonly found matrix form:
\begin{equation}
  \begin{pmatrix}
    \PCM & \MMPCM \\
    \MMPCM^\dagger & \MM
  \end{pmatrix}
  \begin{pmatrix}
   \sigma \\
   \kappa
  \end{pmatrix}
  =
  -
  \begin{pmatrix}
   \esp \\
   \zeta
  \end{pmatrix}
\end{equation}
Finally, let us re-express the equations above in a "supermatrix"
formalism:
\begin{equation}
  U_\mathrm{pol} =
  \frac{1}{2}{}^t\p\V\p + {}^t\p\s
\end{equation}
where:
\begin{alignat}{3}
  \p =
  \begin{pmatrix}
    \sigma \\
    \kappa
  \end{pmatrix},
  \quad&
  \s =
  \begin{pmatrix}
   \esp \\
   \zeta
  \end{pmatrix},
  \quad&
  \V =
  \begin{pmatrix}
    \PCM & \MMPCM \\
    \MMPCM^\dagger & \MM
  \end{pmatrix}
\end{alignat}
and the ${}^t\p$ symbol denotes the transposed supervector $\p$.
The supermatrix form will prove particularly useful in the following
derivations.

\todo[inline]{PUT REFERENCES HERE!!!}
The variational formulation offers a series of advantages.
The classical energy functional:
\begin{equation}
  U_\mathrm{pol} =
   \frac{1}{2}\sigma\PCM\sigma + \sigma\esp
 + \frac{1}{2}\kappa\MM\kappa + \kappa\zeta
 + \sigma\bi{X}\kappa
\end{equation}
renders itself to a straightforward physical interpretation. In the
three bilinear terms, \ie{} the ones mediated by an interaction
operator, we can identify the self-interaction of the classical
polarization with itself, be it purely implicit:
$\frac{1}{2}\sigma\PCM\sigma$, purely explicit
$\frac{1}{2}\kappa\MM\kappa$ or mixed $\sigma\MMPCM\kappa$.
These terms are positive definite and give rise to an unfavorable
contribution, which is counterbalanced by the linear terms. These terms
mediate the interaction between the induced polarization and the
inducing fields: $\esp$ and $\zeta$. While $\esp$ is quite clearly the
\acl{MEP} (\acs{MEP}), $\zeta$ can either be the molecular electric
field (MMpol and PE models) or again the \acs{MEP} (FQ model).
In any case, both will be determined by the quantum mechanical molecular
charge density and can thus be formulated as expectation values of
one-electron operators. Eventually, this achieves the coupling between
the classical -- $\sigma$ and $\kappa$ -- and
the quantum mechanical variational degrees of freedom.

\todo[inline]{Mention the classical equivalent of the Hellmann--Feynman
theorem and why it is particularly useful in formulating response
theory.}
