% mainfile: ../RobertoDiRemigioPhDThesis.tex

\renewcommand{\thefigure}{\arabic{chapter}.\arabic{figure}}

\chapter{A Quick Tour of Quantum Chemistry}\label{ch:QM}

\begin{itemize}
  \item nonrelativistic and relativistic Hamiltonians used in molecular
    quantum mechanics
  \item \ac{BO} approximation and the concept of \ac{PES}
  \item Fock space
  \item single-reference methods: \ac{HF} and \ac{KS} \ac{DFT}
  \item the \ac{CC}  model
\end{itemize}

\pagebreak

\section{Molecular Quantum Mechanical Hamiltonians}\label{sec:hamiltonians}

Konishi and Paffuti\autocite{Konishi2009-zb}
Dyall's book\autocite{Dyall2007-tu}
Reiher's book\autocite{Reiher2014-cp}
The Pink Book\autocite{Helgaker2000-tz}
Shavitt's book\autocite{Shavitt2009-mr}

\section{A Brief Summary of Coupled Cluster Theory}\label{sec:coupled-cluster}

\todo[inline]{MOVED FROM CHAPTER 4 HERE: READJUST!}

The analyses and derivations in this Chapter are preeminently based
on the \acl{CC} wave function Ansatz.
We will assume a closed-shell, \acs{HF} single reference function and
construct our correlated wave function as an exponential mapping:
\begin{equation}
  \ket{\mathrm{CC}} = \expo{T}\ket{\mathrm{HF}}.
\end{equation}
The \emph{cluster} operator appearing in the exponential is given as:
\begin{equation}
  T = \sum_{p=1}^\mathcal{M} T_p = \sum_{p=1}^\mathcal{M}\sum_{\mu_p}\tamp{\mu_p}\cluster{\mu_p}
\end{equation}
that is, as a truncated sum of excitation operators, $\cluster{\mu_p}$,
times the corresponding cluster amplitudes, $\tamp{\mu_p}$.
Here $\mu_p$ is the $\mu$-th excitation at the $p$-th excitation level and
$\mathcal{M}$ is the truncation level.
One of the main strengths of the \acs{CC} model is its size extensivity
which stems directly from the exponential nature of the wave
operator.~\autocite{Shavitt2009-mr, Helgaker2000-tz}

The \acs{CC} method is a \emph{projective} method. Chosen the truncation
level $\mathcal{M}$, one projects the nonrelativistic Schrödinger
equation for this Ansatz on the excitation manifold which comprises the
reference function and all possible excited determinants, up to the
chosen truncation order:
\begin{subequations}
  \begin{align}
    \braket{\mathrm{HF} | H_0\expo{T} | \mathrm{HF}} &= E_\mathrm{CC} \\
    \braket{\mathrm{exc} | H_0\expo{T} | \mathrm{HF}} &=
    E_\mathrm{CC}\braket{\mathrm{exc} | \expo{T} | \mathrm{HF}}
  \end{align}
\end{subequations}
where $\bra{\mathrm{exc}}$ means any of the excited determinants in Fock
space compatible with the excitation level truncation.
This is the \emph{unlinked} form of the \acs{CC}
equations.~\autocite{Helgaker2000-tz} It is usually more convenient to
perform a \emph{similarity transformation} of the Hamiltonian operator,
leading to the \emph{linked} form of the \acs{CC} equations:
\begin{subequations}
  \begin{align}
    \braket{\mathrm{HF} | \expo{-T}H_0\expo{T} | \mathrm{HF}} &= E_\mathrm{CC} \\
    \braket{\mathrm{exc} | \expo{-T}H_0\expo{T} | \mathrm{HF}} &= 0
  \end{align}
\end{subequations}
The linked and unlinked formulations are completely equivalent. However,
the former is size extensive term-by-term.
Use of similarity transformed operators is ubiquitous when dealing with
\acs{CC} theory and we introduce the following notation for it:
\begin{equation}
  \overline{O} =\expo{-T}O\expo{T}
\end{equation}
The cluster operator is not Hermitian. Hence, the similarity transformation
will not, in general, preserve any of the symmetries, such as
hermiticity, that the bare operators might have possessed.
Similarity-transformed operators can be expanded in a commutator series,
also known as the \ac{BCH} expansion:
\begin{equation}
  \overline{O} =
  \sum_{n=0}^{\infty}\frac{1}{n!}[A, B]^n
\end{equation}
where $[A,B]^n$ is the $n$-fold nested commutator. It can be shown that
the \acs{BCH} expansion of the similarity-transformed Hamiltonian
\emph{exactly} truncates after the four-fold nested commutator, greatly
simplifying algebraic derivations and manipulations.

Introducing the \acl{MP} partitioning of the Hamiltonian leads to
further insight:
\begin{equation}
 H_0 = F + \Phi = \sum_P\epsilon_P\cons{P}\anni{P} + (g- V_\mathrm{HF})
\end{equation}
where $F$ if the Fock matrix and $\Phi$ is the fluctuation potential.
Given our initial assumption on the reference function, the Fock
operator is diagonal and expressed in terms of spin-orbital energies and
number operators.
Its similarity transformation truncates after the second term and has a
relatively compact form:
\begin{equation}
  \overline{F} = F + \sum_{i}^4\sum_{\mu_i}\tamp{\mu_i}\cluster{\mu_i}\denom{\mu_i}
\end{equation}
where $\denom{\mu_i}$ is the difference in orbital energies between the
virtual and occupied spin-orbitals of excitation $i$.
For the fluctuation potential the similarity transformation truncates
after the fifth term, including up to four-fold nested commutator.

As formulated so far, the \acs{CC} method is not variational. It is
however possible to introduce a fully variational Lagrangian that
leads to the same amplitudes equations. The Lagrangian is built by
treating the amplitude equations as constraints in the optimization of
the \acs{CC} energy:
\begin{equation}\label{eq:cc-lagrangian}
  \begin{aligned}
    \mathcal{L}^\mathcal{M}(\tamp{}, \tbar{})
    &=
    \braket{\mathrm{HF} | \overline{H_0} | \mathrm{HF}}
    + \sum_{i=1}^{\mathcal{M}}\sum_{\mu_i}
    \tbar{\mu_i}\braket{\mu_i | \overline{H_0} | \mathrm{HF}} \\
    &=
  E_0
  + \sum_{i=1}^{\mathcal{M}}\tbar{\mu_i}\denom{\mu_i}\tamp{\mu_i}
  + \braket{\mathrm{HF} | \overline{\Phi} | \mathrm{HF}}
  + \sum_{i=1}^{\mathcal{M}}\braket{\tbar{i} | \overline{\Phi} | \mathrm{HF}}
  \end{aligned}
\end{equation}
where the shorthand notation for the Lagrangian multiplier state was
introduced:
\begin{equation}
  \bra{\tbar{j}} = \sum_{\mu_j}\tbar{\mu_j}\bra{\mu_j}
\end{equation}
and the \acs{MP}-partitioned form of the Hamiltonian was also exploited.
Differentiation of the Lagrangian with respect to the multipliers
correctly yields the amplitudes equations, while differentiation with
respect to the amplitudes leads to the governing equations for the
multipliers. While these are not needed for the evaluation of the
\acs{CC} energy, their calculation is mandatory when expectation values
and molecular properties in general are sought.
The \acs{CC} expectation values are formed using the left and
right \acs{CC} states and we introduce the following notation:
\begin{equation}
  O(\tamp{}, \tbar{}) = \braket{\mathrm{HF} | \overline{O} | \mathrm{HF} }
  + \sum_{i=1}^{\mathcal{M}}\braket{\tbar{i} | \overline{O} | \mathrm{HF} }
\end{equation}
When required by the context, we will explicitly write at which
truncation level in the cluster operator the expectation was evaluated:
$O^\mathcal{M}(\tamp{}, \tbar{})$.

Having summarized the essential points of \acs{CC} theory and having
introduced the relevant notation, we can proceed to analyze its coupling
with quantum/classical polarizable Hamiltonians.

