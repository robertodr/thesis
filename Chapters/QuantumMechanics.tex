% mainfile: ../RobertoDiRemigioPhDThesis.tex

\renewcommand{\thefigure}{\arabic{chapter}.\arabic{figure}}

\chapter{A Quick Tour of Molecular Electronic Structure Theory}\label{ch:QM}

The purpose of quantum chemistry is to provide models based on first
principles that can help understand and predict molecular properties.
In quantum chemistry, we apply physical models based on
quantum many-body methods to molecular systems, employ their
mathematical realizations and devise computable approximations.
The central idea is, in fact, to be able to obtain an algorithmic
implementation of the methods that can be applied to interesting
chemical systems.

This Chapter presents a brief overview of molecular electronic structure
theory with particular emphasis on the methods used in this thesis.
Section \ref{sec:mqm} is a general introduction to the methods of
molecular quantum mechanics, relativistic or nonrelativistic.
I will provide a brief introduction to the language of second
quantization, used throughout the thesis.
Section \ref{sec:mean-field} presents the mean-field approximation to
the molecular electronic structure problem, the workhorse of quantum
chemistry.
Sections \ref{sec:coupled-cluster}--\ref{sec:cc-approximate} are
concerned with the \gls{CC} and \gls{MBPT} methods for the inclusion of
electron correlation.

\section{Molecular Quantum Mechanics}\label{sec:mqm}

Quantum mechanics is the theory describing the motion and interactions
of microscopic particles. In quantum theory every observable of the
system is represented mathematically by an operator $O$ in the
appropriate Hilbert space of \emph{wave functions}.
A wave function $\psi$ is the mathematical object describing the state of the
system. In the usual Copenhagen interpretation of quantum mechanics, the
modulus square of the wave function $|\psi|^2$ provides the key to
predicting experimentally measurable quantities.\autocite{Konishi2009-zb}
The wave function $\psi$ is the solution to the Schr\"{o}dinger equation:
\begin{equation}\label{eq:schrodinger}
  H\psi = E\psi
\end{equation}
where $H$ is the \emph{Hamiltonian} operator and $E$ is the energy of
the system. The Schr\"{o}dinger equation is an eigenvalue equation for
the Hamiltonian operator and $\psi$ is thus an
eigenfunction.\autocite{Arfken2013-pt}

Molecular quantum mechanics is concerned with the microscopic motion of
nuclei and electrons in molecules and is thus an intrinsically many-body
theory. The problem is clearly very complicated to solve, as the number of
interactions to be considered is large.
Since the nuclei are much heavier than the electrons, the \gls{BO} or
clamped-nuclei approximation is typically assumed. The nuclei are fixed
in a specific configuration (called a \emph{molecular geometry}) and
treated as static electric sources.\autocite{Born1927-ce}
This separation of motions results in an electron-only Hamiltonian and
an electronic wave function that depends \emph{parametrically} on the
nuclear positions. The eigenvalue of this Hamiltonian, also dependent on
the nuclear positions, is the \gls{PES} of the system, that is the
potential function in which the nuclei move.\autocite{Szabo1989-vl,
McWeeny1992-oj}
In first quantization, the general expression for the clamped-nuclei,
$N$-electron, molecular electronic Hamiltonian is:
\begin{alignat}{2}\label{eq:fq-hamiltonian}
  H = \sum_{i=1}^N h(\vect{r}_i)
  + \frac{1}{2}\sum_{i\neq j}^N g(\vect{r}_i, \vect{r}_j)
  + V_\mathrm{NN},
  \quad&
  V_\mathrm{NN} =
  \frac{1}{2}\sum_{A \neq B}^{N_\mathrm{nuclei}}\frac{Z_A Z_B}{|\vect{R}_A - \vect{R}_B|}
\end{alignat}
The last term is the nuclear repulsion energy, a constant shift of the
energy scale in the \acrshort{BO} approximation.
As noted by \citeauthor{Saue2011-qg}, the same general expression for
the clamped-nuclei Hamiltonian Eq.~\eqref{eq:fq-hamiltonian} is valid in
the nonrelativistic and relativistic no-pair
regimes.\autocite{Sucher1980-vf, Saue2011-qg}
In the nonrelativistic domain one defines the one- and two-electron
operators as:\autocite{Szabo1989-vl, McWeeny1992-oj, Konishi2009-zb}
\begin{subequations}
 \begin{align}
   h(\vect{r}_i) &=
   -\frac{1}{2}\nabla^2_i + V_\mathrm{Ne}(\vect{r}_i)
   = -\frac{1}{2}\nabla^2_i - \sum_{A=1}^{N_\mathrm{nuclei}}\frac{Z_A}{|\vect{R}_A -
   \vect{r}_i|} \label{eq:h} \\
   g(\vect{r}_i, \vect{r}_j) &= \frac{1}{|\vect{r}_i - \vect{r}_j|},
   \label{eq:g-coulomb}
 \end{align}
\end{subequations}
while in the relativistic regime with the \gls{DC}
approximation:\autocite{Dyall2007-tu, Konishi2009-zb, Reiher2014-cp}
\begin{subequations}
 \begin{align}
   h(\vect{r}_i) &=
   \begin{pmatrix}
     V_{\mathrm{Ne}}(\vect{r}_i)I_2 & c \vect{\sigma}\cdot\vect{p}_i \\
     \vect{\sigma}\cdot\vect{p}_i & (V_{\mathrm{Ne}}(\vect{r}_i)-2\si{\electronmass}c^2)I_2
   \end{pmatrix} \label{eq:h-dirac} \\
   g(\vect{r}_i, \vect{r}_j) &= \frac{I_4\cdot I_4}{|\vect{r}_i - \vect{r}_j|}
   \label{eq:g-r-coulomb}
 \end{align}
\end{subequations}
and the operators are implicitly assumed to embed the proper no-pair,
positive-energy manifold projectors.\autocite{Sucher1980-vf}
Since these are fermionic Hamiltonians, the spin-statistics theorem
dictates the corresponding eigenfunctions to be
antisymmetric.
Using the expansion theorem,\autocite{Gross1991-hi} we thus seek the wave
function for the $N$-electron system as a linear combination of
$N$-electron \emph{Slater determinants}.

It the language of second quantization, we assume that a, not
necessarily complete, set of orthonormal one-particle states is
available. These are usually called \glspl{MO}.
In general, we would like to embed as many of the symmetries of the
system into the one-particle basis. This is accomplished by requiring
the one-particle space to be the common eigenbasis for a complete set of
suitable, commuting one-particle operators.\autocite{Gross1991-hi}
Any $N$-particle state can be constructed by distributing particles
into one-particle states and accordingly labelling them by the number of
particles per one-particle states present.
This is the so-called \gls{ON} vector representation:
\begin{equation}
  \ket{\vect{n}} = \ket{n1,n2,\ldots},
\end{equation}
and the span of all \acrshort{ON} vectors is called the \emph{Fock
space} $F^N$, a space isomorphic to the Hilbert space $H^N$ of
$N$-particle wave functions.
To preserve consistency, the \emph{vacuum} state with no particles:
\begin{equation}
 \ket{\mathrm{vac}} = \ket{0,0,\ldots}
\end{equation}
is included in the construction of Fock space.
Any $N$-particle state can be generated by application of
\emph{creation} operators $\cons{i}$. Creation operators increase the
occupation number in the one-particle state $i$ by $1$, up to a phase.
Conversely, \emph{annihilation} operators $\anni{i}$, defined as the Hermitian
conjugates of the creation operators, decrease occupation in state $i$
by $1$, again up to a phase. Clearly, annihilation of any one-particle
state in the vacuum results yields $0$.
Pauli's exclusion principle is enforced by the \emph{canonical
anticommutation relations}:
\begin{subequations}
  \begin{align}
    \cons{r}\cons{s} + \cons{s}\cons{r} &= 0 \\
    \anni{r}\anni{s} + \anni{s}\anni{r} &= 0 \\
    \cons{r}\anni{s} + \anni{s}\cons{r} &= \delta_{rs},
  \end{align}
\end{subequations}
showing how the construction of the Fock space representation
embeds antisymmetry at the operator level and not in the wave functions.
Finally, the second-quantized molecular electronic Hamiltonian in the \acrlong{BO}
approximation is:\autocite{Helgaker2000-tz, Shavitt2009-mr}
\begin{equation}\label{eq:sq-hamiltonian}
  H = \sum_{rs}h_{rs}\cons{r}\anni{s}
  + \frac{1}{2}\sum_{rstu}g_{rstu}\cons{r}\cons{t}\anni{u}\anni{s}
  + V_\mathrm{NN}
\end{equation}
where the matrix elements are given as integrals over the chosen
one-particle basis:
\begin{subequations}
 \begin{align}
   h_{rs} &= \int \diff\vect{r} \varphi_r^\dagger(\vect{r}) h(\vect{r})\varphi_s(\vect{r}) \label{eq:one-el}\\
   g_{rstu} &=
   \int\diff\vect{r}\int\diff\vect{r}^\prime
   \Omega_{rs}(\vect{r}) g(\vect{r},\vect{r}^\prime) \Omega_{tu}(\vect{r}^\prime) \label{eq:two-el} \\
  \Omega_{rs}(\vect{r}) &= \varphi_r^\dagger(\vect{r})\varphi_s(\vect{r}). \label{eq:overlap}
 \end{align}
\end{subequations}
The second-quantized Hamiltonian Eq.~\eqref{eq:sq-hamiltonian} is thus a
\emph{projected} operator, exact only within the subspace determined by
the chosen one-particle set.~\autocite{Gross1991-hi, Helgaker2000-tz}

In second quantization, the exact molecular electronic wave function can
be written as a linear combination of \acrshort{ON} vectors:
\begin{equation}
  \ket{\psi} = \sum_{\vect{k}}C_\vect{k}\ket{\vect{k}},
\end{equation}
also known at the \gls{FCI} expansion.
\todo[inline]{Variational property and size-extensivity.}
Coefficients in the expansion corresponding to the molecular ground
state can be determined by seeking the lowest eigenvalue of the matrix
representation of the Hamiltonian in the space of \acrshort{ON} vectors.
The \acrshort{FCI} method is exact, within the chosen one-particle set,
but scales exponentially with the number of electrons in the system and
has thus very limited practical applications.
Approximations can be tailored by appropriate truncations of the
$N$-electron space.
In \emph{single-reference} methods, one first determines an
approximation to the solution by means of a single \acrshort{ON} vector.
This \acrshort{ON} vector is treated as the \emph{physical} vacuum:
vectors in Fock space can be classified according to how different their
occupation is from the reference, \ie~\emph{by excitation
level}.\autocite{Helgaker2000-tz, Shavitt2009-mr}
A truncated expansion can then be constructed by including only those
\acrshort{ON} vectors that differ from the reference up to a certain
number of excitations.
Further details on how the expansions are actually tailored and
algorithmically implemented can be found elsewhere.\autocite{Helgaker2000-tz}
In the rest of this Chapter we will describe how a single \acrshort{ON}
vector approximation can be constructed and how we can improve on it by
means of \acrlong{CC} and \acrlong{MBPT}.

\section{Mean-Field Theory}\label{sec:mean-field}

Basing ourselves on a variational principle, we seek the best
description of the many-electron wave function as a single Slater
determinant. This implies that there is no variational freedom in the
$N$-electron space. Our variational degrees of freedom reside in
the one-electron space, \ie in the space of \gslpl{MO}
$\lbrace\varphi_r\rbrace$, used to build the determinant $\Ket{0}$.
The one-electron space is developed by considering the behaviour of the
single electron in the mean field of all the other electrons, and while
this neglects a smaller part of the interaction energy, the electron
correlation, it provides a suitable starting point for further
variational or perturbational treatments to recover more of the
electron--electron interaction.

\todo[inline]{CURRENTLY VERBATIM FROM MASTER THESIS!!!!}

The problem with introducing more electrons in the theory is that
electron--electron interactions that were previously small now occur to
the same order as the kinetic energy and the interaction with the
nuclear potential. The technique for dealing with this problem is
well-known from nonrelativistic theory of many-electron systems:
mean-field theory.

Basing ourselves on a variational principle, we seek the best
description of the many-electron wave function as a single Slater
determinant. This implies that there is no variational freedom in the
$N$-electron space (\ie the subset of Fock space describing $N$ particle
systems, $\mathcal{F}^N$). Our variational degrees of freedom reside in
the one-electron space, \ie in the space of Molecular Orbitals,
$\lbrace\varphi_r\rbrace$, used to build the determinant $\Ket{0}$.
The one-electron space is developed by considering the behaviour of the
single electron in the mean field of all the other electrons, and while
this neglects a smaller part of the interaction energy, the electron
correlation, it provides a suitable starting point for further
variational or perturbational treatments to recover more of the
electron--electron interaction.

To derive a set of relativistic one-particle functions, \ie 4-spinors,
from a mean-field approach we typically start by making a simple guess
at these functions and then try to refine them iteratively by applying
suitable rotations in the entire one-particle space available. This
iterative procedure normally reaches a stage were further rotations do
not change the spinors, \ie they are \emph{self-consistent}.

As the molecular 4-spinors $\lbrace\varphi_r\rbrace$ are orthonormal,
the resulting $N$-electron space is made of normalized determinants.
Consequently, only norm-conserving transformations of the one-electron
space underlying $\mathcal{F}^N$ must be considered.\footnote{In
general, any suitable transformation of the one-electron space must
enforce \emph{all} the symmetries presented by the system under
consideration.}
This restricts our attention to the group of unitary transformations:
\begin{equation}
 U^\dagger U = id = UU^\dagger.
\end{equation}

Any unitary transformation may be parametrized as the exponential of an
anti-Hermitian operator:\autocite{Arfken2013-pt, Helgaker2000-tz}
\begin{alignat}{2}
 U = \exp(-\kappa) \quad&\quad \kappa^\dagger = -\kappa
\end{alignat}
the anti-Hermitian operator being given as a linear combination of
single excitation operators:
\begin{alignat}{2}\label{eq:orb-rot}
  \kappa = \sum_{rs} \kappa_{rs} \cons{r}\anni{s} -\kappa^*_{rs}\cons{s}\anni{r}
= \sum_{rs}\kappa_{rs}q^\dagger_{rs} - \kappa^*_{rs}q_{rs}
\quad&\quad \kappa_{sr}^* = -\kappa_{rs}
\end{alignat}
and the \emph{orbital rotation} parameters $\kappa_{rs}$ were introduced.
It may be easily verified that the matrix representation of $U$ in the
$\lbrace\varphi_{r}\rbrace$ basis is given by:
\begin{equation}
 \mat{U} = \exp(-\vect{\kappa}).
\end{equation}

Before proceeding with the minimization procedure, let us rewrite the
orbital rotation operator in a more convenient notation. As already
pointed out, orbitals will in general be complex. The usual reduction in
the parametrization of $\kappa$ in the nonrelativistic regime will not
be observed. Instead of working with the real and imaginary parts of the
orbital rotation parameters, we will treat the $\vect{\kappa}$ and
$\vect{\kappa}^*$ matrices as \emph{independent} variables. Hence,
defining the following vectors:
\begin{alignat}{2}
\vect{K} =
 \begin{pmatrix}
  \vect{\kappa} \\
  \vect{\kappa}^*
 \end{pmatrix}
\quad&\quad
\vect{Q} =
 \begin{pmatrix}
  -\vect{q} \\
  \vect{q}^\dagger
 \end{pmatrix}
\end{alignat}
Eq. \eqref{eq:orb-rot} may be rewritten as:
\begin{equation}\label{eq:orb-rot-new}
  \kappa = \frac{1}{2}\left[ \vect{K}^\dagger\vect{Q} - \vect{Q}^\dagger\vect{K}\right]
\end{equation}

Let us note that the parametrization of $\kappa$ is, in general
\emph{redundant}, in the sense that it contains certain rotations which
do not change the energy of the state. Redundant rotations are
inessential in the optimization of the wave function and may lead to
numerical instabilities in second order algorithms.\autocite{Helgaker2000-tz,
Shepard1987-sn}
It is then mandatory to identify and eliminate them. For a single
determinant this is an easy task as all redundant rotations satisfy the
condition:
\begin{equation}
 \cons{r}\anni{s}\Ket{0} = 0.
\end{equation}

For a closed shell only those rotations connecting occupied and virtual
orbitals will be nonredundant. The redundancy-free orbital rotation
generator is then:
\begin{equation}\label{eq:orb-rot-nonred}
  \kappa = \sum_{ai}\kappa_{ai}\cons{a}\anni{i} -\kappa_{ai}^* \cons{i}\anni{a}
\end{equation}
The advantage of this parametrization is that the orbital rotation
operator $\exp(-\kappa)$ ensures orthonormality of the one-particle
orbitals without the need to introduce Lagrange multipliers:
unconstrained optimization approaches can be used.

By means of $\exp(-\kappa)$ the energy is established as a function of
the orbital rotation parameters:
\begin{equation}\label{eq:param-energy}
 E(\vect{K}) = \Braket{0(\vect{K})|H_0|0(\vect{K})} = \Braket{0|\exp(\kappa)H\exp(-\kappa)|0}.
\end{equation}
The optimal 4-spinors are then found by minimizing $E(\vect{K})$. The
Hamiltonian in Eq.~\eqref{eq:param-energy} has the second-quantized
expression of Eq.~\eqref{eq:sq-hamiltonian} and the integrals are given as
in Eqs.~\eqref{eq:one-el}-\eqref{eq:overlap}.

We perform a Taylor expansion up to second-order of the right-hand-side
of Eq.~\eqref{eq:param-energy}, all derivatives are evaluated at the
current expansion point arbitrarily set at
$\vect{K}=0$:
\begin{equation}\label{eq:taylor-exp}
 E(\vect{K}) = E^{[0]} +\vect{K}^\dagger \mat{E}^{[1]} + \frac{1}{2}\vect{K}^\dagger \mat{E}^{[2]}\vect{K} + O(\vect{K}^3)
\end{equation}
where the electronic gradient and Hessian are given as:
\begin{subequations}
 \begin{align}
 \mat{E}^{[1]}
&=
\begin{pmatrix}
 \pderiv{E}{{\vect{\kappa}^*}} \\
 \pderiv{E}{{\vect{\kappa}}}
\end{pmatrix}
\\
\mat{E}^{[2]}
&=
\begin{pmatrix}
 \frac{\partial^2 E}{\partial\boldsymbol{\kappa}^*\partial\boldsymbol{\kappa}} &
 \frac{\partial^2 E}{\partial\boldsymbol{\kappa}^*\partial\boldsymbol{\kappa}^*} \\[0.3em]
 \frac{\partial^2 E}{\partial\boldsymbol{\kappa}\partial\boldsymbol{\kappa}} &
 \frac{\partial^2 E}{\partial\boldsymbol{\kappa}\partial\boldsymbol{\kappa}^*}
\end{pmatrix}
=
\begin{pmatrix}
                  \mat{A} & \mat{B} \\
                  \mat{B}^* & \mat{A}^*
\end{pmatrix}
 \end{align}
\end{subequations}

Expansion of Eq. \eqref{eq:param-energy} in a \gls{BCH}
commutator series yields:\autocite{Arfken2013-pt}
\begin{equation}
 E(\vect{K}) = E(0) + \braket{0|\commutator{\kappa}{H_0}|0} +\frac{1}{2}\braket{0|\commutator{\kappa}{\commutator{\kappa}{H_0}}|0}+O(\vect{\kappa}^3)
\end{equation}

We will now concentrate only on the expression of the electronic energy
gradient, the Hessian will be analyzed in Chapter
\ref{ch:molprop}.
Differentiating the expression above with respect to $\vect{K}$ gives the gradient as:
\begin{equation}\label{eq:brillouin}
 E^{[1]}_{ai} = \Braket{0|\commutator{-\cons{i}\anni{a}}{H_0}|0}.
\end{equation}
The derivative has been evaluated at $\vect{K}=0$, \ie the current
expansion point. The stationarity condition Eq.~\eqref{eq:brillouin} is
also known as \emph{Brillouin's Theorem}.
The explicit expression of the electronic energy gradient follows from a
straightforward application of Wick's theorem:\autocite{Wick1950-iy,
Shavitt2009-mr}
\begin{equation}\label{eq:stationary}
  E^{[1]}_{ai} = -h_{ai} - \sum_{j}[g_{aijj} - g_{ajji}]
\end{equation}

We define the \emph{Fock matrix} $\mat{F}$ with elements:
\begin{equation}
  F_{rs} = h_{rs} + \sum_{j} [g_{rsjj} - g_{rjjs}]
\end{equation}
apart from the one-electron term, whose expression was given in Eq.~\eqref{eq:one-el},
the Fock matrix features two types of two-electron
terms Eq.~\eqref{eq:two-el}.
The former is due to the Coulomb interaction, while the second is the
\emph{exchange} contribution arising because of antisymmetry in the
many-electron wave function, in accordance with Pauli principle.

In terms of the Fock matrix, the stationarity condition Eq.~\eqref{eq:stationary} reads:
\begin{equation}
  f_{ai} = h_{ai} + \sum_{j}[g_{aijj} - g_{ajji}] = 0
\end{equation}
meaning that the Fock matrix is block-diagonal in the basis of the
optimal MO 4-spinors, \ie~the occupied-virtual and virtual-occupied
blocks are zero. Nothing is said about the diagonal blocks (\ie~
Occupied-Occupied and Virtual-Virtual) of this matrix, as the
corresponding orbital rotations are redundant. This arbitrariness may be
used to ease the wave function optimization, \eg~applying level-shifts,
or to require that the eigenvalues of the Fock matrix have particular
properties.\autocite{Szabo1989-vl, McWeeny1992-oj, Helgaker2000-tz}

For example, we can fix the diagonal blocks in order to reinterpret the
eigenvalues of the Fock matrix as ionization potentials and electron
affinities (Koopmans' Theorem).
The optimal 4-spinor rotation will bring the Fock matrix to diagonal
form, what is called its \emph{canonical representation}. Thus the
optimization of the one-particle space may be recast as a
diagonalization problem:
\begin{equation}\label{eq:fock-eq}
 \mat{F}\vect{\varphi} = \vect{\varphi}\mat{\epsilon}
\end{equation}
where the eigenvalues of the Fock matrix are called \emph{orbital
energies}. The theory exposed above goes under several names that we
will use interchangeably in the remainder of this thesis. Apart from
mean-field theory the names \gls{SCF} and
\gls{HF} theory are also in common usage.

%%%% Introduce DFT and KS-DFT
An interesting approach to the quantum mechanical description of
many-electron systems is based on the idea that it should be possible to
find a quantum theory that solely refers to observable quantities.
\gls{DFT}, instead of relying on a wave
function, is based on the electron density.
The Hohenberg--Kohn theorems\autocite{Hohenberg1964-wo, Eschrig2012-as}
and their relativistic counterparts\autocite{Rajagopal1973-ns,
Dreizler2012-ay} establish the energy as a \emph{functional} of the
density:
\begin{equation}
 E = E[\rho]
\end{equation}

In the \gls{KS} approach to \acrshort{DFT}, the density of the system of interest
is supposed to be equal to that of a fictitious noninteracting
system.\autocite{Kohn1965-hg}
The energy is then a sum of five terms:
\begin{equation}\label{eq:ks-energy}
 E[\rho] = T_s[\rho] + V_\mathrm{ext}[\rho] + J[\rho] + E_\mathrm{xc}[\rho] + V_\mathrm{NN}
\end{equation}
the first four of which  are functionals of the density. The first term
$T_s[\rho]$ is the kinetic energy of the fictitious noninteracting
system represented by a single Slater determinant $\ket{\widetilde{0}}$:
\begin{equation}
 T_s[\rho] = \sum_i \braket{\widetilde{0}|h_\mathrm{D}(i)|\widetilde{0}}.
\end{equation}
The second and third term in Eq.~\eqref{eq:ks-energy} represent,
respectively, the classical interaction of the electrons with the
external potential (including the nuclear attraction potential) and the
classical Coulomb interaction of the density with itself:
\begin{subequations}
 \begin{align}
   V_\mathrm{ext}[\rho] &= \int\diff\vect{r} V_\mathrm{ext}(\vect{r})\rho(\vect{r}) \\
   J[\rho] &=
   \frac{1}{2}\int\diff\vect{r}\int\diff\vect{r}^\prime
   \rho(\vect{r})g(\vect{r}, \vect{r}^\prime)\rho(\vect{r}^\prime)
 \end{align}
\end{subequations}
The fourth term in Eq.~\eqref{eq:ks-energy} is the
\gls{XC} functional and is a great unknown. This term
contains all two-electron interactions except for the classical
$J[\rho]$ term, \ie~it contains the effects of electron exchange and
electron correlation. It also corrects for the self-interaction present
in the classical Coulomb term and for the error in the evaluation of the
kinetic energy.
A number of approximations exist for this functional.
In the \gls{LDA}, the functional is expressed in terms of an
\emph{energy density} $e_\mathrm{xc}$ which is a local function of the
density:
\begin{equation}
 E^\mathrm{LDA}_\mathrm{xc}[\rho] = \int e_\mathrm{xc}(\rho(\vect{r}))\diff\vect{r}.
\end{equation}
In the \gls{GGA}, the energy
density is a local function of both the density and the norm of its
gradient $\zeta(\vect{r})=\vect{\nabla}\rho\cdot \vect{\nabla}\rho$:
\begin{equation}
 E^\mathrm{GGA}_\mathrm{xc}[\rho] = \int e_\mathrm{xc}(\rho(\vect{r}),\zeta(\vect{r}))\diff\vect{r}.
\end{equation}
Finally, in \emph{hybrid} functionals, some proportion of the
\acrshort{HF} exchange is included:
\begin{equation}
 E^\mathrm{hybrid}_\mathrm{xc}[\rho] = E^\mathrm{GGA}_\mathrm{xc}[\rho] + \gamma E^\mathrm{HF}_\mathrm{x}[\rho]
\end{equation}
The last term of Eq.~\eqref{eq:ks-energy} is the nuclear--nuclear repulsion energy.

In complete analogy with the mean-field theory exposed above,
\citeauthor{Saue2002-cu} have proposed a unitary parametrization of the
\acrshort{KS} state in terms of orbital rotations:\autocite{Saue2002-cu}
\begin{equation}
 \ket{\widetilde{0}} = \exp(-\kappa)\ket{0}.
\end{equation}
The density matrix is then a function of the orbital rotation parameters:
\begin{equation}
 \widetilde{D}_{rs}(\vect{\kappa}) = \braket{0|\exp(\kappa)\cons{r}\anni{s}\exp(-\kappa)|0}
\end{equation}
thus establishing the density $\rho$ and the norm of its gradient
$\zeta$ as functions of the orbital rotation parameters.
Using the overlap distributions defined in Eq.~\eqref{eq:overlap}:
\begin{subequations}
 \begin{align}
   \rho(\vect{r},\vect{\kappa}) &= \sum_{rs}\widetilde{D}_{rs}(\vect{\kappa})\Omega_{rs}(\vect{r}) \\
   \vect{\nabla} \rho(\vect{r},\vect{\kappa}) &= \sum_{rs}\widetilde{D}_{rs}(\vect{\kappa})\vect{\nabla}\Omega_{rs}(\vect{r}).
 \end{align}
\end{subequations}
The electronic energy gradient leads to the definition of the Fock-like
\emph{KS} matrix as:
\begin{equation}\label{eq:ksmatrix}
  f_{rs} = h_{rs} + \sum_{j}[g_{rsjj} - \gamma g_{rjjs}] + v_{\mathrm{xc};rs}
\end{equation}
the actual form of $v_{\mathrm{xc};rs}$ can be found in
\noparcite[refs.][]{Saue2002-cu}
and \noparcite[][]{Salek2005-at}

In the \acrshort{KS} approach to \acrshort{DFT}, the one-particle space
is then optimized in complete analogy with \acrshort{HF} theory, \ie~by
diagonalization of the \acrshort{KS} matrix:
\begin{equation}
 \mat{F}\vect{\varphi} = \vect{\varphi}\mat{\epsilon}
\end{equation}
unlike \acrshort{HF} theory, the \acrshort{KS} approach enables us to
recover some electron correlation, albeit through the use of \emph{ad
hoc} \acrshort{XC} functionals.\autocite{Koch2015-wq}

\section{The Coupled Cluster Ansatz}\label{sec:coupled-cluster}

\todo[inline]{MOVED FROM CHAPTER 4 HERE: READJUST!}

The analyses and derivations in this Chapter are preeminently based
on the \acrlong*{CC} wave function Ansatz.
We will assume a closed-shell, \acrshort{HF} single reference function and
construct our correlated wave function as an exponential mapping:
\begin{equation}
  \ket{\mathrm{CC}} = \expo{T}\ket{\mathrm{HF}}.
\end{equation}
The \emph{cluster} operator appearing in the exponential is given as:
\begin{equation}
  T = \sum_{p=1}^\mathcal{M} T_p = \sum_{p=1}^\mathcal{M}\sum_{\mu_p}\tamp{\mu_p}\cluster{\mu_p}
\end{equation}
that is, as a truncated sum of excitation operators, $\cluster{\mu_p}$,
times the corresponding cluster amplitudes, $\tamp{\mu_p}$.
Here $\mu_p$ is the $\mu$-th excitation at the $p$-th excitation level and
$\mathcal{M}$ is the truncation level.
One of the main strengths of the \acrshort{CC} model is its size-extensivity
which stems directly from the exponential nature of the wave
operator.~\autocite{Shavitt2009-mr, Helgaker2000-tz}

The \acrshort{CC} method is a \emph{projective} method. Chosen the truncation
level $\mathcal{M}$, one projects the nonrelativistic Schrödinger
equation for this Ansatz on the excitation manifold which comprises the
reference function and all possible excited determinants, up to the
chosen truncation order:
\begin{subequations}
  \begin{align}
    \braket{\mathrm{HF} | H_0\expo{T} | \mathrm{HF}} &= E_\mathrm{CC} \\
    \braket{\mathrm{exc} | H_0\expo{T} | \mathrm{HF}} &=
    E_\mathrm{CC}\braket{\mathrm{exc} | \expo{T} | \mathrm{HF}}
  \end{align}
\end{subequations}
where $\bra{\mathrm{exc}}$ means any of the excited determinants in Fock
space compatible with the excitation level truncation.
This is the \emph{unlinked} form of the \acrshort{CC}
equations.~\autocite{Helgaker2000-tz} It is usually more convenient to
perform a \emph{similarity transformation} of the Hamiltonian operator,
leading to the \emph{linked} form of the \acrshort{CC} equations:
\begin{subequations}
  \begin{align}
    \braket{\mathrm{HF} | \expo{-T}H_0\expo{T} | \mathrm{HF}} &= E_\mathrm{CC} \\
    \braket{\mathrm{exc} | \expo{-T}H_0\expo{T} | \mathrm{HF}} &= 0
  \end{align}
\end{subequations}
The linked and unlinked formulations are completely equivalent. However,
the former is size-extensive term-by-term.
Use of similarity transformed operators is ubiquitous when dealing with
\acrshort{CC} theory and we introduce the following notation for it:
\begin{equation}
  \overline{O} =\expo{-T}O\expo{T}
\end{equation}
The cluster operator is not Hermitian. Hence, the similarity transformation
will not, in general, preserve any of the symmetries, such as
hermiticity, that the bare operators might have possessed.
Similarity-transformed operators can be expanded in a \acrshort{BCH} commutator series:
\begin{equation}
  \overline{O} =
  \sum_{n=0}^{\infty}\frac{1}{n!}[A, B]^n
\end{equation}
where $[A,B]^n$ is the $n$-fold nested commutator. It can be shown that
the \acrshort{BCH} expansion of the similarity-transformed Hamiltonian
\emph{exactly} truncates after the four-fold nested commutator, greatly
simplifying algebraic derivations and manipulations.

Introducing the \acrlong*{MP} partitioning of the Hamiltonian leads to
further insight:
\begin{equation}
 H_0 = F + \Phi = \sum_P\epsilon_P\cons{P}\anni{P} + (g- V_\mathrm{HF})
\end{equation}
where $F$ if the Fock matrix and $\Phi$ is the fluctuation potential.
Given our initial assumption on the reference function, the Fock
operator is diagonal and expressed in terms of spin-orbital energies and
number operators.
Its similarity transformation truncates after the second term and has a
relatively compact form:
\begin{equation}
  \overline{F} = F + \sum_{i}^4\sum_{\mu_i}\tamp{\mu_i}\cluster{\mu_i}\denom{\mu_i}
\end{equation}
where $\denom{\mu_i}$ is the difference in orbital energies between the
virtual and occupied spin-orbitals of excitation $i$.
For the fluctuation potential the similarity transformation truncates
after the fifth term, including up to four-fold nested commutator.

As formulated so far, the \acrshort{CC} method is not variational. It is
however possible to introduce a fully variational Lagrangian that
leads to the same amplitudes equations. The Lagrangian is built by
treating the amplitude equations as constraints in the optimization of
the \acrshort{CC} energy:\autocite{Helgaker1982-xl, Arponen1983-qt,
Helgaker1988-to, Helgaker1989-wl, Koch1990-vr, Kvaal2013-jr}
\begin{equation}\label{eq:cc-lagrangian}
  \begin{aligned}
    \mathcal{L}^\mathcal{M}(\tamp{}, \tbar{})
    &=
    \braket{\mathrm{HF} | \overline{H_0} | \mathrm{HF}}
    + \sum_{i=1}^{\mathcal{M}}\sum_{\mu_i}
    \tbar{\mu_i}\braket{\mu_i | \overline{H_0} | \mathrm{HF}} \\
    &=
  E_0
  + \sum_{i=1}^{\mathcal{M}}\tbar{\mu_i}\denom{\mu_i}\tamp{\mu_i}
  + \braket{\mathrm{HF} | \overline{\Phi} | \mathrm{HF}}
  + \sum_{i=1}^{\mathcal{M}}\braket{\tbar{i} | \overline{\Phi} | \mathrm{HF}}
  \end{aligned}
\end{equation}
where the shorthand notation for the Lagrangian multiplier state was
introduced:
\begin{equation}
  \bra{\tbar{j}} = \sum_{\mu_j}\tbar{\mu_j}\bra{\mu_j}
\end{equation}
and the \acrshort{MP}-partitioned form of the Hamiltonian was also exploited.
Differentiation of the Lagrangian with respect to the multipliers
correctly yields the amplitudes equations, while differentiation with
respect to the amplitudes leads to the governing equations for the
multipliers. While these are not needed for the evaluation of the
\acrshort{CC} energy, their calculation is mandatory when expectation values
and molecular properties in general are sought.
The \acrshort{CC} expectation values are formed using the left and
right \acrshort{CC} states and we introduce the following notation:
\begin{equation}
  O(\tamp{}, \tbar{}) = \braket{\mathrm{HF} | \overline{O} | \mathrm{HF} }
  + \sum_{i=1}^{\mathcal{M}}\braket{\tbar{i} | \overline{O} | \mathrm{HF} }
\end{equation}
When required by the context, we will explicitly write at which
truncation level in the cluster operator the expectation was evaluated:
$O^\mathcal{M}(\tamp{}, \tbar{})$.

\section{Many-Body Perturbation Theory}\label{sec:mbpt}
\section{Approximate Coupled Cluster Methods}\label{sec:cc-approximate}
