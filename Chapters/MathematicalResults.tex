\chapter{Some Mathematical Results}\label{app:mathematical-results}

\begin{epigraphs}
\qitem{\textitalian{Non temerai i terrori della notte non temerai il terrore}}{--- \textsc{CCCP}}
\end{epigraphs}

For brevity's sake, some results and derivations have been omitted from the main
body of the thesis. I collect the ones I judge most relevant in this Appendix.
Section \ref{sec:T1-cc} presents the $T_1$-transformed form of the
\acrshort{CCS}, \acrshort{CCSD} and \acrshort{CCSDT} Lagrangians, the corresponding amplitudes
and multipliers equations and the one-electron operators expectation values.

Basic results in the fields of functional analysis and boundary integral
equations that were omitted from Chapter \ref{ch:CSM} are presented here.
No proofs or examples are given, the interested reader is referred to the
monographs by \citeauthor{Ern2004-oo}, \citeauthor{Hsiao2008-xb} and
\citeauthor{Sauter2011-an}\autocite{Ern2004-oo, Hsiao2008-xb, Sauter2011-an}

\section{The \texorpdfstring{$\T1$}{T1}-Transformation}\label{sec:T1-cc}

Carrying out a similarity transformation of
an operator $O$ by means of the $T_1$ cluster operator will preserve the particle
rank of $O$, since $T_1$ is a one-electron operator.\autocite{Koch1994-vr}
Starting from the arbitrary order Lagrangian \eqref{eq:cc-lagrangian},
we want to derive their $T_1$-transformed
expressions.
For the \acrshort{CCS} model, this is straightforward.
The Lagrangian is:
\begin{equation}\label{eq:ccs-lagrangian}
 \mathcal{L}(\tamp{}, \tbar{})_\mathrm{CCS}
 =
  E_0
  + \tbar{\mu_1}\denom{\mu_1}\tamp{\mu_1}
  + \braket{\mathrm{HF} | \check{\Phi} | \mathrm{HF}}
  + \braket{\tbar{1} | \check{\Phi} | \mathrm{HF}}
\end{equation}
the governing equations:
\begin{subequations}\label{eq:ccs-equations}
  \begin{align}
   \denom{\mu_1}\tamp{\mu_1} &+ \braket{\mu_1 | \check{\Phi} | \mathrm{HF}}
             = 0 \label{eq:ccs-amplitudes}\\
    \denom{\mu_1}\tbar{\mu_1} &+
    \braket{\mathrm{HF} | \commutator{\check{\Phi}}{\cluster{\mu_1}} | \mathrm{HF}} +
    \braket{\tbar{1} |
    \commutator{\check{\Phi}}{\cluster{\mu_1}} | \mathrm{HF}}
             = 0 \label{eq:ccs-multipliers}
  \end{align}
\end{subequations}

When including higher order excitations we will seek simplifications in the commutator expansions
by employing the following result:\autocite{Helgaker2000-tz}
\begin{lemma}[Excitation ranks manifold]\label{lemma:exc-rank}
The $k$-fold nested commutator of a \emph{particle rank} $m_O$ operator $O$ with
cluster operators $T_{n_i}$ of rank $n_i$ acting on the reference determinant:
\begin{equation}
  \commutator{\commutator{\commutator{O}{T_{n_1}}}{\ldots}}{T_{n_k}} \ket{\mathrm{HF}}
\end{equation}
generates a linear combination of determinants with excitation ranks $v$ in the
range:
\begin{equation}
  \sum_{i=1}^k n_i - m_O \leq v \leq \sum_{i=1}^k n_i + m_O -k
\end{equation}
\end{lemma}
\begin{corollary}[Excited states overlaps]
The overlap of a determinant $\mu_n$ with excitation rank $n$ onto the linear
combination of determinants generated by the $k$-fold nested commutator of an
operator $O$ with particle rank $m_O$ with cluster operators $T_{n_i}$ of rank $n_i$ is
nonzero if and only if the sum of cluster operator ranks satisfies:
\begin{equation}
  n - m_O + k \leq \sum_{i=1}^k n_i \leq n + m_O
\end{equation}
In compact form:
\begin{equation}
  \braket{\mu_n |
  \commutator{\commutator{\commutator{O}{T_{n_1}}}{\ldots}}{T_{n_k}} | \mathrm{HF}} \neq 0
  \Longleftrightarrow
  n - m_O + k \leq \sum_{i=1}^k n_i \leq n + m_O
\end{equation}
\end{corollary}

The \acrshort*{CCSD} Lagrangian is then:
\begin{equation}\label{eq:ccsd-lagrangian}
 \begin{aligned}
 \mathcal{L}(\tamp{}, \tbar{})_\mathrm{CCSD}
&=
 E_0
 + \sum_{u=1}^{2}\tbar{\mu_u}\denom{\mu_u}\tamp{\mu_u}
 + \braket{\mathrm{HF} | \check{\Phi} + \commutator{\check{\Phi}}{T_2} | \mathrm{HF}} \\
 &+ \braket{\tbar{1} | \check{\Phi} + \commutator{\check{\Phi}}{T_2} | \mathrm{HF}} \\
 &+ \braket{\tbar{2} | \check{\Phi} + \commutator{\check{\Phi}}{T_2}
   +\frac{1}{2}\commutator{\commutator{\check{\Phi}}{T_2}}{T_2} | \mathrm{HF}}
  \end{aligned}
\end{equation}
with the governing equations:
\begin{subequations}\label{eq:ccsd-equations}
  \begin{align}
   \denom{\mu_1}\tamp{\mu_1} &+ \braket{\mu_1 | \check{\Phi} + \commutator{\check{\Phi}}{T_2} | \mathrm{HF}}
             = 0 \label{eq:ccsd-singles-amplitudes}\\
   \denom{\mu_2}\tamp{\mu_2} &+
   \braket{\mu_2 | \check{\Phi} + \commutator{\check{\Phi}}{T_2} +\frac{1}{2}\commutator{\commutator{\check{\Phi}}{T_2}}{T_2} | \mathrm{HF}}
             = 0 \label{eq:ccsd-doubles-amplitudes}\\
    \denom{\mu_1}\tbar{\mu_1} &+
    \braket{\mathrm{HF} | \commutator{\check{\Phi}}{\cluster{\mu_1}} | \mathrm{HF}} +
    \braket{\tbar{1} |
      \commutator{\check{\Phi}}{\cluster{\mu_1}}
    + \commutator{\commutator{\check{\Phi}}{\cluster{\mu_1}}}{T_2}
    | \mathrm{HF}} \nonumber \\
   &+ \braket{\tbar{2} |
      \commutator{\check{\Phi}}{\cluster{\mu_1}}
    + \commutator{\commutator{\check{\Phi}}{\cluster{\mu_1}}}{T_2}
    | \mathrm{HF}}
             = 0 \label{eq:ccsd-singles-multipliers} \\
    \denom{\mu_2}\tbar{\mu_2} &+
    \braket{\mathrm{HF} | \commutator{\check{\Phi}}{\cluster{\mu_2}} | \mathrm{HF}} +
    \braket{\tbar{1} |
      \commutator{\check{\Phi}}{\cluster{\mu_2}}
    | \mathrm{HF}} \nonumber \\
   &+ \braket{\tbar{2} |
      \commutator{\check{\Phi}}{\cluster{\mu_2}}
    + \commutator{\commutator{\check{\Phi}}{\cluster{\mu_2}}}{T_2}
    | \mathrm{HF}}
             = 0 \label{eq:ccsd-doubles-multipliers}
  \end{align}
\end{subequations}
Equations \eqref{eq:ccsd-doubles-amplitudes},
\eqref{eq:ccsd-singles-multipliers} and \eqref{eq:ccsd-doubles-multipliers} can
be compared with the corresponding \acrshort{CC2} equations
\eqref{eq:cc2-doubles-amplitudes}, \eqref{eq:cc2-singles-multipliers} and
\eqref{eq:cc2-doubles-multipliers}, respectively, to highlight which terms were neglected in
constructing the iterative method for the approximate inclusion of double excitations.

The \acrshort{CCSDT} Lagrangian is built in the same exact manner:
\begin{equation}\label{eq:ccsdt-lagrangian}
 \begin{aligned}
 \mathcal{L}(\tamp{}, \tbar{})_\mathrm{CCSDT}
&=
 E_0
 + \sum_{u=1}^{3}\tbar{\mu_u}\denom{\mu_u}\tamp{\mu_u}
 + \braket{\mathrm{HF} | \check{\Phi} + \commutator{\check{\Phi}}{T_2} | \mathrm{HF}} \\
 &+ \braket{\tbar{1} | \check{\Phi} + \commutator{\check{\Phi}}{T_2} + \commutator{\check{\Phi}}{T_3} | \mathrm{HF}} \\
 &+ \braket{\tbar{2} | \check{\Phi} + \commutator{\check{\Phi}}{T_2}
   +\frac{1}{2}\commutator{\commutator{\check{\Phi}}{T_2}}{T_2} + \commutator{\check{\Phi}}{T_3} | \mathrm{HF}} \\
 &+ \braket{\tbar{3} | \commutator{\check{\Phi}}{T_2}
   + \frac{1}{2}\commutator{\commutator{\check{\Phi}}{T_2}}{T_2} + \commutator{\check{\Phi}}{T_3}
   + \commutator{\commutator{\check{\Phi}}{T_2}}{T_3}
   | \mathrm{HF}}
  \end{aligned}
\end{equation}
Notice that the reference expectation value of the similarity transformed
fluctuation potential $\overline{\Phi}$ is unchanged with respect to the
expression in the \acrshort{CCSD} Lagrangian.
This is a consequence of the well-known fact that the \acrshort*{CC} energy can
be expressed purely in terms of $T_1$ and $T_2$:
\begin{equation}
E_\mathrm{CC} = \braket{\mathrm{HF} | \overline{H} | \mathrm{HF}}
=
\braket{\mathrm{HF} | H(1 + T_1 + T_2 + T_1^2) | \mathrm{HF}}
\end{equation}
The \acrshort{CCSDT} amplitude equations are:
\begin{subequations}\label{eq:ccsdt-amplitudes}
  \begin{align}
   \denom{\mu_1}\tamp{\mu_1} &+ \braket{\mu_1 |
   \check{\Phi} + \commutator{\check{\Phi}}{T_2} + \commutator{\check{\Phi}}{T_3}
   | \mathrm{HF}}
             = 0 \label{eq:ccsdt-singles-amplitudes}\\
   \denom{\mu_2}\tamp{\mu_2} &+ \braket{\mu_2 |
   \check{\Phi} + \commutator{\check{\Phi}}{T_2}
   + \frac{1}{2}\commutator{\commutator{\check{\Phi}}{T_2}}{T_2} + \commutator{\check{\Phi}}{T_3}
   | \mathrm{HF}}
             = 0 \label{eq:ccsdt-doubles-amplitudes} \\
   \denom{\mu_3}\tamp{\mu_3} &+ \braket{\mu_3 |
   \commutator{\check{\Phi}}{T_2}
   + \frac{1}{2}\commutator{\commutator{\check{\Phi}}{T_2}}{T_2} + \commutator{\check{\Phi}}{T_3}
   + \commutator{\commutator{\check{\Phi}}{T_2}}{T_3}
   | \mathrm{HF}}
             = 0 \label{eq:ccsdt-triples-amplitudes}
  \end{align}
\end{subequations}
while the multipliers are determined by solving the following:
\begin{subequations}\label{eq:ccsdt-multipliers}
  \begin{align}
    \denom{\mu_1}\tbar{\mu_1} &+
    \braket{\mathrm{HF} | \commutator{\check{\Phi}}{\cluster{\mu_1}} | \mathrm{HF}} +
    \braket{\tbar{1} |
      \commutator{\check{\Phi}}{\cluster{\mu_1}}
    + \commutator{\commutator{\check{\Phi}}{\cluster{\mu_1}}}{T_2}
    | \mathrm{HF}} \nonumber \\
   &+ \braket{\tbar{2} |
      \commutator{\check{\Phi}}{\cluster{\mu_1}}
    + \commutator{\commutator{\check{\Phi}}{\cluster{\mu_1}}}{T_2}
    + \commutator{\commutator{\check{\Phi}}{\cluster{\mu_1}}}{T_3}
    | \mathrm{HF}} \nonumber \\
   &+ \braket{\tbar{3} |
      \commutator{\commutator{\check{\Phi}}{\cluster{\mu_1}}}{T_2}
    + \frac{1}{2}\commutator{\commutator{\commutator{\check{\Phi}}{\cluster{\mu_1}}}{T_2}}{T_2}
    + \commutator{\commutator{\check{\Phi}}{\cluster{\mu_1}}}{T_3}
    | \mathrm{HF}}
             = 0 \label{eq:ccsdt-singles-multipliers} \\
    \denom{\mu_2}\tbar{\mu_2} &+
    \braket{\mathrm{HF} | \commutator{\check{\Phi}}{\cluster{\mu_2}} | \mathrm{HF}} +
    \braket{\tbar{1} |
      \commutator{\check{\Phi}}{\cluster{\mu_2}}
    | \mathrm{HF}} \nonumber \\
   &+ \braket{\tbar{2} |
      \commutator{\check{\Phi}}{\cluster{\mu_2}}
    + \commutator{\commutator{\check{\Phi}}{\cluster{\mu_2}}}{T_2}
    | \mathrm{HF}} \nonumber \\
   &+ \braket{\tbar{3} |
      \commutator{\check{\Phi}}{\cluster{\mu_2}}
    +  \commutator{\commutator{\check{\Phi}}{\cluster{\mu_2}}}{T_2}
    +  \commutator{\commutator{\check{\Phi}}{\cluster{\mu_2}}}{T_3}
    | \mathrm{HF}}
             = 0 \label{eq:ccsdt-doubles-multipliers} \\
    \denom{\mu_3}\tbar{\mu_3} &+
    \braket{\tbar{1} |
      \commutator{\check{\Phi}}{\cluster{\mu_3}}
    | \mathrm{HF}}
    + \braket{\tbar{2} |
      \commutator{\check{\Phi}}{\cluster{\mu_3}}
    | \mathrm{HF}} \nonumber \\
   &+ \braket{\tbar{3} |
      \commutator{\check{\Phi}}{\cluster{\mu_3}}
    +  \commutator{\commutator{\check{\Phi}}{\cluster{\mu_3}}}{T_2}
    | \mathrm{HF}}
             = 0 \label{eq:ccsdt-triples-multipliers}
  \end{align}
\end{subequations}
A comparison of equations \eqref{eq:ccsdt-triples-amplitudes},
\eqref{eq:ccsdt-singles-multipliers}, \eqref{eq:ccsdt-doubles-multipliers} and
\eqref{eq:ccsdt-triples-multipliers}
with equations \eqref{eq:cc3-triples-amplitudes},
\eqref{eq:cc3-singles-multipliers}, \eqref{eq:cc3-doubles-multipliers} and
\eqref{eq:cc3-triples-multipliers}, respectively, serves to highlight which terms were
neglected in constructing the \acrshort{CC3} method.

\section{Coupled Cluster Expectation Values}\label{sec:CC-expval}

A similar analysis in terms of $T_1$-transformed operators can be given for the
expectation value of one-electron operators, Eq.
\eqref{eq:cc-expectation-value}.
First of all, by virtue of the cluster commutation
condition,\autocite{Helgaker2000-tz} the commutator expansion of the similarity
transformation of a general one-electron operator truncates after the twofold
nested commutator:
\begin{equation}
\overline{O} = O
  + \commutator{O}{T}
  + \frac{1}{2}\commutator{\commutator{O}{T}}{T}
\end{equation}
where $T$ is the complete cluster operator. However, due to Lemma
\ref{lemma:exc-rank}, some of the cluster operators will not contribute to the
expectation value.

For a general truncation level $\mathcal{M}$ one has:
\begin{equation}
  \begin{aligned}
  O(\tamp{}, \tbar{}) &=
  \braket{\mathrm{HF} | O + \BCHfirst{O}{T} + \BCHsecond{O}{T} | \mathrm{HF}} \\
  &+ \sum_{u=1}^{\mathcal{M}}\braket{\tbar{u} | O + \BCHfirst{O}{T} +
  \BCHsecond{O}{T} | \mathrm{HF}},
  \end{aligned}
\end{equation}
and by virtue of Lemma \ref{lemma:exc-rank} and its Corollary:
\begin{equation}
\begin{aligned}
  O(\tamp{}, \tbar{})
  &=
  \braket{\mathrm{HF} | O + \BCHfirst{O}{T_1} | \mathrm{HF}}
  + \braket{\tbar{1} | O | \mathrm{HF}} \\
  &+ \sum_{u=1}^{\mathcal{M}}\braket{\tbar{u} | \BCHfirst{O}{T} + \BCHsecond{O}{T} | \mathrm{HF}}
\end{aligned}
\end{equation}
Note that the singles amplitudes assume a unique role in the \acrshort{CC}
expectation value.
In the following, we give explicit expressions for the \acrshort{CCS}, \acrshort{CCSD}
and \acrshort{CCSDT} models. These results are at the basis of our developments
in Chapter \ref{ch:solvation-correlation}.
We will refer to the $T_1$-transformed expressions as \emph{dressed}, in contrast to the
\emph{bare} expressions, where the operator $O$ appears untransformed.

For the \acrshort{CCS} model, the expectation value of a one-electron operator
is simply:
\begin{equation}\label{eq:ccs-expectation-value}
\begin{aligned}
  O(\tamp{}, \tbar{})_\mathrm{CCS} &=
  \braket{\mathrm{HF} | \check{O} | \mathrm{HF} }
  + \braket{\tbar{1} | \check{O} | \mathrm{HF} } \\
  &=
  \braket{\mathrm{HF} | O + \commutator{O}{T_1} | \mathrm{HF} }
  + \braket{\tbar{1} | O + \commutator{O}{T_1}
  + \frac{1}{2}\commutator{\commutator{O}{T_1}}{T_1} | \mathrm{HF} }
\end{aligned}
\end{equation}
Adding double excitations to the manifold yields:
\begin{equation}\label{eq:ccsd-expectation-value}
  \begin{aligned}
  O(\tamp{}, \tbar{})_\mathrm{CCSD}
  &=
    O(\tamp{}, \tbar{})_\mathrm{CCS}
  + \braket{\tbar{1} | \commutator{\check{O}}{T_2} | \mathrm{HF} }
  + \braket{\tbar{2} | \commutator{\check{O}}{T_2} | \mathrm{HF} } \\
  &=
    O(\tamp{}, \tbar{})_\mathrm{CCS}
  + \braket{\tbar{1} | \commutator{O}{T_2} | \mathrm{HF} }
  + \braket{\tbar{2} | \commutator{O}{T_2}
      + \commutator{\commutator{O}{T_1}}{T_2}
  | \mathrm{HF} },
  \end{aligned}
\end{equation}
where a number of terms was dropped thanks to Lemma \ref{lemma:exc-rank} and its
Corollary.
Eventually, including triples one obtains:
\begin{equation}\label{eq:ccsdt-expectation-value}
\begin{aligned}
  O(\tamp{}, \tbar{})_\mathrm{CCSDT}
  &=
    O(\tamp{}, \tbar{})_\mathrm{CCSD}
  + \braket{\tbar{2} | \commutator{\check{O}}{T_3} | \mathrm{HF} } \\
  &+ \braket{\tbar{3} | \commutator{\check{O}}{T_3}
    + \frac{1}{2}\commutator{\commutator{\check{O}}{T_2}}{T_2}
  | \mathrm{HF} } \\
  &=
    O(\tamp{}, \tbar{})_\mathrm{CCSD}
  + \braket{\tbar{2} | \commutator{O}{T_3} | \mathrm{HF} } \\
  &+ \braket{\tbar{3} | \commutator{O}{T_3}
    + \commutator{\commutator{O}{T_1}}{T_3}
    + \frac{1}{2}\commutator{\commutator{O}{T_2}}{T_2}
  | \mathrm{HF} }
\end{aligned}
\end{equation}
where Lemma \ref{lemma:exc-rank} and its corollary were again extensively employed.

The use of normal-ordered operators is common in \acrlong*{CC} theory. Both for
the dressed and bare representations, the expectation values of such operators
can be formed by simply replacing the operator with its normal-ordered counterpart,
$\check{O}_\mathrm{N}$ or $O$, respectively.
This achieves elimination of the reference expectation value from the first
term in equation \eqref{eq:ccs-expectation-value}.

\section{Selected Results in Functional Analysis}

\begin{defin}[Sobolev spaces]
Let $s$ and $p$ be two integers with $s\geq 0$ and $1\leq p \leq +\infty$.
The so-called \emph{Sobolev space} $W^{s, p}(\Omega)$ is defined as
\begin{equation}
W^{s,p}(\Omega) =
\left\lbrace
u \in \mathcal{D}^\prime(\Omega) | \partial^\alpha u \in L^p(\Omega), |\alpha|\leq s
\right\rbrace
\end{equation}
where $\mathcal{D}^\prime(\Omega)$ is the space of Schwartz distributions and the
derivatives $\partial^\alpha u$ are understood in a distributional sense.
\end{defin}

\begin{lemma}[Hilbert Sobolev spaces]
Let $s\geq 0$. The space $H^s(\Omega) = W^{s, 2}(\Omega)$ is a Hilbert space
when equipped with the scalar product
\begin{equation}
(u, v)_{s, \Omega} = \sum_{|\alpha|\leq s}\int_\Omega\partial^\alpha u \partial^\alpha v.
\end{equation}
The associated norm is denoted by $\| \cdot \|_{s, \Omega}$.
\end{lemma}

\begin{defin}[Fractional Sobolev spaces]
For $0 < s < 1$ and $1 \leq p < +\infty$, the so-called \emph{Sobolev space with fractional
exponent} is defined as
\begin{equation}
W^{s, p}(\Omega) =
\left\lbrace
u \in L^p(\Omega) |
\frac{u(\vect{r}) - u(\vect{r}^\prime)}{\| \vect{r} - \vect{r}^\prime\|^{s+\frac{d}{p}}}
\in L^p(\Omega \times \Omega)
\right\rbrace.
\end{equation}
Furthermore, when $s> 1$ is not integer, letting $\sigma = s - [s]$, where $[s]$ is the integer
part of $s$, $W^{s,p}(\Omega)$ is defined as
\begin{equation}
W^{s, p}(\Omega) =
\left\lbrace
u \in W^{[s], p}(\Omega) |
\partial^\alpha u \in W^{\sigma, p}(\Omega)\, \forall \alpha,\, |\alpha| = [s]
\right\rbrace.
\end{equation}
When $p=2$, we denote $H^s(\Omega) = W^{s,2}(\Omega)$.
\end{defin}

\begin{defin}[Continuity]\label{def:continuity}
  A bilinear form on a normed vector space $V$ is \emph{bounded}, or
  \emph{continuous}, if there is a constant $C$ such that
  $\forall u, v \in V$:
  \[
  a(u, v) \leq C\lVert u\rVert\lVert v\rVert
  \]
\end{defin}

\begin{defin}[Coercivity]\label{def:coercivity}
  A bilinear form on a normed vector space $V$ is \emph{coercive},
  or \emph{elliptic}, if there is a constant $\alpha >0$ such that
  $\forall u \in V$:
  \[
  a(u, u) \geq \alpha\lVert u\rVert^2
  \]
\end{defin}
Coercivity implies that no eigenvalue of the linear operator associated to the
bilinear form can be zero, hence its invertibility.\autocite{Ern2004-oo}

\begin{defin}[Abstract weak problem]\label{def:abstract-weak}
An abstract weak problem is posed as follows:
\begin{equation}
  \left\{
  \begin{aligned}
    &\text{Seek $u \in W$ such that:}\\
    &\forall v \in V \,\,\,\,\,
    a(u, v) = b(v)
  \end{aligned}
  \right.
  \label{eq:weak}
\end{equation}
where:
\begin{itemize}
  \item $W$ and $V$ are normed vector spaces. $W$ is the \emph{solution space},
  $V$ the \emph{test space}.
  \item $a$ is a \emph{continuous} bilinear form on $W \times V$.
  \item $b$ is a \emph{continous} linear form on $V$.
\end{itemize}
\end{defin}

\begin{defin}[Well-posedness]\label{def:hadamard}
The abstract weak problem in Definition \ref{def:abstract-weak} is
\emph{well-posed} if it admits one and only one solution and the solution is
bounded by the \emph{a priori} estimate:
\begin{equation}
\exists c > 0, \forall f \in V^\prime, \|u\|_W \leq c \|f\|_{V^\prime}
\end{equation}
where $V^\prime$ is the dual space of $V$.
\end{defin}

\begin{defin}[Transmission problem]
We assume Euclidean space $\mathbb{R}^3$ to be partitioned into two subdomains
$\Omegai$ and $\Omegae$ sharing a boundary $\Gamma$. We further assume that
$\Omegai$ is a closed domain, entirely contained inside $\Omegae$.
The transmission problem is posed as follows:
\begin{subequations}\label{eq:transmission-general}
\begin{align}[left={\empheqlbrace}]
  \Li u &= f_\mathrm{i} \,\, \text{in}\,\, \Omegai \\
  \Le u &= f_\mathrm{e} \,\, \text{in}\,\, \Omegae \\
  [u] &= \ue - \ui = g_\mathrm{D} \,\, \text{on}\,\, \Gamma
   \\
[\partial_L u] &= \partiale u - \partiali u = g_\mathrm{N} \,\,
\text{on}\,\, \Gamma \\
|u(\vect{r})| &\leq C \|\vect{r} \|^{-1} \,\,\text{for}\,\,\| \vect{r} \|\rightarrow\infty
\end{align}
\end{subequations}
where the differential operators are assumed elliptic and the jump conditions are
given in terms of Dirichlet $g_\mathrm{D}$ and Neumann $g_\mathrm{N}$ data.
\end{defin}

\section{Derivation of the IEF equation}
I will show a detailed derivation of the \acrshort{IEF} equation
for the \acrshort{PCM} transmission problem Eq. \eqref{eq:transmission}:
\begin{subequations}
  \begin{align}[left={\empheqlbrace}]
  \Li u &= \nabla^2 u = -4\pi\rhoi \,\, \text{in}\,\, \Omegai \\
  \Le u &= 0 \,\, \text{in}\,\, \Omegae \\
  [u] &= \ue - \ui = 0 \,\, \text{on}\,\, \Gamma \\
[\partial_L u] &= \partiale u - \partiali u = 0 \,\, \text{on}\,\, \Gamma \\
|u(\vect{r})| &\leq C \|\vect{r} \|^{-1} \,\,\text{for}\,\,\| \vect{r} \|\rightarrow\infty
\end{align}
\end{subequations}

We first state two important results in the theory of boundary integral
equations.
\begin{lemma}[Properties of the boundary integral operators]
  The integral operators introduced in Eqs. \eqref{eq:S}--\eqref{eq:D-dagger}
  enjoy the following properties\autocite{Hsiao2008-xb, Sauter2011-an}:
  \begin{enumerate}
      \item on $L^2(\Gamma)$, $\bi{S}_\star$ is self-adjoint,
        $\bi{D}^\dagger_\star$ is the adjoint operator of
        $\bi{D}_\star$.
      \item The commutation relations hold:
        \begin{alignat}{2}
          \bi{D}_\star\bi{S}_\star = \bi{S}_\star\bi{D}^\dagger_\star, \quad&
          \bi{S}_\star\bi{D}_\star = \bi{D}^\dagger_\star\bi{S}_\star
        \end{alignat}
      \item The boundary integral operators are continuous
        mappings between Sobolev spaces of fractional order:
        \begin{subequations}
          \begin{align}
   \bi{S}_\star &: H^{-\frac{1}{2}}(\Gamma) \rightarrow H^{\frac{1}{2}}(\Gamma) \\
   \bi{D}_\star &: H^{\frac{1}{2}}(\Gamma) \rightarrow H^{\frac{1}{2}}(\Gamma) \\
   \bi{D}^\dagger_\star &: H^{-\frac{1}{2}}(\Gamma) \rightarrow H^{-\frac{1}{2}}(\Gamma)
          \end{align}
        \end{subequations}
      \item The operator $\bi{S}_\star$ is coervice and admits a
        continuous inverse in the aforementioned Sobolev spaces.
      \item The operators $\lambda - \bi{D}_\star$ and $\lambda -
        \bi{D}^\dagger_\star$ with $\lambda \in (-2\pi, +\infty)$
        admit a continuous inverse in the aforementioned
        Sobolev spaces.
  \end{enumerate}
\end{lemma}

\begin{lemma}[Integral Representation]
For the transmission problem \ref{eq:transmission-general} there holds:
\begin{enumerate}
\item $\forall \vect{r} \in \Omegai$
\begin{equation}
u = \bi{S}_\mathrm{i}(\partiali u)
- \bi{D}_\mathrm{i}(\ui) + \int_{\Omegai} \Gi f_\mathrm{i}\diff \vect{r}^\prime
\end{equation}
\item $\forall \vect{r} \in \Omegae$
\begin{equation}
u = -\bi{S}_\mathrm{e}(\partiale u)
+ \bi{D}_\mathrm{e}(\ue) + \int_{\Omegae}\Ge f_\mathrm{e}\diff \vect{r}^\prime
\end{equation}
\item $\forall \vect{r} \in \Gamma$
  \begin{equation}\label{eq:stat3}
    \frac{1}{2}\ui = \bi{S}_\mathrm{i}(\partiali u)
    - \bi{D}_\mathrm{i}(\ui) + \int_{\Omegai}\Gi f_\mathrm{i}\diff \vect{r}^\prime
  \end{equation}
\item $\forall \vect{r} \in \Gamma$
  \begin{equation}\label{eq:stat4}
    \frac{1}{2}\ue = -\bi{S}_\mathrm{e}(\partiale u)
    + \bi{D}_\mathrm{e}(\ue) + \int_{\Omegae}\Ge f_\mathrm{e}\diff \vect{r}^\prime
  \end{equation}
\end{enumerate}
\end{lemma}

We introduce the following auxiliary potential:
\begin{subequations}
\begin{empheq}[left={h(\vect{r}) = \empheqlbrace}]{align}
    &\int_{\mathbb{R}^3}\Gi\rhoi(\vect{r}^\prime)\diff \vect{r}^\prime  \quad \vect{r} \in \Omegai \\
    &\int_{\mathbb{R}^3}\Ge\rhoi(\vect{r}^\prime)\diff \vect{r}^\prime  \quad \vect{r} \in \Omegae
\end{empheq}
\end{subequations}
for which we have:
\begin{subequations}
\begin{empheq}[left={\empheqlbrace}]{align}
    \nabla^2 h &= \rhoi \,\, \text{in}\,\, \Omegai \\
    \Le h &= 0 \,\, \text{in}\,\, \Omegae
\end{empheq}
\end{subequations}
We then define the \emph{reaction potential} as:
\begin{equation}
  \xi = u - h
\end{equation}
such that:
\begin{subequations}
  \begin{align}[left={\empheqlbrace}]
  \nabla^2 \xi &= 0 \,\, \text{in}\,\, \Omegai \\
  \Le \xi &= 0 \,\, \text{in}\,\, \Omegae \\
  -[\xi] &= [h] \,\, \text{on}\,\, \Gamma \\
  -[\partial_L \xi] &= [\partial_L h] \,\, \text{on}\,\, \Gamma
\end{align}
\end{subequations}
inside the cavity, the reaction potential can be represented by a
\emph{single layer potential}:
\begin{equation}
  \xi_\mathrm{i} = \bi{S}_\mathrm{i}\sigma
\end{equation}
where the function $\sigma$ is the, yet unknown, \acrlong*{ASC}.

To derive an equation for $\sigma$, we set up a system of equations containing
the jump conditions and the integral representations of the reaction and auxiliary potentials:
\begin{subequations}
\begin{empheq}[left={\empheqlbrace}]{align}
    &\bi{S}_\mathrm{i}(\partiali \xi) - \bi{D}_\mathrm{i}(\xi_\mathrm{i})
    = \frac{1}{2}\xi_\mathrm{i} \label{eq:first} \\
    &\bi{S}_\mathrm{e}(\partiale \xi) - \bi{D}_\mathrm{e}(\xi_\mathrm{e})
    = -\frac{1}{2}\xi_\mathrm{e} \label{eq:second} \\
  &\bi{S}_\mathrm{e}(\partiale h) - \bi{D}_\mathrm{e}(h_\mathrm{e})
  = -\frac{1}{2}h_\mathrm{e} \label{eq:third} \\
  &\xi_\mathrm{i} - \xi_\mathrm{e} = h_\mathrm{e} - h_\mathrm{i}
  \label{eq:fourth} \\
  &\partiali \xi - \partiale \xi = \partiale h - \partiali h
  \label{eq:fifth}
\end{empheq}
\end{subequations}
The final ingredient is the \gls{DtN} map, which can be derived by employing Eq. \eqref{eq:stat3}
to the Newton potential:
\begin{equation}
  \phi(\vect{r}) = (\bi{N}\rhoi)(\vect{r}) = \int_{\mathbb{R}^3}\Gi\rhoi(\vect{r}^\prime)\diff \vect{r}^\prime
   = \left.h(\vect{r})\right|_{\Omegai}
\end{equation}
which is equal, in $\Omegai$, to the auxiliary potential $h(\vect{r})$.
The \acrshort{DtN} map is thus given as:
\begin{equation}\label{eq:regular-DtN}
   \left(\frac{1}{2} - \bi{D}_\mathrm{i}\right)\phi_\mathrm{i}
      +\bi{S}_\mathrm{i}(\partiali \phi) = 0
\end{equation}
With this last ingredient at hand, algebraic manipulations lead to the \acrshort{IEF}
equation:
\begin{equation}
  \left[ \bi{S}_\mathrm{e}\left(2\pi + \bi{D}^\dagger_\mathrm{i}\right)
  +
  \left(2\pi - \bi{D}_\mathrm{e}\right)\bi{S}_\mathrm{i}
  \right]\sigma =
  -\left[\left(2\pi-\bi{D}_\mathrm{e}\right)
  -\bi{S}_\mathrm{e}\bi{S}_\mathrm{i}^{-1}
  \left(2\pi-\bi{D}_\mathrm{i}\right)
  \right]\esp,
  \tag{\ref{eq:full-IEF} from Chapter \ref{ch:CSM}}
\end{equation}

\section{Weak Formulation of Partial Differential Equations}\label{sec:weak}

The transmission problem can be reformulated in a variational fashion.
Such a formulation allows for a larger vector space, with weaker regularity
conditions, to be explored as solution space for the problem.
We will follow the exposition of \citeauthor{Ern2004-oo} quite closely
in introducing the \emph{weak} formulation of \glspl{PDE}.
For simplicity, we assume Dirichlet boundary conditions for a conductor,
\ie the basic assumption behind \acrshort{COSMO}.
The \emph{strong} formulation of the electrostatic problem then reads:
\begin{equation}\label{eq:poisson-cosmo}
  \nabla^2\esp = -4\pi\rho,
  \quad \esp \in \mathcal{C}^2_0(\Omegai),
\end{equation}
where $\mathcal{C}^2_0(\Omegai)$ is the vector space of twice
continuously differentiable functions in $\Omegai$ with null trace on
$\Gamma$.
We can relax this regularity requirement on $\Psi$ by introducing the
\emph{Hilbert Sobolev space} of test functions $H_0^1(\Omegai)$:
\begin{equation}
  H_0^1(\Omegai) = \lbrace
  f : \Omegai \rightarrow \mathbb{R}
  | f, \nabla f \in L^2(\Omegai), f_\Gamma = 0
  \rbrace.
\end{equation}
Projecting the differential problem onto this space and using the
$\eta\nabla^2\gamma = \nabla\cdot(\eta\nabla\gamma) - \nabla\eta\cdot\nabla\gamma$ identity
one obtains the \emph{weak} formulation of the differential problem:
\begin{equation}\label{eq:poisson-weak}
  \left\{
  \begin{aligned}
    &\text{Seek $\Psi \in H_0^1(\Omegai)$ such that:}\\
    &(\nabla\zeta, \nabla\Psi) =
    -4\pi(\zeta, \rho) \quad
    \forall \zeta\in H_0^1(\Omegai)
  \end{aligned}
  \right.
\end{equation}
The form $(\nabla\cdot, \nabla\cdot)$ is \emph{bilinear} and continuous
in $H_0^1(\Omegai)\times H_0^1(\Omegai)$, while $-4\pi(\cdot, \rho)$ is
\emph{linear} and continuous in $H_0^1(\Omegai)$.
We rewrite Eq. \eqref{eq:poisson-weak} in the abstract form:
\begin{equation}
  \left\{
  \begin{aligned}
    &\text{Seek $u \in V$ such that:}\\
    &\forall v \in V \,\,\,\,\,
    a(u, v) = b(v)
  \end{aligned}
  \right.
  \tag{\ref{eq:weak}}
\end{equation}
and state the following fundamental results:
\begin{lemma}[Lax--Milgram]
  If the bilinear form $a$ is \emph{continous} and \emph{coercive}
  in $V$, then, for any continuous linear form $b$, Problem~\eqref{eq:weak} is
  \emph{well-posed}.
\end{lemma}
\begin{corollary}[Variational property]
  If the bilinear form is \emph{symmetric} and \emph{positive}
  the unique solution to Problem \eqref{eq:weak} is
  the \emph{unique minimum} on $V$ of the functional:
  \[
  \mathcal{F}(u) = \frac{1}{2}a(u, u) - b(u)
  \]
\end{corollary}

It is important to note how \glspl{BIE} can also be reformulated in a
variational framework, as shown in \noparcite[ref.][]{Hsiao2008-xb}

\begin{comment}

\section{The Galerkin Method}

\todo[inline]{Write this section}
Approximability, conformity, consistency and asymptotic consistency
of the approximation setting, $\inf$-$\sup$ condition, Céa's Lemma
and generalized $\inf$-$\sup$ condition.\autocite{Ern2004-oo}

\section{Mixed Blabbering}

\begin{equation}
  \begin{pmatrix}
   \sigma^\mathrm{tot} \\
   \kappa^\mathrm{tot}
  \end{pmatrix}
  =
  \begin{pmatrix}
   \sigma_\mathrm{HF} \\
   \kappa_\mathrm{HF}
  \end{pmatrix}
  +
  \begin{pmatrix}
   \sigma \\
   \kappa
  \end{pmatrix}
\end{equation}
and similarly for the sources:
\begin{equation}
\begin{aligned}
  \begin{pmatrix}
   \esp(\tamp{}, \tbar{}) \\
   \zeta(\tamp{}, \tbar{})
  \end{pmatrix}
  &=
  \begin{pmatrix}
   \esp_\mathrm{HF} \\
   \zeta_\mathrm{HF}
  \end{pmatrix}
  +
  \begin{pmatrix}
    \esp_{\mathrm{N}}(\tamp{}, \tbar{}) \\
    \zeta_{\mathrm{N}}(\tamp{}, \tbar{})
  \end{pmatrix} \\
  &=
  \begin{pmatrix}
  \braket{\mathrm{HF} | \esp  | \mathrm{HF}} \\
  \braket{\mathrm{HF} | \zeta | \mathrm{HF}}
  \end{pmatrix}
  +
  \begin{pmatrix}
   \braket{\mathrm{HF} | \tilde{\esp} | \mathrm{HF}}
  + \sum_{i=1}^{\mathcal{M}}\braket{\tbar{i} | \overline{\esp} | \mathrm{HF}} \\
   \braket{\mathrm{HF} | \tilde{\zeta} | \mathrm{HF}}
  + \sum_{i=1}^{\mathcal{M}}\braket{\tbar{i} | \overline{\zeta} | \mathrm{HF}}
  \end{pmatrix}
\end{aligned}
\end{equation}
The \acrshort{BCH} expansions of the source terms have been
rewritten as:
\begin{subequations}
  \begin{align}
  \overline{\esp} &= \esp + \tilde{\esp} = \esp + \commutator{\esp}{T}
  + \frac{1}{2}\commutator{\commutator{\esp}{T}}{T} \\
  \overline{\kappa} &= \kappa + \tilde{\kappa} = \kappa + \commutator{\kappa}{T}
  + \frac{1}{2}\commutator{\commutator{\kappa}{T}}{T}
  \end{align}
\end{subequations}
taking advantage of the fact that both $\esp$ and $\kappa$ are
nondiagonal, one-electron operators and hence their commutator
expansions truncate at the third term. Moreover, normal-ordering of the
operators has been introduced.\autocite{Crawford2000-ey, Shavitt2009-mr}
This is equivalent to applying a shift to the correlation part of the
source term one-electron operators to remove the reference source terms:
\begin{equation}
  \begin{aligned}
  \begin{pmatrix}
    \esp_{\mathrm{N}}(\tamp{}, \tbar{}) \\
    \zeta_{\mathrm{N}}(\tamp{}, \tbar{})
  \end{pmatrix}
  &=
  \begin{pmatrix}
   \braket{\mathrm{HF} | \tilde{\esp} | \mathrm{HF}}
  + \sum_{i=1}^{\mathcal{M}}\braket{\tbar{i} | \overline{\esp} | \mathrm{HF}} \\
   \braket{\mathrm{HF} | \tilde{\zeta} | \mathrm{HF}}
  + \sum_{i=1}^{\mathcal{M}}\braket{\tbar{i} | \overline{\zeta} | \mathrm{HF}}
  \end{pmatrix}
  \\
  &=
  \begin{pmatrix}
    \braket{\mathrm{HF} | \overline{\esp} - \esp_\mathrm{HF} | \mathrm{HF}}
  + \sum_{i=1}^{\mathcal{M}}\braket{\tbar{i} | \overline{\esp} - \esp_\mathrm{HF} | \mathrm{HF}} \\
  \braket{\mathrm{HF} | \overline{\zeta} - \zeta_\mathrm{HF} | \mathrm{HF}}
  + \sum_{i=1}^{\mathcal{M}}\braket{\tbar{i} | \overline{\zeta} -
  \zeta_\mathrm{HF}| \mathrm{HF}}
  \end{pmatrix}
\end{aligned}
\end{equation}

Eventually, the polarization functional can be rewritten as:
\begin{equation}
  \begin{aligned}
  U_\mathrm{pol} &=
  \frac{1}{2}(\sigma_\mathrm{HF}+\sigma)\PCM(\sigma_\mathrm{HF}+\sigma)
+ (\sigma_\mathrm{HF}+\sigma)(\esp_\mathrm{HF}+\esp_{\mathrm{N}}(\tamp{},\tbar{})) \\
&+ \frac{1}{2}(\kappa_\mathrm{HF}+\kappa)\MM(\kappa_\mathrm{HF}+\kappa)
+ (\kappa_\mathrm{HF}+\kappa)(\zeta_\mathrm{HF}+\zeta_{\mathrm{N}}(\tamp{},\tbar{}))
+ (\sigma_\mathrm{HF}+\sigma)\bi{X}(\kappa_\mathrm{HF}+\kappa) \\
  %%%% Steps
&= \left/ \text{Expand and collect} \right/ \\
&=
    \frac{1}{2}\sigma_\mathrm{HF}\PCM\sigma_\mathrm{HF}
  + \sigma_\mathrm{HF}\esp_\mathrm{HF}
  + \frac{1}{2}\kappa_\mathrm{HF}\MM\kappa_\mathrm{HF}
  + \kappa_\mathrm{HF}\zeta_\mathrm{HF}
  + \sigma_\mathrm{HF}\bi{X}\kappa_\mathrm{HF} \\
&+  \frac{1}{2}\sigma\PCM\sigma
  + \sigma\esp_{\mathrm{N}}(\tamp{},\tbar{})
  + \frac{1}{2}\kappa\MM\kappa
  + \kappa\zeta_{\mathrm{N}}(\tamp{},\tbar{})
  + \sigma\bi{X}\kappa \\
&+  \frac{1}{2}\sigma_\mathrm{HF}\PCM\sigma
  + \sigma_\mathrm{HF}\esp_{\mathrm{N}}(\tamp{},\tbar{})
  + \frac{1}{2}\kappa_\mathrm{HF}\MM\kappa
  + \kappa_\mathrm{HF}\zeta_{\mathrm{N}}(\tamp{},\tbar{})
  + \sigma_\mathrm{HF}\bi{X}\kappa \\
&+  \frac{1}{2}\sigma\PCM\sigma_\mathrm{HF}
  + \sigma\esp_\mathrm{HF}
  + \frac{1}{2}\kappa\MM\kappa_\mathrm{HF}
  + \kappa\zeta_\mathrm{HF}
  + \sigma\bi{X}\kappa_\mathrm{HF} \\
  %%%% Steps
&= \left/
  \frac{1}{2}\sigma_\mathrm{HF}\PCM\sigma_\mathrm{HF}
  +
  \frac{1}{2}\kappa_\mathrm{HF}\MM\kappa_\mathrm{HF}
  +
  \sigma_\mathrm{HF}\esp_\mathrm{HF}
  +
  \kappa_\mathrm{HF}\zeta_\mathrm{HF}
  +
  \sigma_\mathrm{HF}\bi{X}\kappa_\mathrm{HF}
  =
  \frac{1}{2}\sigma_\mathrm{HF}\esp_\mathrm{HF}
  +
  \frac{1}{2}\kappa_\mathrm{HF}\zeta_\mathrm{HF}
  \right/ \\
&=
    \frac{1}{2}\sigma_\mathrm{HF}\esp_\mathrm{HF}
  + \frac{1}{2}\kappa_\mathrm{HF}\zeta_\mathrm{HF} \\
&+  \frac{1}{2}\sigma\PCM\sigma
  + \sigma\esp_{\mathrm{N}}(\tamp{},\tbar{})
  + \frac{1}{2}\kappa\MM\kappa
  + \kappa\zeta_{\mathrm{N}}(\tamp{},\tbar{})
  + \sigma\bi{X}\kappa \\
&+  \frac{1}{2}\sigma_\mathrm{HF}\PCM\sigma
  + \sigma_\mathrm{HF}\esp_{\mathrm{N}}(\tamp{},\tbar{})
  + \frac{1}{2}\kappa_\mathrm{HF}\MM\kappa
  + \kappa_\mathrm{HF}\zeta_{\mathrm{N}}(\tamp{},\tbar{})
  + \sigma_\mathrm{HF}\bi{X}\kappa \\
&+  \frac{1}{2}\sigma\PCM\sigma_\mathrm{HF}
  + \sigma\esp_\mathrm{HF}
  + \frac{1}{2}\kappa\MM\kappa_\mathrm{HF}
  + \kappa\zeta_\mathrm{HF}
  + \sigma\bi{X}\kappa_\mathrm{HF} \\
  %%%% Steps
&= \left/
  \frac{1}{2}\sigma\PCM\sigma_\mathrm{HF}
  =
  \frac{1}{2}\sigma_\mathrm{HF}\PCM\sigma;\quad
  \frac{1}{2}\kappa\MM\kappa_\mathrm{HF}
  =
  \frac{1}{2}\kappa_\mathrm{HF}\MM\kappa; \quad
  \sigma_\mathrm{HF}\bi{X}\kappa
  =
  \kappa\bi{X}^\dagger\sigma_\mathrm{HF}
  \right/ \\
&=
    \frac{1}{2}\sigma_\mathrm{HF}\esp_\mathrm{HF}
  + \frac{1}{2}\kappa_\mathrm{HF}\zeta_\mathrm{HF} \\
&+  \frac{1}{2}\sigma\PCM\sigma
  + \sigma\esp_{\mathrm{N}}(\tamp{},\tbar{})
  + \frac{1}{2}\kappa\MM\kappa
  + \kappa\zeta_{\mathrm{N}}(\tamp{},\tbar{})
  + \sigma\bi{X}\kappa \\
  &+ \sigma(\cancelto{0}{\bi{Y}\sigma_\mathrm{HF} + \bi{X}\kappa_\mathrm{HF} + \esp_\mathrm{HF}})
  + \kappa(\cancelto{0}{\bi{W}\kappa_\mathrm{HF} + \bi{X}^\dagger\sigma_\mathrm{HF} + \zeta_\mathrm{HF}}) \\
&+ \sigma_\mathrm{HF}\esp_{\mathrm{N}}(\tamp{},\tbar{}) + \kappa_\mathrm{HF}\zeta_{\mathrm{N}}(\tamp{},\tbar{}) \\
  %%%% Steps
&= \left/
    \frac{1}{2}\sigma_\mathrm{HF}\esp_\mathrm{HF}
  + \frac{1}{2}\kappa_\mathrm{HF}\zeta_\mathrm{HF}
  = U_\mathrm{pol}^\mathrm{ref}
  \right/ \\
&=
  U_\mathrm{pol}^\mathrm{ref}
  +  \frac{1}{2}\sigma\PCM\sigma
  + \sigma\esp_{\mathrm{N}}(\tamp{},\tbar{})
  + \frac{1}{2}\kappa\MM\kappa
  + \kappa\zeta_{\mathrm{N}}(\tamp{},\tbar{})
  + \sigma\bi{X}\kappa \\
&+ \sigma_\mathrm{HF}\esp_{\mathrm{N}}(\tamp{},\tbar{}) + \kappa_\mathrm{HF}\zeta_{\mathrm{N}}(\tamp{},\tbar{})
  \end{aligned}
\end{equation}

The reference polarization energy $U_\mathrm{pol}^\mathrm{ref}$ still
appears among the terms in the functional but will obviously not enter
in the optimization of the \acrshort{CC} wave function.
Recasting the polarization functional in the supermatrix formalism (see
Section \ref{sec:variational}) will simplify subsequent algebraic
manipulations:
\begin{equation}
  \begin{aligned}
  U_\mathrm{pol} &=
    \frac{1}{2}\sigma\PCM\sigma
    + \sigma\esp_{\mathrm{N}}(\tamp{},\tbar{})
  + \frac{1}{2}\kappa\MM\kappa
  + \kappa\zeta_{\mathrm{N}}(\tamp{},\tbar{})
  + \sigma\bi{X}\kappa
  + \sigma_\mathrm{HF}\esp_{\mathrm{N}}(\tamp{},\tbar{}) +
  \kappa_\mathrm{HF}\zeta_{\mathrm{N}}(\tamp{},\tbar{})
  + U_\mathrm{pol}^\mathrm{ref}
  \\
  &=
  \frac{1}{2}{}^t\p\V\p + {}^t\p\s_{\mathrm{N}}(\tamp{},\tbar{})
  + {}^t\p_\mathrm{HF}\s_{\mathrm{N}}(\tamp{},\tbar{})
  + U_\mathrm{pol}^\mathrm{ref}
\end{aligned}
\end{equation}

\end{comment}
