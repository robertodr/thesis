% mainfile: ../RobertoDiRemigioPhDThesis.tex
%************************************************
\chapter{Treating solvation within correlated wave function models}\label{ch:PCMCC}

\epigraph{Potius sero quam numquam}{--- \textsc{Titus Livius}, \textit{Ab Urbe Condita}}

\epigraph{Non temerai i terrori della notte non temerai il terrore}{--- \textsc{CCCP}}

I will present results on the formulation of quantum/classical
polarizable models when the quantum part of the system includes a
many-body description of electron correlation either by a \ac{MBPT},
\ac{CC} or hybrid approach.
These results leverage the variational formulation of quantum/classical
polarizable Hamiltonians described in Section \ref{sec:variational}.
This is an element of novelty with respect to the existing literature: I
will show how a fully variational framework enables transparent
derivations of \acs{MBPT} and \acs{CC} theory, encompassing previous
work~\cite{Cammi, Olivares del Valle, Caricato, Sneskov}
and solving some inconsistencies.~\cite{Angyan1992}

The contents of this Chapter are the result of an ongoing collaboration
with prof.~T.~Daniel Crawford (Virginia Tech) and Dr.~Andrew
C.~Simmonett (NIH)

After introducing the notation, I will derive the form of the \acs{CC}
effective Lagrangian and proceed to show how the work of
\citeauthor{Cammi2009-gu} and \citeauthor{Caricato2011-tx} on
\acs{PCM}-\acs{CC} and \citeauthor{Sneskov} on \acs{PE}-\acs{CC}
can easily be derived in the variational framework.

The second part of this Chapter will be devoted to the derivation of
\acs{MBPT} for quantum/classical polarizable Hamiltonians.
I will exploit the \acs{CC} Ansatz for the wave function and formulate
\acs{MBPT} in terms of the effective \acs{CC} Lagrangian, an approach
guaranteeing term-by-term size extensivity.

Iterative approximations to the \acs{CCSD} and \acs{CCSDT} models will
be the subject of Section \ref{sec:cc2-and-cc3} where the equations
governing the \acs{CC2} and \acs{CC3} models will be derived.
Eventually, I will describe schemes for the noniterative inclusion of
the effect of connected triples excitations on top of
the \acs{CCSD} wave function. The symmetric and asymmetric [T] and (T)
corrections will be derived.

\pagebreak

\section{A Brief Summary of Coupled Cluster Theory}

The analyses and derivations in this Chapter are preeminently based
on the \acl{CC} wave function Ansatz.
We will assume a closed-shell, \acs{HF} single reference function and
construct our correlated wave function as an exponential mapping:
\begin{equation}
  \ket{\mathrm{CC}} = \expo{T}\ket{\mathrm{HF}}.
\end{equation}
The \emph{cluster} operator appearing in the exponential is given as:
\begin{equation}
  T = \sum_{p=1}^\mathcal{M} T_p = \sum_{p=1}^\mathcal{M}\sum_{\mu_p}\tamp{\mu_p}\cluster{\mu_p}
\end{equation}
that is, as a truncated sum of excitation operators, $\cluster{\mu_p}$,
times the corresponding cluster amplitudes, $\tamp{\mu_p}$.
Here $\mu_p$ is the $\mu$-th excitation at the $p$-th excitation level and
$\mathcal{M}$ is the truncation level.
One of the main strengths of the \acs{CC} model is its size extensivity
which stems directly from the exponential nature of the wave
operator.~\autocite{Shavitt, Helgaker2000-tz}

The \acs{CC} method is a \emph{projective} method. Chosen the truncation
level $\mathcal{M}$, one projects the nonrelativistic Schrödinger
equation for this Ansatz on the excitation manifold which comprises the
reference function and all possible excited determinants, up to the
chosen truncation order:
\begin{subequations}
  \begin{align}
    \braket{\mathrm{HF} | H_0\expo{T} | \mathrm{HF}} &= E_\mathrm{CC} \\
    \braket{\mathrm{exc} | H_0\expo{T} | \mathrm{HF}} &=
    E_\mathrm{CC}\braket{\mathrm{exc} | \expo{T} | \mathrm{HF}}
  \end{align}
\end{subequations}
where $\bra{\mathrm{exc}}$ means any of the excited determinants in Fock
space compatible with the excitation level truncation.
This is the \emph{unlinked} form of the \acs{CC}
equations.~\autocite{Helgaker2000-tz} It is usually more convenient to
perform a \emph{similarity transformation} of the Hamiltonian operator,
leading to the \emph{linked} form of the \acs{CC} equations:
\begin{subequations}
  \begin{align}
    \braket{\mathrm{HF} | \expo{-T}H_0\expo{T} | \mathrm{HF}} &= E_\mathrm{CC} \\
    \braket{\mathrm{exc} | \expo{-T}H_0\expo{T} | \mathrm{HF}} &= 0
  \end{align}
\end{subequations}
The linked and unlinked formulations are completely equivalent. However,
the former is size extensive term-by-term.
Use of similarity transformed operators is ubiquitous when dealing with
\acs{CC} theory and we introduce the following notation for it:
\begin{equation}
  \overline{O} =\expo{-T}O\expo{T}
\end{equation}
The cluster operator is not Hermitian. Hence, the similarity transformation
will not, in general, preserve any of the symmetries, such as
hermiticity, that the bare operators might have possessed.
Similarity-transformed operators can be expanded in a commutator series,
also known as the \ac{BCH} expansion:
\begin{equation}
  \overline{O} =
  \sum_{n=0}^{\infty}\frac{1}{n!}[A, B]^n
\end{equation}
where $[A,B]^n$ is the $n$-fold nested commutator. It can be shown that
the \acs{BCH} expansion of the similarity-transformed Hamiltonian
\emph{exactly} truncates after the four-fold nested commutator, greatly
simplifying algebraic derivations and manipulations.

Introducing the \acl{MP} partitioning of the Hamiltonian leads to
further insight:
\begin{equation}
 H_0 = F + \Phi = \sum_P\epsilon_P\cons{P}\anni{P} + (g- V_\mathrm{HF})
\end{equation}
where $F$ if the Fock matrix and $\Phi$ is the fluctuation potential.
Given our initial assumption on the reference function, the Fock
operator is diagonal and expressed in terms of spin-orbital energies and
number operators.
Its similarity transformation truncates after the second term and has a
relatively compact form:
\begin{equation}
  \overline{F} = F + \sum_{i}^4\sum_{\mu_i}\tamp{\mu_i}\cluster{\mu_i}\denom{\mu_i}
\end{equation}
where $\denom{\mu_i}$ is the difference in orbital energies between the
virtual and occupied spin-orbitals of excitation $i$.
For the fluctuation potential the similarity transformation truncates
after the fifth term, including up to four-fold nested commutator.

As formulated so far, the \acs{CC} method is not variational. It is
however possible to introduce a fully variational Lagrangian that
leads to the same amplitudes equations. The Lagrangian is built by
treating the amplitude equations as constraints in the optimization of
the \acs{CC} energy:
\begin{equation}\label{eq:cc-lagrangian}
  \begin{aligned}
    \mathcal{L}^\mathcal{M}(\tamp{}, \tbar{})
    &=
    \braket{\mathrm{HF} | \overline{H_0} | \mathrm{HF}}
    + \sum_{i=1}^{\mathcal{M}}\sum_{\mu_i}
    \tbar{\mu_i}\braket{\mu_i | \overline{H_0} | \mathrm{HF}} \\
    &=
  E_0
  + \sum_{i=1}^{\mathcal{M}}\tbar{\mu_i}\denom{\mu_i}\tamp{\mu_i}
  + \braket{\mathrm{HF} | \overline{\Phi} | \mathrm{HF}}
  + \sum_{i=1}^{\mathcal{M}}\braket{\tbar{i} | \overline{\Phi} | \mathrm{HF}}
  \end{aligned}
\end{equation}
where the shorthand notation for the Lagrangian multiplier state was
introduced:
\begin{equation}
  \bra{\tbar{j}} = \sum_{\mu_j}\tbar{\mu_j}\bra{\mu_j}
\end{equation}
and the \acs{MP}-partitioned form of the Hamiltonian was also exploited.
Differentiation of the Lagrangian with respect to the multipliers
correctly yields the amplitudes equations, while differentiation with
respect to the amplitudes leads to the governing equations for the
multipliers. While these are not needed for the evaluation of the
\acs{CC} energy, their calculation is mandatory when expectation values
and molecular properties in general are sought.
The \acs{CC} expectation values are formed using the left and
right \acs{CC} states and we introduce the following notation:
\begin{equation}
  O(\tamp{}, \tbar{}) = \braket{\mathrm{HF} | \overline{O} | \mathrm{HF} }
  + \sum_{i=1}^{\mathcal{M}}\braket{\tbar{i} | \overline{O} | \mathrm{HF} }
\end{equation}
When required by the context, we will explicitly write at which
truncation level in the cluster operator the expectation was evaluated:
$O^\mathcal{M}(\tamp{}, \tbar{})$.

Having summarized the essential points of \acs{CC} theory and having
introduced the relevant notation, we can proceed to analyze its coupling
with quantum/classical polarizable Hamiltonians.

\section{Effective Coupled Cluster Lagrangian}

Our purpose in this Section is to derive an \emph{effective} \acs{CC}
Lagrangian that takes into account the coupling of the correlated
electronic state with the classical polarizable
environment.
This can be readily obtained~\autocite{Lipparini2016-mo} by coupling the
usual Lagrangian in Eq.~\eqref{eq:cc-lagrangian} with the classical
polarization functional in Eq.~\eqref{eq:pcm-mm-functional}.
Care must however be taken in the definition of the classical
variational degrees of freedom as to not erroneously include
contributions from the reference state in the subsequent correlated
treatment.

Once again, let us assume that the reference state is a closed-shell
\acs{HF} determinant. Further, we assume that the reference determinant
was optimized in the presence of the classical polarizable environment.
Then the variational degrees of freedom corresponding to the environment
satisfy:
\begin{equation}
  \begin{pmatrix}
    \PCM & \MMPCM \\
    \MMPCM^\dagger & \MM
  \end{pmatrix}
  \begin{pmatrix}
   \sigma_\mathrm{HF} \\
   \kappa_\mathrm{HF}
  \end{pmatrix}
  =
  -
  \begin{pmatrix}
   \esp_\mathrm{HF} \\
   \zeta_\mathrm{HF}
  \end{pmatrix}
\end{equation}
and correspond to the \emph{reference reaction field}.
In the subsequent correlated treatment one can separate the reaction
field into reference and correlated components:
\begin{equation}
  \begin{pmatrix}
   \sigma^\mathrm{tot} \\
   \kappa^\mathrm{tot}
  \end{pmatrix}
  =
  \begin{pmatrix}
   \sigma_\mathrm{HF} \\
   \kappa_\mathrm{HF}
  \end{pmatrix}
  +
  \begin{pmatrix}
   \sigma \\
   \kappa
  \end{pmatrix}
\end{equation}
and similarly for the sources:
\begin{equation}
\begin{aligned}
  \begin{pmatrix}
   \esp(\tamp{}, \tbar{}) \\
   \zeta(\tamp{}, \tbar{})
  \end{pmatrix}
  &=
  \begin{pmatrix}
   \esp_\mathrm{HF} \\
   \zeta_\mathrm{HF}
  \end{pmatrix}
  +
  \begin{pmatrix}
    \tilde{\esp}(\tamp{}, \tbar{}) \\
   \tilde{\zeta}(\tamp{}, \tbar{})
  \end{pmatrix} \\
  &=
  \begin{pmatrix}
  \braket{\mathrm{HF} | \esp  | \mathrm{HF}} \\
  \braket{\mathrm{HF} | \zeta | \mathrm{HF}}
  \end{pmatrix}
  +
  \begin{pmatrix}
   \braket{\mathrm{HF} | \tilde{\esp} | \mathrm{HF}}
  + \sum_{i=1}^{\mathcal{M}}\braket{\tbar{i} | \overline{\esp} | \mathrm{HF}} \\
   \braket{\mathrm{HF} | \tilde{\zeta} | \mathrm{HF}}
  + \sum_{i=1}^{\mathcal{M}}\braket{\tbar{i} | \overline{\zeta} | \mathrm{HF}}
  \end{pmatrix}
\end{aligned}
\end{equation}
The \acs{BCH} expansions of the source terms have been
rewritten as:
\begin{subequations}
  \begin{align}
  \overline{\esp} &= \esp + \tilde{\esp} = \esp + \commutator{\esp}{T}
  + \frac{1}{2}\commutator{\commutator{\esp}{T}}{T} \\
  \overline{\kappa} &= \kappa + \tilde{\kappa} = \kappa + \commutator{\kappa}{T}
  + \frac{1}{2}\commutator{\commutator{\kappa}{T}}{T}
  \end{align}
\end{subequations}
taking advantage of the fact that both $\esp$ and $\kappa$ are
nondiagonal, one-electron operators and hence their commutator
expansions truncate at the third term.
Next, we apply a shift to the correlation part of the source term one-electron
operators to remove the reference source terms:
\begin{equation}
  \begin{aligned}
  \begin{pmatrix}
   \braket{\mathrm{HF} | \tilde{\esp} | \mathrm{HF}}
  + \sum_{i=1}^{\mathcal{M}}\braket{\tbar{i} | \tilde{\esp} | \mathrm{HF}} \\
   \braket{\mathrm{HF} | \tilde{\zeta} | \mathrm{HF}}
  + \sum_{i=1}^{\mathcal{M}}\braket{\tbar{i} | \tilde{\zeta} | \mathrm{HF}}
  \end{pmatrix}
  &=
  \begin{pmatrix}
    \braket{\mathrm{HF} | \overline{\esp} - \esp_\mathrm{HF} | \mathrm{HF}}
  + \sum_{i=1}^{\mathcal{M}}\braket{\tbar{i} | \overline{\esp} - \esp_\mathrm{HF} | \mathrm{HF}} \\
  \braket{\mathrm{HF} | \overline{\zeta} - \zeta_\mathrm{HF} | \mathrm{HF}}
  + \sum_{i=1}^{\mathcal{M}}\braket{\tbar{i} | \overline{\zeta} -
  \zeta_\mathrm{HF}| \mathrm{HF}}
  \end{pmatrix}
  \\
  =
  \begin{pmatrix}
   \braket{\mathrm{HF} | \tilde{\esp} | \mathrm{HF}}
  + \sum_{i=1}^{\mathcal{M}}\braket{\tbar{i} | \overline{\esp} | \mathrm{HF}} \\
   \braket{\mathrm{HF} | \tilde{\zeta} | \mathrm{HF}}
  + \sum_{i=1}^{\mathcal{M}}\braket{\tbar{i} | \overline{\zeta} | \mathrm{HF}}
  \end{pmatrix}
  &=
  \begin{pmatrix}
    \tilde{\esp}(\tamp{}, \tbar{}) \\
    \tilde{\zeta}(\tamp{}, \tbar{})
  \end{pmatrix}
\end{aligned}
\end{equation}
which is equivalent to imposing normal ordering of the
operators.~\autocite{Shavitt}

Eventually, the polarization functional can be rewritten as:
\begin{equation}
  \begin{aligned}
  U_\mathrm{pol} &=
  \frac{1}{2}(\sigma_\mathrm{HF}+\sigma)\PCM(\sigma_\mathrm{HF}+\sigma)
+ (\sigma_\mathrm{HF}+\sigma)(\esp_\mathrm{HF}+\tilde{\esp}(\tamp{},\tbar{})) \\
&+ \frac{1}{2}(\kappa_\mathrm{HF}+\kappa)\MM(\kappa_\mathrm{HF}+\kappa)
+ (\kappa_\mathrm{HF}+\kappa)(\zeta_\mathrm{HF}+\tilde{\zeta}(\tamp{},\tbar{}))
+ (\sigma_\mathrm{HF}+\sigma)\bi{X}(\kappa_\mathrm{HF}+\kappa) \\
  %%%% Steps
&= \left/ \text{Expand and collect} \right/ \\
&=
    \frac{1}{2}\sigma_\mathrm{HF}\PCM\sigma_\mathrm{HF}
  + \sigma_\mathrm{HF}\esp_\mathrm{HF}
  + \frac{1}{2}\kappa_\mathrm{HF}\MM\kappa_\mathrm{HF}
  + \kappa_\mathrm{HF}\zeta_\mathrm{HF}
  + \sigma_\mathrm{HF}\bi{X}\kappa_\mathrm{HF} \\
&+  \frac{1}{2}\sigma\PCM\sigma
  + \sigma\tilde{\esp}(\tamp{},\tbar{})
  + \frac{1}{2}\kappa\MM\kappa
  + \kappa\tilde{\zeta}(\tamp{},\tbar{})
  + \sigma\bi{X}\kappa \\
&+  \frac{1}{2}\sigma_\mathrm{HF}\PCM\sigma
  + \sigma_\mathrm{HF}\tilde{\esp}(\tamp{},\tbar{})
  + \frac{1}{2}\kappa_\mathrm{HF}\MM\kappa
  + \kappa_\mathrm{HF}\tilde{\zeta}(\tamp{},\tbar{})
  + \sigma_\mathrm{HF}\bi{X}\kappa \\
&+  \frac{1}{2}\sigma\PCM\sigma_\mathrm{HF}
  + \sigma\esp_\mathrm{HF}
  + \frac{1}{2}\kappa\MM\kappa_\mathrm{HF}
  + \kappa\zeta_\mathrm{HF}
  + \sigma\bi{X}\kappa_\mathrm{HF} \\
  %%%% Steps
&= \left/
  \frac{1}{2}\sigma_\mathrm{HF}\PCM\sigma_\mathrm{HF}
  +
  \frac{1}{2}\kappa_\mathrm{HF}\MM\kappa_\mathrm{HF}
  +
  \sigma_\mathrm{HF}\esp_\mathrm{HF}
  +
  \kappa_\mathrm{HF}\zeta_\mathrm{HF}
  +
  \sigma_\mathrm{HF}\bi{X}\kappa_\mathrm{HF}
  =
  \frac{1}{2}\sigma_\mathrm{HF}\esp_\mathrm{HF}
  +
  \frac{1}{2}\kappa_\mathrm{HF}\zeta_\mathrm{HF}
  \right/ \\
&=
    \frac{1}{2}\sigma_\mathrm{HF}\esp_\mathrm{HF}
  + \frac{1}{2}\kappa_\mathrm{HF}\zeta_\mathrm{HF} \\
&+  \frac{1}{2}\sigma\PCM\sigma
  + \sigma\tilde{\esp}(\tamp{},\tbar{})
  + \frac{1}{2}\kappa\MM\kappa
  + \kappa\tilde{\zeta}(\tamp{},\tbar{})
  + \sigma\bi{X}\kappa \\
&+  \frac{1}{2}\sigma_\mathrm{HF}\PCM\sigma
  + \sigma_\mathrm{HF}\tilde{\esp}(\tamp{},\tbar{})
  + \frac{1}{2}\kappa_\mathrm{HF}\MM\kappa
  + \kappa_\mathrm{HF}\tilde{\zeta}(\tamp{},\tbar{})
  + \sigma_\mathrm{HF}\bi{X}\kappa \\
&+  \frac{1}{2}\sigma\PCM\sigma_\mathrm{HF}
  + \sigma\esp_\mathrm{HF}
  + \frac{1}{2}\kappa\MM\kappa_\mathrm{HF}
  + \kappa\zeta_\mathrm{HF}
  + \sigma\bi{X}\kappa_\mathrm{HF} \\
  %%%% Steps
&= \left/
  \frac{1}{2}\sigma\PCM\sigma_\mathrm{HF}
  =
  \frac{1}{2}\sigma_\mathrm{HF}\PCM\sigma;\quad
  \frac{1}{2}\kappa\MM\kappa_\mathrm{HF}
  =
  \frac{1}{2}\kappa_\mathrm{HF}\MM\kappa; \quad
  \sigma_\mathrm{HF}\bi{X}\kappa
  =
  \kappa\bi{X}^\dagger\sigma_\mathrm{HF}
  \right/ \\
&=
    \frac{1}{2}\sigma_\mathrm{HF}\esp_\mathrm{HF}
  + \frac{1}{2}\kappa_\mathrm{HF}\zeta_\mathrm{HF} \\
&+  \frac{1}{2}\sigma\PCM\sigma
  + \sigma\tilde{\esp}(\tamp{},\tbar{})
  + \frac{1}{2}\kappa\MM\kappa
  + \kappa\tilde{\zeta}(\tamp{},\tbar{})
  + \sigma\bi{X}\kappa \\
  &+ \sigma(\cancelto{0}{\bi{Y}\sigma_\mathrm{HF} + \bi{X}\kappa_\mathrm{HF} + \esp_\mathrm{HF}})
  + \kappa(\cancelto{0}{\bi{W}\kappa_\mathrm{HF} + \bi{X}^\dagger\sigma_\mathrm{HF} + \zeta_\mathrm{HF}}) \\
&+ \sigma_\mathrm{HF}\tilde{\esp}(\tamp{},\tbar{}) + \kappa_\mathrm{HF}\tilde{\zeta}(\tamp{},\tbar{}) \\
  %%%% Steps
&= \left/
    \frac{1}{2}\sigma_\mathrm{HF}\esp_\mathrm{HF}
  + \frac{1}{2}\kappa_\mathrm{HF}\zeta_\mathrm{HF}
  \coloneq U_\mathrm{pol}^\mathrm{ref}
  \right/ \\
&=
  U_\mathrm{pol}^\mathrm{ref}
  +  \frac{1}{2}\sigma\PCM\sigma
  + \sigma\tilde{\esp}(\tamp{},\tbar{})
  + \frac{1}{2}\kappa\MM\kappa
  + \kappa\tilde{\zeta}(\tamp{},\tbar{})
  + \sigma\bi{X}\kappa \\
&+ \sigma_\mathrm{HF}\tilde{\esp}(\tamp{},\tbar{}) + \kappa_\mathrm{HF}\tilde{\zeta}(\tamp{},\tbar{})
  \end{aligned}
\end{equation}

The reference polarization energy $U_\mathrm{pol}^\mathrm{ref}$ still
appears among the terms in the functional but will obviously not enter
in the optimization of the \acs{CC} wave function.
Recasting the polarization functional in the supermatrix formalism (see
Section \ref{sec:variational}) will simplify subsequent algebraic
manipulations:
\begin{equation}
  \begin{aligned}
  U_\mathrm{pol} &=
    \frac{1}{2}\sigma\PCM\sigma
    + \sigma\tilde{\esp}(\tamp{},\tbar{})
  + \frac{1}{2}\kappa\MM\kappa
  + \kappa\tilde{\zeta}(\tamp{},\tbar{})
  + \sigma\bi{X}\kappa
  + \sigma_\mathrm{HF}\tilde{\esp}(\tamp{},\tbar{}) +
  \kappa_\mathrm{HF}\tilde{\zeta}(\tamp{},\tbar{})
  + U_\mathrm{pol}^\mathrm{ref}
  \\
  &=
  \frac{1}{2}{}^t\p\V\p + {}^t\p\tilde{\s}(\tamp{},\tbar{})
  + {}^t\p_\mathrm{HF}\tilde{\s}(\tamp{},\tbar{})
  + U_\mathrm{pol}^\mathrm{ref}
\end{aligned}
\end{equation}

One can then write the effective Lagrangian as:
\begin{equation}
  \begin{aligned}
  \lag{M}(\tamp{}, \tbar{}, \p) &=
  \braket{\mathrm{HF} | \overline{H_0} | \mathrm{HF}}
  + \sum_{i=1}^{\mathcal{M}}\braket{\tamp{i} | \overline{H_0} | \mathrm{HF}} \\
  &+
  \frac{1}{2}{}^t\p\V\p + {}^t\p\tilde{\s}(\tamp{},\tbar{})
  + {}^t\p_\mathrm{HF}\tilde{\s}(\tamp{},\tbar{})
  + U_\mathrm{pol}^\mathrm{ref}
  \end{aligned}
\end{equation}
In accordance with previous work,\autocite{Cammi2009-gu,
Caricato2011-tx} the Hamiltonian $H_0$ is augmented with the
${}^t\p_\mathrm{HF}\tilde{\s}(\tamp{},\tbar{})$ term to yield the
\acl{PTE} (\acs{PTE}) Hamiltonian -- $H$ -- and eventually the
\acl{PTED} (\acs{PTED}) Lagrangian:~\autocite{Olivares del Valle,
Lipparini}
\begin{equation}\label{eq:pted-cc}
  \lag{M}(\tamp{}, \tbar{}, \p) =
  {}^\mathrm{PTE}\lag{M}
  + \frac{1}{2}{}^t\p\V\p + {}^t\p\tilde{\s}(\tamp{},\tbar{})
\end{equation}
where the \acs{PTE} Lagrangian has been introduced:
\begin{equation}
{}^\mathrm{PTE}\lag{M}
  =
  \braket{\mathrm{HF} | \overline{H} | \mathrm{HF}}
  + \sum_{i=1}^{\mathcal{M}}\braket{\tbar{i} | \overline{H} | \mathrm{HF}}
  + U_\mathrm{pol}^\mathrm{ref}
\end{equation}

As noted by \citeauthor{Cammi2009-gu} for the implicit \acs{PCM} model,
the \acs{PTE} Hamiltonian is computed in \acs{MP}-partitioned form by
employing the ``solvated'' orbitals and Fock matrix:
\begin{equation}
  H = F + \Phi.
\end{equation}
$F$ is a diagonal one-electron operator and $\Phi$ is the
fluctuation potential.
The simple nature of the Fock operator lets us re-express the \acs{PTE}
Lagrangian as follows:
\begin{equation}
{}^\mathrm{PTE}\lag{M}
  =
  G_0
  + \sum_{i=1}^{\mathcal{M}}\tbar{\mu_i}\denom{\mu_i}\tamp{\mu_i}
  + \braket{\mathrm{HF} | \overline{\Phi} | \mathrm{HF}}
  + \sum_{i=1}^{\mathcal{M}}\braket{\tbar{i} | \overline{\Phi} | \mathrm{HF}}
\end{equation}
where $G_0 = E_0 + U_\mathrm{pol}^\mathrm{ref}$ is the reference free
energy.

Before delving into the details of the governing equations for
amplitudes, multipliers and polarization, we need to take a closer look
at the source term in classical energy functional. This is the purpose
of the next Section.

\section{Source terms in the classical energy functional}

In the supermatrix formalism, the similarity-transformed terms will be
written as:
\begin{equation}
  \overline{\s} = \s + \tilde{\s} = \s +
  \commutator{\s}{T} +
  \frac{1}{2}\commutator{\commutator{\s}{T}}{T}
\end{equation}
It is important to single out which terms contribute in the various
expectation values where the similarity-transformed sources will be
involved.
We develop this analysis in terms of excitation ranks of the operators
involved, based on the following:
\begin{equation}
  \braket{\mu_n |
  \commutator{\commutator{\commutator{O}{T_{n_1}}}{\ldots}}{T_{n_k}} | \mathrm{HF}} \neq 0
  \Longleftrightarrow
  n - m_O + k \leq \sum_{i=1}^k n_i \leq n + m_O
\end{equation}
where $n$ is the excitation rank of the $\ket{\mu_n}$ determinant, $k$
is the total number of cluster operators appearing in the nested
commutator, $n_i$ are their respective excitations ranks and $m_O$
is the \emph{particle rank} of the operator $O$.\autocite{Helgaker2000-tz}
This follows from the realization that:
\begin{equation}
  \commutator{\commutator{\commutator{O}{T_{n_1}}}{\ldots}}{T_{n_k}} \ket{\mathrm{HF}}
\end{equation}
is a linear combination of determinants with excitation ranks $v$ in the
range:
\begin{equation}
  \sum_{i=1}^k n_i - m_O \leq v \leq \sum_{i=1}^k n_i + m_O -k
\end{equation}

We first expand the source terms entering the Lagrangian as follows:
\begin{equation}
  \tilde{\s}(\tamp{}, \tbar{}) =
  \braket{\mathrm{HF} | \BCHfirst{\s}{T} + \BCHsecond{\s}{T} | \mathrm{HF}}
  + \sum_{i=1}^{\mathcal{M}}\braket{\tbar{i} | \s + \BCHfirst{\s}{T} +
  \BCHsecond{\s}{T} | \mathrm{HF}},
\end{equation}
and since $\s$ has particle rank $m_\s = 1$ we can conclude that the
following relationships hold:
\begin{subequations}
\begin{align}
  \braket{\mathrm{HF} |
  \BCHfirst{\s}{T_t}
  | \mathrm{HF}}
  &\quad\quad\text{nonzero contributions} \quad 0 \leq t \leq 1
  \Longleftrightarrow t = 1 \\
  \braket{\mathrm{HF} |
  \commutator{\commutator{\s}{T_t}}{T_u}
  | \mathrm{HF}}
  &\quad\quad\text{nonzero contributions} \quad t + u = 1
  \label{eq:HF-double-commutator} \\
  \braket{\tbar{i} |
  \s
  | \mathrm{HF} }
  &\quad\quad\text{nonzero contributions} \quad i = 1 \\
  \braket{\tbar{i} |
  \BCHfirst{\s}{T_t}
  | \mathrm{HF} }
  &\quad\quad\text{nonzero contributions} \quad i \leq t \leq i + 1 \\
  \braket{\tbar{i} |
  \commutator{\commutator{\s}{T_t}}{T_u}
  | \mathrm{HF} }
  &\quad\quad\text{nonzero contributions} \quad t + u = i + 1
\end{align}
\end{subequations}
By virtue of Eq.~\eqref{eq:HF-double-commutator} the \acs{HF}
expectation value of the double commutator can be dropped. Thus, for a
general truncation level we obtain:
\begin{equation}\label{eq:source-term}
  \tilde{\s}(\tamp{}, \tbar{}) =
  \braket{\mathrm{HF} | \BCHfirst{\s}{T_1} | \mathrm{HF}}
  + \braket{\tbar{1} | \s | \mathrm{HF}}
  + \sum_{i=1}^{\mathcal{M}}\braket{\tbar{i} | \BCHfirst{\s}{T} + \BCHsecond{\s}{T} | \mathrm{HF}}
\end{equation}
Note that the singles amplitudes assume a unique role in the \acs{CC}
expression of the source term.
In addition, double commutators involving the highest order cluster
operators can only lead to nonzero terms when $T_1$ is also involved and
by projection onto the highest order multiplier state, \eg
$\braket{\tbar{2} | \commutator{\commutator{\s}{T_1}}{T_2} |
\mathrm{HF}} \neq 0$ and $\braket{\tbar{3} |
\commutator{\commutator{\s}{T_1}}{T_3} | \mathrm{HF}} \neq 0$
More explicitly, for the \ac{CCS} model ($\mathcal{M} = 1$) one has:
\begin{equation}
  \tilde{\s}^1(\tamp{}, \tbar{}) =
  \braket{\mathrm{HF} | \BCHfirst{\s}{T_1} | \mathrm{HF}}
  + \braket{\tbar{1} | \s | \mathrm{HF}}
  + \braket{\tbar{1} | \BCHfirst{\s}{T_1} | \mathrm{HF}}
  + \frac{1}{2}\braket{\tbar{1} | \commutator{\commutator{\s}{T_1}}{T_1} | \mathrm{HF}}
\end{equation}
while for the \ac{CCSD} model ($\mathcal{M} = 2$):
\begin{equation}
  \tilde{\s}^2(\tamp{}, \tbar{}) =
  \tilde{\s}^1(\tamp{}, \tbar{})
  + \braket{\tbar{1} | \BCHfirst{\s}{T_2} | \mathrm{HF}}
  + \braket{\tbar{2} | \BCHfirst{\s}{T_2} | \mathrm{HF}}
  + \braket{\tbar{2} | \commutator{\commutator{\s}{T_1}}{T_2} | \mathrm{HF}}
\end{equation}
Eventually, within the \ac{CCSDT} model ($\mathcal{M} = 3$) four more
terms are added:
\begin{equation}
  \begin{aligned}
  \tilde{\s}^3(\tamp{}, \tbar{}) &=
  \tilde{\s}^2(\tamp{}, \tbar{})
  + \braket{\tbar{2} | \BCHfirst{\s}{T_3} | \mathrm{HF}} \\
  &+ \braket{\tbar{3} | \BCHfirst{\s}{T_3} | \mathrm{HF}}
  + \frac{1}{2}
  \braket{\tbar{3} | \commutator{\commutator{\s}{T_2}}{T_2} |
  \mathrm{HF}}
  + \braket{\tbar{3} | \commutator{\commutator{\s}{T_1}}{T_3} |
  \mathrm{HF}}
  \end{aligned}
\end{equation}

\section{Governing Equations and Their Approximations}\label{sec:pcm-cc-models}

Differentiation of the effective Lagrangian in Eq.~\eqref{eq:pted-cc}
with respect to the variational parameters $\tamp{\mu_i}$,
$\tbar{\mu_i}$ and $\p$ yields the \acs{PTED}-\acs{CC}
equations:
\begin{subequations}
  \begin{align}
   \tampEq{\mu_i}(\tamp{}, \tbar{}, \p)  &=
   \denom{\mu_i}\tamp{\mu_i} + \braket{\mu_i | \overline{\Phi} | \mathrm{HF}}
%   + {}^t\p\braket{\mu_i | \s | \mathrm{HF}}\delta_{\mu_i\mu_1} \nonumber \\
+ {}^t\p\braket{\mu_i | \overline{\s} | \mathrm{HF}}
             = 0 \label{eq:pted-cc-amplitudes}\\
   \tbarEq{\mu_i}(\tamp{}, \tbar{}, \p)
    &=
    \denom{\mu_i}\tbar{\mu_i} +
    \braket{\mathrm{HF} | \commutator{\overline{\Phi}}{\cluster{\mu_i}} | \mathrm{HF}} +
    \sum_{j}^{\mathcal{M}}\braket{\tbar{j} |
    \commutator{\overline{\Phi}}{\cluster{\mu_i}} | \mathrm{HF}}
    \nonumber \\
    &+
    {}^t\p\braket{\mathrm{HF} | \commutator{\s}{\cluster{\mu_i}} | \mathrm{HF}}\delta_{\mu_i\mu_1} \nonumber \\
    &+
    {}^t\p\sum_{j}^{\mathcal{M}}\braket{\tbar{j} | \commutator{\s}{\cluster{\mu_i}} | \mathrm{HF}}
    +
    {}^t\p\sum_{j}^{\mathcal{M}}\braket{\tbar{j} | \commutator{\commutator{\s}{T}}{\cluster{\mu_i}} | \mathrm{HF}}
             = 0 \label{eq:pted-cc-multipliers}\\
    \Omega_\p(\tamp{}, \tbar{}, \p)
    &=
    \V\p + \tilde{\s}(\tamp{}, \tbar{}) = 0
    \label{eq:pted-cc-polarization}
  \end{align}
\end{subequations}
A coupling of the amplitudes and multipliers equations is introduced by
the presence of the $\tilde{\s}(\tamp{}, \tbar{})$ expectation value
in the polarization equation.
This requires a proper macroiteration/microiteration self-consistency
scheme for its implementation.~\autocite{Cammi2009-gu, Caricato2010-hx}
As a consequence, a single-point \acs{PTED}-\acs{CC} will suffer from a
$2\times$ prefactor in its computational cost with respect to an
\emph{in vacuo} \acs{CC} calculation.

It is thus convenient to devise approximations that are able to simplify the
Lagrangian and the governing equations by breaking or weakening the coupling.
While Eq.~\eqref{eq:pted-cc-polarization} is \emph{directly} coupled to
the multipliers equation, since the source terms directly depends on
amplitudes and multipliers, it is only \emph{indirectly} coupled to the
amplitudes equation, where only the polarization enters.
Moreover, the leading terms in the correlated source operator
expectation value, Eq.~\eqref{eq:source-term}, prominently involve the
singles cluster operator and the singles multiplier state.
Three approximation have currently been proposed: \acs{PTES}, \acs{PTE(S)}
and \acs{PTE}.

\subsection*{PTES Scheme}

\todo[inline]{MY CLAIM
The first approximate scheme truncates the source term to its \acs{CCS}
expectation value.}

\subsection*{PTE(S) Scheme}

If we further truncate the correlated source operator expectation value to the
first term in Eq.~\eqref{eq:source-term}, the coupling between the
\acs{CC} right and left states is broken, while still improving upon the
reference reaction field with correlated contributions.
This is the essence of the \acs{PTE(S)} approximation.~\autocite{Caricato2011-tx}
The effective Lagrangian would then read as:
\begin{equation}\label{eq:ptes-cc}
  {}^{\mathrm{PTE(S)}}\lag{M}(\tamp{}, \tbar{}, \p) =
  \braket{\mathrm{HF} | \overline{H} | \mathrm{HF}}
  + \sum_{i=1}^{\mathcal{M}}\braket{\tbar{i} | \overline{H} | \mathrm{HF}}
  + \frac{1}{2}{}^t\p\V\p + {}^t\p
  \braket{\mathrm{HF} | \BCHfirst{\s}{T_1} | \mathrm{HF}}
\end{equation}
which uncouples all equations:
\begin{subequations}
  \begin{align}
  {}^{\mathrm{PTE(S)}}\tampEq{\mu_i}(\tamp{}, \tbar{})  &=
   \denom{\mu_i}\tamp{\mu_i} + \braket{\mu_i | \overline{\Phi} | \mathrm{HF}}
   = 0 \\
   {}^{\mathrm{PTE(S)}}\tbarEq{\mu_i}(\tamp{}, \tbar{}, \p) &=
    \denom{\mu_i}\tbar{\mu_i} +
    \braket{\mathrm{HF} | \commutator{\overline{\Phi}}{\cluster{\mu_i}} | \mathrm{HF}} +
    \sum_{j}^{\mathcal{M}}\braket{\tbar{j} |
    \commutator{\overline{\Phi}}{\cluster{\mu_i}} | \mathrm{HF}}
    \nonumber \\
    &+
    {}^t\p\braket{\mathrm{HF} | \commutator{\s}{\cluster{\mu_i}} | \mathrm{HF}}\delta_{\mu_i\mu_1}
    = 0 \\
    {}^{\mathrm{PTE(S)}}\Omega_\p(\tamp{}, \tbar{}, \p)
    &=
    \V\p + \braket{\mathrm{HF} | \BCHfirst{\s}{T_1} | \mathrm{HF}} = 0
  \end{align}
\end{subequations}
The amplitude equations are formally equivalent to those derived in the
\acs{PTE} scheme. The equations for multipliers are also formally
equivalent to the \acs{PTE} equations, apart from the equation
determining the singles multipliers which is augmented by an additional term:
${}^t\p\braket{\mathrm{HF} | \commutator{\s}{\cluster{\mu_1}} |
\mathrm{HF}}$
This term is not, however, coupled to the other equations given that
$\p$ is fixed once the singles amplitudes have been determined.
Practical implementation of the \acs{PTE(S)} scheme will solve the
\acs{PTE}-like amplitude equation, form the source term
$\braket{\mathrm{HF} | \BCHfirst{\s}{T_1} | \mathrm{HF}}$, solve for
the polarization degrees of freedom and calculate the polarization
energy as a correction to the \acs{CC} energy.
Only when a first-order molecular property, such as the molecular
gradient, is needed, will the multipliers equations need to be solved.
In other words, only the $T_1$-dependent part of the \acs{CC} density is
used to define the classical sources.

\subsection*{PTE Scheme}

The simplest approach is to neglect altogether the polarization equation
and the polarization degree of freedom. This is equivalent to
differentiating the polarization-independent \acs{PTE} effective
Lagrangian:
\begin{subequations}
  \begin{align}
   \tampEq{\mu_i}(\tamp{}, \tbar{}, \p)  &=
   \denom{\mu_i}\tamp{\mu_i} + \braket{\mu_i | \overline{\Phi} | \mathrm{HF}}
    = 0 \\
   \tbarEq{\mu_i}(\tamp{}, \tbar{}, \p)
    &=
    \denom{\mu_i}\tbar{\mu_i} +
    \braket{\mathrm{HF} | \commutator{\overline{\Phi}}{\cluster{\mu_i}} | \mathrm{HF}} +
    \sum_{j}^{\mathcal{M}}\braket{\tbar{j} |
    \commutator{\overline{\Phi}}{\cluster{\mu_i}} | \mathrm{HF}}
    = 0
  \end{align}
\end{subequations}
The \acs{PTE} scheme naturally preserves the scaling of the underlying
\acs{CC} method.
The \acs{PTE} scheme is advantageous since no coupling between the
\acs{CC} amplitude and multiplier equations is introduced.
The model is also easily implemented, as it only requires access to a
reference state optimized in the presence of the classical medium.
The \acs{PTE} model thus offers an efficient approximation to the full
\acs{PTED} model with a computational cost comparable to that of
\emph{in vacuo} \acs{CC} theory.

However, the polarization included in the correlation treatment is the
same as for the reference determinant, an approximation which appears
questionable from the point of view of classical electrostatics. There
is, in fact, no relaxation of the reference reaction field due to the
correlated description of the electronic density.

\section[Coupled Cluster Perturbation Theory]{
Coupled Cluster Perturbation Theory for a Quantum/Classical Polarizable Hamiltonian}\label{sec:ccpt-polarizable}

We will now develop a perturbative expansion of the effective \acs{CC}
Lagrangian and derive the \acs{CC}\acs{PT} series when a
quantum/classical polarizable Hamiltonian is used.
The fluctuation potential $\Phi$ is assumed as the perturbation, hence
it will be considered of order 1: $\Phi = O(1)$.
Orders will be counted assuming a closed-shell \acs{HF} reference:
we will use square brackets $[i]$ for an $i$-th order contributions.
We further assume that real wave functions are used.

The structure of the stationarity conditions
Eqs.~\eqref{eq:pted-cc-amplitudes} and \eqref{eq:pted-cc-multipliers},
already shows that $\tamp{\mu_i}^{[0]} = 0, \quad \forall i$ and
$\tbar{\mu_i}^{[0]} = 0, \quad \forall i$.

The \acs{PTED}-\acs{CC} equations are expanded in orders of the
perturbation and terms are collected order by order. The use of an
effective, variational Lagrangian implies the validity of the $2n+1$
rule for the amplitudes $\tamp{\mu_i}$ and polarization $\p$ and of the
$2n+2$ rule for the multipliers $\tbar{\mu_i}$.
We can thus derive energy corrections up to fifth order by means of the
amplitudes and polarization up to and including second order terms and
the multipliers up to and including second order.

The polarization equation couples to the multipliers and amplitudes
equations \emph{via} the source term, see Eq.~\eqref{eq:source-term}.
Its perturbative expansion will be given as:
\begin{equation}\label{eq:source-term-pt}
  \tilde{\s}(\tamp{},\tbar{}) =
  \tilde{\s}^{[0]}
  + \tilde{\s}^{[1]}
  + \tilde{\s}^{[2]}
  + \ldots
\end{equation}
and correspondingly for the polarization:
\begin{equation}\label{eq:pol-eq-pt}
  \V\p^{[i]} + \tilde{\s}^{[i]} = 0
\end{equation}

The source operator $\s$ will be considered as \emph{zeroth-order} in
the perturbation.
The order of the contributions to $\tilde{\s}(\tamp{},\tbar{})$ will be
solely determined by the amplitudes and multipliers.
From the structure of \eqref{eq:pted-cc-polarization}, it is already
clear that $\tilde{\s}(\tamp{},\tbar{})$ is at least first order in the
fluctuation potential:
\begin{equation}
  \tilde{\s}(\tamp{},\tbar{}) =
  \tilde{\s}^{[1]}
  + \tilde{\s}^{[2]}
  + \ldots
\end{equation}
which also implies:
\begin{equation}
  \V\p^{[0]} + \tilde{\s}^{[0]} = 0 \Rightarrow  \p^{[0]} = 0
\end{equation}

\paragraph*{Perturbative expansion of $\tilde{\s}$}

\begin{equation}\label{eq:source-1st-order}
  \tilde{\s}^{[1]}(\tamp{}, \tbar{}) =
  \braket{\mathrm{HF} | \BCHfirst{\s}{T^{[1]}_1} | \mathrm{HF}}
  + \braket{\tbar{1}^{[1]} | \s | \mathrm{HF}}
\end{equation}
In the free energy corrections we will use the $2n+1$ and $2n+2$ complying
expression:
\begin{equation}\label{eq:source-1st-order-rules}
  \tilde{\s}^{[1]}(\tamp{}, \tbar{}) =
  \braket{\mathrm{HF} | \BCHfirst{\s}{T^{[1]}_1} | \mathrm{HF}}
\end{equation}

\begin{equation}\label{eq:source-2nd-order}
  \tilde{\s}^{[1]}(\tamp{}, \tbar{}) =
  \braket{\mathrm{HF} | \BCHfirst{\s}{T^{[2]}_1} | \mathrm{HF}}
  + \braket{\tbar{1}^{[2]} | \s | \mathrm{HF}}
  + \sum_{i=1}^{\mathcal{M}}\braket{\tbar{i}^{[1]} | \BCHfirst{\s}{T^{[1]}} | \mathrm{HF}}
\end{equation}
In the free energy corrections we will use the $2n+1$ and $2n+2$ complying
expression:
\begin{equation}\label{eq:source-2nd-order-rules}
  \tilde{\s}^{[2]}(\tamp{}, \tbar{}) =
  \braket{\mathrm{HF} | \BCHfirst{\s}{T^{[2]}_1} | \mathrm{HF}}
  + \braket{\tbar{i}^{[1]} | \BCHfirst{\s}{T^{[1]}} | \mathrm{HF}}
\end{equation}



\subsection{First order equations}\label{sec:first-order-pt}

\paragraph*{Amplitudes}
\begin{equation}\label{eq:1st-order-amp}
  \Omega_{\mu_i}^{[1]} = \denom{\mu_i}\tamp{\mu_i}^{[1]}
  + \braket{\mu_i | \Phi | \mathrm{HF}}
  + {}^t\p^{[1]}\braket{\mu_i | \s | \mathrm{HF}}\delta_{\mu_i\mu_1}
  = 0
\end{equation}
Considering the singles, doubles and triples excitation manifolds:
\begin{subequations}
  \begin{align}
  \Omega_{\mu_1}^{[1]} &= \denom{\mu_1}\tamp{\mu_1}^{[1]}
  + {}^t\p^{[1]}\braket{\mu_1 | \s | \mathrm{HF}}
  = 0 \\
  \Omega_{\mu_2}^{[1]} &= \denom{\mu_2}\tamp{\mu_2}^{[1]}
  + \braket{\mu_2 | \Phi | \mathrm{HF}}
  = 0 \\
  \Omega_{\mu_3}^{[1]} &= \denom{\mu_3}\tamp{\mu_3}^{[1]} = 0
  \end{align}
\end{subequations}
Thus, despite the closed-shell \acs{HF} reference, the singles
amplitudes will already appear in first order, due to the
quantum/classical coupling.

\paragraph*{Multipliers}
\begin{equation}%\label{eq:1st-order-mult}
  \bar{\Omega}_{\mu_i}^{[1]} =
    \denom{\mu_i}\tbar{\mu_i}^{[1]}
    + \braket{\mathrm{HF} | \commutator{\Phi}{\cluster{\mu_i}} | \mathrm{HF}}
    + {}^t\p^{[1]}\braket{\mathrm{HF} |
    \commutator{\s}{\cluster{\mu_i}} | \mathrm{HF}}\delta_{\mu_i\mu_1} =0
\end{equation}
and expanding the commutators:
\begin{equation}\label{eq:1st-order-mult}
  \bar{\Omega}_{\mu_i}^{[1]} =
    \denom{\mu_i}\tbar{\mu_i}^{[1]}
    + \braket{\mathrm{HF} | \Phi | \mu_i}
    + {}^t\p^{[1]}\braket{\mathrm{HF} | \s | \mu_i}\delta_{\mu_i\mu_1} =0
\end{equation}
This clearly shows that Eq.~\eqref{eq:1st-order-amp}
and Eq.~\eqref{eq:1st-order-mult} are complex conjugates. Under the
assumption of real wave functions we can also conclude that:
\begin{equation}
  \tamp{\mu_i}^{[1]} = \tbar{\mu_i}^{[1]},\,\,\forall i
\end{equation}
Thus, as for the amplitudes, despite the closed-shell \acs{HF} reference, the singles
multipliers will already appear in first order, due to the
quantum/classical coupling.
Moreover, as is the case \emph{in vacuo}, there are no triples contributions to
the first order amplitudes and multipliers: $\tamp{\mu_3}^{[1]} = \tbar{\mu_3}^{[1]} = 0,\,\,\forall \mu$.

\paragraph*{Polarization}
To first order, the polarization degree of freedom will be given as:
\begin{equation}%\label{eq:1st-order-pol}
  \V\p^{[1]} + \braket{\mathrm{HF} | \commutator{\s}{T_1^{[1]}}| \mathrm{HF}}
  + \braket{\tbar{1}^{[1]} | \s | \mathrm{HF} } = 0
\end{equation}
Under the assumption of real wave functions, the equivalence of
first order amplitudes and multipliers yields:
\begin{equation}
  \begin{aligned}
  \braket{\mathrm{HF} | \commutator{\s}{T_1^{[1]}}| \mathrm{HF}}
  %%%% Steps
  &= \left/ T_1^{[1]} = \sum_{\mu_1}\ket{\mu_1}\tamp{\mu_1} \right/
  =
  \sum_{\mu_1}\braket{\mathrm{HF}|\s|\mu_1}\tamp{\mu_1} \\
  %%%% Steps
  &= \left/ \text{Complex conjugation and real wave functions} \right/ \\
  &= \sum_{\mu_1}\braket{\mu_1 | \s | \mathrm{HF}}\tbar{\mu_1}
  = \braket{\tbar{\mu_1}^{[1]} | \s | \mathrm{HF}}
 \end{aligned}
\end{equation}
so that the first order polarization equation becomes:
\begin{equation}\label{eq:1st-order-pol}
  \V\p^{[1]} + 2\braket{\mathrm{HF} | \commutator{\s}{T_1^{[1]}}| \mathrm{HF}} = 0
\end{equation}

\subsection{Second order equations}\label{sec:second-order-pt}

\paragraph*{Amplitudes}
\begin{equation}\label{eq:2nd-order-amp}
  \Omega_{\mu_i}^{[2]} = \denom{\mu_i}\tamp{\mu_i}^{[2]}
  + \braket{\mu_i | \commutator{\Phi}{T^{[1]}} | \mathrm{HF}}
  + {}^t\p^{[2]}\braket{\mu_i | \s | \mathrm{HF}}\delta_{\mu_i\mu_1}
  + {}^t\p^{[1]}\braket{\mu_i | \commutator{\s}{T^{[1]}} |
  \mathrm{HF}}
  = 0
\end{equation}
Considering the singles, doubles and triples excitation manifolds:
\begin{subequations}
  \begin{align}
  \Omega_{\mu_1}^{[2]} &= \denom{\mu_1}\tamp{\mu_1}^{[2]}
  + \braket{\mu_1 | \commutator{\Phi}{T^{[1]}} | \mathrm{HF}}
  + {}^t\p^{[2]}\braket{\mu_1 | \s | \mathrm{HF}}
  + {}^t\p^{[1]}\braket{\mu_1 | \commutator{\s}{T^{[1]}} |
  \mathrm{HF}}
  = 0 \\
  \Omega_{\mu_2}^{[2]} &= \denom{\mu_2}\tamp{\mu_2}^{[2]}
  + \braket{\mu_2 | \commutator{\Phi}{T^{[1]}} | \mathrm{HF}}
  + {}^t\p^{[1]}\braket{\mu_2 | \commutator{\s}{T_2^{[1]}} |
  \mathrm{HF}}
  = 0 \\
  \Omega_{\mu_3}^{[2]} &= \denom{\mu_3}\tamp{\mu_3}^{[2]}
  + \braket{\mu_3 | \commutator{\Phi}{T_2^{[1]}} | \mathrm{HF}}
  = 0
  \end{align}
\end{subequations}
As already noted elsewhere~\autocite{Koch1997-nm, Helgaker2000-tz},
the second term in Eq.~\eqref{eq:2nd-order-amp} can involve no higher
than triple excitations.
Moreover, the triples first appear to second order and are not \emph{directly}
affected by the quantum/classical coupling.

\paragraph*{Multipliers}
\begin{equation}
  \begin{aligned}
  \bar{\Omega}_{\mu_i}^{[2]} &=
    \denom{\mu_i}\tbar{\mu_i}^{[2]}
    + \braket{\mathrm{HF} | \commutator{\commutator{\Phi}{T^{[1]}}}{\cluster{\mu_i}} | \mathrm{HF}}
    + \sum_j\braket{\tbar{j}^{[1]} |
    \commutator{\Phi}{\cluster{\mu_i}} | \mathrm{HF}} \\
    &+ {}^t\p^{[2]}\braket{\mathrm{HF} |
    \commutator{\s}{\cluster{\mu_i}} | \mathrm{HF}}\delta_{\mu_i\mu_1}
    + {}^t\p^{[1]}
    \sum_j\braket{\tbar{j}^{[1]} |
    \commutator{\s}{\cluster{\mu_i}} | \mathrm{HF}}
    =0
  \end{aligned}
\end{equation}
The double commutator term is null thus:~\autocite{Koch1997-nm, Helgaker2000-tz}
\todo[inline]{Check this statement... I am not 100\% sure and I have
found no derivation...}
\begin{equation}\label{eq:2nd-order-mult}
  \begin{aligned}
  \bar{\Omega}_{\mu_i}^{[2]} &=
    \denom{\mu_i}\tbar{\mu_i}^{[2]}
    + \sum_j\braket{\tbar{j}^{[1]} |
    \commutator{\Phi}{\cluster{\mu_i}} | \mathrm{HF}} \\
    &+ {}^t\p^{[2]}\braket{\mathrm{HF} |
    \commutator{\s}{\cluster{\mu_i}} | \mathrm{HF}}\delta_{\mu_i\mu_1}
    + {}^t\p^{[1]}
    \sum_j\braket{\tbar{j}^{[1]} |
    \commutator{\s}{\cluster{\mu_i}} | \mathrm{HF}}
    =0
  \end{aligned}
\end{equation}
\citeauthor{Koch1997-nm} showed that the \emph{in vacuo} second order
amplitudes and multipliers equations are the complex conjugates of each
other for the singles, doubles and tripled manifold. Under the further
assumption of real wave functions thus $\tamp{\mu_i}^{[2]} =
\tbar{\mu_i}^{[2]},\,\,\forall i = 1 - 3$.
A similar conclusion holds also for the quantum/classical polarizable
Hamiltonians here considered.
First of all, the terms involving the second order polarization degree of
freedom are identical:
\begin{equation}
    {}^t\p^{[2]}\braket{\mathrm{HF} | \commutator{\s}{\cluster{\mu_i}} | \mathrm{HF}}\delta_{\mu_i\mu_1}
    =
    {}^t\p^{[2]}\braket{\mathrm{HF} | \s|\mu_i}\delta_{\mu_i\mu_1}
    =
    {}^t\p^{[2]}\braket{\mu_i| \s|\mathrm{HF}}\delta_{\mu_i\mu_1}
\end{equation}
since they are complex conjugates and real wave functions are assumed.
Under the same set of assumptions, the first order terms are also identical:
\begin{equation}
  \begin{aligned}
  {}^t\p^{[1]} \sum_j\braket{\tbar{j}^{[1]} | \commutator{\s}{\cluster{\mu_i}} | \mathrm{HF}}
  %%%% Steps
  &= \left/ \sum_j\bra{\tbar{j}^{[1]}} = \sum_j\sum_{\mu_j}\tbar{\mu_j}\bra{\mu_j}  \right/ \\
  &=
  {}^t\p^{[1]} \sum_j\sum_{\mu_j}\tbar{\mu_j}^{[1]} \braket{\mu_j| \s | \mu_i}
  %%%% Steps
  = \left/ \text{Complex conjugation and real wave functions} \right/ \\
  &=
  {}^t\p^{[1]} \sum_j\sum_{\mu_j} \braket{\mu_i| \s | \mu_j} \tamp{\mu_j}^{[1]}
  %%%% Steps
  = \left/ \sum_j\sum_{\mu_j}\tamp{\mu_j}\ket{\mu_j}  = T^{[1]}\right/ \\
  &=
  {}^t\p^{[1]}\braket{\mu_i | \commutator{\s}{T^{[1]}} | \mathrm{HF}}
 \end{aligned}
\end{equation}

Considering the singles, doubles and triples excitation manifolds:
\begin{subequations}
  \begin{align}
    \bar{\Omega}_{\mu_1}^{[2]} &=
      \denom{\mu_1}\tbar{\mu_1}^{[2]}
      + \sum_j\braket{\tbar{j}^{[1]} |
      \commutator{\Phi}{\cluster{\mu_1}} | \mathrm{HF}} \nonumber \\
      &+ {}^t\p^{[2]}\braket{\mathrm{HF} |
      \commutator{\s}{\cluster{\mu_1}} | \mathrm{HF}}
      + {}^t\p^{[1]}
      \braket{\tbar{1}^{[1]} |
      \commutator{\s}{\cluster{\mu_1}} | \mathrm{HF}}
      =0 \\
    \bar{\Omega}_{\mu_2}^{[2]} &=
      \denom{\mu_2}\tbar{\mu_2}^{[2]}
      + \sum_j\braket{\tbar{j}^{[1]} |
      \commutator{\Phi}{\cluster{\mu_2}} | \mathrm{HF}}
      + {}^t\p^{[1]}
      \braket{\tbar{1}^{[1]} + \tbar{2}^{[1]} |
      \commutator{\s}{\cluster{\mu_2}} | \mathrm{HF}}
      =0 \\
    \bar{\Omega}_{\mu_3}^{[2]} &=
      \denom{\mu_3}\tbar{\mu_3}^{[2]}
      + \sum_j\braket{\tbar{j}^{[1]} |
      \commutator{\Phi}{\cluster{\mu_3}} | \mathrm{HF}}
      =0
  \end{align}
\end{subequations}
The triples first appear to second order and are not \emph{directly}
affected by the quantum/classical coupling.

\paragraph*{Polarization}
\begin{equation}
  \V\p^{[2]} +
  \braket{\mathrm{HF} | \commutator{\s}{T_1^{[2]}}| \mathrm{HF}}
  + \braket{\tbar{1}^{[2]} | \s | \mathrm{HF} }
  + \braket{\tbar{1}^{[1]} | \commutator{\s}{T^{[1]}} | \mathrm{HF}}
  + \braket{\tbar{2}^{[1]} | \commutator{\s}{T_2^{[1]}} | \mathrm{HF}}
  = 0
\end{equation}
There are no explicit contributions from triples. This means that the
\acs{CCSD} polarization is correct through fourth order in the
fluctuation potential.
Moreover, since the second order amplitudes and multipliers are
identical up to the triples one has $\braket{\mathrm{HF} | \commutator{\s}{T_1^{[2]}}| \mathrm{HF}} = \braket{\tbar{1}^{[2]} | \s | \mathrm{HF} }$
and thus:
\begin{equation}\label{eq:2nd-order-pol}
  \V\p^{[2]} +
  2\braket{\mathrm{HF} | \commutator{\s}{T_1^{[2]}}| \mathrm{HF}}
  + \braket{\tbar{1}^{[1]} | \commutator{\s}{T^{[1]}} | \mathrm{HF}}
  + \braket{\tbar{2}^{[1]} | \commutator{\s}{T_2^{[1]}} | \mathrm{HF}}
  = 0
\end{equation}

\subsection{Free energies up to fifth order}\label{sec:energies-pt}

From the expansion of the effective Lagrangian we can obtain free energy
corrections up to the desired order. Given the variational nature of the
Lagrangian, only terms fulfilling the $2n+1$ and $2n+2$ rules will
appear in the free energy corrections.
To zeroth-order all variational parameters are zero. To first order we
thus have:
\begin{equation}\label{eq:first-order-G}
  G^{[1]} = \braket{\mathrm{HF} | \Phi | \mathrm{HF}}
\end{equation}
The reference energy can thus be recovered as usual:
\begin{equation}
  G_\mathrm{HF} = G^{[0]} + G^{[1]}
\end{equation}
where $G^{[0]}$ includes the quantum/classical coupling.

We introduce the following notation:
\begin{equation}
  \tilde{\s}^{[m]}(\tamp{}^{[p]}, \tbar{}^{[q]})
\end{equation}
for a source term of $m$-th order formed by amplitudes up to and
including $p$-th order and by multipliers up to and including $q$-th order.
Green boxes will be put around terms that are formally equivalent to the
vacuum expression for the energy corrections. Red boxes will appear
around quantum/classical coupling terms that involve the triples
manifold.

The second order energy correction will be formed including first order
amplitudes and polarization, while using zeroth order multipliers:
\begin{equation}\label{eq:second-order-G}
  \begin{aligned}
  G^{[2]} &=
  \highlight{green}{
  \braket{\mathrm{HF} | \commutator{\Phi}{T^{[1]}} | \mathrm{HF}}
  }
  + \frac{1}{2}{}^t\p^{[1]}\V\p^{[1]} +
  {}^t\p^{[1]}\tilde{\s}^{[1]}(\tamp{}^{[1]}, \tbar{}^{[0]}) \\
  &= \highlight{green}{E^{[2]}}
  +
  \frac{1}{2}{}^t\p^{[1]}\V\p^{[1]}
  +
  {}^t\p^{[1]}\braket{\mathrm{HF} | \commutator{\s}{T^{[1]}_1} | \mathrm{HF}} \\
  &= \left/
  \V\p^{[1]} + 2\braket{\mathrm{HF} | \commutator{\s}{T_1^{[1]}}| \mathrm{HF}} = 0
  \right/ \\
  &= \highlight{green}{E^{[2]}}
  \end{aligned}
\end{equation}
\todo[inline]{I feel like there is some important lesson/conclusion to
draw from this.}

The third order energy correction will be formed including first order
amplitudes, multipliers and polarization:
\begin{equation}\label{eq:third-order-G}
  \begin{aligned}
    G^{[3]} &=
    \highlight{green}{\sum_j\braket{\tbar{j}^{[1]} | \commutator{\Phi}{T^{[1]}} | \mathrm{HF}}}
    + {}^t\p^{[1]}\tilde{\s}^{[2]}(\tamp{}^{[1]}, \tbar{}^{[1]}) \\
    &=
    \highlight{green}{E^{[3]}}
    + {}^t\p^{[1]}
    \lbrace
      \braket{\tbar{1}^{[1]} | \commutator{\s}{T^{[1]}} | \mathrm{HF}}
    + \braket{\tbar{2}^{[1]} | \commutator{\s}{T^{[1]}_2} | \mathrm{HF}}
    \rbrace
  \end{aligned}
\end{equation}

The fourth order energy corrections will be formed including second
order amplitudes and polarization, while using first order multipliers:
\begin{equation}\label{eq:fourth-order-G}
  \begin{aligned}
    G^{[4]} &=
    \highlight{green}
    {
    \sum_j\braket{\tbar{j}^{[1]} | \commutator{\Phi}{T^{[2]}} | \mathrm{HF}}
    +
    \sum_j\braket{\tbar{j}^{[1]} |
    \frac{1}{2}\commutator{\commutator{\Phi}{T^{[1]}}}{T^{[1]}} | \mathrm{HF}}
    }
    \\
    &+\frac{1}{2}{}^t\p^{[2]}\V\p^{[2]}
    + {}^t\p^{[2]}\tilde{\s}^{[2]}(\tamp{}^{[2]}, \tbar{}^{[1]})
    + {}^t\p^{[1]}\tilde{\s}^{[3]}(\tamp{}^{[2]}, \tbar{}^{[1]}) \\
    &= \highlight{green}{E^{[4]}} + \frac{1}{2}{}^t\p^{[2]}\V\p^{[2]}
    +{}^t\p^{[2]}
    \lbrace
    \braket{\mathrm{HF} | \commutator{\s}{T_1^{[2]}} | \mathrm{HF}}
    + \sum_i\braket{\tbar{i}^{[1]}| \commutator{\s}{T^{[1]}} | \mathrm{HF} }
    \rbrace \\
    &+{}^t\p^{[1]}
    \lbrace
    \sum_i\braket{\tbar{i}^{[1]} | \commutator{\s}{T^{[2]}} | \mathrm{HF}}
    +
    \sum_i\frac{1}{2}\braket{\tbar{i}^{[1]} |
    \commutator{\commutator{\s}{T^{[1]}}}{T^{[1]}}
    |
    \mathrm{HF}}
    \rbrace \\
  %%%% Steps
  &= \left/ \text{Remove null terms} \right/ \\
  &= \highlight{green}{E^{[4]}} \\
    &+ \frac{1}{2}{}^t\p^{[2]}\V\p^{[2]}
    +{}^t\p^{[2]}
    \lbrace
    \braket{\mathrm{HF} | \commutator{\s}{T_1^{[2]}} | \mathrm{HF}}
    + \braket{\tbar{1}^{[1]}| \commutator{\s}{T^{[1]}} | \mathrm{HF} }
    + \braket{\tbar{2}^{[1]}| \commutator{\s}{T_2^{[1]}} | \mathrm{HF} }
    \rbrace \\
    &+
    {}^t\p^{[1]}
    \lbrace
    \braket{\tbar{1}^{[1]} | \commutator{\s}{T_1^{[2]} + T_2^{[2]}} | \mathrm{HF}}
    +
    \braket{\tbar{2}^{[1]} | \commutator{\s}{T_2^{[2]} + T_3^{[2]}} | \mathrm{HF}}
    \rbrace \\
    &+
    {}^t\p^{[1]}
    \lbrace
    \frac{1}{2}\braket{\tbar{1}^{[1]} |
    \commutator{\commutator{\s}{T_1^{[1]}}}{T_1^{[1]}}
    |
    \mathrm{HF}}
    +
    \braket{\tbar{2}^{[1]} |
    \commutator{\commutator{\s}{T_1^{[1]}}}{T_2^{[1]}}
    |
    \mathrm{HF}}
    \rbrace \\
  %%%% Steps
  &= \left/
  \V\p^{[2]} +
  2\braket{\mathrm{HF} | \commutator{\s}{T_1^{[2]}}| \mathrm{HF}}
  + \braket{\tbar{1}^{[1]} | \commutator{\s}{T^{[1]}} | \mathrm{HF}}
  + \braket{\tbar{2}^{[1]} | \commutator{\s}{T_2^{[1]}} | \mathrm{HF}}
  = 0
  \right/ \\
  &= \highlight{green}{E^{[4]}}
    + \frac{1}{2}{}^t\p^{[2]}\braket{\tbar{1}^{[1]}| \commutator{\s}{T^{[1]}} | \mathrm{HF} }
    + \frac{1}{2}{}^t\p^{[2]}\braket{\tbar{2}^{[1]}| \commutator{\s}{T_2^{[1]}} | \mathrm{HF} } \\
    &+
    {}^t\p^{[1]}
    \braket{\tbar{1}^{[1]} | \commutator{\s}{T_1^{[2]} + T_2^{[2]}} | \mathrm{HF}}
    +
    \highlight{red}{
    {}^t\p^{[1]}
    \braket{\tbar{2}^{[1]} | \commutator{\s}{T_2^{[2]} + T_3^{[2]}} | \mathrm{HF}}
    }
    \\
    &+
    {}^t\p^{[1]}
    \lbrace
    \frac{1}{2}\braket{\tbar{1}^{[1]} |
    \commutator{\commutator{\s}{T_1^{[1]}}}{T_1^{[1]}}
    |
    \mathrm{HF}}
    +
    \braket{\tbar{2}^{[1]} |
    \commutator{\commutator{\s}{T_1^{[1]}}}{T_2^{[1]}}
    |
    \mathrm{HF}}
    \rbrace \\
  \end{aligned}
\end{equation}
\todo[inline]{There is one term involving second order triples and first
order polarization}

Eventually, the fifth order energy correction is formed including second order
amplitudes, multipliers and polarization:
\begin{equation}\label{eq:fifth-order-G}
  \begin{aligned}
    G^{[5]} &=
    \highlight{green}
    {
    \frac{1}{2}\braket{\mathrm{HF} |
    \commutator{\commutator{\Phi}{T^{[2]}}}{T^{[2]}}
    | \mathrm{HF}}
    +
    \sum_j \braket{\tbar{j}^{[1]} |
    \commutator{\commutator{\Phi}{T^{[2]}}}{T^{[1]}}
    | \mathrm{HF}}
    }
    \\
    &+ \highlight{green}{\sum_j \braket{\tbar{j}^{[2]} |
      \commutator{\Phi}{T^{[2]}}
    | \mathrm{HF}}
    +
    \frac{1}{2}\sum_j\braket{\tbar{j}^{[2]} |
    \commutator{\commutator{\Phi}{T^{[1]}}}{T^{[1]}}
    | \mathrm{HF}}
    }
    + {}^t\p^{[2]}\tilde{\s}^{[3]}(\tamp{}^{[2]}, \tbar{}^{[2]})
    + {}^t\p^{[1]}\tilde{\s}^{[4]}(\tamp{}^{[2]}, \tbar{}^{[2]}) \\
    &= \highlight{green}{E^{[5]}}
    + {}^t\p^{[2]}
    \lbrace
    \sum_i\braket{\tbar{i}^{[1]} | \commutator{\s}{T^{[2]}} | \mathrm{HF}}
    +
    \sum_i\braket{\tbar{i}^{[1]} |
    \frac{1}{2}\commutator{\commutator{\s}{T^{[1]}}}{T^{[1]}}
    | \mathrm{HF}}
    +
    \sum_i\braket{\tbar{i}^{[2]} | \commutator{\s}{T^{[1]}} | \mathrm{HF}}
    \rbrace \\
    &+ {}^t\p^{[1]}
    \lbrace
    \sum_i\braket{\tbar{i}^{[2]} | \commutator{\s}{T^{[2]}} | \mathrm{HF}}
    +
    \sum_i\braket{\tbar{i}^{[2]} |
    \frac{1}{2}\commutator{\commutator{\s}{T^{[1]}}}{T^{[1]}}
    | \mathrm{HF}}
    +
    \sum_i\braket{\tbar{i}^{[1]} |
    \commutator{\commutator{\s}{T^{[1]}}}{T^{[2]}}
    | \mathrm{HF}}
    \rbrace \\
  %%%% Steps
  &= \left/ \text{Remove null terms} \right/ \\
  &= \highlight{green}{E^{[5]}}
    + {}^t\p^{[2]}\braket{\tbar{1}^{[1]} | \commutator{\s}{T_1^{[2]} + T_2^{[2]}} | \mathrm{HF}}
    + \highlight{red}{
    {}^t\p^{[2]}\braket{\tbar{2}^{[1]} | \commutator{\s}{T_2^{[2]} + T_3^{[2]}} | \mathrm{HF}}
    } \\
    &+ {}^t\p^{[2]}\braket{\tbar{1}^{[2]} | \commutator{\s}{T^{[1]}} | \mathrm{HF}}
    + {}^t\p^{[2]}
    \braket{\tbar{1}^{[1]} |
    \frac{1}{2}\commutator{\commutator{\s}{T_1^{[1]}}}{T_1^{[1]}}
    | \mathrm{HF}}
    + {}^t\p^{[2]}
    \braket{\tbar{2}^{[1]} |
    \commutator{\commutator{\s}{T_1^{[1]}}}{T_2^{[1]}}
    | \mathrm{HF}}
    \\
    &+
    {}^t\p^{[1]}
    \braket{\tbar{1}^{[2]} | \commutator{\s}{T_1^{[2]} + T_2^{[2]}} | \mathrm{HF}}
    +
    \highlight{red}{
    {}^t\p^{[1]}
    \braket{\tbar{2}^{[2]} | \commutator{\s}{T_2^{[2]} + T_3^{[2]}} | \mathrm{HF}}
    }
    +
    \highlight{red}{
    {}^t\p^{[1]}
    \braket{\tbar{3}^{[2]} | \commutator{\s}{T_3^{[2]}} | \mathrm{HF}}
    } \\
    &+
    {}^t\p^{[1]}
    \braket{\tbar{1}^{[2]} |
    \frac{1}{2}\commutator{\commutator{\s}{T_1^{[1]}}}{T_1^{[1]}}
    | \mathrm{HF}}
    +
    {}^t\p^{[1]}
    \braket{\tbar{2}^{[2]} |
    \commutator{\commutator{\s}{T_1^{[1]}}}{T_2^{[1]}}
    | \mathrm{HF}} \\
    &+
    \highlight{red}{
    {}^t\p^{[1]}
    \braket{\tbar{3}^{[2]} |
    \frac{1}{2}\commutator{\commutator{\s}{T_2^{[1]}}}{T_2^{[1]}}
    | \mathrm{HF}}
    } \\
    &+ {}^t\p^{[1]}
    \braket{\tbar{1}^{[1]} |
    \commutator{\commutator{\s}{T_1^{[1]}}}{T_1^{[2]}}
    | \mathrm{HF}}
    + {}^t\p^{[1]}
    \braket{\tbar{2}^{[1]} |
    \commutator{\commutator{\s}{T_1^{[1]}}}{T_2^{[2]}}
    | \mathrm{HF}} \\
    &+ {}^t\p^{[1]}
    \braket{\tbar{2}^{[1]} |
    \commutator{\commutator{\s}{T_2^{[1]}}}{T_1^{[2]}}
    | \mathrm{HF}}
  \end{aligned}
\end{equation}

\section[Noniterative Triples Corrections to PCM-CCSD]{
Noniterative Perturbative Triples Corrections to PCM-CCSD}\label{sec:ccsd-t-noniterative}

The \acs{PTED}-\acs{CCSD} scheme for the quantum/classical coupling is
only includes singles and doubles in its excitation manifold and is thus
correct up to third order in perturbation theory:
\begin{subequations}
  \begin{align}
    \tamp{\mu_i}^{*} &= \tamp{\mu_i}^{[1]} + \tamp{\mu_i}^{[2]} + O(3)
    \label{eq:tamp-star} \\
    \tbar{\mu_i}^{*} &= \tbar{\mu_i}^{[1]} + \tbar{\mu_i}^{[2]} + O(3)
    \label{eq:tbar-star} \\
    \p^{*} &= \p^{[1]} + \p^{[2]} + \p^{[3]} + O(4)
    \label{eq:p-star}
  \end{align}
\end{subequations}
We will denote the \acs{CCSD} converged parameters with a $*$ superscript.
\todo[inline]{Check the statement above about $\p^{*}$.}

Already in fourth order connected triples make their appearance and any
model going beyond third order must then take their effect into account.
We first select the fourth and fifth order contributions to the energy
involving connected triples.
\begin{equation}\label{eq:triples-fourth-order}
  G_T^{[4]} =
  \highlight{red}{\braket{\tbar{1}^{[1]} | \commutator{\Phi}{T_3^{[2]}} | \mathrm{HF}}}
  + \braket{\tbar{2}^{[1]}| \commutator{\Phi}{T_3^{[2]}} | \mathrm{HF}}
  + {}^t\p^{[1]}\braket{\tbar{2}^{[1]} | \commutator{\s}{T_3^{[2]}} | \mathrm{HF}}
\end{equation}
\begin{equation}\label{eq:triples-fifth-order}
  \begin{aligned}
  G_T^{[5]} &=
    \highlight{red}{
    \braket{\tbar{2}^{[1]} |
    \commutator{\commutator{\Phi}{T_3^{[2]}}}{T_1^{[1]}}
    | \mathrm{HF}}
    } \\
    &+
    \braket{\tbar{1}^{[2]} | \commutator{\Phi}{T_3^{[2]}} | \mathrm{HF}}
  + \braket{\tbar{2}^{[2]} | \commutator{\Phi}{T_3^{[2]}} | \mathrm{HF}}
  + \braket{\tbar{3}^{[2]} | \commutator{\Phi}{T_2^{[2]}} | \mathrm{HF}}
  + \braket{\tbar{3}^{[2]} | \commutator{\Phi}{T_3^{[2]}} | \mathrm{HF}}
  \\
    &+
    \frac{1}{2}\braket{\tbar{3}^{[2]} |
    \commutator{\commutator{\Phi}{T_2^{[1]}}}{T_2^{[1]}}
    | \mathrm{HF}}
  +
  \highlight{red}{
    \braket{\tbar{3}^{[2]} |
    \commutator{\commutator{\Phi}{T_1^{[1]}}}{T_2^{[1]}}
    | \mathrm{HF}}
   }
   \\
   &+
    {}^t\p^{[2]}\braket{\tbar{2}^{[1]} | \commutator{\s}{T_3^{[2]}} | \mathrm{HF}}
  + {}^t\p^{[1]}\braket{\tbar{2}^{[2]} | \commutator{\s}{T_3^{[2]}} | \mathrm{HF}} \\
  &+ {}^t\p^{[1]}\braket{\tbar{3}^{[2]} | \commutator{\s}{T_3^{[2]}} | \mathrm{HF}}
  + {}^t\p^{[1]}\braket{\tbar{3}^{[2]} |
    \frac{1}{2}\commutator{\commutator{\s}{T_2^{[1]}}}{T_2^{[1]}}
    | \mathrm{HF}}
  \end{aligned}
\end{equation}
Terms in red boxes are vacuum-like terms that appear due to nonzero
first order singles amplitudes and multipliers.

\todo[inline]{WARNING. I am not 100\% sure that the multipliers
equations are complex conjugate to the amplitudes equations also in
second order!}

\subsection{The \texorpdfstring{$\Lambda\text{CCSD}[\text{T}]$}{CCSD[aT]} and \texorpdfstring{$\text{CCSD}[\text{T}]$}{CCSD[T]} models}

\todo[inline]{References for the a-CCSD[T] and CCSD[T] models!!!}

The [T] noniterative triples corrections are formed by adding the fourth
order \acs{CC}\acs{PT} terms arising from connected triples on top of
the \acs{CCSD} energy. The \emph{converged} \acs{CCSD} amplitudes,
multipliers and polarization are however used instead of the
perturbative quantities in Eq.~\eqref{eq:triples-fourth-order}.
The correction term is thus improved, since the \acs{CCSD} quantities
are determined to infinite order, within the singles and doubles space.
The asymmetric $\Lambda\text{CCSD}[\text{T}]$ correction is then
obtained as:
\begin{equation}\label{eq:4-lambda-bracket-t}
  G_{\Lambda[\text{T}]} =
  {}^*G_{\Lambda[\text{T}]}^{[4]} =
  \highlight{red}{\braket{\tbar{1}^{*} | \commutator{\Phi}{T_3^{*}} | \mathrm{HF}}}
  + \braket{\tbar{2}^{*}| \commutator{\Phi}{T_3^{*}} | \mathrm{HF}}
  + {}^t\p^{*}\braket{\tbar{2}^{*} | \commutator{\s}{T_3^{*}} | \mathrm{HF}}
\end{equation}
Replacing the perturbation multipliers by converged \acs{CCSD}
amplitudes leads to the symmetric CCSD[T] model:
\begin{equation}\label{eq:4-bracket-t}
  G_{[\text{T}]} =
  {}^*G_{[\text{T}]}^{[4]} =
  \highlight{red}{\braket{\tamp{1}^{*} | \commutator{\Phi}{T_3^{*}} | \mathrm{HF}}}
  + \braket{\tamp{2}^{*}| \commutator{\Phi}{T_3^{*}} | \mathrm{HF}}
  + {}^t\p^{*}\braket{\tamp{2}^{*} | \commutator{\s}{T_3^{*}} | \mathrm{HF}}
\end{equation}

\subsection{The
\texorpdfstring{$\Lambda\text{CCSD}(\text{T})$}{CCSD(aT)} and CCSD(T) models}

\todo[inline]{References for the a-CCSD(T) and CCSD(T) models!!!}

The (T) noniterative triples corrections are formed by adding some fifth
order terms to the fourth order [T] correction given above.
Also for the (T) corrections, the \emph{converged} \acs{CCSD} amplitudes, multipliers and polarization
are used instead of the perturbative quantities.
The fifth order terms involving projection of
$\commutator{\Phi}{T_3^{*}}$ on the singles and doubles manifold are
chosen to form the additional perturbative correction.

The correction term is thus improved, since the \acs{CCSD} quantities
are determined to infinite order, within the singles and doubles space.
The asymmetric $\Lambda\text{CCSD}(\text{T})$ correction is then
obtained as:
\begin{equation}\label{eq:5-lambda-parens-t}
 \begin{aligned}
  G_{\Lambda(\text{T})} =
  {}^*G_{\Lambda[\text{T}]}^{[4]} +
  {}^*G_{\Lambda(\text{T})}^{[5]} =
  \sum_{j=1}^2
    \braket{\tbar{j}^{*} | \commutator{\Phi}{T_3^{*}} | \mathrm{HF}}
  + {}^t\p^{*}\braket{\tbar{2}^{*} | \commutator{\s}{T_3^{*}} | \mathrm{HF}}
 \end{aligned}
\end{equation}
Replacing the perturbation multipliers by converged \acs{CCSD}
amplitudes leads to the symmetric CCSD(T) model:
\begin{equation}\label{eq:5-parens-t}
 \begin{aligned}
  G_{(\text{T})} =
  {}^*G_{[\text{T}]}^{[4]} +
  {}^*G_{(\text{T})}^{[5]} =
  \sum_{j=1}^2
    \braket{\tamp{j}^{*} | \commutator{\Phi}{T_3^{*}} | \mathrm{HF}}
  + {}^t\p^{*}\braket{\tamp{2}^{*} | \commutator{\s}{T_3^{*}} | \mathrm{HF}}
 \end{aligned}
\end{equation}

\subsection{MBPT Terms in the
\texorpdfstring{$\Lambda\text{CCSD}(\text{T})$}{CCSD(aT)} and CCSD(T) models}

Triples excitations enter the \acs{MBPT} equations for the singles and
doubles amplitudes and multipliers at third order. Thus the \acs{CCSD}
amplitudes and multipliers are correct through second order in the
fluctuation potential, while the polarization is correct through third
order. Moreover, amplitudes and multipliers are equal to one another.

We can rework the energy correction given above so that the perturbative
counterparts clearly appear.
First of all, we notice that $T_3^{*} = T_3^{[2]} + \tilde{T}_3$ contains higher than second order
contributions where $\tilde{T}_3$ contains third and higher order contributions to the
connected triples. In particular, the following holds:
\begin{equation}
  \denom{\mu_3}\tamp{\mu_3}^{*} =
  - \braket{\mu_3 |\commutator{\Phi}{T_2^{1}} | \mathrm{HF}}
  - \braket{\mu_3 | \commutator{\Phi}{T_2^[2]} | \mathrm{HF}}
  + O(3)
\end{equation}
such that:
\begin{equation}
  \tilde{T}_3
  = \sum_{\mu_3}\tilde{t}_{\mu_3}\cluster{\mu_3}
  \coloneq \sum_{\mu_3}
  \left(-\denom{\mu_3}^{-1}\braket{\mu_3 |
  \commutator{\Phi}{T_3^{[2]}}
  | \mathrm{HF}}\right)
  \cluster{\mu_3}
\end{equation}

Using Eqs.~\eqref{eq:tamp-star}--\eqref{eq:p-star} one can expand the
energy correction as follows:
\begin{equation}
  \begin{aligned}
  G_{\Lambda(\text{T})} &=
  \sum_{j=1}^2 \braket{\tbar{j}^{*} | \commutator{\Phi}{T_3^{*}} | \mathrm{HF}}
  + {}^t\p^{*}\braket{\tbar{2}^{*} | \commutator{\s}{T_3^{*}} | \mathrm{HF}} \\
  &= \left/ \text{Insert perturbative expansions of parameters} \right/ \\
  &=
  \sum_{j=1}^2 \braket{\tbar{j}^{[1]}+\tbar{j}^{[2]} |
  \commutator{\Phi}{T_3^{[2]} + \tilde{T}_3} | \mathrm{HF}} \\
  &+ ({}^t\p^{[1]} + {}^t\p^{[2]})
  \braket{\tbar{2}^{[1]} + \tbar{2}^{[2]} | \commutator{\s}{T_3^{[2]} + \tilde{T}_3} | \mathrm{HF}} \\
  &= \left/ \text{Expand} \right/ \\
  &\highlight{green}{
  \sum_{j=1}^2 \braket{\tbar{j}^{[1]} | \commutator{\Phi}{T_3^{[2]}} | \mathrm{HF}}
  }
  + \highlight{blue}{
  \sum_{j=1}^2 \braket{\tbar{j}^{[1]} | \commutator{\Phi}{\tilde{T}_3} | \mathrm{HF}}
  }
  + \highlight{blue}{
  \sum_{j=1}^2 \braket{\tbar{j}^{[2]} | \commutator{\Phi}{T_3^{[2]}} | \mathrm{HF}}
  } \\
  &+
  \highlight{green}{
  {}^t\p^{[1]} \braket{\tbar{2}^{[1]} | \commutator{\s}{T_3^{[2]}} | \mathrm{HF}}
  }
  + \highlight{blue}{
  {}^t\p^{[1]} \braket{\tbar{2}^{[2]} | \commutator{\s}{T_3^{[2]}} | \mathrm{HF}}
  } \\
  &+ \highlight{blue}{
  {}^t\p^{[1]} \braket{\tbar{2}^{[1]} | \commutator{\s}{\tilde{T}_3} | \mathrm{HF}}
  }
  + \highlight{blue}{
  {}^t\p^{[2]} \braket{\tbar{2}^{[1]} | \commutator{\s}{T_3^{[2]}} | \mathrm{HF}}
  }
  + O(6) \\
  &= \left/ \text{Use definition of $\tilde{T}_3$ and triples multipliers equation} \right/ \\
  &\highlight{green}{
  \sum_{j=1}^2 \braket{\tbar{j}^{[1]} | \commutator{\Phi}{T_3^{[2]}} | \mathrm{HF}}
  }
  + \highlight{blue}{
  \braket{\tbar{3}^{[2]} | \commutator{\Phi}{T_2^{[2]}} | \mathrm{HF}}
  }
  + \highlight{blue}{
  \sum_{j=1}^2 \braket{\tbar{j}^{[2]} | \commutator{\Phi}{T_3^{[2]}} | \mathrm{HF}}
  } \\
  &+
  \highlight{green}{
  {}^t\p^{[1]} \braket{\tbar{2}^{[1]} | \commutator{\s}{T_3^{[2]}} | \mathrm{HF}}
  }
  + \highlight{blue}{
  {}^t\p^{[1]} \braket{\tbar{2}^{[2]} | \commutator{\s}{T_3^{[2]}} | \mathrm{HF}}
  } \\
  &+ \highlight{blue}{
  {}^t\p^{[1]} \braket{\tbar{2}^{[1]} | \commutator{\s}{\tilde{T}_3} | \mathrm{HF}}
  }
  + \highlight{blue}{
  {}^t\p^{[2]} \braket{\tbar{2}^{[1]} | \commutator{\s}{T_3^{[2]}} | \mathrm{HF}}
  }
  + O(6) \\
  \end{aligned}
\end{equation}

\todo[inline]{Don't really know what to do with the
${}^t\p^{[1]} \braket{\tbar{2}^{[1]} | \commutator{\s}{\tilde{T}_3} | \mathrm{HF}}$ term.
I think it is correctly identified as $O(5)$, but that it's among the
discarded terms.}
