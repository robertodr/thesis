\documentclass[notitlepage,a4paper,11pt,dvipsnames]{article}
\usepackage{amsmath, amssymb}
\usepackage[inline]{showlabels}
\usepackage[utf8]{inputenc}
\usepackage[T1]{fontenc}
\usepackage[english]{babel}
\usepackage[dvips]{graphicx}
\usepackage[style=phys,
maxcitenames=3,
maxbibnames=5,
minbibnames=3,
firstinits=true,
url=false,
isbn=false,
eprint=false,
texencoding=utf8,
bibencoding=utf8,
autocite=superscript,
backend=biber
]{biblatex}
\usepackage{booktabs}
\usepackage{caption}
\usepackage{subcaption}
\captionsetup{format=hang, font=footnotesize,labelfont=bf,labelsep=space}
\captionsetup[subfigure]{format=hang, font=footnotesize,labelfont=bf,labelsep=space}
\usepackage{colortbl}
\usepackage{tcolorbox}
\usepackage[version=4]{mhchem}

\usepackage[nohyperlinks,nolist]{acronym}
\usepackage{csquotes}
\usepackage{braket}
\usepackage{bm}
\usepackage{mathtools}
\usepackage{hyperref}
\hypersetup{colorlinks,
            breaklinks,
            urlcolor=blue,
            linkcolor=blue,
            citecolor=blue} % Set link colors

\newcommand*{\paper}[1]{\textbf{Paper #1}}

\title{Summary of Contributions to the Scientific Articles Included in the Doctoral Dissertation \\ {\itshape The Polarizable Continuum Model Goes Viral!}}

\author{Roberto Di Remigio}
\date{\today}

\begin{document}
\maketitle

The doctoral dissertation {\itshape The Polarizable Continuum Model Goes
Viral!} contains 5 Chapters, 1 Appendix and 5 scientific papers. Of these, 3
have been published, 1 has been accepted for publication and 1 has been
submitted for publication.

My contributions to the scientific papers are summarized in the following.

\section*{\paper{I}}

\begin{tcolorbox}
  {\small
  \textsf{Four-Component Relativistic Calculations in Solution with the
  Polarizable Continuum Model of Solvation: Theory,
  Implementation, and Application to the Group 16 Dihydrides
  \ce{H2X} (\ce{X} = \ce{O}, \ce{S}, \ce{Se}, \ce{Te},
  \ce{Po})
  }
  \\
  \textbf{R. Di Remigio}, R. Bast, L. Frediani, and T. Saue
  \\
  \textit{J. Phys. Chem. A}, \textrm{2015}, \textbf{119}, 5061--5077
  \\
  DOI: \url{10.1063/1.4943782}
  }
\end{tcolorbox}

I contributed the theoretical derivation of the quantum/classical polarizable
terms in a four-component self-consistent field framework, for energies and linear response
properties. I devised the coupling of the four-component program DIRAC with
PCMSolver by providing the implementation and testing of:
\begin{enumerate}
  \item molecular electrostatic potential integrals for four-component wave functions,
  \item the additional Fock matrix contributions, and
  \item the additional terms in the response equations.
\end{enumerate}
I performed all the calculations and large part of the data analysis
for the results reported in the paper.
Finally, I wrote the first draft of the paper and coordinated editing of
all subsequent versions.
My coauthors have all read and approved the description of contributions.

\section*{\paper{II}}

\begin{tcolorbox}
  {\small
  \textsf{Wavelet Formulation of the Polarizable Continuum Model. II. Use of
  Piecewise Bilinear Boundary Elements
  }
  \\
  M. Bugeanu, \textbf{R. Di Remigio}, K. Mozgawa, S. S. Reine, H.
  Harbrecht,  and L. Frediani
  \\
  \textit{Phys. Chem. Chem. Phys.}, \textrm{2015}, \textbf{17},
  31566--31581
  \\
  DOI: \url{10.1039/C5CP03410H}
  }
\end{tcolorbox}

I provided template interface and test sets for the cavity
generator and wavelet Galerkin
boundary element method solver with PCMSolver.
These were used to interface with the new C++ implementation of the wavelet
Galerkin solvers of Monica Bugeanu.
I implemented the interface between the LSDALTON quantum chemistry software
package and the PCMSolver software library. The interface allows to run Hartree--Fock (HF) and
Kohn--Shame density-functional theory (KS-DFT)
single-point and linear response calculations.
Together with coauthor Krzysztof Mozgawa, I performed the benchmark quantum
chemical calculations presented in the paper.
Finally, I coordinated the editing of all manuscript drafts.
In particular, I wrote the first draft of Sections 2.1 and 2.3.
The first draft of Sections 3 and 4 was co-written with the first author, Monica Bugeanu.
I performed most of the data analysis and produced tables and graphs.

Finally, the interface to LSDALTON was later released in the 2016 version,
providing polarizable continuum model (PCM) capabilities to the software package.
My coauthors have all read and approved the description of contributions.

\section*{\paper{III}}

\begin{tcolorbox}
  {\small
  \textsf{A Polarizable Continuum Model for Molecules at Spherical
  Diffuse Interfaces
  }
  \\
  \textbf{R. Di Remigio}, K. Mozgawa, H. Cao, V. Weijo, and L.
  Frediani
  \\
  \textit{J. Chem. Phys.}, \textrm{2016}, \textbf{144}, 124103
  \\
  DOI: \url{10.1063/1.4943782}
  }
\end{tcolorbox}

I contributed the theoretical work for \paper{III}, based on earlier drafts from coauthors
Ville Weijo and Hui Cao. In particular, I derived the separation of
the Coulomb singularity in its final form.
Moreover, I contributed the implementation and testing of the Green's function code.
The interface to the LSDALTON program package, developed within
\paper{II}, was
also used for this paper.
I wrote the first draft of the paper and coordinated all subsequent editing stages.
My coauthors have all read and approved the description of contributions.

\section*{\paper{IV}}

\begin{tcolorbox}
  {\small
  \textsf{Four-Component Relativistic Density Functional Theory with the
  Polarizable Continuum Model: Application to EPR Parameters
  and Paramagnetic NMR Shifts
  }
  \\
  \textbf{R. Di Remigio}, M. Repisky, S. Komorovsky, P. Hrobarik, L.
  Frediani, and K. Ruud
  \\
  Accepted for publication in \textit{Mol. Phys.}
  \\
  DOI: \url{10.1080/00268976.2016.1239846}
  }
\end{tcolorbox}

My contributions to this paper include prototyping the interface between the
PCMSolver library and the ReSpect quantum chemistry code.
The interface is maintained in collaboration with coauthor Michal Repisky, who also
refined the implementation to achieve better computational performance.
I tested the interface against one-component and four-component results obtained with
the LSDALTON and DIRAC codes, respectively.
I helped coauthors Michal Repisky, Stanislav Komorovsky and Peter Hrobarik
with setting up the PCM calculations described in the paper.
Finally, I provided the first draft for Section 2 of the paper and took part in all editing stages.
The interface to ReSpect will be released in the next public version of the software
package, providing PCM and COSMO capabilities.
My coauthors have all read and approved the description of contributions.

\section*{\paper{IV}}

\begin{tcolorbox}
  {\small
  \textsf{Open-Ended Formulation of Self-Consistent Field Response Theory with
  the Polarizable Continuum Model for Solvation
  }
  \\
  \textbf{R. Di Remigio}, M. T. P. Beerepoot, Y. Cornaton, M. Ringholm,
  A. H. S. Steindal, K. Ruud, and L. Frediani
  \\
  Submitted to \textit{Phys. Chem. Chem. Phys.}
  }
\end{tcolorbox}
I developed the theoretical framework for the open-ended SCF formulation
of molecular response properties when a quantum/classical polarizable continuum
Hamiltonian is used.
I provided its implementation within the DALTON code, by interfacing the
PCMSolver library and the open-ended SCF response code of Ringholm \emph{et al}.
I performed extensive testing of the code by comparing with previously
published implementations of the PCM-SCF response functions within
DALTON.
Together with coauthors Maarten T. P. Beerepoot and Yann Cornaton, I carried out
the multiphoton absorption calculations presented in the paper. I contributed
to data collection and data analysis.
I drafted the initial versions of Sections 2 and 3 of the manuscript and coordinated all
editing stages with coauthor Maarten T. P. Beerepoot.
My coauthors have all read and approved the description of contributions.

\begin{flushright}
    \begin{tabular}{m{5cm}}
      \textit{Troms{\o}, \today} \\
        {\small \textbf{Candidate}} \\
      \includegraphics[width=0.35\textwidth]{gfx/signature.png} \\
        \hline
        {\small Roberto \textsc{Di Remigio}} \\
    \end{tabular}
    \\
    \begin{tabular}{m{5cm}}
        {\small \textbf{Advisor} }
        \\
      \includegraphics[width=0.3\textwidth]{gfx/LucaSignature.png} \\
        \hline
        {\small Luca \textsc{Frediani}} \\
    \end{tabular}
\end{flushright}

\end{document}
