%*******************************************************
% Publications
%*******************************************************
\thispagestyle{empty}
\refstepcounter{dummy}
\pdfbookmark[1]{Publications}{Publications}
\chapter*{Publications}

This thesis is based on the following scientific papers:

\begin{enumerate}[label=\textbf{\Roman{*} }, ref=\Roman{*}]

\item
  \textbf{R. Di Remigio}, R. Bast, L. Frediani, and T. Saue
  \\
  \textsc{
  Four-Component Relativistic Calculations in Solution with the
  Polarizable Continuum Model of Solvation: Theory,
  Implementation, and Application to the Group 16 Dihydrides
  \ce{H2X} (\ce{X} = \ce{O}, \ce{S}, \ce{Se}, \ce{Te},
  \ce{Po})
  }
  \\
\textit{J. Phys. Chem. A}, \textrm{2015}, \textbf{119}, 5061--5077
\label{relapcm}

\item
  M. Bugeanu, \textbf{R. Di Remigio}, K. Mozgawa, S. S. Reine, H.
  Harbrecht,  and L. Frediani
  \\
  \textsc{
  Wavelet Formulation of the Polarizable Continuum Model. II. Use of
  Piecewise Bilinear Boundary Elements
  }
  \\
  \textit{Phys. Chem. Chem. Phys.}, \textrm{2015}, \textbf{17},
  31566--31581
\label{wemlin}

\item
  \textbf{R. Di Remigio}, K. Mozgawa, H. Cao, V. Weijo, and L.
  Frediani
  \\
  \textsc{
  A Polarizable Continuum Model for Molecules at Spherical
  Diffuse Interfaces
  }
  \\
  \textit{J. Chem. Phys.}, \textrm{2016}, \textbf{144}, 124103
\label{spherical}

\item
  \textbf{R. Di Remigio}, M. Repisky, S. Komorovsky, P. Hrobarik, L.
  Frediani, and K. Ruud
  \\
  \textsc{
  Four-Component Relativistic Density Functional Theory with the
  Polarizable Continuum Model: Application to EPR Parameters
  and Paramagnetic NMR Shifts
  }
  \\
  Accepted for publication in \textit{Mol. Phys.}
\label{pcmepr}

\item
  \textbf{R. Di Remigio}, M. T. P. Beerepoot, Y. Cornaton, M. Ringholm,
  A. H. S. Steindal, K. Ruud, and L. Frediani
  \\
  \textsc{
  Open-ended formulation of self-consistent field response theory with
  the polarizable continuum model for solvation
  }
  \\
  In preparation for submission to \textit{Phys. Chem. Chem. Phys.}
\label{pcmopenrsp}
\end{enumerate}
