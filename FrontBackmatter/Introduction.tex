\chapter*{Introduction}
\addtocontents{toc}{\protect\vspace{\beforebibskip}} % to have the bib a bit from the rest in the toc
\addcontentsline{toc}{chapter}{\tocEntry{Introduction}}

\renewcommand{\thefigure}{\Alph{figure}}

\begin{epigraphs}
\qitem{
pity this busy monster, manunkind,
    }{
    --- \textsc{E. E. Cummings}}
\qitem{
[...] \textnorsk{og fordi jeg alltid har hatt en dragning mot det skjulte og hemmelige.}
}{
--- \textnorsk{\textsc{K. O. Knausgård}, \textit{Om Høsten}}}
\end{epigraphs}


\section*{What Are the Electrons Really Doing in Molecules?}

\todo[inline]{Quantum chemistry as they study of
molecular properties by means of \emph{ab initio} physico-mathematical
theories and computer programs.
Introduce the concept of model chemistries, but then mention that we are
interested in different aspects and will look at different "axes".}

This question was posed by R.~S.~Mulliken over a half-century ago, as
lecture title for the acceptance of the Gilbert N.~Lewis award in 1960.

\begin{figure}[tb]
\centering
\scalebox{0.7}{%Angle Definitions
%-----------------

%set the plot display orientation
%synatax: \tdplotsetdisplay{\theta_d}{\phi_d}
\tdplotsetmaincoords{60}{110}

%define polar coordinates for some vector
%TODO: look into using 3d spherical coordinate system
\pgfmathsetmacro{\rvec}{.8}
\pgfmathsetmacro{\thetavec}{30}
\pgfmathsetmacro{\phivec}{60}

%start tikz picture, and use the tdplot_main_coords style to implement the display 
%coordinate transformation provided by 3dplot
\begin{tikzpicture}[scale=5,tdplot_main_coords]

%set up some coordinates 
%-----------------------
\coordinate (O) at (0,0,0);

%determine a coordinate (P) using (r,\theta,\phi) coordinates.  This command
%also determines (Pxy), (Pxz), and (Pyz): the xy-, xz-, and yz-projections
%of the point (P).
%syntax: \tdplotsetcoord{Coordinate name without parentheses}{r}{\theta}{\phi}
\tdplotsetcoord{P}{\rvec}{\thetavec}{\phivec}

%draw figure contents
%--------------------
% HAMILTONIAN AXIS
\draw[thick,->] (0,0,0) -- (1.5,0,0) node[anchor=north east]{Hamiltonian $(\begingroup\color{brewerRed}{y}\endgroup)$};
\draw (1/3, .01pt, 0pt) -- (1/3, -.04pt, 0pt) node[anchor=east] {\begingroup\color{brewerRed}{one-component}\endgroup};
\draw (2/3, .01pt, 0pt) -- (2/3, -.05pt, 0pt) node[anchor=east] {\begingroup\color{brewerRed}{two-component}\endgroup};
\draw (1, .01pt, 0pt) -- (1, -.05pt, 0pt) node[anchor=east] {\begingroup\color{brewerRed}{four-component}\endgroup};

% BASIS AXIS
\draw[thick,->] (0,0,0) -- (0,1.5,0) node[anchor=north west]{Basis $(\begingroup\color{brewerYellow}{M}\endgroup)$};
\draw (0pt, 1/3, -.01pt) -- (0pt, 1/3, +.05pt) node[anchor=south] {\begingroup\color{brewerYellow}{double zeta}\endgroup};
\draw (0pt, 2/3, .01pt) -- (0pt, 2/3, -.05pt) node[anchor=north] {\begingroup\color{brewerYellow}{triple zeta}\endgroup};
\draw (0pt, 1, -.01pt) -- (0pt, 1, +.05pt) node[anchor=south] {\begingroup\color{brewerYellow}{quadruple zeta}\endgroup};

% METHOD AXIS
\draw[thick,->] (0,0,0) -- (0,0,1.5) node[anchor=south]{Method $(\begingroup\color{brewerBlue}{x}\endgroup)$};
\draw (0pt, .01pt, 1/3) -- (0pt, -.05pt, 1/3) node[anchor=east] {\begingroup\color{brewerBlue}{Hartree--Fock}\endgroup};
\draw (0pt, .01pt, 2/3) -- (0pt, -.05pt, 2/3) node[anchor=east] {\begingroup\color{brewerBlue}{MP2}\endgroup};
\draw (0pt, .01pt, 1) -- (0pt, -.05pt, 1) node[anchor=east] {\begingroup\color{brewerBlue}{CCSD(T)}\endgroup};

\end{tikzpicture}
}
\caption[Pictorial depiction of the concept of \emph{model chemistries}.]{
Pictorial depiction of the concept of \emph{model chemistries} as the
three dimensions of quantum chemistry.~\autocite{Pople1999-gt, Saue2011-qg}
The computational scaling of a model chemistry is
$S=\textcolor{PMS1797}{y}(\textcolor{PMS2229}{x})f(\textcolor{PMS2229}{x})\textcolor{PMS130}{M}^{\textcolor{PMS2229}{x}}$.
}
\label{fig:RQC-axis}
\end{figure}

This will be the subject of Chapter \ref{ch:QM}

\section*{The Problem of Solvation or Taming Complexity with Models}

Chemistry can be largely considered a \emph{wet} science: almost always
chemical phenomena happen in a liquid environment.~\autocite{Reichardt2010-le}
We hereby define a "solution", or more generally an "environment", as
a system were the number of solvent molecules exceeds by far the number
of solute molecules.~\autocite{Tomasi2004-dc, Tomasi2007-es}
It is then clear that theoretical and computational approaches to such a
problem will necessarily suffer from a \emph{dimensionality disease}.
The number of degrees of freedom to be taken into account is, in
principle, so large, that even the most advanced computing machines
would have a hard time computing the desired observables.
Moreover, on an interpretive level, it would not even be desirable to
have such a detailed insight.
As is well known from statistical mechanics, microscopic detail cannot
account for the macroscopic behaviour.~\autocite{Hill1960-ql,
Hansen2013-io}
To tame this complexity and cure the disease, one must devise
\emph{models} that simplify physical reality, while offering tools for
understanding reality and predicting new and exciting
phenomena.~\autocite{Anderson1972-ai, Winsberg2010-sy, Kovac2011-ew}
One of the earlier attempts at tackling the problem of solvation is due
to \citeauthor{Onsager1936-wf}. His was a rather crude model, but one
that has had a lasting impact and informs much of the developments that
will be presented in this thesis.~\autocite{Onsager1936-wf}

Before introducing our model of choice, let us consider how an
environment might affect molecular observables of interest.
Environment effects are usually classified as:
\begin{itemize}
\item \emph{direct}.
  These effect stem straightforwardly from the modification underwent by
  the solute electronic density when interacting with the environment.
\item \emph{Indirect}.
  It is common for solutes to exhibit different minimum-energy
  conformations in different environments. These effects are commonly
  labelled as indirect.
\item \emph{Local field}.
  Light-matter interactions are also affected by the environment. Local
  modifications of externally applied fields subtly influence molecular
  responses.~\autocite{Cammi1998-jp, Pipolo2014-sd}
\item \emph{Dynamic}.
  Probing the nature of molecular excited states has enormous
  technological impact. The presence of the environment radically
  influences excited states, since relaxation processes in the medium
  become important.
\item \emph{Specific}. This catch-all category includes all effects
  stemming from the peculiar solute-solvent pair interactions that
  cannot be fully described under any of the previous labels.
  In general, modelling such effects demands an atomistic level of
  detail.
\end{itemize}

Faced with the problem of describing such a diverse array of effects,
two main models have emerged in the past decades, each with its
strengths and weaknesses.
Both can be classified as \emph{multiscale} (or \emph{focused})
models~\autocite{Nobel2013} and hinge on the same idea: treat different
parts of the system with different methods and couple these methods by
bridging "scales" at the boundary, \emph{vide infra}.
While both models treat the molecular degrees of freedom at the quantum
mechanical level, their approach to the microscopic description of the
degrees of freedom of the environment differs:
\begin{itemize}
 \item
   \emph{Discrete} (or \emph{explicit}) models explicitly treat those
   degrees of freedom.
   This is either achieved by a cheaper quantum mechanical
   method~\autocite{Vreven2006-gx} or by \gls{MM}.~\autocite{Senn2009-sk}
   In the latter approach, commonly dubbed \acrshort{QM}/\acrshort{MM}, the \acrshort{MM}
   region can either be polarizable~\autocite{Mennucci2013-go,
   Olsen2010-wa, Lipparini2011-rd} or non
   polarizable. While the former method allows for mutual polarization
   between the \acrshort*{QM} and \acrshort*{MM} subsystems, the latter
   treats the \acrshort*{MM} region as fixed.
 \item
   \emph{Continuum} (or \emph{implicit}) completely remove the degrees
   of freedom of the environment from the picture, replacing it with a
   structureless continuum.
   The effect of this continuum is described, classically, \emph{via}
   its bulk properties.~\autocite{Onsager1936-wf, Miertus1981-mm}
\end{itemize}
\acrshort{QM}/\acrshort{MM} models can capture, albeit approximately, the effect
of the atomistic nature of the environment on the active part of the
system.
However, they demand a statistical average of environment configurations
to yield results of any significance. Moreover, a large number
\emph{cutoff} for the \acrshort{MM} region is usually required to converge
long-range electrostatic interactions.\autocite{Steindal2011-ki}
Continuum models avoid both problems at once. Statistical averaging is
built into the model \emph{via} their parametrization by
means of the environment's bulk properties, such as the permittivity.
In addition, long-range electrostatics is treated exactly.
Unfortunately, atomistic detail is lost and it is then impossible to
recover a satisfactory description of specific effects.
To partly alleviate these sources of error, the
\acrshort{QM}/\acrshort{MM} and \acrshort{QM}/Continuum methods can and
have been successfully combined to yield the three-layer
\acrshort{QM}/\acrshort{MM}/Continuum method.\autocite{Steindal2011-ki,
Lipparini2011-rd, Caprasecca2012-ir, Lipparini2013-ud}
Figure \ref{fig:qm-to-multiscale} schematically portrays the transition
from a full \acrshort{QM} model of the relevant system to its multiscale
representations.

\begin{figure}[tb]
\missingfigure{Add picture with multiscale models}
\caption[From quantum mechanical to multiscale models]{
From quantum mechanical to multiscale models.
The images used in the scheme are reproduced courtesy of Dr.~Stefano
Caprasecca (MoLEcoLab, Università di Pisa).}
\label{fig:qm-to-multiscale}
\end{figure}

Notice that we have deliberately ruled out so-called \emph{cluster}
models from the above discussion.
These approaches replace the actual physical setting with a suitable
truncation of the whole solute+solvent system, the \emph{model system}
and treat it within a chosen quantum mechanical level of theory.
Cluster models can be used to benchmark more approximate multiscale
models, but their description is outside the scope of this thesis.

Chapter \ref{ch:CSM} will present an overview of the \gls{PCM} for
solvation.
I will present a nontechnical discussion of the mathematical details of
the model and an outline of current methodologies for the solution of
the associated governing equations.
Borrowing from the work of \citeauthor{Lipparini2010-be},\autocite{Lipparini2010-be,
Lipparini2015-lq} I will introduce a unifying theoretical formalism for
\acrshort{QM}/Continuum, \acrshort{QM}/\acrshort{MM} and \acrshort{QM}/\acrshort{MM}/Continuum models that will
be extensively used throughout the thesis.

\section*{The Road to Reality or Molecular Response Properties}

Characterizing, explaining and predicting the effect of the chemical
environment on a large number of measurable properties requires,
however, a synergistic experimental and theoretical approach.

This will be the subject of Chapter \ref{ch:molprop}

\section*{Accurate Methods for Accurate Properties}

This will be the subject of Chapter \ref{ch:solvation-correlation}
