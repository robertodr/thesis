\chapter*{Notations, conventions and units}
\addcontentsline{toc}{section}{\tocEntry{Notations, conventions and units}}

%A number of notations and typographic conventions has been adopted in order to maintain consistency throughout. We summarize them here:
%\begin{table*}[!h]
%\begin{tabular}{c l}
% $\mathcal{A}, \mathcal{D}, \ldots$ & integral operators over a closed subset of $\mathbb{R}^2$ \\
% $H, \varPhi, \ldots$ & first or second quantized $N$-electron operators \\
% $\vect{r}, \vect{r}^\prime, \ldots$ & position vectors in $\mathbb{R}^3$ \\
% $\vect{s}, \vect{s}^\prime, \ldots$ & position vectors in a closed subset of $\mathbb{R}^2$ \\
% $\vect{v}, \vect{q}, \ldots$ & vectors in an arbitrary vector space \\
% $\mat{K}, \mat{F}, \ldots$ & matrices in an arbitrary vector space \\
% $0_N, I_N$ & the zero and the identity in an $N$-dimensional vector space
%\end{tabular}
%\end{table*}
%
%Summation over repeated indices, both discrete and continuous, is always to be understood:
%\begin{alignat*}{2}
% f_r = C_{\kappa r}g_{\kappa} \equiv \sum_{\kappa}C_{\kappa r}g_{\kappa}, \quad&\quad
% f(\vect{r}) = U(\vect{r},\vect{r}^\prime)g(\vect{r}^\prime) \equiv \int\diff\vect{r}^\prime U(\vect{r},\vect{r}^\prime)g(\vect{r}^\prime).
%\end{alignat*}
%
%Further conventions have been adopted for the indices of functions. Lower case Latin letters are used for MO 4-spinors, while lower case
%Greek letters ($\kappa, \lambda, \mu, \ldots$) are reserved for one-electron basis functions in 2-spinor or scalar form.
%Specific ranges of letters are used as follows:
%\begin{table*}[!h]
%\begin{tabular}{c l}
% $r, s, t, \ldots$ & general MO 4-spinor indices \\
% $i, j, k, \ldots$ & occupied MO 4-spinor indices \\
% $a, b, c, \ldots$ & virtual MO 4-spinor indices
%\end{tabular}
%\end{table*}
%
%Upper case Latin letters always refer to functions in vector spaces other than the two listed above. Complex conjugation will always be shown using
%a dagger ($\dagger$) instead of a star ($*$).
%
%SI-based atomic units have been used \cite{Whiffen1978}:
%\begin{equation*}
% \si{\electronmass} = \si{\elementarycharge} = \hslash = \frac{1}{4\pi\diel_0} = 1
%\end{equation*}
%the unit of length is the Bohr \si{\bohr}, while that of energy is the Hartree \si{\hartree}.
%The speed of light is then:
%\begin{equation*}
% c = \SI{137.035999074}{\bohr\hartree\per\planckbar}
%\end{equation*}
%
%
%Some useful conversion constants to SI units are here listed \cite{CODATA2010}:
%\begin{subequations}
% \begin{align*}
%  \SI{1}{\bohr} &= \SI{0.52917721092}{\angstrom} = \SI{5.2917721092e-11}{\meter} \\
%  \SI{1}{\hartree} &= \SI{27.21138505}{\electronvolt} = \SI{2625.4996404}{\kilo\joule\per\mole} \\
%  \SI{1}{\debye} &= \SI{3.3356410E-30}{\coulomb\meter} \\
%  \SI{1}{\angstrom\cubed} &= \SI{1.11264984E-40}{\coulomb\squared\meter\squared\per\joule}
% \end{align*}
%\end{subequations}
