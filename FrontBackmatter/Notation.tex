\renewcommand\nomgroup[1]{%
  \item[\bfseries
  \ifstrequal{#1}{A}{Basic Notation}{%
  \ifstrequal{#1}{B}{Vectors and Matrices}{%
  \ifstrequal{#1}{C}{Operators}{}}{
  \ifstrequal{#1}{D}{Function Spaces}{}}{
  \ifstrequal{#1}{E}{Molecular Quantum Mechanics}{}}}%
]}

\renewcommand{\nomname}{Notations, Conventions and Units}

\phantomsection
\refstepcounter{dummy}
\addcontentsline{toc}{chapter}{\tocEntry{\nomname}}

\renewcommand{\nompreamble}{
A number of notations and typographic conventions has been adopted in
order to maintain consistency throughout. We summarize them here.
Hartree atomic units are used throughout:\autocite{Whiffen1978-xx, Mohr2012-zc}
\begin{equation*}
  \si{\electronmass} = \si{\elementarycharge} = \hslash = 4\pi\diel_0 = 1
\end{equation*}
the unit of length is the Bohr \si{\bohr}, while that of energy is the Hartree \si{\hartree}.
The speed of light is then:
\begin{equation*}
  c = \SI{137.035999074}{\bohr\hartree\per\planckbar}
\end{equation*}
Complex conjugation will always be shown using a dagger ($\dagger$)
instead of a star ($*$).
}

% Basic Notation
\nomenclature[A, 01]{$u_E$}{Restriction (trace) of the function $u$ to the set $E$}
\nomenclature[A, 02]{$\Span{\vect{v}_1, \ldots, \vect{v}_n}$}{Vector space spanned by the vectors $\vect{v}_1, \ldots, \vect{v}_n$}
\nomenclature[A, 03]{$\delta_{ij}$}{Kronecker symbol: $\delta_{ij}=1i$ if $i=j$ and 0 otherwise}
\nomenclature[A, 04]{$\scalprod[X]{u}{v}$}{Scalar product of $u$ and $v$ in the Hilbert space $X$}
\nomenclature[A, 05]{$\norm{u}{X}$}{Norm of $u$ in the normed space $X$}
\nomenclature[A, 06]{$O(N)$}{A quantity of order $N$ or higher}
% Vectors and Matrices
\nomenclature[B, 01]{$\vect{v}, \vect{q}$}{Vectors}
\nomenclature[B, 02]{$\mat{K}, \mat{F}$}{Matrices}
\nomenclature[B, 03]{$\mathbb{R}^{m, n}$}{Vector space of real-valued $m\times n$ matrices}
\nomenclature[B, 04]{$\Tr\mat{A}$}{Trace of $\mat{A}$. For $\mat{A}\in\mathbb{R}^{n, n}$, $\Tr\mat{A}=\sum_{i=1}^n A_{ii}$}
\nomenclature[B, 05]{$\eqtr$}{The expression following is to be interpreted as a trace. $\mathcal{E} \eqtr \mat{h}\mat{D} = \Tr\mat{h}\mat{D}$}
\nomenclature[B, 06]{$\eqavetr$}{The expression following is to be interpreted as a trace followed by time-averaging over a period $T$.
$\mathcal{E} \eqavetr \mat{h}\mat{D} = \frac{1}{T}\int_{0}^{T}\diff t\Tr\mat{h}\mat{D}$}
\nomenclature[B, 07]{$0_N, I_N$}{The zero and the identity in an $N$-dimensional vector space}
% Operators
\nomenclature[C, 01]{$\bi{A}, \bi{D}$}{Integral operators}
\nomenclature[C, 02]{$H, \varPhi$}{First or second quantized $N$-electron operators}
% Function Spaces
\nomenclature[D, 02]{$\mathcal{C}^0(\Omega)$}{Space of continuous functions on $\Omega\subset\mathbb{R}^d$}
\nomenclature[D, 03]{$\mathcal{C}^k(\Omega)$}{Space of $k$ times continuously differentiable functions on $\Omega$}
\nomenclature[D, 04]{$L^p(\Omega)$}{Space of functions whose $p$-th power is Lebesgue integrable on $\Omega$}
\nomenclature[D, 05]{$\norm{u}{0, p, \Omega}$}{Norm in $L^p(\Omega)$: $\norm{u}{0, p, \Omega} = (\int_\Omega\diff\vect{r} |u(\vect{r})|^p)^\frac{1}{p}$}
\nomenclature[D, 06]{$\norm{u}{\Omega}$}{Norm in $L^2(\Omega)$: $\norm{u}{\Omega} = \norm{u}{0, 2,\Omega} = \sqrt{\int_\Omega\diff\vect{r} |u(\vect{r})|^2}$}
\nomenclature[D, 07]{$H^s(\Omega)$}{Sobolev space of order $s$ of functions whose derivative up to order $s$ are in $L^2(\Omega)$}
\nomenclature[D, 08]{$H_0^s(\Omega)$}{Space of functions in $H^2(\Omega)$ with null trace on the boundary of $\Omega$}
\nomenclature[D, 09]{$|u|_{s, \Omega}$}{Seminorm in $H^s(\Omega)$: $|u|_{s, \Omega} = \sum_{|\alpha|=s}\norm{\partial^\alpha u}{\Omega}$}
\nomenclature[D, 10]{$\norm{u}{s, \Omega}$}{Norm in $H^s(\Omega)$: $\norm{u}{s, \Omega} = \sum_{l \leq s} |u|_{l, \Omega}$}
% Molecular Quantum Mechanics
\nomenclature[E, 01]{$r, s, t$}{General \acrlong*{MO} indices}
\nomenclature[E, 02]{$i, j, k$}{Occupied \acrlong*{MO} indices}
\nomenclature[E, 03]{$a, b, c$}{Virtual \acrlong*{MO} indices}
\nomenclature[E, 04]{$\kappa, \lambda, \mu$}{One-electron basis functions indices}

\printnomenclature
