\chapter*{Notations, conventions and units}
\addcontentsline{toc}{section}{\tocEntry{Notations, conventions and units}}

A number of notations and typographic conventions has been adopted in
order to maintain consistency throughout. We summarize them here:
\begin{table*}[!h]
\begin{tabular}{c l}
 $\bi{A}, \bi{D}, \ldots$ & integral operators over a closed subset of $\mathbb{R}^2$ \\
 $H, \varPhi, \ldots$ & first or second quantized $N$-electron operators \\
 $\vect{r}, \vect{r}^\prime, \ldots$ & position vectors in $\mathbb{R}^3$ \\
 $\vect{s}, \vect{s}^\prime, \ldots$ & position vectors in a closed subset of $\mathbb{R}^2$ \\
 $\vect{v}, \vect{q}, \ldots$ & vectors in an arbitrary vector space \\
 $\mat{K}, \mat{F}, \ldots$ & matrices in an arbitrary vector space \\
 $0_N, I_N$ & the zero and the identity in an $N$-dimensional vector space
\end{tabular}
\end{table*}

\todo[inline]{Clarify the notation for the 4-component basis in
accordance with the EPR paper.}
Further conventions have been adopted for the indices of functions.
Lower case Latin letters are used for \ac{MO} 4-spinors, while lower case
Greek letters ($\kappa, \lambda, \mu, \ldots$) are reserved for
one-electron basis functions in 2-spinor or scalar form.
Specific ranges of letters are used as follows:
\begin{table*}[!h]
\begin{tabular}{c l}
 $r, s, t, \ldots$ & general MO 4-spinor indices \\
 $i, j, k, \ldots$ & occupied MO 4-spinor indices \\
 $a, b, c, \ldots$ & virtual MO 4-spinor indices
\end{tabular}
\end{table*}

Upper case Latin letters always refer to functions in vector spaces
other than the two listed above. Complex conjugation will always be
shown using a dagger ($\dagger$) instead of a star ($*$).

Hartree atomic units have been used:~\autocite{Whiffen1978-xx}
\begin{equation*}
 \si{\electronmass} = \si{\elementarycharge} = \hslash = 4\pi\diel_0 = 1
\end{equation*}
the unit of length is the Bohr \si{\bohr}, while that of energy is the Hartree \si{\hartree}.
The speed of light is then:
\begin{equation*}
 c = \SI{137.035999074}{\bohr\hartree\per\planckbar}
\end{equation*}

Some useful conversion constants to SI units are here
listed:~\autocite{Mohr2012-zc}
\begin{subequations}
 \begin{align*}
  \SI{1}{\bohr} &= \SI{0.52917721092}{\angstrom} = \SI{5.2917721092e-11}{\meter} \\
  \SI{1}{\hartree} &= \SI{27.21138505}{\electronvolt} = \SI{2625.4996404}{\kilo\joule\per\mole} \\
  \SI{1}{\debye} &= \SI{3.3356410E-30}{\coulomb\meter} \\
  \SI{1}{\angstrom\cubed} &= \SI{1.11264984E-40}{\coulomb\squared\meter\squared\per\joule}
 \end{align*}
\end{subequations}
