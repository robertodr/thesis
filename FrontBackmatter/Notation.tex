\chapter*{Notations, conventions and units}
\addcontentsline{toc}{section}{\tocEntry{Notations, conventions and units}}

A number of notations and typographic conventions has been adopted in
order to maintain consistency throughout. We summarize them here:

\section*{Basic Notation}
\begin{table*}[!h]
\begin{tabular}{l l}
  $u_E$ & Restriction (trace) of the function $u$ to the set $E$ \\
  $\Span{\vect{v}_1, \ldots, \vect{v}_n}$ & Vector space spanned by the
  vectors $\vect{v}_1, \ldots, \vect{v}_n$ \\
  $\delta_{ij}$ & Kronecker symbol: $\delta_{ij}=1i$ if $i=j$ and 0
  otherwise
\end{tabular}
\end{table*}

\section*{Vectors and Matrices}
\begin{tabular}{l l}
 $\vect{v}, \vect{q}$ & Vectors \\
 $\mat{K}, \mat{F}$ & Matrices \\
 $\mathbb{R}^{m, n}$ & Vector space of real-valued $m\times n$ matrices
 \\
 $\Tr\mat{A}$ & Trace of $\mat{A}$. For $\mat{A}\in\mathbb{R}^{n, n}$,
 $\Tr\mat{A}=\sum_{i=1}^n A_{ii}$ \\
 $0_N, I_N$ & the zero and the identity in an $N$-dimensional vector space
\end{tabular}

\section*{Operators}
\begin{tabular}{c l}
 $\bi{A}, \bi{D}$ & integral operators \\
 $H, \varPhi$ & first or second quantized $N$-electron operators
\end{tabular}

\section*{Function Spaces}
\begin{tabular}{c l}
 $\norm{u}{X}$ & Norm of $u$ in the normed space $X$ \\
 $\mathcal{C}^0(\Omega)$ & Space of continuous
 functions on $\Omega\subset\mathbb{R}^d$ \\
 $\mathcal{C}^k(\Omega)$ & Space of $k$ times
 continuously differentiable functions on $\Omega$ \\
 $L^p(\Omega)$ & Space of functions whose $p$-th power is Lebesgue
 integrable on $\Omega$ \\
 $\norm{u}{0, p, \Omega}$ & Norm in $L^p(\Omega)$: $\norm{u}{0, p,
 \Omega} = (\int_\Omega\diff\vect{r} |u(\vect{r})|^p)^\frac{1}{p}$ \\
 $\norm{u}{\Omega}$ & Norm in $L^2(\Omega)$:
 $\norm{u}{\Omega} = \norm{u}{0, 2,\Omega} = \sqrt{\int_\Omega\diff\vect{r} |u(\vect{r})|^2}$ \\
 $H^s(\Omega)$ & Sobolev space of order $s$ of functions whose
 derivative up to order $s$ are in $L^2(\Omega)$ \\
 $H_0^s(\Omega)$ & Space of functions in $H^2(\Omega)$ with null trace on the boundary of $\Omega$ \\
 $|u|_{s, \Omega}$ & Seminorm in $H^s(\Omega)$:
 $|u|_{s, \Omega} \coloneq \sum_{|\alpha|=s}\norm{\partial^\alpha u}{\Omega}$ \\
 $\norm{u}{s, \Omega}$ & Norm in $H^s(\Omega)$:
 $\norm{u}{s, \Omega} \coloneq \sum_{l \leq s} |u|_{l, \Omega}$
\end{tabular}

\todo[inline]{Clarify the notation for the 4-component basis in
accordance with the EPR paper.}
Further conventions have been adopted for the indices of functions.
Lower case Latin letters are used for \ac{MO} 4-spinors, while lower case
Greek letters ($\kappa, \lambda, \mu, \ldots$) are reserved for
one-electron basis functions in 2-spinor or scalar form.
Specific ranges of letters are used as follows:
\begin{table*}[!h]
\begin{tabular}{c l}
 $r, s, t, \ldots$ & general MO 4-spinor indices \\
 $i, j, k, \ldots$ & occupied MO 4-spinor indices \\
 $a, b, c, \ldots$ & virtual MO 4-spinor indices
\end{tabular}
\end{table*}

Upper case Latin letters always refer to functions in vector spaces
other than the two listed above. Complex conjugation will always be
shown using a dagger ($\dagger$) instead of a star ($*$).

Hartree atomic units have been used:~\autocite{Whiffen1978-xx}
\begin{equation*}
 \si{\electronmass} = \si{\elementarycharge} = \hslash = 4\pi\diel_0 = 1
\end{equation*}
the unit of length is the Bohr \si{\bohr}, while that of energy is the Hartree \si{\hartree}.
The speed of light is then:
\begin{equation*}
 c = \SI{137.035999074}{\bohr\hartree\per\planckbar}
\end{equation*}

Some useful conversion constants to SI units are here
listed:~\autocite{Mohr2012-zc}
\begin{subequations}
 \begin{align*}
  \SI{1}{\bohr} &= \SI{0.52917721092}{\angstrom} = \SI{5.2917721092e-11}{\meter} \\
  \SI{1}{\hartree} &= \SI{27.21138505}{\electronvolt} = \SI{2625.4996404}{\kilo\joule\per\mole} \\
  \SI{1}{\debye} &= \SI{3.3356410E-30}{\coulomb\meter} \\
  \SI{1}{\angstrom\cubed} &= \SI{1.11264984E-40}{\coulomb\squared\meter\squared\per\joule}
 \end{align*}
\end{subequations}
