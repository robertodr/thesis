%*******************************************************
% Abstract
%*******************************************************
%\renewcommand{\abstractname}{Abstract}
\pdfbookmark[1]{Abstract}{Abstract}
\begingroup
\let\clearpage\relax
\let\cleardoublepage\relax
\let\cleardoublepage\relax

\chapter*{Abstract}

Synergistic theoretical and experimental approaches to challenging
chemical problems have become more and more widespread, due to the
availability of efficient and accurate \emph{ab initio} quantum chemical
models.
Limitations to such an approach do, however, still exist.
The vast majority of chemical phenomena happens in complex environments,
where the molecule of interest can interact with a large number of other
moieties, solvent molecules or residues in a protein.
These systems represent an ongoing challenge to our modelling
capabilities, especially when high accuracy is required for the
prediction of exotic and novel molecular properties.
How to achieve the insight needed to understand and predict the physics
and chemistry of such complex systems is still an open question.

I will present our efforts in answering this question based on the
development of the polarizable continuum model for solvation.
While the solute is described by a quantum mechanical method, the
surrounding environment is replaced by a structureless continuum
dielectric.
The mutual polarization of the solute-environment system is described by
classical electrostatics.
Despite its inherent simplifications, the model contains the basic
mathematical features of more refined explicit quantum/classical
polarizable models.
Leveraging this fundamental similarity, we show how the
inclusion of environment effects for relativistic and nonrelativistic
quantum mechanical Hamiltonians, arbitrary order response properties and
high-level electron correlation methods can be transparently derived and
implemented.
We implemented an application programming interface that can provide the
continuum solvation functionality to any quantum chemistry software.
As examples of the flexibility of our implementation approach, we
present results for the continuum modelling of non homogeneous environments and
how wavelet-based numerical methods greatly outperform the accuracy of
traditional methods usually employed in continuum solvation models.

\endgroup

\begin{flushright}
    \begin{tabular}{m{5cm}}
      \textit{\myLocation, \myTime} \\
      \includegraphics[width=0.4\textwidth]{gfx/signature.png} \\
        \hline
        \centering\myName \\
    \end{tabular}
\end{flushright}
