% ****************************************************************************************************
% classicthesis-config.tex
% formerly known as loadpackages.sty, classicthesis-ldpkg.sty, and classicthesis-preamble.sty
% Use it at the beginning of your ClassicThesis.tex, or as a LaTeX Preamble
% in your ClassicThesis.{tex,lyx} with % ****************************************************************************************************
% classicthesis-config.tex
% formerly known as loadpackages.sty, classicthesis-ldpkg.sty, and classicthesis-preamble.sty
% Use it at the beginning of your ClassicThesis.tex, or as a LaTeX Preamble
% in your ClassicThesis.{tex,lyx} with % ****************************************************************************************************
% classicthesis-config.tex
% formerly known as loadpackages.sty, classicthesis-ldpkg.sty, and classicthesis-preamble.sty
% Use it at the beginning of your ClassicThesis.tex, or as a LaTeX Preamble
% in your ClassicThesis.{tex,lyx} with % ****************************************************************************************************
% classicthesis-config.tex
% formerly known as loadpackages.sty, classicthesis-ldpkg.sty, and classicthesis-preamble.sty
% Use it at the beginning of your ClassicThesis.tex, or as a LaTeX Preamble
% in your ClassicThesis.{tex,lyx} with \input{classicthesis-config}
% ****************************************************************************************************
% If you like the classicthesis, then I would appreciate a postcard.
% My address can be found in the file ClassicThesis.pdf. A collection
% of the postcards I received so far is available online at
% http://postcards.miede.de
% ****************************************************************************************************


% ****************************************************************************************************
% 0. Set the encoding of your files. UTF-8 is the only sensible encoding nowadays. If you can't read
% äöüßáéçèê∂åëæƒÏ€ then change the encoding setting in your editor, not the line below. If your editor
% does not support utf8 use another editor!
% ****************************************************************************************************
%\PassOptionsToPackage{utf8}{inputenc}
%      \usepackage{inputenc}

% ****************************************************************************************************
% 1. Configure classicthesis for your needs here, e.g., remove "drafting" below
% in order to deactivate the time-stamp on the pages
% ****************************************************************************************************
\PassOptionsToPackage{dottedtoc,
                      linedheaders,
                      floatperchapter,
                      listings%,drafting
                     }{classicthesis}
% ********************************************************************
% Available options for classicthesis.sty
% (see ClassicThesis.pdf for more information):
% drafting
% parts nochapters linedheaders
% eulerchapternumbers beramono eulermath pdfspacing minionprospacing
% tocaligned dottedtoc manychapters
% listings floatperchapter subfig
% ********************************************************************


% ****************************************************************************************************
% 2. Personal data and user ad-hoc commands
% ****************************************************************************************************
\newcommand{\myTitle}{The Polarizable Continuum Model Goes Viral\xspace}
\newcommand{\mySubtitle}{Advances in the Theory and Implementation of the
Polarizable Continuum Model\xspace}
\newcommand{\myDegree}{Philosphiae Doctor (Ph.~D.)\xspace}
\newcommand{\myName}{Roberto Di Remigio\xspace}
\newcommand{\myProf}{Luca Frediani\xspace}
\newcommand{\myOtherProf}{Benedetta Mennucci\xspace}
\newcommand{\mySupervisor}{Luca Frediani\xspace}
\newcommand{\myFaculty}{Fakultet for Naturvitenskap og Teknologi\xspace}
\newcommand{\myDepartment}{Institutt for Kjemi\xspace}
\newcommand{\myUni}{UiT -- Norges Arktiske Universitet\xspace}
\newcommand{\myLocation}{Tromsø\xspace}
\newcommand{\myTime}{January 2017\xspace}
\newcommand{\myVersion}{}

% ********************************************************************
% Setup, finetuning, and useful commands
% ********************************************************************
\newcounter{dummy} % necessary for correct hyperlinks (to index, bib, etc.)
\newlength{\abcd} % for ab..z string length calculation
\providecommand{\mLyX}{L\kern-.1667em\lower.25em\hbox{Y}\kern-.125emX\@}
\newcommand{\ie}{i.\,e.}
\newcommand{\Ie}{I.\,e.}
\newcommand{\eg}{e.\,g.}
\newcommand{\Eg}{E.\,g.}
% ****************************************************************************************************


% ****************************************************************************************************
% 3. Loading some handy packages
% ****************************************************************************************************
% ********************************************************************
% Packages with options that might require adjustments
% ********************************************************************
\usepackage{polyglossia}
\setmainlanguage{english}
\setotherlanguages{italian, french, norsk, greek}
\setdefaultlanguage{english}
\usepackage{csquotes}
\PassOptionsToPackage{%
style=phys,
maxcitenames=1,
mincitenames=1,
maxbibnames=100,
firstinits=true,
url=false,
isbn=false,
eprint=false,
texencoding=utf8,
bibencoding=utf8,
autocite=superscript,
backend=biber,
sorting=none,
backref=false,
hyperref=true,
block=none,
date=long,
urldate=long
}{biblatex}
    \usepackage{biblatex}
\renewcommand{\bibfont}{\normalfont\footnotesize\raggedright}
\AtBeginBibliography{
\DeclareFieldFormat{prefixnumber}{\mkbibbold{#1}}
\DeclareFieldFormat{labelnumber}{\mkbibbold{#1}}
}
\AtEveryBibitem{%
  \clearlist{language}%
}
\DeclareFieldFormat[article]{title}{\textsf{#1}}
\DeclareFieldFormat[inbook]{title}{\textsf{#1}}
\DeclareFieldFormat[incollection]{title}{\textsf{#1}}
\DeclareFieldFormat[inproceedings]{title}{\textsf{#1}}
\DeclareFieldFormat[inproceedings]{booktitle}{\textit{#1}}
\DeclareFieldFormat[inproceedings]{note}{#1}
\DeclareFieldFormat[unpublished]{title}{\textsf{#1}}

\DeclareCiteCommand{\noparcite}%[\mkbibbrackets] CITATION LIKE in ref. 6 WITHOUT SQUARE BRACKETS
  {\usebibmacro{cite:init}%
   \usebibmacro{prenote}}
  {\usebibmacro{citeindex}%
   \usebibmacro{cite:comp}}
  {}
  {\usebibmacro{cite:dump}%
   \usebibmacro{postnote}}

\PassOptionsToPackage{fleqn}{amsmath}       % math environments and more by the AMS
    \usepackage{amsmath}

% ********************************************************************
% General useful packages
% ********************************************************************
\usepackage{textcomp} % fix warning with missing font shapes
\usepackage{scrhack} % fix warnings when using KOMA with listings package
\usepackage{xspace} % to get the spacing after macros right
\usepackage{mparhack} % get marginpar right
\usepackage{fixltx2e} % fixes some LaTeX stuff --> since 2015 in the LaTeX kernel (see below)
%\usepackage[latest]{latexrelease} % will be used once available in more distributions (ISSUE #107)

% ****************************************************************************************************


% ****************************************************************************************************
% 4. Setup floats: tables, (sub)figures, and captions
% ****************************************************************************************************
\usepackage{tabularx} % better tables
    \setlength{\extrarowheight}{3pt} % increase table row height
\newcommand{\tableheadline}[1]{\multicolumn{1}{c}{\spacedlowsmallcaps{#1}}}
\newcommand{\myfloatalign}{\centering} % to be used with each float for alignment
\usepackage{caption}
% Thanks to cgnieder and Claus Lahiri
% http://tex.stackexchange.com/questions/69349/spacedlowsmallcaps-in-caption-label
% [REMOVED DUE TO OTHER PROBLEMS, SEE ISSUE #82]
%\DeclareCaptionLabelFormat{smallcaps}{\bothIfFirst{#1}{~}\MakeTextLowercase{\textsc{#2}}}
%\captionsetup{font=small,labelformat=smallcaps} % format=hang,
\captionsetup{format=hang,indention=-1.1cm,
              font=footnotesize,labelfont=bf,labelsep=space}
\usepackage{subcaption}
\captionsetup[sub]{format=hang,indention=0cm,
              font=footnotesize,labelfont=bf,labelsep=space}
% ****************************************************************************************************


% ****************************************************************************************************
% 5. Setup code listings
% ****************************************************************************************************
\usepackage{listings}
% ****************************************************************************************************


% ****************************************************************************************************
% 6. PDFLaTeX, hyperreferences and citation backreferences
% ****************************************************************************************************
% ********************************************************************
% Using PDFLaTeX
% ********************************************************************
\PassOptionsToPackage{hyperfootnotes=false,pdfpagelabels}{hyperref}
    \usepackage{hyperref}  % backref linktocpage pagebackref
\usepackage{graphicx}
\graphicspath{{gfx/}}

% ********************************************************************
% Setup autoreferences
% ********************************************************************
% There are some issues regarding autorefnames
% http://www.ureader.de/msg/136221647.aspx
% http://www.tex.ac.uk/cgi-bin/texfaq2html?label=latexwords
% you have to redefine the makros for the
% language you use, e.g., american, ngerman
% (as chosen when loading babel/AtBeginDocument)
% ********************************************************************
%\PassOptionsToPackage{american,italian}{babel}   % change this to your language(s)
% Spanish languages need extra options in order to work with this template
%\PassOptionsToPackage{spanish,es-lcroman}{babel}
% \usepackage{babel}
\makeatletter
\@ifpackageloaded{babel}%
    {%
       \addto\extrasamerican{%
			\renewcommand*{\figureautorefname}{Figure}%
			\renewcommand*{\tableautorefname}{Table}%
			\renewcommand*{\partautorefname}{Part}%
			\renewcommand*{\chapterautorefname}{Chapter}%
			\renewcommand*{\sectionautorefname}{Section}%
			\renewcommand*{\subsectionautorefname}{Section}%
			\renewcommand*{\subsubsectionautorefname}{Section}%
                }%
       \addto\extrasngerman{%
			\renewcommand*{\paragraphautorefname}{Absatz}%
			\renewcommand*{\subparagraphautorefname}{Unterabsatz}%
			\renewcommand*{\footnoteautorefname}{Fu\"snote}%
			\renewcommand*{\FancyVerbLineautorefname}{Zeile}%
			\renewcommand*{\theoremautorefname}{Theorem}%
			\renewcommand*{\appendixautorefname}{Anhang}%
			\renewcommand*{\equationautorefname}{Gleichung}%
			\renewcommand*{\itemautorefname}{Punkt}%
                }%
            % Fix to getting autorefs for subfigures right (thanks to Belinda Vogt for changing the definition)
            \providecommand{\subfigureautorefname}{\figureautorefname}%
    }{\relax}
\makeatother


% ****************************************************************************************************
% 7. Last calls before the bar closes
% ****************************************************************************************************
% ********************************************************************
% Development Stuff
% ********************************************************************
%\listfiles
%\PassOptionsToPackage{l2tabu,orthodox,abort}{nag}
%   \usepackage{nag}
%\PassOptionsToPackage{warning, all}{onlyamsmath}
%   \usepackage{onlyamsmath}

% ********************************************************************
% Last, but not least...
% ********************************************************************
\usepackage{classicthesis}
% ****************************************************************************************************


% ****************************************************************************************************
% 8. Further adjustments (experimental)
% ****************************************************************************************************
% ********************************************************************
% Changing the text area
% ********************************************************************
%\linespread{1.05} % a bit more for Palatino
\areaset[2cm]{410pt}{700pt}
%\setlength{\marginparwidth}{7em}%
%\setlength{\marginparsep}{2em}%

% ********************************************************************
% Using different fonts
% ********************************************************************
%\usepackage[oldstylenums]{kpfonts} % oldstyle notextcomp
%\usepackage[osf]{libertine}
%\usepackage[light,condensed,math]{iwona}
%\renewcommand{\sfdefault}{iwona}
%\usepackage{lmodern} % <-- no osf support :-(
%\usepackage{cfr-lm} %
%\usepackage[urw-garamond]{mathdesign} <-- no osf support :-(
%\usepackage[default,osfigures]{opensans} % scale=0.95
%\usepackage[sfdefault]{FiraSans}
% ****************************************************************************************************

% ****************************************************************************************************
% 9. MY OWN ADJUSTMENTS
% ****************************************************************************************************

\newlength{\drop}% for my convenience

\usepackage{xltxtra}
%%% Font stuff
\usepackage{amssymb}
\usepackage{fontspec}
\setmainfont[Mapping=tex-text]{MinionPro-Regular}
\setsansfont[Scale=0.85]{Open Sans}
\setmonofont[Scale=0.8]{Bitstream Vera Sans Mono}
\usepackage[math-style=ISO,
            bold-style=ISO]{unicode-math}
\setmathfont{xits-math}[
Path           = fonts/,
Extension      = .otf,
BoldFont       = *bold,
]
\usepackage{microtype}
\usepackage{booktabs}
\usepackage{url}
\usepackage[colorinlistoftodos,
            textsize=small]{todonotes}
\usepackage{wrapfig}

\usepackage{siunitx}

%%% To include the published papers
\usepackage{pdfpages}

\usepackage{epigraph}
\setlength\epigraphwidth{7cm}
\setlength\epigraphrule{0pt}
\renewcommand{\textflush}{flushleft}
\renewcommand{\epigraphsize}{\footnotesize}

\usepackage[dvipsnames]{xcolor}
\usepackage[overload]{empheq}
\usepackage{braket}
\usepackage{cancel}
\usepackage{amsthm}
\allowdisplaybreaks[4]

% Comment before compiling final version
%\usepackage{showkeys}
\usepackage{verbatim}

\usepackage[bottom]{footmisc}

\usepackage[inline]{enumitem}

\usepackage[version=4]{mhchem}
\usepackage{xpatch}

\usepackage{tikz}
\usepackage{tikz-3dplot}
\usetikzlibrary{shapes,arrows,shadows,positioning}
\usepackage{tcolorbox}

\usepackage[style=long,
            nolist,
            nonumberlist,
            acronym,
            shortcuts,
            nopostdot]{glossaries}
\makeglossaries
\usepackage[noprefix]{nomencl}
\makenomenclature

%%% Colors
% Blue
\definecolor{PMS2229}{RGB}{0, 156, 182}
% Red
\definecolor{PMS1797}{RGB}{203, 51, 59}
% Orange
\definecolor{PMS138}{RGB}{222, 124, 0}
% Yellow
\definecolor{PMS130}{RGB}{242, 169, 0}

% ********************************************************************
% Hyperreferences
% ********************************************************************
\hypersetup{%
    %draft, % = no hyperlinking at all (useful in b/w printouts)
    colorlinks=true, linktocpage=true, pdfstartpage=3, pdfstartview=FitV,%
    % uncomment the following line if you want to have black links (e.g., for printing)
    %colorlinks=false, linktocpage=false, pdfstartpage=3, pdfstartview=FitV, pdfborder={0 0 0},%
    breaklinks=true, pdfpagemode=UseNone, pageanchor=true, pdfpagemode=UseOutlines,%
    plainpages=false, bookmarksnumbered, bookmarksopen=true, bookmarksopenlevel=1,%
    hypertexnames=true, pdfhighlight=/O,%nesting=true,%frenchlinks,%
    urlcolor=PMS138, linkcolor=PMS2229, citecolor=webgreen, %pagecolor=RoyalBlue,%
    %urlcolor=Black, linkcolor=Black, citecolor=Black, %pagecolor=Black,%
    pdftitle={\myTitle},%
    pdfauthor={\textcopyright\ \myName, \myUni, \myFaculty},%
    pdfsubject={},%
    pdfkeywords={},%
    pdfcreator={XeLaTeX},%
    pdfproducer={LaTeX with hyperref and classicthesis}%
}

% ********************************************************************
% Listings
% ********************************************************************
\definecolor{mygreen}{rgb}{0,0.6,0}
\definecolor{mygray}{rgb}{0.5,0.5,0.5}
\definecolor{mymauve}{rgb}{0.58,0,0.82}

\lstset{ %
  backgroundcolor=\color{white},     % choose the background color; you must add \usepackage{color} or \usepackage{xcolor}
  basicstyle=\scriptsize\ttfamily, % the size of the fonts that are used for the code
  breakatwhitespace=false,         % sets if automatic breaks should only happen at whitespace
  breaklines=true,                 % sets automatic line breaking
  captionpos=t,                    % sets the caption-position to bottom
  commentstyle=\color{mygreen},    % comment style
  deletekeywords={...},            % if you want to delete keywords from the given language
  escapeinside={\%*}{*)},          % if you want to add LaTeX within your code
  extendedchars=true,              % lets you use non-ASCII characters; for 8-bits encodings only, does not work with UTF-8
  frame=single,	                   % adds a frame around the code
  keepspaces=true,                 % keeps spaces in text, useful for keeping indentation of code (possibly needs columns=flexible)
  keywordstyle=\color{blue},       % keyword style
  otherkeywords={*,...},           % if you want to add more keywords to the set
  numbers=none,                    % where to put the line-numbers; possible values are (none, left, right)
  rulecolor=\color{black},         % if not set, the frame-color may be changed on line-breaks within not-black text (e.g. comments (green here))
  showspaces=false,                % show spaces everywhere adding particular underscores; it overrides 'showstringspaces'
  showstringspaces=false,          % underline spaces within strings only
  showtabs=false,                  % show tabs within strings adding particular underscores
  stepnumber=2,                    % the step between two line-numbers. If it's 1, each line will be numbered
  stringstyle=\color{mymauve},     % string literal style
  tabsize=2,	                   % sets default tabsize to 2 spaces
  title=\lstname                   % show the filename of files included with \lstinputlisting; also try caption instead of title
}

% Change font for sections in TOC
\renewcommand\cftsecfont{\footnotesize}
\renewcommand\cftsecpagefont{\footnotesize}

\input{newcommands}

% ****************************************************************************************************
% If you like the classicthesis, then I would appreciate a postcard.
% My address can be found in the file ClassicThesis.pdf. A collection
% of the postcards I received so far is available online at
% http://postcards.miede.de
% ****************************************************************************************************


% ****************************************************************************************************
% 0. Set the encoding of your files. UTF-8 is the only sensible encoding nowadays. If you can't read
% äöüßáéçèê∂åëæƒÏ€ then change the encoding setting in your editor, not the line below. If your editor
% does not support utf8 use another editor!
% ****************************************************************************************************
%\PassOptionsToPackage{utf8}{inputenc}
%      \usepackage{inputenc}

% ****************************************************************************************************
% 1. Configure classicthesis for your needs here, e.g., remove "drafting" below
% in order to deactivate the time-stamp on the pages
% ****************************************************************************************************
\PassOptionsToPackage{dottedtoc,
                      linedheaders,
                      floatperchapter,
                      listings%,drafting
                     }{classicthesis}
% ********************************************************************
% Available options for classicthesis.sty
% (see ClassicThesis.pdf for more information):
% drafting
% parts nochapters linedheaders
% eulerchapternumbers beramono eulermath pdfspacing minionprospacing
% tocaligned dottedtoc manychapters
% listings floatperchapter subfig
% ********************************************************************


% ****************************************************************************************************
% 2. Personal data and user ad-hoc commands
% ****************************************************************************************************
\newcommand{\myTitle}{The Polarizable Continuum Model Goes Viral\xspace}
\newcommand{\mySubtitle}{Advances in the Theory and Implementation of the
Polarizable Continuum Model\xspace}
\newcommand{\myDegree}{Philosphiae Doctor (Ph.~D.)\xspace}
\newcommand{\myName}{Roberto Di Remigio\xspace}
\newcommand{\myProf}{Luca Frediani\xspace}
\newcommand{\myOtherProf}{Benedetta Mennucci\xspace}
\newcommand{\mySupervisor}{Luca Frediani\xspace}
\newcommand{\myFaculty}{Fakultet for Naturvitenskap og Teknologi\xspace}
\newcommand{\myDepartment}{Institutt for Kjemi\xspace}
\newcommand{\myUni}{UiT -- Norges Arktiske Universitet\xspace}
\newcommand{\myLocation}{Tromsø\xspace}
\newcommand{\myTime}{January 2017\xspace}
\newcommand{\myVersion}{}

% ********************************************************************
% Setup, finetuning, and useful commands
% ********************************************************************
\newcounter{dummy} % necessary for correct hyperlinks (to index, bib, etc.)
\newlength{\abcd} % for ab..z string length calculation
\providecommand{\mLyX}{L\kern-.1667em\lower.25em\hbox{Y}\kern-.125emX\@}
\newcommand{\ie}{i.\,e.}
\newcommand{\Ie}{I.\,e.}
\newcommand{\eg}{e.\,g.}
\newcommand{\Eg}{E.\,g.}
% ****************************************************************************************************


% ****************************************************************************************************
% 3. Loading some handy packages
% ****************************************************************************************************
% ********************************************************************
% Packages with options that might require adjustments
% ********************************************************************
\usepackage{polyglossia}
\setmainlanguage{english}
\setotherlanguages{italian, french, norsk, greek}
\setdefaultlanguage{english}
\usepackage{csquotes}
\PassOptionsToPackage{%
style=phys,
maxcitenames=1,
mincitenames=1,
maxbibnames=100,
firstinits=true,
url=false,
isbn=false,
eprint=false,
texencoding=utf8,
bibencoding=utf8,
autocite=superscript,
backend=biber,
sorting=none,
backref=false,
hyperref=true,
block=none,
date=long,
urldate=long
}{biblatex}
    \usepackage{biblatex}
\renewcommand{\bibfont}{\normalfont\footnotesize\raggedright}
\AtBeginBibliography{
\DeclareFieldFormat{prefixnumber}{\mkbibbold{#1}}
\DeclareFieldFormat{labelnumber}{\mkbibbold{#1}}
}
\AtEveryBibitem{%
  \clearlist{language}%
}
\DeclareFieldFormat[article]{title}{\textsf{#1}}
\DeclareFieldFormat[inbook]{title}{\textsf{#1}}
\DeclareFieldFormat[incollection]{title}{\textsf{#1}}
\DeclareFieldFormat[inproceedings]{title}{\textsf{#1}}
\DeclareFieldFormat[inproceedings]{booktitle}{\textit{#1}}
\DeclareFieldFormat[inproceedings]{note}{#1}
\DeclareFieldFormat[unpublished]{title}{\textsf{#1}}

\DeclareCiteCommand{\noparcite}%[\mkbibbrackets] CITATION LIKE in ref. 6 WITHOUT SQUARE BRACKETS
  {\usebibmacro{cite:init}%
   \usebibmacro{prenote}}
  {\usebibmacro{citeindex}%
   \usebibmacro{cite:comp}}
  {}
  {\usebibmacro{cite:dump}%
   \usebibmacro{postnote}}

\PassOptionsToPackage{fleqn}{amsmath}       % math environments and more by the AMS
    \usepackage{amsmath}

% ********************************************************************
% General useful packages
% ********************************************************************
\usepackage{textcomp} % fix warning with missing font shapes
\usepackage{scrhack} % fix warnings when using KOMA with listings package
\usepackage{xspace} % to get the spacing after macros right
\usepackage{mparhack} % get marginpar right
\usepackage{fixltx2e} % fixes some LaTeX stuff --> since 2015 in the LaTeX kernel (see below)
%\usepackage[latest]{latexrelease} % will be used once available in more distributions (ISSUE #107)

% ****************************************************************************************************


% ****************************************************************************************************
% 4. Setup floats: tables, (sub)figures, and captions
% ****************************************************************************************************
\usepackage{tabularx} % better tables
    \setlength{\extrarowheight}{3pt} % increase table row height
\newcommand{\tableheadline}[1]{\multicolumn{1}{c}{\spacedlowsmallcaps{#1}}}
\newcommand{\myfloatalign}{\centering} % to be used with each float for alignment
\usepackage{caption}
% Thanks to cgnieder and Claus Lahiri
% http://tex.stackexchange.com/questions/69349/spacedlowsmallcaps-in-caption-label
% [REMOVED DUE TO OTHER PROBLEMS, SEE ISSUE #82]
%\DeclareCaptionLabelFormat{smallcaps}{\bothIfFirst{#1}{~}\MakeTextLowercase{\textsc{#2}}}
%\captionsetup{font=small,labelformat=smallcaps} % format=hang,
\captionsetup{format=hang,indention=-1.1cm,
              font=footnotesize,labelfont=bf,labelsep=space}
\usepackage{subcaption}
\captionsetup[sub]{format=hang,indention=0cm,
              font=footnotesize,labelfont=bf,labelsep=space}
% ****************************************************************************************************


% ****************************************************************************************************
% 5. Setup code listings
% ****************************************************************************************************
\usepackage{listings}
% ****************************************************************************************************


% ****************************************************************************************************
% 6. PDFLaTeX, hyperreferences and citation backreferences
% ****************************************************************************************************
% ********************************************************************
% Using PDFLaTeX
% ********************************************************************
\PassOptionsToPackage{hyperfootnotes=false,pdfpagelabels}{hyperref}
    \usepackage{hyperref}  % backref linktocpage pagebackref
\usepackage{graphicx}
\graphicspath{{gfx/}}

% ********************************************************************
% Setup autoreferences
% ********************************************************************
% There are some issues regarding autorefnames
% http://www.ureader.de/msg/136221647.aspx
% http://www.tex.ac.uk/cgi-bin/texfaq2html?label=latexwords
% you have to redefine the makros for the
% language you use, e.g., american, ngerman
% (as chosen when loading babel/AtBeginDocument)
% ********************************************************************
%\PassOptionsToPackage{american,italian}{babel}   % change this to your language(s)
% Spanish languages need extra options in order to work with this template
%\PassOptionsToPackage{spanish,es-lcroman}{babel}
% \usepackage{babel}
\makeatletter
\@ifpackageloaded{babel}%
    {%
       \addto\extrasamerican{%
			\renewcommand*{\figureautorefname}{Figure}%
			\renewcommand*{\tableautorefname}{Table}%
			\renewcommand*{\partautorefname}{Part}%
			\renewcommand*{\chapterautorefname}{Chapter}%
			\renewcommand*{\sectionautorefname}{Section}%
			\renewcommand*{\subsectionautorefname}{Section}%
			\renewcommand*{\subsubsectionautorefname}{Section}%
                }%
       \addto\extrasngerman{%
			\renewcommand*{\paragraphautorefname}{Absatz}%
			\renewcommand*{\subparagraphautorefname}{Unterabsatz}%
			\renewcommand*{\footnoteautorefname}{Fu\"snote}%
			\renewcommand*{\FancyVerbLineautorefname}{Zeile}%
			\renewcommand*{\theoremautorefname}{Theorem}%
			\renewcommand*{\appendixautorefname}{Anhang}%
			\renewcommand*{\equationautorefname}{Gleichung}%
			\renewcommand*{\itemautorefname}{Punkt}%
                }%
            % Fix to getting autorefs for subfigures right (thanks to Belinda Vogt for changing the definition)
            \providecommand{\subfigureautorefname}{\figureautorefname}%
    }{\relax}
\makeatother


% ****************************************************************************************************
% 7. Last calls before the bar closes
% ****************************************************************************************************
% ********************************************************************
% Development Stuff
% ********************************************************************
%\listfiles
%\PassOptionsToPackage{l2tabu,orthodox,abort}{nag}
%   \usepackage{nag}
%\PassOptionsToPackage{warning, all}{onlyamsmath}
%   \usepackage{onlyamsmath}

% ********************************************************************
% Last, but not least...
% ********************************************************************
\usepackage{classicthesis}
% ****************************************************************************************************


% ****************************************************************************************************
% 8. Further adjustments (experimental)
% ****************************************************************************************************
% ********************************************************************
% Changing the text area
% ********************************************************************
%\linespread{1.05} % a bit more for Palatino
\areaset[2cm]{410pt}{700pt}
%\setlength{\marginparwidth}{7em}%
%\setlength{\marginparsep}{2em}%

% ********************************************************************
% Using different fonts
% ********************************************************************
%\usepackage[oldstylenums]{kpfonts} % oldstyle notextcomp
%\usepackage[osf]{libertine}
%\usepackage[light,condensed,math]{iwona}
%\renewcommand{\sfdefault}{iwona}
%\usepackage{lmodern} % <-- no osf support :-(
%\usepackage{cfr-lm} %
%\usepackage[urw-garamond]{mathdesign} <-- no osf support :-(
%\usepackage[default,osfigures]{opensans} % scale=0.95
%\usepackage[sfdefault]{FiraSans}
% ****************************************************************************************************

% ****************************************************************************************************
% 9. MY OWN ADJUSTMENTS
% ****************************************************************************************************

\newlength{\drop}% for my convenience

\usepackage{xltxtra}
%%% Font stuff
\usepackage{amssymb}
\usepackage{fontspec}
\setmainfont[Mapping=tex-text]{MinionPro-Regular}
\setsansfont[Scale=0.85]{Open Sans}
\setmonofont[Scale=0.8]{Bitstream Vera Sans Mono}
\usepackage[math-style=ISO,
            bold-style=ISO]{unicode-math}
\setmathfont{xits-math}[
Path           = fonts/,
Extension      = .otf,
BoldFont       = *bold,
]
\usepackage{microtype}
\usepackage{booktabs}
\usepackage{url}
\usepackage[colorinlistoftodos,
            textsize=small]{todonotes}
\usepackage{wrapfig}

\usepackage{siunitx}

%%% To include the published papers
\usepackage{pdfpages}

\usepackage{epigraph}
\setlength\epigraphwidth{7cm}
\setlength\epigraphrule{0pt}
\renewcommand{\textflush}{flushleft}
\renewcommand{\epigraphsize}{\footnotesize}

\usepackage[dvipsnames]{xcolor}
\usepackage[overload]{empheq}
\usepackage{braket}
\usepackage{cancel}
\usepackage{amsthm}
\allowdisplaybreaks[4]

% Comment before compiling final version
%\usepackage{showkeys}
\usepackage{verbatim}

\usepackage[bottom]{footmisc}

\usepackage[inline]{enumitem}

\usepackage[version=4]{mhchem}
\usepackage{xpatch}

\usepackage{tikz}
\usepackage{tikz-3dplot}
\usetikzlibrary{shapes,arrows,shadows,positioning}
\usepackage{tcolorbox}

\usepackage[style=long,
            nolist,
            nonumberlist,
            acronym,
            shortcuts,
            nopostdot]{glossaries}
\makeglossaries
\usepackage[noprefix]{nomencl}
\makenomenclature

%%% Colors
% Blue
\definecolor{PMS2229}{RGB}{0, 156, 182}
% Red
\definecolor{PMS1797}{RGB}{203, 51, 59}
% Orange
\definecolor{PMS138}{RGB}{222, 124, 0}
% Yellow
\definecolor{PMS130}{RGB}{242, 169, 0}

% ********************************************************************
% Hyperreferences
% ********************************************************************
\hypersetup{%
    %draft, % = no hyperlinking at all (useful in b/w printouts)
    colorlinks=true, linktocpage=true, pdfstartpage=3, pdfstartview=FitV,%
    % uncomment the following line if you want to have black links (e.g., for printing)
    %colorlinks=false, linktocpage=false, pdfstartpage=3, pdfstartview=FitV, pdfborder={0 0 0},%
    breaklinks=true, pdfpagemode=UseNone, pageanchor=true, pdfpagemode=UseOutlines,%
    plainpages=false, bookmarksnumbered, bookmarksopen=true, bookmarksopenlevel=1,%
    hypertexnames=true, pdfhighlight=/O,%nesting=true,%frenchlinks,%
    urlcolor=PMS138, linkcolor=PMS2229, citecolor=webgreen, %pagecolor=RoyalBlue,%
    %urlcolor=Black, linkcolor=Black, citecolor=Black, %pagecolor=Black,%
    pdftitle={\myTitle},%
    pdfauthor={\textcopyright\ \myName, \myUni, \myFaculty},%
    pdfsubject={},%
    pdfkeywords={},%
    pdfcreator={XeLaTeX},%
    pdfproducer={LaTeX with hyperref and classicthesis}%
}

% ********************************************************************
% Listings
% ********************************************************************
\definecolor{mygreen}{rgb}{0,0.6,0}
\definecolor{mygray}{rgb}{0.5,0.5,0.5}
\definecolor{mymauve}{rgb}{0.58,0,0.82}

\lstset{ %
  backgroundcolor=\color{white},     % choose the background color; you must add \usepackage{color} or \usepackage{xcolor}
  basicstyle=\scriptsize\ttfamily, % the size of the fonts that are used for the code
  breakatwhitespace=false,         % sets if automatic breaks should only happen at whitespace
  breaklines=true,                 % sets automatic line breaking
  captionpos=t,                    % sets the caption-position to bottom
  commentstyle=\color{mygreen},    % comment style
  deletekeywords={...},            % if you want to delete keywords from the given language
  escapeinside={\%*}{*)},          % if you want to add LaTeX within your code
  extendedchars=true,              % lets you use non-ASCII characters; for 8-bits encodings only, does not work with UTF-8
  frame=single,	                   % adds a frame around the code
  keepspaces=true,                 % keeps spaces in text, useful for keeping indentation of code (possibly needs columns=flexible)
  keywordstyle=\color{blue},       % keyword style
  otherkeywords={*,...},           % if you want to add more keywords to the set
  numbers=none,                    % where to put the line-numbers; possible values are (none, left, right)
  rulecolor=\color{black},         % if not set, the frame-color may be changed on line-breaks within not-black text (e.g. comments (green here))
  showspaces=false,                % show spaces everywhere adding particular underscores; it overrides 'showstringspaces'
  showstringspaces=false,          % underline spaces within strings only
  showtabs=false,                  % show tabs within strings adding particular underscores
  stepnumber=2,                    % the step between two line-numbers. If it's 1, each line will be numbered
  stringstyle=\color{mymauve},     % string literal style
  tabsize=2,	                   % sets default tabsize to 2 spaces
  title=\lstname                   % show the filename of files included with \lstinputlisting; also try caption instead of title
}

% Change font for sections in TOC
\renewcommand\cftsecfont{\footnotesize}
\renewcommand\cftsecpagefont{\footnotesize}

%%%%% Units of measure
\DeclareSIUnit[number-unit-product = \;]\debye{D}
\DeclareSIUnit[number-unit-product = \;]\au{a.u.}
\DeclareSIUnit{\calorie}{cal}

%%%%% URLs
\newcommand{\GitHub}{\href{https://github.com/robertodr}{GitHub}\xspace}
\newcommand{\GitLab}{\href{https://gitlab.com/u/robertodr}{GitLab}\xspace}
\newcommand{\pcmsolver}{\href{http://pcmsolver.readthedocs.io}{\textsc{PCMSolver}}\xspace}
\newcommand{\psicode}{\href{http://www.psicode.org}{\textsc{Psi4}}\xspace}
\newcommand{\DIRAC}{\href{http://www.diracprogram.org}{\textsc{DIRAC}}\xspace}
\newcommand{\DALTON}{\href{http://www.daltonprogram.org}{\textsc{Dalton}}\xspace}
\newcommand{\LSDALTON}{\href{http://www.daltonprogram.org}{\textsc{LSDalton}}\xspace}
\newcommand{\ReSpect}{\href{http://respect.readthedocs.io/}{\textsc{ReSpect}}\xspace}
\newcommand{\git}{\href{https://git-scm.com/}{\texttt{git}}\xspace}
\newcommand{\cmake}{\href{https://cmake.org/}{\texttt{CMake}}\xspace}
\newcommand{\readthedocs}{\href{https://readthedocs.org/}{Read The Docs}\xspace}
\newcommand{\GPL}{\href{https://www.gnu.org/licenses/gpl.html}{GPL}\xspace}
\newcommand{\LGPL}{\href{https://www.gnu.org/licenses/lgpl.html}{LGPL}\xspace}
\newcommand{\GetKw}{\href{https://github.com/juselius/libgetkw}{GetKw}\xspace}
\newcommand{\Boost}{\href{http://www.boost.org/}{Boost}\xspace}
\newcommand{\Eigen}{\href{http://eigen.tuxfamily.org/index.php?title=Main_Page}{Eigen}\xspace}
\newcommand{\Taylor}{\href{https://github.com/uekstrom/libtaylor}{Taylor}\xspace}

%%%%% General
\newcounter{dummy} % necessary for correct hyperlinks (to index, bib, etc.)
\newcommand*{\diff}{\mathop{}\!\mathrm{d}} % Differential
\newcommand*{\Supp}{\mathop{}\!\mathrm{Supp}}
\newcommand*{\deriv}[3][]{% Derivative
\frac{\diff^{#1}#2}{\diff #3^{#1}}}
\newcommand*{\pderiv}[3][]{%
\frac{\partial^{#1}#2}% Partial derivative
{\partial #3^{#1}}}
\newcommand*{\vect}[1]{\symbf{#1}} % Vector
\newcommand*{\mat}[1]{\symbf{#1}}% Matrix
\newcommand*{\Tr}{\mathrm{Tr}}% Trace
\newcommand*{\supp}[1]{\Supp(#1)}
\newcommand*{\conv}{\mathop{}\!\mathrm{Conv}\,}
\newcommand*{\hull}{\mathop{}\!\mathrm{Hull}\,}
\newcommand*{\sing}{\mathop{}\!\mathrm{Sing}\,}
\newcommand*{\dist}{\mathop{}\!\mathrm{Dist}}
\newcommand*{\Span}[1]{\mathop{}\!\mathrm{Span}\{#1\}}
\newcommand*{\norm}[2]{\| #1 \|_{#2}}
\newcommand*{\paper}[1]{\textbf{Paper #1}}
\newcommand{\finalVersion}{Final version as of \today}
\newcommand{\submittedVersion}{Final version as of October 5, 2016}
\DeclareRobustCommand{\T1}{T_1}
% Inserts a blank page
\newcommand{\blankpage}{\newpage\hbox{}\thispagestyle{empty}\newpage}
\newcommand{\hairsp}{\hspace{1pt}}% hair space
\newcommand{\ie}{\textit{i.\hairsp{}e.}\xspace}
\newcommand{\Ie}{\textit{I.\hairsp{}e.}\xspace}
\newcommand{\eg}{\textit{e.\hairsp{}g.}\xspace}
\newcommand{\Eg}{\textit{E.\hairsp{}g.}\xspace}

%%
% Amount of space to skip before \newthought or after title block
\newskip\tufteskipamount
\tufteskipamount=1.0\baselineskip plus 0.5ex minus 0.2ex

\newcommand{\tuftebreak}{\par\ifdim\lastskip<\tufteskipamount
  \removelastskip\penalty-100\tufteskip\fi}

\newcommand{\tufteskip}{\vspace\tufteskipamount}

%%
% A newthought
\providecommand\newthought[1]{%
   \tuftebreak
   \noindent\textsc{#1}%
}

%%
% A newthought, with a small arrow on the margin
\providecommand\arrowthought[1]{%
   \tuftebreak
   \begin{tikzpicture}
   \node[rotate=270] {\tikz\draw[draw=brewerCyan, fill=brewerCyan, -triangle 45](0,0) ;};
   \end{tikzpicture}
   \noindent\textsc{#1}%
}

%%
% A newthought without tuftebreak
\providecommand\begthought[1]{%
   \noindent\textsc{#1}%
}

%%
% A newthought without tuftebreak but with a small arrow on the margin
\providecommand\begarrowthought[1]{%
   \begin{tikzpicture}
   \node[rotate=270] {\tikz\draw[draw=brewerCyan, fill=brewerCyan, -triangle 45](0,0) ;};
   \end{tikzpicture}
   \noindent\textsc{#1}%
}

%%%%% PCM stuff
\newcommand*{\diel}{\varepsilon} % Permittivity
\newcommand*{\diels}{\varepsilon_\mathrm{s}}
\newcommand*{\dield}{\varepsilon_\infty}
\newcommand*{\bi}[1]{\hat{\mathcal{#1}}} % Boundary integral operator
\newcommand*{\PCM}{\bi{Y}} % The PCM boundary integral operator
\newcommand*{\MM}{\bi{W}} % The MM classical matrix
\newcommand*{\MMPCM}{\bi{X}} % The MM/PCM interaction kernel
\newcommand*{\scalprod}[3][\Gamma]{\left(#2, #3\right)_{#1}} % Scalar product
\newcommand*{\s}{\mathrm{s}} % Sources
\newcommand*{\p}{\mathrm{p}} % Polarization degree of freedom
\newcommand*{\V}{\mathbb{V}} % Supermatrix
\newcommand*{\asc}{\sigma} % ASC
\newcommand*{\esp}{\varphi} % MEP
\newcommand*{\Ylm}[4]{\mathcal{Y}_{#1}^{#2}(#3, #4)}
\newcommand*{\BarYlm}[4]{\overline{\mathcal{Y}}_{#1}^{#2}(#3, #4)}
\newcommand*{\Li}{L_\mathrm{i}}
\newcommand*{\ui}{u_\mathrm{i}}
\newcommand*{\rhoi}{\rho_\mathrm{i}}
\newcommand*{\Omegai}{\Omega_\mathrm{i}}
\newcommand*{\partiali}{\partial_{L_\mathrm{i}}}
\newcommand*{\Gi}{G_\mathrm{i}(\vect{r}, \vect{r}^\prime)}
\newcommand*{\Le}{L_\mathrm{e}}
\newcommand*{\ue}{u_\mathrm{e}}
\newcommand*{\rhoe}{\rho_\mathrm{e}}
\newcommand*{\Omegae}{\Omega_\mathrm{e}}
\newcommand*{\partiale}{\partial_{L_\mathrm{e}}}
\newcommand*{\Ge}{G_\mathrm{e}(\vect{r}, \vect{r}^\prime)}

%%%%% Response theory stuff
\newcommand*{\aveQ}{\left\lbrace Q(t)\right\rbrace_T}% Time-averaged quasi-energy
\newcommand*{\response}[3]{\braket{\braket{#1; #2}}_{#3}}% Response function
\newcommand*{\pertmat}[1]{\tilde{\mat{#1}}}
\newcommand*{\aveL}{\lbrace\tilde{L}(\pertmat{C}, \pertmat{\lambda}, \tilde{\sigma}, t)\rbrace_T}% Time-averaged quasi-energy
\newcommand*{\aveLa}{\lbrace\tilde{L}^a(\pertmat{C}, \pertmat{\lambda}, \tilde{\sigma}, t)\rbrace_T}% Time-averaged quasi-energy
\newcommand*{\pertPCM}[1]{\PCM^{(#1)}}
\newcommand*{\eqtr}{\overset{\Tr}{=}}
\newcommand*{\eqavetr}{\overset{\lbrace\Tr\rbrace_T}{=}}
\newcommand*{\herm}[1]{\left[#1\right]^{\oplus}}
\newcommand*{\aherm}[1]{\left[#1\right]^{\ominus}}
\newcommand*{\hermitian}[1]{\mat{#1} + \mat{#1}^\dagger}
\newcommand*{\ahermitian}[1]{\mat{#1} - \mat{#1}^\dagger}
\newcommand*{\breveFock}[1]{\breve{\mat{\mathcal{F}}}_\omega^{#1}}
\newcommand*{\rspParam}[1]{\mat{X}^{#1}}
\newcommand*{\RHS}[1]{\mat{M}_\mathrm{RHS}^{#1}}
\newcommand*{\Dp}[1]{\mat{D}_\mathrm{P}^{#1}}
\newcommand*{\Dh}[1]{\mat{D}_\mathrm{H}^{#1}}
\newcommand*{\ASCp}[1]{\asc_\mathrm{P}^{#1}}
\newcommand*{\ASCh}[1]{\asc_\mathrm{H}^{#1}}
\newcommand*{\Gn}[1]{\mathcal{G}^{0, #1}}
\newcommand*{\Gderiv}[2]{\mat{\mathcal{G}}^{#1, #2}}
\newcommand*{\genHessian}{\mat{E}^{[2]}}
\newcommand*{\genMetric}{\mat{S}^{[2]}}

%%%%% Coupled cluster stuff
\newcommand*{\commutator}[3][]{[ #2, #3 ]^{#1}} % Commutator
\newcommand*{\anticommutator}[2]{% Anticommutator
[ #1, #2 ]_{+}}
\newcommand*{\lag}{\mathcal{L}_\mathrm{eff}}
%\newcommand*{\lagCC2}{\mathcal{L}_\mathrm{eff}(\tamp{}, \tbar{}, \p)_\mathrm{CC2}}
%\newcommand*{\lagCC3}{\mathcal{L}_\mathrm{eff}(\tamp{}, \tbar{}, \p)_\mathrm{CC3}}
\newcommand*{\expo}[1]{\mathrm{e}^{#1}}
\newcommand*{\twoel}[2]{\Braket{#1 \| #2}}
\newcommand*{\tamp}[1]{t_{#1}} % Cluster amplitude
\newcommand*{\tampEq}[1]{\Omega_{#1}} % Cluster amplitude vector function
\newcommand*{\sd}[1]{{}^*#1}
\newcommand*{\tbar}[1]{\bar{t}_{#1}} % Multiplier
\newcommand*{\tbarEq}[1]{\bar{\Omega}_{#1}} % Multiplier vector function
\newcommand*{\cluster}[1]{\tau_{#1}} % Cluster operator
\newcommand*{\denom}[1]{\epsilon_{#1}} % Energy denominator
\newcommand*{\cons}[1]{a_{#1}^{\dagger}}% Constructor operator
\newcommand*{\anni}[1]{a_{#1}} % Annihilation operator
\newcommand*{\sigmaEq}{\Omega^{\sigma}} % Multiplier vector function
\newcommand*{\BCH}[2]{#1+\commutator{#1}{#2}+\frac{1}{2}\commutator{\commutator{#1}{#2}}{#2}
+ \ldots} % BCH expansion up to second order
\newcommand*{\BCHfirst}[2]{\commutator{#1}{#2}}% BCH 1st order
\newcommand*{\BCHsecond}[2]{\frac{1}{2}\commutator{\commutator{#1}{#2}}{#2}}% BCH 2nd order
\newcommand*{\BCHthird}[2]{\frac{1}{6}\commutator{\commutator{\commutator{#1}{#2}}{#2}}{#2}} %BCH 3rd order
% BCH 4th order
\newcommand*{\BCHfourth}[2]{\frac{1}{24}\commutator{\commutator{\commutator{\commutator{#1}{#2}}{#2}}{#2}}{#2}}
% \Phi
\newcommand*{\RTerm}[3]{{}^{#1}R^{#2}_{#3}}
\newcommand*{\STerm}[3]{{}^{#1}S^{#2}_{#3}}
\newcommand*{\UTerm}[4]{{}^{#1}U^{#2}_{#3 #4}}
% \esp
\newcommand*{\MEPRTerm}[3]{{}^{#1}\underline{R}^{#2}_{#3}}
\newcommand*{\MEPSTerm}[3]{{}^{#1}\underline{S}^{#2}_{#3}}
\newcommand*{\MEPUTerm}[4]{{}^{#1}\underline{U}^{#2}_{#3 #4}}

% Box around equation
\newcommand{\highlight}[2]{%
  \colorbox{#1!50}{$\displaystyle#2$}}
% Color terms in equations
\newcommand*{\mathcolor}{}
\def\mathcolor#1#{\mathcoloraux{#1}}
\newcommand*{\mathcoloraux}[3]{%
  \protect\leavevmode
    \begingroup
        \color#1{#2}#3%
          \endgroup
}

% Theorem environments
%%% Definitions
\newtheoremstyle{classicdef}%             % Name
  {12pt}%                                 % Space above
  {12pt}%                                 % Space below
  {}%                                     % Body font
  {}%                                     % Indent amount
  {\bfseries\scshape}%                    % Theorem head font
  {.}%                                    % Punctuation after theorem head
  { }%                                    % Space after theorem head, ' ', or \newline
  {\thmname{#1}\thmnumber{ #2}\thmnote{ (#3)}}%

%%% Theorems, Lemmas, Corollaries
\newtheoremstyle{classicthm}%             % Name
  {12pt}%                                 % Space above
  {12pt}%                                 % Space below
  {\itshape}%                             % Body font
  {}%                                     % Indent amount
  {\bfseries\scshape}%                    % Theorem head font
  {.}%                                    % Punctuation after theorem head
  { }%                                    % Space after theorem head, ' ', or \newline
  {\thmname{#1}\thmnumber{ #2}\thmnote{ (#3)}}%

\theoremstyle{classicthm}
\newtheorem{theorem}{Theorem}
\newtheorem{lemma}{Lemma}
\newtheorem{proposition}{Proposition}
\newtheorem*{corollary}{Corollary}
\newtheorem{postulate}{Postulate}

\theoremstyle{classicdef}
\newtheorem{defin}{Definition}
\newtheorem{conjecture}{Conjecture}
\newtheorem{example}{Example}

\theoremstyle{remark}
\newtheorem*{rem}{Remark}
\newtheorem*{note}{Note}
\newtheorem{case}{Case}

\newenvironment{Proof}{\vspace*{1.5pt}\begin{proof} \small{}}{\end{proof}\vspace*{1.5pt}}
\renewenvironment{example}[1][]{\subparagraph*{Example} #1 \small{}}{}



% ****************************************************************************************************
% If you like the classicthesis, then I would appreciate a postcard.
% My address can be found in the file ClassicThesis.pdf. A collection
% of the postcards I received so far is available online at
% http://postcards.miede.de
% ****************************************************************************************************


% ****************************************************************************************************
% 0. Set the encoding of your files. UTF-8 is the only sensible encoding nowadays. If you can't read
% äöüßáéçèê∂åëæƒÏ€ then change the encoding setting in your editor, not the line below. If your editor
% does not support utf8 use another editor!
% ****************************************************************************************************
%\PassOptionsToPackage{utf8}{inputenc}
%      \usepackage{inputenc}

% ****************************************************************************************************
% 1. Configure classicthesis for your needs here, e.g., remove "drafting" below
% in order to deactivate the time-stamp on the pages
% ****************************************************************************************************
\PassOptionsToPackage{dottedtoc,
                      linedheaders,
                      floatperchapter,
                      listings%,drafting
                     }{classicthesis}
% ********************************************************************
% Available options for classicthesis.sty
% (see ClassicThesis.pdf for more information):
% drafting
% parts nochapters linedheaders
% eulerchapternumbers beramono eulermath pdfspacing minionprospacing
% tocaligned dottedtoc manychapters
% listings floatperchapter subfig
% ********************************************************************


% ****************************************************************************************************
% 2. Personal data and user ad-hoc commands
% ****************************************************************************************************
\newcommand{\myTitle}{The Polarizable Continuum Model Goes Viral\xspace}
\newcommand{\mySubtitle}{Advances in the Theory and Implementation of the
Polarizable Continuum Model\xspace}
\newcommand{\myDegree}{Philosphiae Doctor (Ph.~D.)\xspace}
\newcommand{\myName}{Roberto Di Remigio\xspace}
\newcommand{\myProf}{Luca Frediani\xspace}
\newcommand{\myOtherProf}{Benedetta Mennucci\xspace}
\newcommand{\mySupervisor}{Luca Frediani\xspace}
\newcommand{\myFaculty}{Fakultet for Naturvitenskap og Teknologi\xspace}
\newcommand{\myDepartment}{Institutt for Kjemi\xspace}
\newcommand{\myUni}{UiT -- Norges Arktiske Universitet\xspace}
\newcommand{\myLocation}{Tromsø\xspace}
\newcommand{\myTime}{January 2017\xspace}
\newcommand{\myVersion}{}

% ********************************************************************
% Setup, finetuning, and useful commands
% ********************************************************************
\newcounter{dummy} % necessary for correct hyperlinks (to index, bib, etc.)
\newlength{\abcd} % for ab..z string length calculation
\providecommand{\mLyX}{L\kern-.1667em\lower.25em\hbox{Y}\kern-.125emX\@}
\newcommand{\ie}{i.\,e.}
\newcommand{\Ie}{I.\,e.}
\newcommand{\eg}{e.\,g.}
\newcommand{\Eg}{E.\,g.}
% ****************************************************************************************************


% ****************************************************************************************************
% 3. Loading some handy packages
% ****************************************************************************************************
% ********************************************************************
% Packages with options that might require adjustments
% ********************************************************************
\usepackage{polyglossia}
\setmainlanguage{english}
\setotherlanguages{italian, french, norsk, greek}
\setdefaultlanguage{english}
\usepackage{csquotes}
\PassOptionsToPackage{%
style=phys,
maxcitenames=1,
mincitenames=1,
maxbibnames=100,
firstinits=true,
url=false,
isbn=false,
eprint=false,
texencoding=utf8,
bibencoding=utf8,
autocite=superscript,
backend=biber,
sorting=none,
backref=false,
hyperref=true,
block=none,
date=long,
urldate=long
}{biblatex}
    \usepackage{biblatex}
\renewcommand{\bibfont}{\normalfont\footnotesize\raggedright}
\AtBeginBibliography{
\DeclareFieldFormat{prefixnumber}{\mkbibbold{#1}}
\DeclareFieldFormat{labelnumber}{\mkbibbold{#1}}
}
\AtEveryBibitem{%
  \clearlist{language}%
}
\DeclareFieldFormat[article]{title}{\textsf{#1}}
\DeclareFieldFormat[inbook]{title}{\textsf{#1}}
\DeclareFieldFormat[incollection]{title}{\textsf{#1}}
\DeclareFieldFormat[inproceedings]{title}{\textsf{#1}}
\DeclareFieldFormat[inproceedings]{booktitle}{\textit{#1}}
\DeclareFieldFormat[inproceedings]{note}{#1}
\DeclareFieldFormat[unpublished]{title}{\textsf{#1}}

\DeclareCiteCommand{\noparcite}%[\mkbibbrackets] CITATION LIKE in ref. 6 WITHOUT SQUARE BRACKETS
  {\usebibmacro{cite:init}%
   \usebibmacro{prenote}}
  {\usebibmacro{citeindex}%
   \usebibmacro{cite:comp}}
  {}
  {\usebibmacro{cite:dump}%
   \usebibmacro{postnote}}

\PassOptionsToPackage{fleqn}{amsmath}       % math environments and more by the AMS
    \usepackage{amsmath}

% ********************************************************************
% General useful packages
% ********************************************************************
\usepackage{textcomp} % fix warning with missing font shapes
\usepackage{scrhack} % fix warnings when using KOMA with listings package
\usepackage{xspace} % to get the spacing after macros right
\usepackage{mparhack} % get marginpar right
\usepackage{fixltx2e} % fixes some LaTeX stuff --> since 2015 in the LaTeX kernel (see below)
%\usepackage[latest]{latexrelease} % will be used once available in more distributions (ISSUE #107)

% ****************************************************************************************************


% ****************************************************************************************************
% 4. Setup floats: tables, (sub)figures, and captions
% ****************************************************************************************************
\usepackage{tabularx} % better tables
    \setlength{\extrarowheight}{3pt} % increase table row height
\newcommand{\tableheadline}[1]{\multicolumn{1}{c}{\spacedlowsmallcaps{#1}}}
\newcommand{\myfloatalign}{\centering} % to be used with each float for alignment
\usepackage{caption}
% Thanks to cgnieder and Claus Lahiri
% http://tex.stackexchange.com/questions/69349/spacedlowsmallcaps-in-caption-label
% [REMOVED DUE TO OTHER PROBLEMS, SEE ISSUE #82]
%\DeclareCaptionLabelFormat{smallcaps}{\bothIfFirst{#1}{~}\MakeTextLowercase{\textsc{#2}}}
%\captionsetup{font=small,labelformat=smallcaps} % format=hang,
\captionsetup{format=hang,indention=-1.1cm,
              font=footnotesize,labelfont=bf,labelsep=space}
\usepackage{subcaption}
\captionsetup[sub]{format=hang,indention=0cm,
              font=footnotesize,labelfont=bf,labelsep=space}
% ****************************************************************************************************


% ****************************************************************************************************
% 5. Setup code listings
% ****************************************************************************************************
\usepackage{listings}
% ****************************************************************************************************


% ****************************************************************************************************
% 6. PDFLaTeX, hyperreferences and citation backreferences
% ****************************************************************************************************
% ********************************************************************
% Using PDFLaTeX
% ********************************************************************
\PassOptionsToPackage{hyperfootnotes=false,pdfpagelabels}{hyperref}
    \usepackage{hyperref}  % backref linktocpage pagebackref
\usepackage{graphicx}
\graphicspath{{gfx/}}

% ********************************************************************
% Setup autoreferences
% ********************************************************************
% There are some issues regarding autorefnames
% http://www.ureader.de/msg/136221647.aspx
% http://www.tex.ac.uk/cgi-bin/texfaq2html?label=latexwords
% you have to redefine the makros for the
% language you use, e.g., american, ngerman
% (as chosen when loading babel/AtBeginDocument)
% ********************************************************************
%\PassOptionsToPackage{american,italian}{babel}   % change this to your language(s)
% Spanish languages need extra options in order to work with this template
%\PassOptionsToPackage{spanish,es-lcroman}{babel}
% \usepackage{babel}
\makeatletter
\@ifpackageloaded{babel}%
    {%
       \addto\extrasamerican{%
			\renewcommand*{\figureautorefname}{Figure}%
			\renewcommand*{\tableautorefname}{Table}%
			\renewcommand*{\partautorefname}{Part}%
			\renewcommand*{\chapterautorefname}{Chapter}%
			\renewcommand*{\sectionautorefname}{Section}%
			\renewcommand*{\subsectionautorefname}{Section}%
			\renewcommand*{\subsubsectionautorefname}{Section}%
                }%
       \addto\extrasngerman{%
			\renewcommand*{\paragraphautorefname}{Absatz}%
			\renewcommand*{\subparagraphautorefname}{Unterabsatz}%
			\renewcommand*{\footnoteautorefname}{Fu\"snote}%
			\renewcommand*{\FancyVerbLineautorefname}{Zeile}%
			\renewcommand*{\theoremautorefname}{Theorem}%
			\renewcommand*{\appendixautorefname}{Anhang}%
			\renewcommand*{\equationautorefname}{Gleichung}%
			\renewcommand*{\itemautorefname}{Punkt}%
                }%
            % Fix to getting autorefs for subfigures right (thanks to Belinda Vogt for changing the definition)
            \providecommand{\subfigureautorefname}{\figureautorefname}%
    }{\relax}
\makeatother


% ****************************************************************************************************
% 7. Last calls before the bar closes
% ****************************************************************************************************
% ********************************************************************
% Development Stuff
% ********************************************************************
%\listfiles
%\PassOptionsToPackage{l2tabu,orthodox,abort}{nag}
%   \usepackage{nag}
%\PassOptionsToPackage{warning, all}{onlyamsmath}
%   \usepackage{onlyamsmath}

% ********************************************************************
% Last, but not least...
% ********************************************************************
\usepackage{classicthesis}
% ****************************************************************************************************


% ****************************************************************************************************
% 8. Further adjustments (experimental)
% ****************************************************************************************************
% ********************************************************************
% Changing the text area
% ********************************************************************
%\linespread{1.05} % a bit more for Palatino
\areaset[2cm]{410pt}{700pt}
%\setlength{\marginparwidth}{7em}%
%\setlength{\marginparsep}{2em}%

% ********************************************************************
% Using different fonts
% ********************************************************************
%\usepackage[oldstylenums]{kpfonts} % oldstyle notextcomp
%\usepackage[osf]{libertine}
%\usepackage[light,condensed,math]{iwona}
%\renewcommand{\sfdefault}{iwona}
%\usepackage{lmodern} % <-- no osf support :-(
%\usepackage{cfr-lm} %
%\usepackage[urw-garamond]{mathdesign} <-- no osf support :-(
%\usepackage[default,osfigures]{opensans} % scale=0.95
%\usepackage[sfdefault]{FiraSans}
% ****************************************************************************************************

% ****************************************************************************************************
% 9. MY OWN ADJUSTMENTS
% ****************************************************************************************************

\newlength{\drop}% for my convenience

\usepackage{xltxtra}
%%% Font stuff
\usepackage{amssymb}
\usepackage{fontspec}
\setmainfont[Mapping=tex-text]{MinionPro-Regular}
\setsansfont[Scale=0.85]{Open Sans}
\setmonofont[Scale=0.8]{Bitstream Vera Sans Mono}
\usepackage[math-style=ISO,
            bold-style=ISO]{unicode-math}
\setmathfont{xits-math}[
Path           = fonts/,
Extension      = .otf,
BoldFont       = *bold,
]
\usepackage{microtype}
\usepackage{booktabs}
\usepackage{url}
\usepackage[colorinlistoftodos,
            textsize=small]{todonotes}
\usepackage{wrapfig}

\usepackage{siunitx}

%%% To include the published papers
\usepackage{pdfpages}

\usepackage{epigraph}
\setlength\epigraphwidth{7cm}
\setlength\epigraphrule{0pt}
\renewcommand{\textflush}{flushleft}
\renewcommand{\epigraphsize}{\footnotesize}

\usepackage[dvipsnames]{xcolor}
\usepackage[overload]{empheq}
\usepackage{braket}
\usepackage{cancel}
\usepackage{amsthm}
\allowdisplaybreaks[4]

% Comment before compiling final version
%\usepackage{showkeys}
\usepackage{verbatim}

\usepackage[bottom]{footmisc}

\usepackage[inline]{enumitem}

\usepackage[version=4]{mhchem}
\usepackage{xpatch}

\usepackage{tikz}
\usepackage{tikz-3dplot}
\usetikzlibrary{shapes,arrows,shadows,positioning}
\usepackage{tcolorbox}

\usepackage[style=long,
            nolist,
            nonumberlist,
            acronym,
            shortcuts,
            nopostdot]{glossaries}
\makeglossaries
\usepackage[noprefix]{nomencl}
\makenomenclature

%%% Colors
% Blue
\definecolor{PMS2229}{RGB}{0, 156, 182}
% Red
\definecolor{PMS1797}{RGB}{203, 51, 59}
% Orange
\definecolor{PMS138}{RGB}{222, 124, 0}
% Yellow
\definecolor{PMS130}{RGB}{242, 169, 0}

% ********************************************************************
% Hyperreferences
% ********************************************************************
\hypersetup{%
    %draft, % = no hyperlinking at all (useful in b/w printouts)
    colorlinks=true, linktocpage=true, pdfstartpage=3, pdfstartview=FitV,%
    % uncomment the following line if you want to have black links (e.g., for printing)
    %colorlinks=false, linktocpage=false, pdfstartpage=3, pdfstartview=FitV, pdfborder={0 0 0},%
    breaklinks=true, pdfpagemode=UseNone, pageanchor=true, pdfpagemode=UseOutlines,%
    plainpages=false, bookmarksnumbered, bookmarksopen=true, bookmarksopenlevel=1,%
    hypertexnames=true, pdfhighlight=/O,%nesting=true,%frenchlinks,%
    urlcolor=PMS138, linkcolor=PMS2229, citecolor=webgreen, %pagecolor=RoyalBlue,%
    %urlcolor=Black, linkcolor=Black, citecolor=Black, %pagecolor=Black,%
    pdftitle={\myTitle},%
    pdfauthor={\textcopyright\ \myName, \myUni, \myFaculty},%
    pdfsubject={},%
    pdfkeywords={},%
    pdfcreator={XeLaTeX},%
    pdfproducer={LaTeX with hyperref and classicthesis}%
}

% ********************************************************************
% Listings
% ********************************************************************
\definecolor{mygreen}{rgb}{0,0.6,0}
\definecolor{mygray}{rgb}{0.5,0.5,0.5}
\definecolor{mymauve}{rgb}{0.58,0,0.82}

\lstset{ %
  backgroundcolor=\color{white},     % choose the background color; you must add \usepackage{color} or \usepackage{xcolor}
  basicstyle=\scriptsize\ttfamily, % the size of the fonts that are used for the code
  breakatwhitespace=false,         % sets if automatic breaks should only happen at whitespace
  breaklines=true,                 % sets automatic line breaking
  captionpos=t,                    % sets the caption-position to bottom
  commentstyle=\color{mygreen},    % comment style
  deletekeywords={...},            % if you want to delete keywords from the given language
  escapeinside={\%*}{*)},          % if you want to add LaTeX within your code
  extendedchars=true,              % lets you use non-ASCII characters; for 8-bits encodings only, does not work with UTF-8
  frame=single,	                   % adds a frame around the code
  keepspaces=true,                 % keeps spaces in text, useful for keeping indentation of code (possibly needs columns=flexible)
  keywordstyle=\color{blue},       % keyword style
  otherkeywords={*,...},           % if you want to add more keywords to the set
  numbers=none,                    % where to put the line-numbers; possible values are (none, left, right)
  rulecolor=\color{black},         % if not set, the frame-color may be changed on line-breaks within not-black text (e.g. comments (green here))
  showspaces=false,                % show spaces everywhere adding particular underscores; it overrides 'showstringspaces'
  showstringspaces=false,          % underline spaces within strings only
  showtabs=false,                  % show tabs within strings adding particular underscores
  stepnumber=2,                    % the step between two line-numbers. If it's 1, each line will be numbered
  stringstyle=\color{mymauve},     % string literal style
  tabsize=2,	                   % sets default tabsize to 2 spaces
  title=\lstname                   % show the filename of files included with \lstinputlisting; also try caption instead of title
}

% Change font for sections in TOC
\renewcommand\cftsecfont{\footnotesize}
\renewcommand\cftsecpagefont{\footnotesize}

%%%%% Units of measure
\DeclareSIUnit[number-unit-product = \;]\debye{D}
\DeclareSIUnit[number-unit-product = \;]\au{a.u.}
\DeclareSIUnit{\calorie}{cal}

%%%%% URLs
\newcommand{\GitHub}{\href{https://github.com/robertodr}{GitHub}\xspace}
\newcommand{\GitLab}{\href{https://gitlab.com/u/robertodr}{GitLab}\xspace}
\newcommand{\pcmsolver}{\href{http://pcmsolver.readthedocs.io}{\textsc{PCMSolver}}\xspace}
\newcommand{\psicode}{\href{http://www.psicode.org}{\textsc{Psi4}}\xspace}
\newcommand{\DIRAC}{\href{http://www.diracprogram.org}{\textsc{DIRAC}}\xspace}
\newcommand{\DALTON}{\href{http://www.daltonprogram.org}{\textsc{Dalton}}\xspace}
\newcommand{\LSDALTON}{\href{http://www.daltonprogram.org}{\textsc{LSDalton}}\xspace}
\newcommand{\ReSpect}{\href{http://respect.readthedocs.io/}{\textsc{ReSpect}}\xspace}
\newcommand{\git}{\href{https://git-scm.com/}{\texttt{git}}\xspace}
\newcommand{\cmake}{\href{https://cmake.org/}{\texttt{CMake}}\xspace}
\newcommand{\readthedocs}{\href{https://readthedocs.org/}{Read The Docs}\xspace}
\newcommand{\GPL}{\href{https://www.gnu.org/licenses/gpl.html}{GPL}\xspace}
\newcommand{\LGPL}{\href{https://www.gnu.org/licenses/lgpl.html}{LGPL}\xspace}
\newcommand{\GetKw}{\href{https://github.com/juselius/libgetkw}{GetKw}\xspace}
\newcommand{\Boost}{\href{http://www.boost.org/}{Boost}\xspace}
\newcommand{\Eigen}{\href{http://eigen.tuxfamily.org/index.php?title=Main_Page}{Eigen}\xspace}
\newcommand{\Taylor}{\href{https://github.com/uekstrom/libtaylor}{Taylor}\xspace}

%%%%% General
\newcounter{dummy} % necessary for correct hyperlinks (to index, bib, etc.)
\newcommand*{\diff}{\mathop{}\!\mathrm{d}} % Differential
\newcommand*{\Supp}{\mathop{}\!\mathrm{Supp}}
\newcommand*{\deriv}[3][]{% Derivative
\frac{\diff^{#1}#2}{\diff #3^{#1}}}
\newcommand*{\pderiv}[3][]{%
\frac{\partial^{#1}#2}% Partial derivative
{\partial #3^{#1}}}
\newcommand*{\vect}[1]{\symbf{#1}} % Vector
\newcommand*{\mat}[1]{\symbf{#1}}% Matrix
\newcommand*{\Tr}{\mathrm{Tr}}% Trace
\newcommand*{\supp}[1]{\Supp(#1)}
\newcommand*{\conv}{\mathop{}\!\mathrm{Conv}\,}
\newcommand*{\hull}{\mathop{}\!\mathrm{Hull}\,}
\newcommand*{\sing}{\mathop{}\!\mathrm{Sing}\,}
\newcommand*{\dist}{\mathop{}\!\mathrm{Dist}}
\newcommand*{\Span}[1]{\mathop{}\!\mathrm{Span}\{#1\}}
\newcommand*{\norm}[2]{\| #1 \|_{#2}}
\newcommand*{\paper}[1]{\textbf{Paper #1}}
\newcommand{\finalVersion}{Final version as of \today}
\newcommand{\submittedVersion}{Final version as of October 5, 2016}
\DeclareRobustCommand{\T1}{T_1}
% Inserts a blank page
\newcommand{\blankpage}{\newpage\hbox{}\thispagestyle{empty}\newpage}
\newcommand{\hairsp}{\hspace{1pt}}% hair space
\newcommand{\ie}{\textit{i.\hairsp{}e.}\xspace}
\newcommand{\Ie}{\textit{I.\hairsp{}e.}\xspace}
\newcommand{\eg}{\textit{e.\hairsp{}g.}\xspace}
\newcommand{\Eg}{\textit{E.\hairsp{}g.}\xspace}

%%
% Amount of space to skip before \newthought or after title block
\newskip\tufteskipamount
\tufteskipamount=1.0\baselineskip plus 0.5ex minus 0.2ex

\newcommand{\tuftebreak}{\par\ifdim\lastskip<\tufteskipamount
  \removelastskip\penalty-100\tufteskip\fi}

\newcommand{\tufteskip}{\vspace\tufteskipamount}

%%
% A newthought
\providecommand\newthought[1]{%
   \tuftebreak
   \noindent\textsc{#1}%
}

%%
% A newthought, with a small arrow on the margin
\providecommand\arrowthought[1]{%
   \tuftebreak
   \begin{tikzpicture}
   \node[rotate=270] {\tikz\draw[draw=brewerCyan, fill=brewerCyan, -triangle 45](0,0) ;};
   \end{tikzpicture}
   \noindent\textsc{#1}%
}

%%
% A newthought without tuftebreak
\providecommand\begthought[1]{%
   \noindent\textsc{#1}%
}

%%
% A newthought without tuftebreak but with a small arrow on the margin
\providecommand\begarrowthought[1]{%
   \begin{tikzpicture}
   \node[rotate=270] {\tikz\draw[draw=brewerCyan, fill=brewerCyan, -triangle 45](0,0) ;};
   \end{tikzpicture}
   \noindent\textsc{#1}%
}

%%%%% PCM stuff
\newcommand*{\diel}{\varepsilon} % Permittivity
\newcommand*{\diels}{\varepsilon_\mathrm{s}}
\newcommand*{\dield}{\varepsilon_\infty}
\newcommand*{\bi}[1]{\hat{\mathcal{#1}}} % Boundary integral operator
\newcommand*{\PCM}{\bi{Y}} % The PCM boundary integral operator
\newcommand*{\MM}{\bi{W}} % The MM classical matrix
\newcommand*{\MMPCM}{\bi{X}} % The MM/PCM interaction kernel
\newcommand*{\scalprod}[3][\Gamma]{\left(#2, #3\right)_{#1}} % Scalar product
\newcommand*{\s}{\mathrm{s}} % Sources
\newcommand*{\p}{\mathrm{p}} % Polarization degree of freedom
\newcommand*{\V}{\mathbb{V}} % Supermatrix
\newcommand*{\asc}{\sigma} % ASC
\newcommand*{\esp}{\varphi} % MEP
\newcommand*{\Ylm}[4]{\mathcal{Y}_{#1}^{#2}(#3, #4)}
\newcommand*{\BarYlm}[4]{\overline{\mathcal{Y}}_{#1}^{#2}(#3, #4)}
\newcommand*{\Li}{L_\mathrm{i}}
\newcommand*{\ui}{u_\mathrm{i}}
\newcommand*{\rhoi}{\rho_\mathrm{i}}
\newcommand*{\Omegai}{\Omega_\mathrm{i}}
\newcommand*{\partiali}{\partial_{L_\mathrm{i}}}
\newcommand*{\Gi}{G_\mathrm{i}(\vect{r}, \vect{r}^\prime)}
\newcommand*{\Le}{L_\mathrm{e}}
\newcommand*{\ue}{u_\mathrm{e}}
\newcommand*{\rhoe}{\rho_\mathrm{e}}
\newcommand*{\Omegae}{\Omega_\mathrm{e}}
\newcommand*{\partiale}{\partial_{L_\mathrm{e}}}
\newcommand*{\Ge}{G_\mathrm{e}(\vect{r}, \vect{r}^\prime)}

%%%%% Response theory stuff
\newcommand*{\aveQ}{\left\lbrace Q(t)\right\rbrace_T}% Time-averaged quasi-energy
\newcommand*{\response}[3]{\braket{\braket{#1; #2}}_{#3}}% Response function
\newcommand*{\pertmat}[1]{\tilde{\mat{#1}}}
\newcommand*{\aveL}{\lbrace\tilde{L}(\pertmat{C}, \pertmat{\lambda}, \tilde{\sigma}, t)\rbrace_T}% Time-averaged quasi-energy
\newcommand*{\aveLa}{\lbrace\tilde{L}^a(\pertmat{C}, \pertmat{\lambda}, \tilde{\sigma}, t)\rbrace_T}% Time-averaged quasi-energy
\newcommand*{\pertPCM}[1]{\PCM^{(#1)}}
\newcommand*{\eqtr}{\overset{\Tr}{=}}
\newcommand*{\eqavetr}{\overset{\lbrace\Tr\rbrace_T}{=}}
\newcommand*{\herm}[1]{\left[#1\right]^{\oplus}}
\newcommand*{\aherm}[1]{\left[#1\right]^{\ominus}}
\newcommand*{\hermitian}[1]{\mat{#1} + \mat{#1}^\dagger}
\newcommand*{\ahermitian}[1]{\mat{#1} - \mat{#1}^\dagger}
\newcommand*{\breveFock}[1]{\breve{\mat{\mathcal{F}}}_\omega^{#1}}
\newcommand*{\rspParam}[1]{\mat{X}^{#1}}
\newcommand*{\RHS}[1]{\mat{M}_\mathrm{RHS}^{#1}}
\newcommand*{\Dp}[1]{\mat{D}_\mathrm{P}^{#1}}
\newcommand*{\Dh}[1]{\mat{D}_\mathrm{H}^{#1}}
\newcommand*{\ASCp}[1]{\asc_\mathrm{P}^{#1}}
\newcommand*{\ASCh}[1]{\asc_\mathrm{H}^{#1}}
\newcommand*{\Gn}[1]{\mathcal{G}^{0, #1}}
\newcommand*{\Gderiv}[2]{\mat{\mathcal{G}}^{#1, #2}}
\newcommand*{\genHessian}{\mat{E}^{[2]}}
\newcommand*{\genMetric}{\mat{S}^{[2]}}

%%%%% Coupled cluster stuff
\newcommand*{\commutator}[3][]{[ #2, #3 ]^{#1}} % Commutator
\newcommand*{\anticommutator}[2]{% Anticommutator
[ #1, #2 ]_{+}}
\newcommand*{\lag}{\mathcal{L}_\mathrm{eff}}
%\newcommand*{\lagCC2}{\mathcal{L}_\mathrm{eff}(\tamp{}, \tbar{}, \p)_\mathrm{CC2}}
%\newcommand*{\lagCC3}{\mathcal{L}_\mathrm{eff}(\tamp{}, \tbar{}, \p)_\mathrm{CC3}}
\newcommand*{\expo}[1]{\mathrm{e}^{#1}}
\newcommand*{\twoel}[2]{\Braket{#1 \| #2}}
\newcommand*{\tamp}[1]{t_{#1}} % Cluster amplitude
\newcommand*{\tampEq}[1]{\Omega_{#1}} % Cluster amplitude vector function
\newcommand*{\sd}[1]{{}^*#1}
\newcommand*{\tbar}[1]{\bar{t}_{#1}} % Multiplier
\newcommand*{\tbarEq}[1]{\bar{\Omega}_{#1}} % Multiplier vector function
\newcommand*{\cluster}[1]{\tau_{#1}} % Cluster operator
\newcommand*{\denom}[1]{\epsilon_{#1}} % Energy denominator
\newcommand*{\cons}[1]{a_{#1}^{\dagger}}% Constructor operator
\newcommand*{\anni}[1]{a_{#1}} % Annihilation operator
\newcommand*{\sigmaEq}{\Omega^{\sigma}} % Multiplier vector function
\newcommand*{\BCH}[2]{#1+\commutator{#1}{#2}+\frac{1}{2}\commutator{\commutator{#1}{#2}}{#2}
+ \ldots} % BCH expansion up to second order
\newcommand*{\BCHfirst}[2]{\commutator{#1}{#2}}% BCH 1st order
\newcommand*{\BCHsecond}[2]{\frac{1}{2}\commutator{\commutator{#1}{#2}}{#2}}% BCH 2nd order
\newcommand*{\BCHthird}[2]{\frac{1}{6}\commutator{\commutator{\commutator{#1}{#2}}{#2}}{#2}} %BCH 3rd order
% BCH 4th order
\newcommand*{\BCHfourth}[2]{\frac{1}{24}\commutator{\commutator{\commutator{\commutator{#1}{#2}}{#2}}{#2}}{#2}}
% \Phi
\newcommand*{\RTerm}[3]{{}^{#1}R^{#2}_{#3}}
\newcommand*{\STerm}[3]{{}^{#1}S^{#2}_{#3}}
\newcommand*{\UTerm}[4]{{}^{#1}U^{#2}_{#3 #4}}
% \esp
\newcommand*{\MEPRTerm}[3]{{}^{#1}\underline{R}^{#2}_{#3}}
\newcommand*{\MEPSTerm}[3]{{}^{#1}\underline{S}^{#2}_{#3}}
\newcommand*{\MEPUTerm}[4]{{}^{#1}\underline{U}^{#2}_{#3 #4}}

% Box around equation
\newcommand{\highlight}[2]{%
  \colorbox{#1!50}{$\displaystyle#2$}}
% Color terms in equations
\newcommand*{\mathcolor}{}
\def\mathcolor#1#{\mathcoloraux{#1}}
\newcommand*{\mathcoloraux}[3]{%
  \protect\leavevmode
    \begingroup
        \color#1{#2}#3%
          \endgroup
}

% Theorem environments
%%% Definitions
\newtheoremstyle{classicdef}%             % Name
  {12pt}%                                 % Space above
  {12pt}%                                 % Space below
  {}%                                     % Body font
  {}%                                     % Indent amount
  {\bfseries\scshape}%                    % Theorem head font
  {.}%                                    % Punctuation after theorem head
  { }%                                    % Space after theorem head, ' ', or \newline
  {\thmname{#1}\thmnumber{ #2}\thmnote{ (#3)}}%

%%% Theorems, Lemmas, Corollaries
\newtheoremstyle{classicthm}%             % Name
  {12pt}%                                 % Space above
  {12pt}%                                 % Space below
  {\itshape}%                             % Body font
  {}%                                     % Indent amount
  {\bfseries\scshape}%                    % Theorem head font
  {.}%                                    % Punctuation after theorem head
  { }%                                    % Space after theorem head, ' ', or \newline
  {\thmname{#1}\thmnumber{ #2}\thmnote{ (#3)}}%

\theoremstyle{classicthm}
\newtheorem{theorem}{Theorem}
\newtheorem{lemma}{Lemma}
\newtheorem{proposition}{Proposition}
\newtheorem*{corollary}{Corollary}
\newtheorem{postulate}{Postulate}

\theoremstyle{classicdef}
\newtheorem{defin}{Definition}
\newtheorem{conjecture}{Conjecture}
\newtheorem{example}{Example}

\theoremstyle{remark}
\newtheorem*{rem}{Remark}
\newtheorem*{note}{Note}
\newtheorem{case}{Case}

\newenvironment{Proof}{\vspace*{1.5pt}\begin{proof} \small{}}{\end{proof}\vspace*{1.5pt}}
\renewenvironment{example}[1][]{\subparagraph*{Example} #1 \small{}}{}



% ****************************************************************************************************
% If you like the classicthesis, then I would appreciate a postcard.
% My address can be found in the file ClassicThesis.pdf. A collection
% of the postcards I received so far is available online at
% http://postcards.miede.de
% ****************************************************************************************************


% ****************************************************************************************************
% 0. Set the encoding of your files. UTF-8 is the only sensible encoding nowadays. If you can't read
% äöüßáéçèê∂åëæƒÏ€ then change the encoding setting in your editor, not the line below. If your editor
% does not support utf8 use another editor!
% ****************************************************************************************************
%\PassOptionsToPackage{utf8}{inputenc}
%      \usepackage{inputenc}

% ****************************************************************************************************
% 1. Configure classicthesis for your needs here, e.g., remove "drafting" below
% in order to deactivate the time-stamp on the pages
% ****************************************************************************************************
\PassOptionsToPackage{dottedtoc,
                      floatperchapter,
                      listings
                     }{classicthesis}
% ********************************************************************
% Available options for classicthesis.sty
% (see ClassicThesis.pdf for more information):
% drafting
% parts nochapters linedheaders
% eulerchapternumbers beramono eulermath pdfspacing minionprospacing
% tocaligned dottedtoc manychapters
% listings floatperchapter subfig
% ********************************************************************


% ****************************************************************************************************
% 2. Personal data and user ad-hoc commands
% ****************************************************************************************************
\newcommand{\myTitle}{The Polarizable Continuum Model Goes Viral!\xspace}
\newcommand{\mySubtitle}{Extensible, Modular and Sustainable Development of Quantum Mechanical Continuum Solvation Models\xspace}
\newcommand{\myDegree}{Philosphiae Doctor (Ph.~D.)\xspace}
\newcommand{\myName}{Roberto Di Remigio\xspace}
\newcommand{\myProf}{Luca Frediani\xspace}
\newcommand{\myOtherProf}{Benedetta Mennucci\xspace}
\newcommand{\mySupervisor}{Luca Frediani\xspace}
\newcommand{\myFaculty}{Fakultet for Naturvitenskap og Teknologi\xspace}
\newcommand{\myDepartment}{Institutt for Kjemi\xspace}
\newcommand{\myUni}{UiT -- Norges Arktiske Universitet\xspace}
\newcommand{\myLocation}{Tromsø\xspace}
\newcommand{\myTime}{January 2017\xspace}
\newcommand{\myVersion}{}

% ********************************************************************
% Setup, finetuning, and useful commands
% ********************************************************************
\newcounter{dummy} % necessary for correct hyperlinks (to index, bib, etc.)
\newlength{\abcd} % for ab..z string length calculation
\providecommand{\mLyX}{L\kern-.1667em\lower.25em\hbox{Y}\kern-.125emX\@}
% ****************************************************************************************************


% ****************************************************************************************************
% 3. Loading some handy packages
% ****************************************************************************************************
% ********************************************************************
% Packages with options that might require adjustments
% ********************************************************************
\usepackage{polyglossia}
\setmainlanguage{english}
\setotherlanguages{italian, french, norsk, greek}
\setdefaultlanguage{english}
\usepackage{csquotes}
\PassOptionsToPackage{%
style=phys,
maxcitenames=1,
mincitenames=1,
maxbibnames=100,
firstinits=true,
url=false,
isbn=false,
eprint=false,
texencoding=utf8,
bibencoding=utf8,
autocite=superscript,
backend=biber,
sorting=none,
backref=false,
hyperref=true,
block=none,
date=long,
urldate=long
}{biblatex}
    \usepackage{biblatex}
\renewcommand{\bibfont}{\normalfont\footnotesize\raggedright}
\AtBeginBibliography{
\DeclareFieldFormat{prefixnumber}{\mkbibbold{#1}}
\DeclareFieldFormat{labelnumber}{\mkbibbold{#1}}
}
\AtEveryBibitem{%
  \clearlist{language}%
}
\DeclareFieldFormat[article]{title}{\textsf{#1}}
\DeclareFieldFormat[inbook]{title}{\textsf{#1}}
\DeclareFieldFormat[incollection]{title}{\textsf{#1}}
\DeclareFieldFormat[inproceedings]{title}{\textsf{#1}}
\DeclareFieldFormat[inproceedings]{booktitle}{\textit{#1}}
\DeclareFieldFormat[inproceedings]{note}{#1}
\DeclareFieldFormat[unpublished]{title}{\textsf{#1}}

\DeclareCiteCommand{\noparcite}%[\mkbibbrackets] CITATION LIKE in ref. 6 WITHOUT SQUARE BRACKETS
  {\usebibmacro{cite:init}%
   \usebibmacro{prenote}}
  {\usebibmacro{citeindex}%
   \usebibmacro{cite:comp}}
  {}
  {\usebibmacro{cite:dump}%
   \usebibmacro{postnote}}

\PassOptionsToPackage{fleqn}{amsmath}       % math environments and more by the AMS
    \usepackage{amsmath}

% ********************************************************************
% General useful packages
% ********************************************************************
\usepackage{textcomp} % fix warning with missing font shapes
\usepackage{scrhack} % fix warnings when using KOMA with listings package
\usepackage{xspace} % to get the spacing after macros right
\usepackage{mparhack} % get marginpar right
\usepackage{fixltx2e} % fixes some LaTeX stuff --> since 2015 in the LaTeX kernel (see below)
%\usepackage[latest]{latexrelease} % will be used once available in more distributions (ISSUE #107)

% ****************************************************************************************************


% ****************************************************************************************************
% 4. Setup floats: tables, (sub)figures, and captions
% ****************************************************************************************************
\usepackage{tabularx} % better tables
    \setlength{\extrarowheight}{3pt} % increase table row height
\newcommand{\tableheadline}[1]{\multicolumn{1}{c}{\spacedlowsmallcaps{#1}}}
\newcommand{\myfloatalign}{\centering} % to be used with each float for alignment
\usepackage{caption}
% Thanks to cgnieder and Claus Lahiri
% http://tex.stackexchange.com/questions/69349/spacedlowsmallcaps-in-caption-label
% [REMOVED DUE TO OTHER PROBLEMS, SEE ISSUE #82]
%\DeclareCaptionLabelFormat{smallcaps}{\bothIfFirst{#1}{~}\MakeTextLowercase{\textsc{#2}}}
%\captionsetup{font=small,labelformat=smallcaps} % format=hang,
\captionsetup{format=hang,indention=-1.1cm,
              font=footnotesize,labelfont=bf,labelsep=space}
\usepackage{subcaption}
\captionsetup[sub]{format=hang,indention=0cm,
              font=footnotesize,labelfont=bf,labelsep=space}
% ****************************************************************************************************


% ****************************************************************************************************
% 5. Setup code listings
% ****************************************************************************************************
\usepackage{listings}
% ****************************************************************************************************


% ****************************************************************************************************
% 6. PDFLaTeX, hyperreferences and citation backreferences
% ****************************************************************************************************
% ********************************************************************
% Using PDFLaTeX
% ********************************************************************
\PassOptionsToPackage{hyperfootnotes=false,pdfpagelabels}{hyperref}
    \usepackage{hyperref}  % backref linktocpage pagebackref
\usepackage{graphicx}
\graphicspath{{gfx/}}

% ********************************************************************
% Setup autoreferences
% ********************************************************************
% There are some issues regarding autorefnames
% http://www.ureader.de/msg/136221647.aspx
% http://www.tex.ac.uk/cgi-bin/texfaq2html?label=latexwords
% you have to redefine the makros for the
% language you use, e.g., american, ngerman
% (as chosen when loading babel/AtBeginDocument)
% ********************************************************************
%\PassOptionsToPackage{american,italian}{babel}   % change this to your language(s)
% Spanish languages need extra options in order to work with this template
%\PassOptionsToPackage{spanish,es-lcroman}{babel}
% \usepackage{babel}
\makeatletter
\@ifpackageloaded{babel}%
    {%
       \addto\extrasamerican{%
			\renewcommand*{\figureautorefname}{Figure}%
			\renewcommand*{\tableautorefname}{Table}%
			\renewcommand*{\partautorefname}{Part}%
			\renewcommand*{\chapterautorefname}{Chapter}%
			\renewcommand*{\sectionautorefname}{Section}%
			\renewcommand*{\subsectionautorefname}{Section}%
			\renewcommand*{\subsubsectionautorefname}{Section}%
                }%
       \addto\extrasngerman{%
			\renewcommand*{\paragraphautorefname}{Absatz}%
			\renewcommand*{\subparagraphautorefname}{Unterabsatz}%
			\renewcommand*{\footnoteautorefname}{Fu\"snote}%
			\renewcommand*{\FancyVerbLineautorefname}{Zeile}%
			\renewcommand*{\theoremautorefname}{Theorem}%
			\renewcommand*{\appendixautorefname}{Anhang}%
			\renewcommand*{\equationautorefname}{Gleichung}%
			\renewcommand*{\itemautorefname}{Punkt}%
                }%
            % Fix to getting autorefs for subfigures right (thanks to Belinda Vogt for changing the definition)
            \providecommand{\subfigureautorefname}{\figureautorefname}%
    }{\relax}
\makeatother


% ****************************************************************************************************
% 7. Last calls before the bar closes
% ****************************************************************************************************
% ********************************************************************
% Development Stuff
% ********************************************************************
%\listfiles
%\PassOptionsToPackage{l2tabu,orthodox,abort}{nag}
%   \usepackage{nag}
%\PassOptionsToPackage{warning, all}{onlyamsmath}
%   \usepackage{onlyamsmath}

% ********************************************************************
% Last, but not least...
% ********************************************************************
\usepackage{classicthesis}
% ****************************************************************************************************


% ****************************************************************************************************
% 8. Further adjustments (experimental)
% ****************************************************************************************************
% ********************************************************************
% Changing the text area
% ********************************************************************
%\linespread{1.05} % a bit more for Palatino
\RequirePackage{geometry}
\geometry{paperwidth=170mm, paperheight=240mm,
          inner=42pt, %outer=,
          top=80pt, %bottom=,
          textheight=500pt, textwidth=280pt,
          marginparsep=20pt, marginparwidth=100pt,
          footskip=40pt
        }
\savegeometry{Tufte}
%\areaset[2cm]{410pt}{700pt}
%\setlength{\marginparwidth}{7em}%
%\setlength{\marginparsep}{2em}%

% ********************************************************************
% Using different fonts
% ********************************************************************
%\usepackage[oldstylenums]{kpfonts} % oldstyle notextcomp
%\usepackage[osf]{libertine}
%\usepackage[light,condensed,math]{iwona}
%\renewcommand{\sfdefault}{iwona}
%\usepackage{lmodern} % <-- no osf support :-(
%\usepackage{cfr-lm} %
%\usepackage[urw-garamond]{mathdesign} <-- no osf support :-(
%\usepackage[default,osfigures]{opensans} % scale=0.95
%\usepackage[sfdefault]{FiraSans}
% ****************************************************************************************************

% ****************************************************************************************************
% 9. MY OWN ADJUSTMENTS
% ****************************************************************************************************

\newlength{\drop}% for my convenience

\usepackage{metalogo}
%%% Font stuff
\usepackage{amssymb}
\usepackage{fontspec}
\setmainfont[Ligatures=TeX, Numbers=OldStyle, Contextuals=Swash]{TeX Gyre Termes}
\newfontfamily\greekfont[Script=Greek]{Linux Libertine O}
\newfontfamily\greekfontsf[Script=Greek]{Linux Libertine O}
\setsansfont[Scale=0.85]{TeX Gyre Heros}
\setmonofont[Scale=0.8]{Bitstream Vera Sans Mono}
\usepackage[math-style=ISO,
            bold-style=ISO]{unicode-math}
\setmathfont{xits-math}[
Path           = fonts/,
Extension      = .otf,
BoldFont       = *bold,
]
\microtypesetup{activate={true,nocompatibility},final,factor=1100,stretch=10,shrink=10}
\usepackage{booktabs}
\usepackage{url}
\usepackage[colorinlistoftodos,
            textsize=small]{todonotes}
\usepackage{wrapfig}

\usepackage{siunitx}

%%% To include the published papers
\usepackage{pdfpages}

\usepackage[symbol, side, flushmargin]{footmisc}
\renewcommand*{\footnotelayout}{\scriptsize}
\setfnsymbol{bringhurst}

\usepackage{epigraph}
%\setlength\epigraphwidth{7cm}
\setlength\epigraphrule{0pt}
%\renewcommand{\textflush}{flushleft}
\renewcommand{\epigraphsize}{\footnotesize}

\usepackage[dvipsnames]{xcolor}
\usepackage[overload]{empheq}
\usepackage{braket}
\usepackage{cancel}
\usepackage{amsthm}

% Comment before compiling final version
%\usepackage{showkeys}
\usepackage{verbatim}

\usepackage[inline]{enumitem}

\usepackage[version=4]{mhchem}
\usepackage{xpatch}

\usepackage{tikz}
\usepackage{tikz-3dplot}
\usetikzlibrary{shapes,arrows,shadows,positioning}
\usepackage{tcolorbox}

\usepackage[style=long,
            nolist,
            nonumberlist,
            acronym,
            shortcuts,
            nopostdot]{glossaries}
\makeglossaries
\usepackage[noprefix]{nomencl}
\makenomenclature

%%% Colors
% Blue
\definecolor{PMS2229}{RGB}{0, 156, 182}
% Red
\definecolor{PMS1797}{RGB}{203, 51, 59}
% Orange
\definecolor{PMS138}{RGB}{222, 124, 0}
% Yellow
\definecolor{PMS130}{RGB}{242, 169, 0}

% ********************************************************************
% Hyperreferences
% ********************************************************************
\hypersetup{%
    %draft, % = no hyperlinking at all (useful in b/w printouts)
    colorlinks=true, linktocpage=true, pdfstartpage=3, pdfstartview=FitV,%
    % uncomment the following line if you want to have black links (e.g., for printing)
    %colorlinks=false, linktocpage=false, pdfstartpage=3, pdfstartview=FitV, pdfborder={0 0 0},%
    breaklinks=true, pdfpagemode=UseNone, pageanchor=true, pdfpagemode=UseOutlines,%
    plainpages=false, bookmarksnumbered, bookmarksopen=true, bookmarksopenlevel=1,%
    hypertexnames=true, pdfhighlight=/O,%nesting=true,%frenchlinks,%
    urlcolor=PMS138, linkcolor=PMS2229, citecolor=webgreen, %pagecolor=RoyalBlue,%
    %urlcolor=Black, linkcolor=Black, citecolor=Black, %pagecolor=Black,%
    pdftitle={\myTitle},%
    pdfauthor={\textcopyright\ \myName, \myUni, \myFaculty},%
    pdfsubject={},%
    pdfkeywords={},%
    pdfcreator={XeLaTeX},%
    pdfproducer={LaTeX with hyperref and classicthesis}%
}

% ********************************************************************
% Listings
% ********************************************************************
\definecolor{mygreen}{rgb}{0,0.6,0}
\definecolor{mygray}{rgb}{0.5,0.5,0.5}
\definecolor{mymauve}{rgb}{0.58,0,0.82}

\lstset{ %
  backgroundcolor=\color{white},     % choose the background color; you must add \usepackage{color} or \usepackage{xcolor}
  basicstyle=\scriptsize\ttfamily, % the size of the fonts that are used for the code
  breakatwhitespace=false,         % sets if automatic breaks should only happen at whitespace
  breaklines=true,                 % sets automatic line breaking
  captionpos=t,                    % sets the caption-position to bottom
  commentstyle=\color{mygreen},    % comment style
  deletekeywords={...},            % if you want to delete keywords from the given language
  escapeinside={\%*}{*)},          % if you want to add LaTeX within your code
  extendedchars=true,              % lets you use non-ASCII characters; for 8-bits encodings only, does not work with UTF-8
  frame=single,	                   % adds a frame around the code
  keepspaces=true,                 % keeps spaces in text, useful for keeping indentation of code (possibly needs columns=flexible)
  keywordstyle=\color{blue},       % keyword style
  otherkeywords={*,...},           % if you want to add more keywords to the set
  numbers=none,                    % where to put the line-numbers; possible values are (none, left, right)
  rulecolor=\color{black},         % if not set, the frame-color may be changed on line-breaks within not-black text (e.g. comments (green here))
  showspaces=false,                % show spaces everywhere adding particular underscores; it overrides 'showstringspaces'
  showstringspaces=false,          % underline spaces within strings only
  showtabs=false,                  % show tabs within strings adding particular underscores
  stepnumber=2,                    % the step between two line-numbers. If it's 1, each line will be numbered
  stringstyle=\color{mymauve},     % string literal style
  tabsize=2,	                   % sets default tabsize to 2 spaces
  title=\lstname                   % show the filename of files included with \lstinputlisting; also try caption instead of title
}

% Change font for sections in TOC
\renewcommand\cftsecfont{\footnotesize}
\renewcommand\cftsecpagefont{\footnotesize}

%%%%% Units of measure
\DeclareSIUnit[number-unit-product = \;]\debye{D}
\DeclareSIUnit[number-unit-product = \;]\au{a.u.}
\DeclareSIUnit{\calorie}{cal}

%%%%% URLs
\newcommand{\GitHub}{\href{https://github.com/robertodr}{GitHub}\xspace}
\newcommand{\GitLab}{\href{https://gitlab.com/u/robertodr}{GitLab}\xspace}
\newcommand{\pcmsolver}{\href{http://pcmsolver.readthedocs.io}{\textsc{PCMSolver}}\xspace}
\newcommand{\psicode}{\href{http://www.psicode.org}{\textsc{Psi4}}\xspace}
\newcommand{\DIRAC}{\href{http://www.diracprogram.org}{\textsc{DIRAC}}\xspace}
\newcommand{\DALTON}{\href{http://www.daltonprogram.org}{\textsc{Dalton}}\xspace}
\newcommand{\LSDALTON}{\href{http://www.daltonprogram.org}{\textsc{LSDalton}}\xspace}
\newcommand{\ReSpect}{\href{http://respect.readthedocs.io/}{\textsc{ReSpect}}\xspace}
\newcommand{\git}{\href{https://git-scm.com/}{\texttt{git}}\xspace}
\newcommand{\cmake}{\href{https://cmake.org/}{\texttt{CMake}}\xspace}
\newcommand{\readthedocs}{\href{https://readthedocs.org/}{Read The Docs}\xspace}
\newcommand{\GPL}{\href{https://www.gnu.org/licenses/gpl.html}{GPL}\xspace}
\newcommand{\LGPL}{\href{https://www.gnu.org/licenses/lgpl.html}{LGPL}\xspace}
\newcommand{\GetKw}{\href{https://github.com/juselius/libgetkw}{GetKw}\xspace}
\newcommand{\Boost}{\href{http://www.boost.org/}{Boost}\xspace}
\newcommand{\Eigen}{\href{http://eigen.tuxfamily.org/index.php?title=Main_Page}{Eigen}\xspace}
\newcommand{\Taylor}{\href{https://github.com/uekstrom/libtaylor}{Taylor}\xspace}

%%%%% General
\newcounter{dummy} % necessary for correct hyperlinks (to index, bib, etc.)
\newcommand*{\diff}{\mathop{}\!\mathrm{d}} % Differential
\newcommand*{\Supp}{\mathop{}\!\mathrm{Supp}}
\newcommand*{\deriv}[3][]{% Derivative
\frac{\diff^{#1}#2}{\diff #3^{#1}}}
\newcommand*{\pderiv}[3][]{%
\frac{\partial^{#1}#2}% Partial derivative
{\partial #3^{#1}}}
\newcommand*{\vect}[1]{\symbf{#1}} % Vector
\newcommand*{\mat}[1]{\symbf{#1}}% Matrix
\newcommand*{\Tr}{\mathrm{Tr}}% Trace
\newcommand*{\supp}[1]{\Supp(#1)}
\newcommand*{\conv}{\mathop{}\!\mathrm{Conv}\,}
\newcommand*{\hull}{\mathop{}\!\mathrm{Hull}\,}
\newcommand*{\sing}{\mathop{}\!\mathrm{Sing}\,}
\newcommand*{\dist}{\mathop{}\!\mathrm{Dist}}
\newcommand*{\Span}[1]{\mathop{}\!\mathrm{Span}\{#1\}}
\newcommand*{\norm}[2]{\| #1 \|_{#2}}
\newcommand*{\paper}[1]{\textbf{Paper #1}}
\newcommand{\finalVersion}{Final version as of \today}
\newcommand{\submittedVersion}{Final version as of October 5, 2016}
\DeclareRobustCommand{\T1}{T_1}
% Inserts a blank page
\newcommand{\blankpage}{\newpage\hbox{}\thispagestyle{empty}\newpage}
\newcommand{\hairsp}{\hspace{1pt}}% hair space
\newcommand{\ie}{\textit{i.\hairsp{}e.}\xspace}
\newcommand{\Ie}{\textit{I.\hairsp{}e.}\xspace}
\newcommand{\eg}{\textit{e.\hairsp{}g.}\xspace}
\newcommand{\Eg}{\textit{E.\hairsp{}g.}\xspace}

%%
% Amount of space to skip before \newthought or after title block
\newskip\tufteskipamount
\tufteskipamount=1.0\baselineskip plus 0.5ex minus 0.2ex

\newcommand{\tuftebreak}{\par\ifdim\lastskip<\tufteskipamount
  \removelastskip\penalty-100\tufteskip\fi}

\newcommand{\tufteskip}{\vspace\tufteskipamount}

%%
% A newthought
\providecommand\newthought[1]{%
   \tuftebreak
   \noindent\textsc{#1}%
}

%%
% A newthought, with a small arrow on the margin
\providecommand\arrowthought[1]{%
   \tuftebreak
   \begin{tikzpicture}
   \node[rotate=270] {\tikz\draw[draw=brewerCyan, fill=brewerCyan, -triangle 45](0,0) ;};
   \end{tikzpicture}
   \noindent\textsc{#1}%
}

%%
% A newthought without tuftebreak
\providecommand\begthought[1]{%
   \noindent\textsc{#1}%
}

%%
% A newthought without tuftebreak but with a small arrow on the margin
\providecommand\begarrowthought[1]{%
   \begin{tikzpicture}
   \node[rotate=270] {\tikz\draw[draw=brewerCyan, fill=brewerCyan, -triangle 45](0,0) ;};
   \end{tikzpicture}
   \noindent\textsc{#1}%
}

%%%%% PCM stuff
\newcommand*{\diel}{\varepsilon} % Permittivity
\newcommand*{\diels}{\varepsilon_\mathrm{s}}
\newcommand*{\dield}{\varepsilon_\infty}
\newcommand*{\bi}[1]{\hat{\mathcal{#1}}} % Boundary integral operator
\newcommand*{\PCM}{\bi{Y}} % The PCM boundary integral operator
\newcommand*{\MM}{\bi{W}} % The MM classical matrix
\newcommand*{\MMPCM}{\bi{X}} % The MM/PCM interaction kernel
\newcommand*{\scalprod}[3][\Gamma]{\left(#2, #3\right)_{#1}} % Scalar product
\newcommand*{\s}{\mathrm{s}} % Sources
\newcommand*{\p}{\mathrm{p}} % Polarization degree of freedom
\newcommand*{\V}{\mathbb{V}} % Supermatrix
\newcommand*{\asc}{\sigma} % ASC
\newcommand*{\esp}{\varphi} % MEP
\newcommand*{\Ylm}[4]{\mathcal{Y}_{#1}^{#2}(#3, #4)}
\newcommand*{\BarYlm}[4]{\overline{\mathcal{Y}}_{#1}^{#2}(#3, #4)}
\newcommand*{\Li}{L_\mathrm{i}}
\newcommand*{\ui}{u_\mathrm{i}}
\newcommand*{\rhoi}{\rho_\mathrm{i}}
\newcommand*{\Omegai}{\Omega_\mathrm{i}}
\newcommand*{\partiali}{\partial_{L_\mathrm{i}}}
\newcommand*{\Gi}{G_\mathrm{i}(\vect{r}, \vect{r}^\prime)}
\newcommand*{\Le}{L_\mathrm{e}}
\newcommand*{\ue}{u_\mathrm{e}}
\newcommand*{\rhoe}{\rho_\mathrm{e}}
\newcommand*{\Omegae}{\Omega_\mathrm{e}}
\newcommand*{\partiale}{\partial_{L_\mathrm{e}}}
\newcommand*{\Ge}{G_\mathrm{e}(\vect{r}, \vect{r}^\prime)}

%%%%% Response theory stuff
\newcommand*{\aveQ}{\left\lbrace Q(t)\right\rbrace_T}% Time-averaged quasi-energy
\newcommand*{\response}[3]{\braket{\braket{#1; #2}}_{#3}}% Response function
\newcommand*{\pertmat}[1]{\tilde{\mat{#1}}}
\newcommand*{\aveL}{\lbrace\tilde{L}(\pertmat{C}, \pertmat{\lambda}, \tilde{\sigma}, t)\rbrace_T}% Time-averaged quasi-energy
\newcommand*{\aveLa}{\lbrace\tilde{L}^a(\pertmat{C}, \pertmat{\lambda}, \tilde{\sigma}, t)\rbrace_T}% Time-averaged quasi-energy
\newcommand*{\pertPCM}[1]{\PCM^{(#1)}}
\newcommand*{\eqtr}{\overset{\Tr}{=}}
\newcommand*{\eqavetr}{\overset{\lbrace\Tr\rbrace_T}{=}}
\newcommand*{\herm}[1]{\left[#1\right]^{\oplus}}
\newcommand*{\aherm}[1]{\left[#1\right]^{\ominus}}
\newcommand*{\hermitian}[1]{\mat{#1} + \mat{#1}^\dagger}
\newcommand*{\ahermitian}[1]{\mat{#1} - \mat{#1}^\dagger}
\newcommand*{\breveFock}[1]{\breve{\mat{\mathcal{F}}}_\omega^{#1}}
\newcommand*{\rspParam}[1]{\mat{X}^{#1}}
\newcommand*{\RHS}[1]{\mat{M}_\mathrm{RHS}^{#1}}
\newcommand*{\Dp}[1]{\mat{D}_\mathrm{P}^{#1}}
\newcommand*{\Dh}[1]{\mat{D}_\mathrm{H}^{#1}}
\newcommand*{\ASCp}[1]{\asc_\mathrm{P}^{#1}}
\newcommand*{\ASCh}[1]{\asc_\mathrm{H}^{#1}}
\newcommand*{\Gn}[1]{\mathcal{G}^{0, #1}}
\newcommand*{\Gderiv}[2]{\mat{\mathcal{G}}^{#1, #2}}
\newcommand*{\genHessian}{\mat{E}^{[2]}}
\newcommand*{\genMetric}{\mat{S}^{[2]}}

%%%%% Coupled cluster stuff
\newcommand*{\commutator}[3][]{[ #2, #3 ]^{#1}} % Commutator
\newcommand*{\anticommutator}[2]{% Anticommutator
[ #1, #2 ]_{+}}
\newcommand*{\lag}{\mathcal{L}_\mathrm{eff}}
%\newcommand*{\lagCC2}{\mathcal{L}_\mathrm{eff}(\tamp{}, \tbar{}, \p)_\mathrm{CC2}}
%\newcommand*{\lagCC3}{\mathcal{L}_\mathrm{eff}(\tamp{}, \tbar{}, \p)_\mathrm{CC3}}
\newcommand*{\expo}[1]{\mathrm{e}^{#1}}
\newcommand*{\twoel}[2]{\Braket{#1 \| #2}}
\newcommand*{\tamp}[1]{t_{#1}} % Cluster amplitude
\newcommand*{\tampEq}[1]{\Omega_{#1}} % Cluster amplitude vector function
\newcommand*{\sd}[1]{{}^*#1}
\newcommand*{\tbar}[1]{\bar{t}_{#1}} % Multiplier
\newcommand*{\tbarEq}[1]{\bar{\Omega}_{#1}} % Multiplier vector function
\newcommand*{\cluster}[1]{\tau_{#1}} % Cluster operator
\newcommand*{\denom}[1]{\epsilon_{#1}} % Energy denominator
\newcommand*{\cons}[1]{a_{#1}^{\dagger}}% Constructor operator
\newcommand*{\anni}[1]{a_{#1}} % Annihilation operator
\newcommand*{\sigmaEq}{\Omega^{\sigma}} % Multiplier vector function
\newcommand*{\BCH}[2]{#1+\commutator{#1}{#2}+\frac{1}{2}\commutator{\commutator{#1}{#2}}{#2}
+ \ldots} % BCH expansion up to second order
\newcommand*{\BCHfirst}[2]{\commutator{#1}{#2}}% BCH 1st order
\newcommand*{\BCHsecond}[2]{\frac{1}{2}\commutator{\commutator{#1}{#2}}{#2}}% BCH 2nd order
\newcommand*{\BCHthird}[2]{\frac{1}{6}\commutator{\commutator{\commutator{#1}{#2}}{#2}}{#2}} %BCH 3rd order
% BCH 4th order
\newcommand*{\BCHfourth}[2]{\frac{1}{24}\commutator{\commutator{\commutator{\commutator{#1}{#2}}{#2}}{#2}}{#2}}
% \Phi
\newcommand*{\RTerm}[3]{{}^{#1}R^{#2}_{#3}}
\newcommand*{\STerm}[3]{{}^{#1}S^{#2}_{#3}}
\newcommand*{\UTerm}[4]{{}^{#1}U^{#2}_{#3 #4}}
% \esp
\newcommand*{\MEPRTerm}[3]{{}^{#1}\underline{R}^{#2}_{#3}}
\newcommand*{\MEPSTerm}[3]{{}^{#1}\underline{S}^{#2}_{#3}}
\newcommand*{\MEPUTerm}[4]{{}^{#1}\underline{U}^{#2}_{#3 #4}}

% Box around equation
\newcommand{\highlight}[2]{%
  \colorbox{#1!50}{$\displaystyle#2$}}
% Color terms in equations
\newcommand*{\mathcolor}{}
\def\mathcolor#1#{\mathcoloraux{#1}}
\newcommand*{\mathcoloraux}[3]{%
  \protect\leavevmode
    \begingroup
        \color#1{#2}#3%
          \endgroup
}

% Theorem environments
%%% Definitions
\newtheoremstyle{classicdef}%             % Name
  {12pt}%                                 % Space above
  {12pt}%                                 % Space below
  {}%                                     % Body font
  {}%                                     % Indent amount
  {\bfseries\scshape}%                    % Theorem head font
  {.}%                                    % Punctuation after theorem head
  { }%                                    % Space after theorem head, ' ', or \newline
  {\thmname{#1}\thmnumber{ #2}\thmnote{ (#3)}}%

%%% Theorems, Lemmas, Corollaries
\newtheoremstyle{classicthm}%             % Name
  {12pt}%                                 % Space above
  {12pt}%                                 % Space below
  {\itshape}%                             % Body font
  {}%                                     % Indent amount
  {\bfseries\scshape}%                    % Theorem head font
  {.}%                                    % Punctuation after theorem head
  { }%                                    % Space after theorem head, ' ', or \newline
  {\thmname{#1}\thmnumber{ #2}\thmnote{ (#3)}}%

\theoremstyle{classicthm}
\newtheorem{theorem}{Theorem}
\newtheorem{lemma}{Lemma}
\newtheorem{proposition}{Proposition}
\newtheorem*{corollary}{Corollary}
\newtheorem{postulate}{Postulate}

\theoremstyle{classicdef}
\newtheorem{defin}{Definition}
\newtheorem{conjecture}{Conjecture}
\newtheorem{example}{Example}

\theoremstyle{remark}
\newtheorem*{rem}{Remark}
\newtheorem*{note}{Note}
\newtheorem{case}{Case}

\newenvironment{Proof}{\vspace*{1.5pt}\begin{proof} \small{}}{\end{proof}\vspace*{1.5pt}}
\renewenvironment{example}[1][]{\subparagraph*{Example} #1 \small{}}{}


