%%
% Microtypographic stuff
\usepackage[activate={true,nocompatibility},
            final,
            factor=1100,
            stretch=10,
            shrink=10]{microtype}

%%
% Chapter numbers
\newcommand{\chapterNumber}{%
  \fontsize{70}{70}\usefont{\encodingdefault}{\rmdefault}{b}{n}}

%%
% Own Stuff
% Disable single lines at the start of a paragraph (Schusterjungen)
\clubpenalty = 10000
% Disable single lines at the end of a paragraph (Hurenkinder)
\widowpenalty = 10000
\displaywidowpenalty = 10000 % formulas

%%
% Fancy Stuff
\usepackage{booktabs} % for better rules in tables
\usepackage{textcase} % for \MakeTextUppercase

\usepackage{soul} % for letterspacing
\sodef\allcapsspacing{\upshape}{0.15em}{0.65em}{0.6em}%
\sodef\lowsmallcapsspacing{\scshape}{0.075em}{0.5em}{0.6em}%
\DeclareRobustCommand{\spacedallcaps}[1]{\MakeTextUppercase{\allcapsspacing{#1}}}%
\DeclareRobustCommand{\spacedlowsmallcaps}[1]{\MakeTextLowercase{\textsc{\lowsmallcapsspacing{#1}}}}%\protect

%%
% Headlines
\usepackage[automark]{scrlayer-scrpage}
\clearscrheadings
\renewcommand{\chaptermark}[1]{\markboth{\spacedlowsmallcaps{#1}}{\spacedlowsmallcaps{#1}}}
\renewcommand{\sectionmark}[1]{\markright{\thesection\enspace\spacedlowsmallcaps{#1}}}
\lehead{\mbox{\llap{\small\thepage\kern2em}\headmark\hfil}}
\rohead{\mbox{\hfil{\headmark}\rlap{\small\kern2em\thepage}}}
\renewcommand{\headfont}{\small}
\def\toc@heading{%
    \chapter*{\contentsname}
    \@mkboth{\spacedlowsmallcaps{\contentsname}}{\spacedlowsmallcaps{\contentsname}}}

%%
% Listings
\usepackage{listings}
\lstset{ %
  backgroundcolor=\color{white},   % choose the background color; you must add \usepackage{color} or \usepackage{xcolor}
  basicstyle=\scriptsize\ttfamily, % the size of the fonts that are used for the code
  breakatwhitespace=false,         % sets if automatic breaks should only happen at whitespace
  breaklines=true,                 % sets automatic line breaking
  captionpos=t,                    % sets the caption-position to bottom
  commentstyle=\color{brewerComment},    % comment style
  deletekeywords={...},            % if you want to delete keywords from the given language
  escapeinside={\%*}{*)},          % if you want to add LaTeX within your code
  extendedchars=true,              % lets you use non-ASCII characters; for 8-bits encodings only, does not work with UTF-8
  frame=single,	                   % adds a frame around the code
  keepspaces=true,                 % keeps spaces in text, useful for keeping indentation of code (possibly needs columns=flexible)
  keywordstyle=\color{blue},       % keyword style
  otherkeywords={*,...},           % if you want to add more keywords to the set
  numbers=none,                    % where to put the line-numbers; possible values are (none, left, right)
  numberbychapter=true,            % numberbychapter works in listings>=1.4
  rulecolor=\color{black},         % if not set, the frame-color may be changed on line-breaks within not-black text (e.g. comments (green here))
  showspaces=false,                % show spaces everywhere adding particular underscores; it overrides 'showstringspaces'
  showstringspaces=false,          % underline spaces within strings only
  showtabs=false,                  % show tabs within strings adding particular underscores
  stepnumber=2,                    % the step between two line-numbers. If it's 1, each line will be numbered
  stringstyle=\color{brewerString},     % string literal style
  tabsize=2,                       % sets default tabsize to 2 spaces
  title=\lstname                   % show the filename of files included with \lstinputlisting; also try caption instead of title
}

%%
% Layout of the chapter-, section-, subsection-, subsubsection-,
% paragraph and description-headings
\usepackage{titlesec}
    % Chapters
    % something like Bringhurst
    \titleformat{\chapter}[display]%
        {\relax}{\raggedright{\color{brewerGray}\chapterNumber\thechapter} \\ }{0pt}%
        {\raggedright\color{brewerBlue}\LARGE\bfseries\itshape}[\color{black}\normalsize\vspace*{.8\baselineskip}\titlerule]%
    % sections \FloatBarrier
    \titleformat{\section}
        {\relax}{\textsc{\MakeTextLowercase{\thesection}}}{1em}{\spacedlowsmallcaps}
    % subsections
    \titleformat{\subsection}
        {\relax}{\textsc{\MakeTextLowercase{\thesubsection}}}{1em}{\normalsize\itshape}
    % subsubsections
    \titleformat{\subsubsection}
        {\relax}{\textsc{\MakeTextLowercase{\thesubsubsection}}}{1em}{\normalsize\itshape}
    % paragraphs
    \titleformat{\paragraph}[runin]
        {\normalfont\normalsize}{\theparagraph}{0pt}{\spacedlowsmallcaps}
    % descriptionlabels
        \renewcommand{\descriptionlabel}[1]{\hspace*{\labelsep}\spacedlowsmallcaps{#1}}   % spacedlowsmallcaps textit textsc
    % spacing
    \titlespacing*{\chapter}{0pt}{1\baselineskip}{1.2\baselineskip}
    \titlespacing*{\section}{0pt}{1.25\baselineskip}{1\baselineskip}
    \titlespacing*{\subsection}{0pt}{1.25\baselineskip}{1\baselineskip}
    \titlespacing*{\paragraph}{0pt}{1\baselineskip}{1\baselineskip}

%%
% Layout of the TOC, LOF and LOT (LOL-workaround see next section)
\usepackage[titles]{tocloft}
    % avoid page numbers being right-aligned in fixed-size box
    \newlength{\newnumberwidth}
    \settowidth{\newnumberwidth}{999} % yields overfull hbox warnings for pages > 999
    \cftsetpnumwidth{\newnumberwidth}

    % have the bib neatly positioned after the rest
    \newlength{\beforebibskip}
    \setlength{\beforebibskip}{0em}

    % space for more than nine chapters
    \newlength{\newchnumberwidth}
    \settowidth{\newchnumberwidth}{.} % <--- tweak here if more space required
    % pagenumbers right after the titles
    % chapters
                        \renewcommand{\cftchappresnum}{\scshape\MakeTextLowercase}%
            \renewcommand{\cftchapfont}{\normalfont}%
            \renewcommand{\cftchappagefont}{\normalfont}%
            %\setlength{\cftbeforechapskip}{.1em}%
    % sections
        \renewcommand{\cftsecpresnum}{\scshape\MakeTextLowercase}%
        \renewcommand{\cftsecfont}{\normalfont}%
      \renewcommand{\cftsecpagefont}{\normalfont}%
    % subsections
        \renewcommand{\cftsubsecpresnum}{\scshape\MakeTextLowercase}%
        \renewcommand{\cftsubsecfont}{\normalfont}%
    % subsubsections
        \renewcommand{\cftsubsubsecpresnum}{\scshape\MakeTextLowercase}%
        \renewcommand{\cftsubsubsecfont}{\normalfont}%
    % figures
        \renewcommand{\cftfigpresnum}{\scshape\MakeTextLowercase}%
        \renewcommand{\cftfigfont}{\normalfont}%
      \renewcommand{\cftfigpresnum}{\figurename~}%Fig.~}
      \newlength{\figurelabelwidth}
      \settowidth{\figurelabelwidth}{\cftfigpresnum~999}
      \addtolength{\figurelabelwidth}{2.5em}
      \cftsetindents{figure}{0em}{\figurelabelwidth}
    % tables
        \renewcommand{\cfttabpresnum}{\scshape\MakeTextLowercase}%
        \renewcommand{\cfttabfont}{\normalfont}%
      \renewcommand{\cfttabpresnum}{\tablename~}%Tab.~}
      \newlength{\tablelabelwidth}
      \settowidth{\tablelabelwidth}{\cfttabpresnum~999}
      \addtolength{\tablelabelwidth}{2.5em}
      %\cftsetindents{table}{0em}{\tablelabelwidth}
      \cftsetindents{table}{0em}{\figurelabelwidth}

    % dirty work-around to get the spacing after the toc/lot/lof-titles right
            \AtBeginDocument{\addtocontents{toc}{\protect\vspace{-\cftbeforechapskip}}}

    % another dirty work-around to get the spaced low small caps into the toc ;-(
%% use modified \chapter (thanks to Hinrich Harms)
         \let\oldchap=\chapter
         \renewcommand*{\chapter}{%
                 \secdef{\Chap}{\ChapS}%
         }
         \newcommand\ChapS[1]{\oldchap*{#1}}%
         \newcommand\Chap[2][]{%
                 \ifpdf\oldchap[\texorpdfstring{\spacedlowsmallcaps{#1}}{#1}]{#2}%
                 \else\oldchap[\spacedlowsmallcaps{#1}]{#2}%
                 \fi%
         }%

    \newcommand{\tocEntry}[1]{% for bib, etc.
        \ifpdf\texorpdfstring{\spacedlowsmallcaps{#1}}{#1}%
        \else{#1}\fi%
    }

        \DeclareRobustCommand*{\deactivateaddvspace}{\let\addvspace\@gobble}%
        \AtBeginDocument{%
            \addtocontents{lof}{\deactivateaddvspace}%
            \addtocontents{lot}{\deactivateaddvspace}%
                \addtocontents{lol}{\deactivateaddvspace}%
        }%
%    }

% Change font for sections in TOC
\renewcommand\cftsecfont{\footnotesize}
\renewcommand\cftsecpagefont{\footnotesize}

%%
% Drafting Stuff
\usepackage{scrtime} % time access
\newcommand{\finalVersionString}{\relax}
\providecommand{\myVersion}{$\!\!$} % w/o classicthesis-config.tex
\renewcommand{\finalVersionString}{\emph{Final Version} as of \today\ (\texttt{classicthesis}~\myVersion).}
\usepackage{scrhack}

\usepackage{polyglossia}
\setmainlanguage{english}
\setotherlanguages{italian, french, norsk, greek}
\setdefaultlanguage{english}
\usepackage[autostyle=true]{csquotes}
\usepackage[%
style=phys,
maxcitenames=1,
mincitenames=1,
maxbibnames=100,
firstinits=true,
url=false,
isbn=false,
eprint=false,
texencoding=utf8,
bibencoding=utf8,
autocite=superscript,
backend=biber,
sorting=none,
backref=false,
hyperref=true,
block=none,
date=long,
urldate=long
]{biblatex}
\renewcommand{\bibfont}{\normalfont\footnotesize\raggedright}
\AtBeginBibliography{
\DeclareFieldFormat{prefixnumber}{\mkbibbold{#1}}
\DeclareFieldFormat{labelnumber}{\mkbibbold{#1}}
}
\AtEveryBibitem{%
  \clearlist{language}%
}
\DeclareFieldFormat[article]{title}{\textsf{#1}}
\DeclareFieldFormat[inbook]{title}{\textsf{#1}}
\DeclareFieldFormat[incollection]{title}{\textsf{#1}}
\DeclareFieldFormat[inproceedings]{title}{\textsf{#1}}
\DeclareFieldFormat[inproceedings]{booktitle}{\textit{#1}}
\DeclareFieldFormat[inproceedings]{note}{#1}
\DeclareFieldFormat[unpublished]{title}{\textsf{#1}}
\DeclareFieldFormat*{citetitle}{\textsf{#1}}

\DeclareCiteCommand{\noparcite}%[\mkbibbrackets] CITATION LIKE in ref. 6 WITHOUT SQUARE BRACKETS
  {\usebibmacro{cite:init}%
   \usebibmacro{prenote}}
  {\usebibmacro{citeindex}%
   \usebibmacro{cite:comp}}
  {}
  {\usebibmacro{cite:dump}%
   \usebibmacro{postnote}}

\PassOptionsToPackage{fleqn}{amsmath}       % math environments and more by the AMS
    \usepackage{amsmath}

%%
% General useful packages
\usepackage{tabularx} % better tables
    \setlength{\extrarowheight}{3pt} % increase table row height
\newcommand{\tableheadline}[1]{\multicolumn{1}{c}{\spacedlowsmallcaps{#1}}}
\newcommand{\myfloatalign}{\centering} % to be used with each float for alignment
\usepackage{caption}
\captionsetup{format=hang,indention=-1.1cm,
              font=footnotesize,labelfont=bf,labelsep=space}
\usepackage{subcaption}
\captionsetup[sub]{format=hang,indention=0cm,
              font=footnotesize,labelfont=bf,labelsep=space}

%%
% Hyperreferences
\usepackage[hyperfootnotes=false, pdfpagelabels]{hyperref}
\hypersetup{%
    %draft, % = no hyperlinking at all (useful in b/w printouts)
    colorlinks=true, linktocpage=true, pdfstartpage=3, pdfstartview=FitV,%
    % uncomment the following line if you want to have black links (e.g., for printing)
    %colorlinks=false, linktocpage=false, pdfstartpage=3, pdfstartview=FitV, pdfborder={0 0 0},%
    breaklinks=true, pdfpagemode=UseNone, pageanchor=true, pdfpagemode=UseOutlines,%
    plainpages=false, bookmarksnumbered, bookmarksopen=true, bookmarksopenlevel=1,%
    hypertexnames=true, pdfhighlight=/O,%nesting=true,%frenchlinks,%
    urlcolor=brewerURL, linkcolor=brewerGray, citecolor=brewerCitation, %pagecolor=RoyalBlue,%
    %urlcolor=Black, linkcolor=Black, citecolor=Black, %pagecolor=Black,%
    pdftitle={\myTitle},%
    pdfauthor={\textcopyright\ \myName},%
    pdfsubject={},%
    pdfkeywords={},%
    pdfcreator={LuaLaTeX},%
    pdfproducer={LuaLaTeX with hyperref}%
}

\usepackage{graphicx}
\graphicspath{{gfx/}}

\RequirePackage{geometry}
\geometry{paperwidth=170mm, paperheight=240mm,
          inner=52pt, %outer=,
          top=80pt, %bottom=,
          textheight=500pt, textwidth=280pt,
          marginparsep=15pt, marginparwidth=120pt,
          footskip=40pt
        }
\savegeometry{Tufte}

\newlength{\drop}% for my convenience

\usepackage{metalogo}
%%
% Font stuff
%\usepackage{amssymb}
\usepackage{fontspec}
\setmainfont[Ligatures=TeX, Numbers=OldStyle, Scale=0.95]{TeX Gyre Termes}
\newfontfamily\greekfont[Script=Greek, Scale=0.95]{Linux Libertine O}
\newfontfamily\greekfontsf[Script=Greek, Scale=0.95]{Linux Libertine O}
\setsansfont[Scale=0.85]{TeX Gyre Heros}
\setmonofont[Scale=0.8]{Bitstream Vera Sans Mono}
\usepackage[math-style=ISO,
            bold-style=ISO]{unicode-math}
\setmathfont{XITS Math}
\removenolimits{\int}
\usepackage{url}
\usepackage[colorinlistoftodos,
            textsize=small]{todonotes}

\usepackage{siunitx}

%%% To include the published papers
\usepackage{pdfpages}

\usepackage[symbol, side, flushmargin, perpage]{footmisc}
\renewcommand*{\footnotelayout}{\scriptsize}
\setfnsymbol{bringhurst}
%%
% For you reference, these are Bringhurst's footnote markers:
% \textasteriskcentered
% \textdagger
% \textdaggerdbl
% \textsection
% \textbardbl
% \textparagraph

\usepackage{epigraph}
\setlength\epigraphwidth{4.5cm}
\setlength\epigraphrule{0pt}
\renewcommand{\textflush}{flushleft}
\renewcommand{\epigraphsize}{\footnotesize}

\usepackage[overload]{empheq}
\usepackage{braket}
\usepackage{cancel}
\usepackage{amsthm}

% Comment before compiling final version
%\usepackage{showkeys}
\usepackage{verbatim}

\usepackage[inline]{enumitem}

\usepackage[version=4]{mhchem}
\usepackage{xpatch}

\usepackage{tikz}
\usepackage{tikz-3dplot}
\usetikzlibrary{shapes, arrows, shadows, positioning}
\usepackage{tcolorbox}

\usepackage[style=long,
            nolist,
            nonumberlist,
            acronym,
            shortcuts,
            nopostdot]{glossaries}
\makeglossaries
\usepackage[noprefix]{nomencl}
\makenomenclature
