%%
% Microtypographic stuff
\usepackage[activate={true,nocompatibility},
            final,
            factor=1100,
            stretch=10,
            shrink=10]{microtype}

%%
% Chapter numbers
\newcommand{\chapterNumber}{%
  \fontsize{70}{70}\usefont{\encodingdefault}{\rmdefault}{b}{n}}

%%
% Own Stuff
% Disable single lines at the start of a paragraph (Schusterjungen)
\clubpenalty = 10000
% Disable single lines at the end of a paragraph (Hurenkinder)
\widowpenalty = 10000
\displaywidowpenalty = 10000 % formulas

%%
% Fancy Stuff
\usepackage{booktabs} % for better rules in tables
\usepackage{textcase} % for \MakeTextUppercase

\usepackage{soul} % for letterspacing
\sodef\allcapsspacing{\upshape}{0.15em}{0.65em}{0.6em}%
\sodef\lowsmallcapsspacing{\scshape}{0.075em}{0.5em}{0.6em}%
\DeclareRobustCommand{\spacedallcaps}[1]{\MakeTextUppercase{\allcapsspacing{#1}}}%
\DeclareRobustCommand{\spacedlowsmallcaps}[1]{\MakeTextLowercase{\textsc{\lowsmallcapsspacing{#1}}}}%\protect

%%
% Headlines
\usepackage[automark]{scrlayer-scrpage}
\clearscrheadings
\renewcommand{\chaptermark}[1]{\markboth{\spacedlowsmallcaps{#1}}{\spacedlowsmallcaps{#1}}}
\renewcommand{\sectionmark}[1]{\markright{\thesection\enspace\spacedlowsmallcaps{#1}}}
\lehead{\mbox{\llap{\small\thepage\kern2em}\headmark\hfil}}
\rohead{\mbox{\hfil{\headmark}\rlap{\small\kern2em\thepage}}}
\renewcommand{\headfont}{\small}
\def\toc@heading{%
    \chapter*{\contentsname}
    \@mkboth{\spacedlowsmallcaps{\contentsname}}{\spacedlowsmallcaps{\contentsname}}}

%%
% Listings
\usepackage{listings}
\lstset{ %
  backgroundcolor=\color{white},   % choose the background color; you must add \usepackage{color} or \usepackage{xcolor}
  basicstyle=\scriptsize\ttfamily, % the size of the fonts that are used for the code
  breakatwhitespace=false,         % sets if automatic breaks should only happen at whitespace
  breaklines=true,                 % sets automatic line breaking
  captionpos=t,                    % sets the caption-position to bottom
  commentstyle=\color{brewerComment},    % comment style
  deletekeywords={...},            % if you want to delete keywords from the given language
  escapeinside={\%*}{*)},          % if you want to add LaTeX within your code
  extendedchars=true,              % lets you use non-ASCII characters; for 8-bits encodings only, does not work with UTF-8
  frame=single,	                   % adds a frame around the code
  keepspaces=true,                 % keeps spaces in text, useful for keeping indentation of code (possibly needs columns=flexible)
  keywordstyle=\color{blue},       % keyword style
  otherkeywords={*,...},           % if you want to add more keywords to the set
  numbers=none,                    % where to put the line-numbers; possible values are (none, left, right)
  numberbychapter=true,            % numberbychapter works in listings>=1.4
  rulecolor=\color{black},         % if not set, the frame-color may be changed on line-breaks within not-black text (e.g. comments (green here))
  showspaces=false,                % show spaces everywhere adding particular underscores; it overrides 'showstringspaces'
  showstringspaces=false,          % underline spaces within strings only
  showtabs=false,                  % show tabs within strings adding particular underscores
  stepnumber=2,                    % the step between two line-numbers. If it's 1, each line will be numbered
  stringstyle=\color{brewerString},     % string literal style
  tabsize=2,                       % sets default tabsize to 2 spaces
  title=\lstname                   % show the filename of files included with \lstinputlisting; also try caption instead of title
}

%%
% Layout of the chapter-, section-, subsection-, subsubsection-,
% paragraph and description-headings
\usepackage{titlesec}
    % Chapters
    % something like Bringhurst
    \titleformat{\chapter}[display]%
        {\relax}{\raggedright{\color{brewerGray}\chapterNumber\thechapter} \\ }{0pt}%
        {\raggedright\color{brewerBlue}\LARGE\bfseries\itshape}[\color{black}\normalsize\vspace*{.8\baselineskip}\titlerule]%
    % sections \FloatBarrier
    \titleformat{\section}
        {\relax}{\textsc{\MakeTextLowercase{\thesection}}}{1em}{\spacedlowsmallcaps}
    % subsections
    \titleformat{\subsection}
        {\relax}{\textsc{\MakeTextLowercase{\thesubsection}}}{1em}{\normalsize\itshape}
    % subsubsections
    \titleformat{\subsubsection}
        {\relax}{\textsc{\MakeTextLowercase{\thesubsubsection}}}{1em}{\normalsize\itshape}
    % paragraphs
    \titleformat{\paragraph}[runin]
        {\normalfont\normalsize}{\theparagraph}{0pt}{\spacedlowsmallcaps}
    % descriptionlabels
        \renewcommand{\descriptionlabel}[1]{\hspace*{\labelsep}\spacedlowsmallcaps{#1}}   % spacedlowsmallcaps textit textsc
    % spacing
    \titlespacing*{\chapter}{0pt}{1\baselineskip}{1.2\baselineskip}
    \titlespacing*{\section}{0pt}{1.25\baselineskip}{1\baselineskip}
    \titlespacing*{\subsection}{0pt}{1.25\baselineskip}{1\baselineskip}
    \titlespacing*{\paragraph}{0pt}{1\baselineskip}{1\baselineskip}

%%
% Layout of the TOC, LOF and LOT (LOL-workaround see next section)
\usepackage[titles]{tocloft}
    % avoid page numbers being right-aligned in fixed-size box
    \newlength{\newnumberwidth}
    \settowidth{\newnumberwidth}{999} % yields overfull hbox warnings for pages > 999
    \cftsetpnumwidth{\newnumberwidth}

    % have the bib neatly positioned after the rest
    \newlength{\beforebibskip}
    \setlength{\beforebibskip}{0em}

    % space for more than nine chapters
    \newlength{\newchnumberwidth}
    \settowidth{\newchnumberwidth}{.} % <--- tweak here if more space required
    % pagenumbers right after the titles
    % chapters
                        \renewcommand{\cftchappresnum}{\scshape\MakeTextLowercase}%
            \renewcommand{\cftchapfont}{\normalfont}%
            \renewcommand{\cftchappagefont}{\normalfont}%
            %\setlength{\cftbeforechapskip}{.1em}%
    % sections
        \renewcommand{\cftsecpresnum}{\scshape\MakeTextLowercase}%
        \renewcommand{\cftsecfont}{\normalfont}%
      \renewcommand{\cftsecpagefont}{\normalfont}%
    % subsections
        \renewcommand{\cftsubsecpresnum}{\scshape\MakeTextLowercase}%
        \renewcommand{\cftsubsecfont}{\normalfont}%
    % subsubsections
        \renewcommand{\cftsubsubsecpresnum}{\scshape\MakeTextLowercase}%
        \renewcommand{\cftsubsubsecfont}{\normalfont}%
    % figures
        \renewcommand{\cftfigpresnum}{\scshape\MakeTextLowercase}%
        \renewcommand{\cftfigfont}{\normalfont}%
      \renewcommand{\cftfigpresnum}{\figurename~}%Fig.~}
      \newlength{\figurelabelwidth}
      \settowidth{\figurelabelwidth}{\cftfigpresnum~999}
      \addtolength{\figurelabelwidth}{2.5em}
      \cftsetindents{figure}{0em}{\figurelabelwidth}
    % tables
        \renewcommand{\cfttabpresnum}{\scshape\MakeTextLowercase}%
        \renewcommand{\cfttabfont}{\normalfont}%
      \renewcommand{\cfttabpresnum}{\tablename~}%Tab.~}
      \newlength{\tablelabelwidth}
      \settowidth{\tablelabelwidth}{\cfttabpresnum~999}
      \addtolength{\tablelabelwidth}{2.5em}
      %\cftsetindents{table}{0em}{\tablelabelwidth}
      \cftsetindents{table}{0em}{\figurelabelwidth}

    % dirty work-around to get the spacing after the toc/lot/lof-titles right
            \AtBeginDocument{\addtocontents{toc}{\protect\vspace{-\cftbeforechapskip}}}

    % another dirty work-around to get the spaced low small caps into the toc ;-(
%% use modified \chapter (thanks to Hinrich Harms)
         \let\oldchap=\chapter
         \renewcommand*{\chapter}{%
                 \secdef{\Chap}{\ChapS}%
         }
         \newcommand\ChapS[1]{\oldchap*{#1}}%
         \newcommand\Chap[2][]{%
                 \ifpdf\oldchap[\texorpdfstring{\spacedlowsmallcaps{#1}}{#1}]{#2}%
                 \else\oldchap[\spacedlowsmallcaps{#1}]{#2}%
                 \fi%
         }%

    \newcommand{\tocEntry}[1]{% for bib, etc.
        \ifpdf\texorpdfstring{\spacedlowsmallcaps{#1}}{#1}%
        \else{#1}\fi%
    }

        \DeclareRobustCommand*{\deactivateaddvspace}{\let\addvspace\@gobble}%
        \AtBeginDocument{%
            \addtocontents{lof}{\deactivateaddvspace}%
            \addtocontents{lot}{\deactivateaddvspace}%
                \addtocontents{lol}{\deactivateaddvspace}%
        }%
%    }

%%
% Footnotes setup
\ifdefined\deffootnote
  % KOMA-command, footnotemark not superscripted at the bottom
  \deffootnote{0em}{0em}{\thefootnotemark\hspace*{.5em}}%
  \message{Using KOMA-command "deffootnote" for footnote setup}%
\else
    \PassOptionsToPackage{flushmargin}{footmisc}%
    \RequirePackage{footmisc}%
    \setlength{\footnotemargin}{-.5em}%
    \PackageWarningNoLine{classicthesis}{Using package "footmisc" with option %
        "flushmargin" for footnote setup (not 100\% the same as with KOMA)}%
\fi

%%
% Drafting Stuff
\usepackage{scrtime} % time access
\newcommand{\finalVersionString}{\relax}
\providecommand{\myVersion}{$\!\!$} % w/o classicthesis-config.tex
\renewcommand{\finalVersionString}{\emph{Final Version} as of \today\ (\texttt{classicthesis}~\myVersion).}
